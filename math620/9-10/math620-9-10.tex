\documentclass[12pt, psamsfonts]{amsart}

%-------Packages---------
\usepackage{amssymb,amsfonts}
\usepackage{fullpage}
\usepackage{physics}
\usepackage[all,arc]{xy}
\usepackage{enumerate}
\usepackage{mathrsfs}
\usepackage{theoremref}
\usepackage{graphicx}
\usepackage[bookmarks]{hyperref}

%--------Theorem Environments--------
%theoremstyle{plain} --- default
\newtheorem{thm}{Theorem}[section]
\newtheorem{cor}[thm]{Corollary}
\newtheorem{prop}[thm]{Proposition}
\newtheorem{lem}[thm]{Lemma}
\newtheorem{conj}[thm]{Conjecture}
\newtheorem{quest}[thm]{Question}

\theoremstyle{definition}
\newtheorem{defn}[thm]{Definition}
\newtheorem{defns}[thm]{Definitions}
\newtheorem{con}[thm]{Construction}
\newtheorem{exmp}[thm]{Example}
\newtheorem{exmps}[thm]{Examples}
\newtheorem{notn}[thm]{Notation}
\newtheorem{notns}[thm]{Notations}
\newtheorem{addm}[thm]{Addendum}
\newtheorem*{exer}{Exercise}

\theoremstyle{remark}
\newtheorem{rem}[thm]{Remark}
\newtheorem{rems}[thm]{Remarks}
\newtheorem{warn}[thm]{Warning}
\newtheorem{sch}[thm]{Scholium}

\DeclareMathOperator{\Hom}{Hom}
\DeclareMathOperator{\Id}{Id}
\DeclareMathOperator{\sgn}{sgn}

\makeatletter
\let\c@equation\c@thm
\makeatother
\numberwithin{equation}{section}

\bibliographystyle{plain}

\begin{document}

\title{Math 620 Homework (Due 9/10)}
\author{Hidenori Shinohara}
\maketitle

\begin{exer}
  Show that $F_*: T_p\mathbb{R}^n \rightarrow T_q\mathbb{R}^m$.
\end{exer}

\begin{proof}
  Let $v_1, v_2 \in T_pU, c \in \mathbb{R}$.
  Then $v_1 = c_1^j\frac{\partial}{\partial x^j} \mid_p, v_2 = c_2^j\frac{\partial}{\partial x^j} \mid_p$ where $c_i^j \in \mathbb{R}$.
  Let $\gamma_1(t) = p + t(c_1^1, \cdots, c_1^n), \gamma_2(t) = p + t(c_2^1, \cdots, c_2^n), \gamma = c\gamma_1 + \gamma_2$.
  Then there exist unique $b_1^1, \cdots, b_1^m, b_2^1, \cdots, b_2^m, b^1, \cdots, b^m \in \mathbb{R}$ such that
  \begin{itemize}
    \item
      $F_*(v_1) = b_1^s\frac{\partial}{\partial y^s}$.
    \item
      $F_*(v_2) = b_2^s\frac{\partial}{\partial y^s}$.
    \item
      $F_*(cv_1 + v_2) = b^s\frac{\partial}{\partial y^s}$.
  \end{itemize}
  For each $s$,
  \begin{align*}
    b_s
      &= (F_*(cv_1 + v_2))(y^s) \\
      &= \frac{d}{dt} y^s \circ F \circ \gamma(t) \Big\vert_{t = 0} \\
      &= \frac{d}{dt} F^s \circ \gamma(t) \Big\vert_{t = 0} & \text{(Let $F^s = y^s \circ F$.)} \\
      &= \frac{\partial F^s}{\partial x^j} \Big\vert_p (cc_1^j + c_2^j) \\
      &= c\frac{\partial F^s}{\partial x^j} \Big\vert_p c_1^j + \frac{\partial F^s}{\partial x^j} \Big\vert_p c_2^j \\
      &= c\frac{d}{dt} F^s \circ \gamma_1(t) \Big\vert_p c_1^j + \frac{d}{dt} F^s \circ \gamma_2(t) \Big\vert_p c_2^j \\
      &= c(F_*v_1)(y^s) + (F_*v_2)(y^s) \\
      &= cb_1^s+ b_2^s.
  \end{align*}
  Therefore, $F_*(cv_1 + v_2) = cF_*(v_1) + F_*(v_2)$.
\end{proof}

\begin{exer}
  Prove that if $f_I \in \mathscr{C}^{\infty}$, then $f_Idx^I \in \mathcal{A}^k$.
\end{exer}

\begin{proof}
  Let $\eta = \sum_{I} f_Idx^I$.
  Let $X_1, \cdots, X_k \in \mathfrak{X}(\mathbb{R}^n)$.
  We must show that $F: \mathbb{R}^n \rightarrow \mathbb{R}$ defined by $F(p) = \eta_p(X_{1, p}, \cdots, X_{k, p})$ is smooth.
  For any $p \in \mathbb{R}^n$,
  \begin{align*}
    F(p)
      &= \sum_{I} \eta_p(X_{1, p}, \cdots, X_{k, p}) \\
      &= \sum_{I} f_I(p)(dx^{i_1}\vert_p \wedge \cdots \wedge dx^{i_k}\vert_p)(X_{1, p}, \cdots, X_{k, p}) \\
      &= \sum_{I} f_I(p) \sum_{\sigma \in S_k} (dx^{i_{\sigma_1}}\vert_p)(X_{1, p}) \cdots (dx^{i_{\sigma_k}}\vert_p(X_{k, p})).
  \end{align*}

  Since products and sums of smooth functions are smooth, it suffices to show $p \mapsto dx^{i}\vert_p(X_{j, p})$ is smooth for each $i, j$.
  Then $dx^i\vert_p(X_{j, p}) = X_{j, p}(x^i)$, which is smooth because $\mathfrak{X}$ is defined to be the collection of all smooth vector fields.
\end{proof}

\begin{exer}
  Given $\eta \in \mathscr{A}^k(V), \omega \in \mathscr{A}^l(V)$, prove that $F^*(\eta \wedge \omega) = (F^*\eta) \wedge (F^*\omega)$.
\end{exer}

\begin{proof}
  Let $p \in V, v_1, \cdots, v_{k + l} \in V$.
  \begin{align*}
    (F^*(\eta \wedge \omega))_p(v_1, \cdots, v_{k + l})
      &= (\eta \wedge \omega)_p(F_*v_1, \cdots, F_*v_{k + l}) \\
      &= \frac{1}{k!l!}\sum_{\sigma \in S_{k + l}} \sgn(\sigma) \eta_p(F_*v_{\sigma_1}, \cdots, F_*v_{\sigma_k})\omega_p(F_*v_{\sigma_{k + 1}}, \cdots, F_*v_{\sigma_{k + l}}) \\
      &= \frac{1}{k!l!}\sum_{\sigma \in S_{k + l}} \sgn(\sigma) (F^*\eta)_p(v_{\sigma_1}, \cdots, v_{\sigma_k})(F^*\omega)_p(v_{\sigma_{k + 1}}, \cdots, v_{\sigma_{k + l}}) \\
      &= ((F^*\eta) \wedge (F^*\omega))_p(v_1, \cdots, v_{k + l}).
  \end{align*}
\end{proof}

\begin{exer}
  Define $F: \mathbb{R}^2_{(s, t)} \rightarrow \mathbb{R}^3_{(x, y, z)}$ such that $F(s, t) = (s^2, st, t^2)$.
  Compute the following:
  \begin{enumerate}
    \item
      $F^*(xyz)$.
    \item
      $F^*(xydz + yzdx + zxdy)$.
    \item
      $F^*(dx \wedge dy - zdx \wedge dz + y^2dy \wedge dz)$.
    \item
      $F^*(dx \wedge dy \wedge dz)$.
  \end{enumerate}
\end{exer}

\begin{proof}
  We have
  \begin{itemize}
    \item
      $F^*x = s^2$,
    \item
      $F^*y = st$,
    \item
      $F^*z = t^2$.
  \end{itemize}

  Therefore, 
  \begin{itemize}
    \item
      $F^*dx = 2sds$,
    \item
      $F^*dy = tds + sdt$,
    \item
      $F^*dz = 2tdt$.
  \end{itemize}

  \begin{enumerate}
    \item
      $F^*(xyz) = (s^2)(st)(t^2) = (st)^3$.
    \item
      \begin{align*}
        F^*(xydz + yzdx + zxdy)
          &= s^2(st)(2tdt) + (st)t^2(2sds) + t^2s^2(tds + sdt) \\
          &= 3t^2s^3dt + 3s^2t^3ds.
      \end{align*}
    \item
      \begin{align*}
        &F^*(dx \wedge dy - zdx \wedge dz + y^2dy \wedge dz) \\
          &= F^*(dx) \wedge F^*(dy) - F^*(zdx) \wedge F^*(dz) + F^*(y^2dy) \wedge F^*(dz) \\
          &= 2sds \wedge (tds + sdt) - (2st^2ds) \wedge 2tdt + (st)^2(tds + sdt) \wedge 2tdt. \\
          &= 2sds \wedge sdt - (2st^2ds) \wedge 2tdt + s^2t^3ds \wedge 2tdt. \\
          &= 2s^2(ds \wedge dt) - (4st^3)(ds \wedge dt) + 2s^2t^4(ds \wedge dt). \\
          &= (2s^2 - 4st^3 + 2s^2t^4)(ds \wedge dt).
      \end{align*}
    \item
      $F^*(dx \wedge dy \wedge dz) = F^*(dx) \wedge F^*(dy) \wedge F^*(dz) = 2sds \wedge (tds + sdt) \wedge 2tdt = 0$ because the dimension of the vector space is 2 and that is smaller than the number of variables, 3.
  \end{enumerate}
\end{proof}

\end{document}
