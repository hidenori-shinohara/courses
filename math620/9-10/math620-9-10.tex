\documentclass[12pt, psamsfonts]{amsart}

%-------Packages---------
\usepackage{amssymb,amsfonts}
\usepackage{fullpage}
\usepackage{physics}
\usepackage[all,arc]{xy}
\usepackage{enumerate}
\usepackage{mathrsfs}
\usepackage{theoremref}
\usepackage{graphicx}
\usepackage[bookmarks]{hyperref}

%--------Theorem Environments--------
%theoremstyle{plain} --- default
\newtheorem{thm}{Theorem}[section]
\newtheorem{cor}[thm]{Corollary}
\newtheorem{prop}[thm]{Proposition}
\newtheorem{lem}[thm]{Lemma}
\newtheorem{conj}[thm]{Conjecture}
\newtheorem{quest}[thm]{Question}

\theoremstyle{definition}
\newtheorem{defn}[thm]{Definition}
\newtheorem{defns}[thm]{Definitions}
\newtheorem{con}[thm]{Construction}
\newtheorem{exmp}[thm]{Example}
\newtheorem{exmps}[thm]{Examples}
\newtheorem{notn}[thm]{Notation}
\newtheorem{notns}[thm]{Notations}
\newtheorem{addm}[thm]{Addendum}
\newtheorem*{exer}{Exercise}

\theoremstyle{remark}
\newtheorem{rem}[thm]{Remark}
\newtheorem{rems}[thm]{Remarks}
\newtheorem{warn}[thm]{Warning}
\newtheorem{sch}[thm]{Scholium}

\DeclareMathOperator{\Hom}{Hom}
\DeclareMathOperator{\Id}{Id}

\makeatletter
\let\c@equation\c@thm
\makeatother
\numberwithin{equation}{section}

\bibliographystyle{plain}

\begin{document}

\title{Math 620 Homework (Due 9/10)}
\author{Hidenori Shinohara}
\maketitle

\begin{exer}
  Show that $F_*: T_p\mathbb{R}^n \rightarrow T_q\mathbb{R}^m$.
\end{exer}

\begin{proof}
  Let $v_1, v_2 \in T_pU, c \in \mathbb{R}$.
  Then $v_1 = c_1^j\frac{\partial}{\partial x^j} \mid_p, v_2 = c_2^j\frac{\partial}{\partial x^j} \mid_p$ where $c_i^j \in \mathbb{R}$.
  Let $\gamma_1(t) = p + t(c_1^1, \cdots, c_1^n), \gamma_2(t) = p + t(c_2^1, \cdots, c_2^n), \gamma = c\gamma_1 + \gamma_2$.
  Then there exist unique $b_1^1, \cdots, b_1^m, b_2^1, \cdots, b_2^m, b^1, \cdots, b^m \in \mathbb{R}$ such that
  \begin{itemize}
    \item
      $F_*(v_1) = b_1^s\frac{\partial}{\partial y^s}$.
    \item
      $F_*(v_2) = b_2^s\frac{\partial}{\partial y^s}$.
    \item
      $F_*(cv_1 + v_2) = b^s\frac{\partial}{\partial y^s}$.
  \end{itemize}
  For each $s$,
  \begin{align*}
    b_s
      &= (F_*(cv_1 + v_2))(y^s) \\
      &= \frac{d}{dt} y^s \circ F \circ \gamma(t) \Big\vert_{t = 0} \\
      &= \frac{d}{dt} F^s \circ \gamma(t) \Big\vert_{t = 0} & \text{(Let $F^s = y^s \circ F$.)} \\
      &= \frac{\partial F^s}{\partial x^j} \Big\vert_p (cc_1^j + c_2^j) \\
      &= c\frac{\partial F^s}{\partial x^j} \Big\vert_p c_1^j + \frac{\partial F^s}{\partial x^j} \Big\vert_p c_2^j \\
      &= c\frac{d}{dt} F^s \circ \gamma_1(t) \Big\vert_p c_1^j + \frac{d}{dt} F^s \circ \gamma_2(t) \Big\vert_p c_2^j \\
      &= c(F_*v_1)(y^s) + (F_*v_2)(y^s) \\
      &= cb_1^s+ b_2^s.
  \end{align*}
  Therefore, $F_*(cv_1 + v_2) = cF_*(v_1) + F_*(v_2)$.
\end{proof}

\begin{exer}
  Prove that if $f_i \in \mathscr{C}^{\infty}$, then $f_Idx^I \in \mathcal{A}^k$.
\end{exer}

\begin{proof}
  Let $\eta = f_Idx^I$ and let $\zeta = dx^I$.
  Let $X_1, \cdots, X_k \in \mathfrak{X}(\mathbb{R}^n)$.
  We must show that $F: \mathbb{R}^n \rightarrow \mathbb{R}$ defined by $F(p) = \eta_p(X_{1, p}, \cdots, X_{k, p})$ is smooth.
  For any $p \in \mathbb{R}^n$,
  \begin{align*}
    F(p)
      &= \eta_p(X_{1, p}, \cdots, X_{k, p}) \\
      &= (f_{i_1}(p)dx^{i_1}\vert_p \wedge \cdots \wedge f_{i_k}(p)dx^{i_k}\vert_p)(X_{1, p}, \cdots, X_{k, p}) \\
      &= \sum_{\sigma \in S_k} (f_{i_{\sigma_1}}(p) dx^{i_{\sigma_1}}\vert_p)(X_{1, p}) \cdots (f_{i_{\sigma_k}}(p) dx^{i_{\sigma_k}}\vert_p)(X_{k, p}) \\
      &= (f_1(p) \cdots f_k(p))\sum_{\sigma \in S_k} (dx^{i_{\sigma_1}}\vert_p)(X_{1, p}) \cdots (dx^{i_{\sigma_k}}\vert_p)(X_{k, p}) \\
      &= (f_1(p) \cdots f_k(p))\zeta_p(X_{1, p}, \cdots, X_{k, p}).
  \end{align*}

  As discussed in the lecture, $\zeta = dx^I \in \mathcal{A}^k(\mathbb{R}^n)$.
  Thus the mapping $p \mapsto \zeta_p(X_{1, p}, \cdots, X_{k, p})$ must be smooth.
  Since each $f_i$ is smooth and the product of smooth functions is smooth, $p \mapsto f_1(p) \cdots f_k(p)\zeta_p(X_{1, p}, \cdots, X_{k, p})$ is smooth.
  Therefore, $F$ is smooth, so $\eta = f_Idx^I \in \mathcal{A}^k$.
\end{proof}


\end{document}


