\documentclass[12pt, psamsfonts]{amsart}

%-------Packages---------
\usepackage{amssymb,amsfonts}
\usepackage{fullpage}
\usepackage{physics}
\usepackage[all,arc]{xy}
\usepackage{enumerate}
\usepackage{mathrsfs}
\usepackage{theoremref}
\usepackage{graphicx}
\usepackage[bookmarks]{hyperref}

%--------Theorem Environments--------
%theoremstyle{plain} --- default
\newtheorem{thm}{Theorem}[section]
\newtheorem{cor}[thm]{Corollary}
\newtheorem{prop}[thm]{Proposition}
\newtheorem{lem}[thm]{Lemma}
\newtheorem{conj}[thm]{Conjecture}
\newtheorem{quest}[thm]{Question}

\theoremstyle{definition}
\newtheorem{defn}[thm]{Definition}
\newtheorem{defns}[thm]{Definitions}
\newtheorem{con}[thm]{Construction}
\newtheorem{exmp}[thm]{Example}
\newtheorem{exmps}[thm]{Examples}
\newtheorem{notn}[thm]{Notation}
\newtheorem{notns}[thm]{Notations}
\newtheorem{addm}[thm]{Addendum}
\newtheorem*{exer}{Exercise}

\theoremstyle{remark}
\newtheorem{rem}[thm]{Remark}
\newtheorem{rems}[thm]{Remarks}
\newtheorem{warn}[thm]{Warning}
\newtheorem{sch}[thm]{Scholium}

\DeclareMathOperator{\Hom}{Hom}
\DeclareMathOperator{\Id}{Id}

\makeatletter
\let\c@equation\c@thm
\makeatother
\numberwithin{equation}{section}

\bibliographystyle{plain}

\begin{document}

\title{Math 620 Homework (Due 9/10)}
\author{Hidenori Shinohara}
\maketitle

\begin{exer}
  Show that $F_*: T_p\mathbb{R}^n \rightarrow T_q\mathbb{R}^m$.
\end{exer}

\begin{proof}
** TODO **
This is probably wrong because $F$ is not multi-valued function, so the expression like $F'(p)v_2)$ makes little sense?
  Let $v_1, v_2 \in T_p\mathbb{R}^n, c \in \mathbb{R}$ be given.
  We will show that $F_*(cv_1 + v_2) = cF_*(v_1) + F_*(V_2)$.
  Choose $\gamma_1, \gamma_2$ be paths in $U$ defined on a neighborhood of $0$ in $\mathbb{R}$ such that $\gamma_1(0) = \gamma_2(0) = p, \gamma_1'(0) = v_1$ and $\gamma_2'(0) = v_2$.
  Then $F_*(v_1) = (F(\gamma_1(t)))'\mid_{t = 0} = F'(\gamma_1(0))\gamma_1'(0) = F'(p)v_1$, and $F_*(v_2) = (F(\gamma_2(t)))'\mid_{t = 0} = F'(\gamma_2(0))\gamma_2'(0) = F'(p)v_2$.
  Let $\gamma_3: \mathbb{R} \rightarrow U$ be the constant path at $p$.
  Let $\gamma = c(\gamma_1 - \gamma_3) + \gamma_2$.
  \begin{itemize}
    \item
      $\gamma(0) = c\gamma_1(0) - c\gamma(0) + \gamma_2(0) = p$.
    \item
      $\gamma'(0) = c\gamma_1'(0) + \gamma_2'(0) = cv_1 + v_2$.
  \end{itemize}
  Therefore, $F_*(cv_1 + v_2) = (F \circ (c(\gamma_1 - \gamma_3) + \gamma_2))'(0) = F'(p)(c\gamma_1'(0) + \gamma_2'(0)) = F'(p)(cv_1 + v_2)$.

  Hence, $F_*(cv_1 + v_2) = cF_*(v_1) + F_*(v_2)$, so $F_*$ is indeed linear.
\end{proof}

\end{document}


