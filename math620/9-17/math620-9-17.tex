\documentclass[12pt, psamsfonts]{amsart}

%-------Packages---------
\usepackage{amssymb,amsfonts}
\usepackage{fullpage}
\usepackage{todonotes}
\usepackage{physics}
\usepackage[all,arc]{xy}
\usepackage{enumerate}
\usepackage{mathrsfs}
\usepackage{theoremref}
\usepackage{graphicx}
\usepackage[bookmarks]{hyperref}

%--------Theorem Environments--------
%theoremstyle{plain} --- default
\newtheorem{thm}{Theorem}[section]
\newtheorem{cor}[thm]{Corollary}
\newtheorem{prop}[thm]{Proposition}
\newtheorem{lem}[thm]{Lemma}
\newtheorem{conj}[thm]{Conjecture}
\newtheorem{quest}[thm]{Question}

\theoremstyle{definition}
\newtheorem{defn}[thm]{Definition}
\newtheorem{defns}[thm]{Definitions}
\newtheorem{con}[thm]{Construction}
\newtheorem{exmp}[thm]{Example}
\newtheorem{exmps}[thm]{Examples}
\newtheorem{notn}[thm]{Notation}
\newtheorem{notns}[thm]{Notations}
\newtheorem{addm}[thm]{Addendum}
\newtheorem*{exer}{Exercise}

\theoremstyle{remark}
\newtheorem{rem}[thm]{Remark}
\newtheorem{rems}[thm]{Remarks}
\newtheorem{warn}[thm]{Warning}
\newtheorem{sch}[thm]{Scholium}

\DeclareMathOperator{\Hom}{Hom}
\DeclareMathOperator{\Aut}{Aut}
\DeclareMathOperator{\Id}{Id}

\makeatletter
\let\c@equation\c@thm
\makeatother
\numberwithin{equation}{section}

\bibliographystyle{plain}

\begin{document}

\title{Math 620 (9/17)}
\author{Hidenori Shinohara}
\maketitle

\begin{exer}
  Prove $\omega = g^{-1}dg$.
\end{exer}

\begin{proof}
  \begin{align*}
    g^{-1}dg
      &= g^{-1}d\begin{bmatrix} 1 & 0 \\ x & A \end{bmatrix} \\
      &= g^{-1}\begin{bmatrix} 0 & 0 \\ dx & dA \end{bmatrix} \\
      &= \begin{bmatrix} 1 & 0 \\ x & A \end{bmatrix}^{-1}\begin{bmatrix} 0 & 0 \\ dx & dA \end{bmatrix} \\
      &= \begin{bmatrix} 1 & 0 \\ x & A \end{bmatrix}^{-1}\begin{bmatrix} 0 & 0 \\ \omega^iA_i & \omega^j_i A_j \end{bmatrix} \\
      &= \begin{bmatrix} 1 & 0 \\ -A^{-1}x & A^{-1} \end{bmatrix}\begin{bmatrix} 0 & 0 \\ \omega^iA_i & \omega^j_i A_j \end{bmatrix} \\
      &= \begin{bmatrix} 0 & 0 \\ \omega^i & \omega^i_j \end{bmatrix} \\
      &= \omega.
  \end{align*}
\end{proof}

\begin{exer}
  Prove that $V \times \Aut(V)$ is a group.
\end{exer}

\begin{proof}
  $ $
  \begin{itemize}
    \item
      Associativity.
      Let $(u, \phi), (v, \psi), (w, \rho)$ be given.
      \begin{align*}
        ((u, \phi) \cdot (v, \psi)) \cdot (w, \rho)
          &= (u + \phi v, \phi \circ \psi) \cdot (w, \rho) \\
          &= (u + \phi v + (\phi \circ \psi) w, (\phi \circ \psi) \circ \rho) \\
          &= (u + \phi v + (\phi \circ \psi) w, \phi \circ (\psi \circ \rho)) \\
          &= (u + \phi v + \phi(\psi w), \phi \circ (\psi \circ \rho)) \\
          &= (u + \phi(v + \psi w), \phi \circ (\psi \circ \rho)) \\
          &= (u, \phi) \cdot (v + \psi w, \psi \circ \rho) \\
          &= (u, \phi) \cdot ((v, \psi) \cdot (w, \rho)).
      \end{align*}
    \item
      Identity.
      Let $u = 0, \phi = \Id$.
      Then for any $(v, \psi) \in V \times \Aut(V)$,
      \begin{itemize}
        \item
          $(u, \phi) \cdot (v, \psi) = (u + \phi v, \phi \circ \psi) = (0 + \Id \circ v, \Id \circ \psi) = (v, \psi)$,
        \item
          $(v, psi) \cdot (u, \phi) = (v + \psi u, \psi \circ \phi) = (v + \Id \circ u, \psi \circ \Id) = (v, \psi)$.
      \end{itemize}

      Thus $(u, \phi)$ is the identity.
    \item
      Inverse.
      Let $(u, \phi) \in V \times \Aut(V)$ be given.
      Then $\phi^{-1} \in \Aut(V)$, and thus $-\phi^{-1}(u) \in V$.
      \begin{itemize}
        \item
          $(-\phi^{-1}(u), \phi^{-1}) \cdot (u, \phi) = (-\phi^{-1}(u) + \phi^{-1}(u), \phi^{-1} \circ \phi) = (0, \Id)$.
        \item
          $(u, \phi) \cdot (-\phi^{-1}(u), \phi^{-1}) = (u + \phi(-\phi^{-1}(u)), \phi \circ \phi^{-1}) = (0, \Id)$.
      \end{itemize}
  \end{itemize}
  Therefore, $V \times \Aut(V)$ forms a group.
\end{proof}

\end{document}


