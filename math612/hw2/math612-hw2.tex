\documentclass[12pt, psamsfonts]{amsart}

%-------Packages---------
\usepackage{amssymb,amsfonts}
\usepackage{semantic}
\usepackage{fullpage}
\usepackage{tikz-cd}
\usepackage{todonotes}
\usepackage{physics}
\usepackage[all,arc]{xy}
\usepackage{enumerate}
\usepackage{enumitem}
\usepackage{mathrsfs}
\usepackage{theoremref}
\usepackage{graphicx}
\usepackage[bookmarks]{hyperref}

%--------Theorem Environments--------
%theoremstyle{plain} --- default
\newtheorem{thm}{Theorem}[section]
\newtheorem{cor}[thm]{Corollary}
\newtheorem{prop}[thm]{Proposition}
\newtheorem{lem}[thm]{Lemma}
\newtheorem{conj}[thm]{Conjecture}
\newtheorem{quest}[thm]{Question}

\theoremstyle{definition}
\newtheorem{defn}[thm]{Definition}
\newtheorem{defns}[thm]{Definitions}
\newtheorem{con}[thm]{Construction}
\newtheorem{exmp}[thm]{Example}
\newtheorem{exmps}[thm]{Examples}
\newtheorem{notn}[thm]{Notation}
\newtheorem{notns}[thm]{Notations}
\newtheorem{addm}[thm]{Addendum}
\newtheorem*{exer}{Exercise}

\theoremstyle{remark}
\newtheorem{rem}[thm]{Remark}
\newtheorem{rems}[thm]{Remarks}
\newtheorem{warn}[thm]{Warning}
\newtheorem{sch}[thm]{Scholium}

\DeclareMathOperator{\Hom}{Hom}
\DeclareMathOperator{\Id}{Id}
\DeclareMathOperator{\End}{End}
\DeclareMathOperator{\ord}{ord}
\DeclareMathOperator{\Aut}{Aut}
\DeclareMathOperator{\Gal}{Gal}
\DeclareMathOperator{\Ext}{Ext}
\DeclareMathOperator{\Hom}{Hom}

\makeatletter
\let\c@equation\c@thm
\makeatother
\numberwithin{equation}{section}

\bibliographystyle{plain}

\begin{document}

\title{Math 612 (Homework 2)}
\author{Hidenori Shinohara}
\maketitle

\begin{exer}{(Exercise 1)}
  Fix $G$ and let $\alpha: H \rightarrow H'$ be given.
  Let $0 \rightarrow F_1 \xrightarrow{f_1} F_0 \xrightarrow{f_0} H \rightarrow 0, 0 \rightarrow G_1 \xrightarrow{g_1} G_0 \xrightarrow{g_0} H \rightarrow 0$ be free resolutions.
  By Lemma 3.1(a), we obtain two homomorphisms $\alpha_1: F_1 \rightarrow G_1, \alpha_0: F_0 \rightarrow G_0$ which commutes with $f_i, g_i, \alpha$.
  Then we obtain two chain complexes

  \begin{align*}
    &0 \leftarrow \Hom(F_1, G) \xleftarrow{f_1^{\ast}} \Hom(F_0, G) \xleftarrow{f_0^{\ast}} \Hom(H, G) \leftarrow 0 \\
    &0 \leftarrow \Hom(F_1, G') \xleftarrow{f_1^{\ast}} \Hom(F_0, G') \xleftarrow{f_0^{\ast}} \Hom(H, G') \leftarrow 0.
  \end{align*}

  with induced maps $\alpha_1^{\ast}, \alpha_0^{\ast}, \alpha^{\ast}$ forming a chain map from the chain complex on the bottom to the one on the top.
  Then $\alpha_1^{\ast}$ induces a map from $\Ext(H', G) \rightarrow \Ext(H, G)$.

  Fix $H$ and let $f: G \rightarrow G'$ be given.
  Let $0 \rightarrow F_1 \xrightarrow{f_1} F_0 \xrightarrow{f_0} H \rightarrow 0$ be a free resolution of $H$.
  We obtain two cochain complexes where $f_{\ast}$ is a chain map from the top one to the bottom one.

  \begin{align*}
    &0 \leftarrow \Hom(F_1, G) \xleftarrow{f_1^{\ast}} \Hom(F_0, G) \xleftarrow{f_0^{\ast}} \Hom(H, G) \leftarrow 0 \\
    &0 \leftarrow \Hom(F_1, G') \xleftarrow{f_1^{\ast}} \Hom(F_0, G') \xleftarrow{f_0^{\ast}} \Hom(H, G') \leftarrow 0.
  \end{align*}

  $f_{\ast}$ indeed makes the diagram commute because for any $\sigma \in \Hom(H, G)$,
  \begin{align*} 
    f_{\ast}(f_0^{\ast}(\sigma))
      &= f_{\ast}(\sigma \circ f_0) \\
      &= f \circ (\sigma \circ f_0) \\
      &= (f \circ \sigma) \circ f_0 \\
      &= f_0^{\ast}(f \circ \sigma) \\
      &= f_0^{\ast}(f_{\ast}(\sigma)).
  \end{align*}
  Similarly, $f_{\ast}(f_1^{\ast}(\sigma)) = f_1^{\ast}(f_{\ast}(\sigma))$ for every $\sigma \in \Hom(F_0, G)$.
  Since a chain map induces a homomorphism on cohomology groups, $f$ induces a map from $\Ext(H, G) \rightarrow \Ext(H, G')$.
\end{exer}

\end{document}


