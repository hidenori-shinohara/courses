\documentclass[12pt, psamsfonts]{amsart}

%-------Packages---------
\usepackage{amssymb,amsfonts}
\usepackage{semantic}
\usepackage{fullpage}
\usepackage{tikz-cd}
\usepackage{todonotes}
\usepackage{physics}
\usepackage[all,arc]{xy}
\usepackage{enumerate}
\usepackage{enumitem}
\usepackage{mathrsfs}
\usepackage{theoremref}
\usepackage{graphicx}
\usepackage[bookmarks]{hyperref}

%--------Theorem Environments--------
%theoremstyle{plain} --- default
\newtheorem{thm}{Theorem}[section]
\newtheorem{cor}[thm]{Corollary}
\newtheorem{prop}[thm]{Proposition}
\newtheorem{lem}[thm]{Lemma}
\newtheorem{conj}[thm]{Conjecture}
\newtheorem{quest}[thm]{Question}

\theoremstyle{definition}
\newtheorem{defn}[thm]{Definition}
\newtheorem{defns}[thm]{Definitions}
\newtheorem{con}[thm]{Construction}
\newtheorem{exmp}[thm]{Example}
\newtheorem{exmps}[thm]{Examples}
\newtheorem{notn}[thm]{Notation}
\newtheorem{notns}[thm]{Notations}
\newtheorem{addm}[thm]{Addendum}
\newtheorem*{exer}{Exercise}

\theoremstyle{remark}
\newtheorem{rem}[thm]{Remark}
\newtheorem{rems}[thm]{Remarks}
\newtheorem{warn}[thm]{Warning}
\newtheorem{sch}[thm]{Scholium}

\DeclareMathOperator{\Hom}{Hom}
\DeclareMathOperator{\Id}{Id}
\DeclareMathOperator{\End}{End}
\DeclareMathOperator{\ord}{ord}
\DeclareMathOperator{\Aut}{Aut}
\DeclareMathOperator{\Gal}{Gal}
\DeclareMathOperator{\Ext}{Ext}
\DeclareMathOperator{\Hom}{Hom}
\DeclareMathOperator{\im}{im}

\makeatletter
\let\c@equation\c@thm
\makeatother
\numberwithin{equation}{section}

\bibliographystyle{plain}

\begin{document}

\title{Math 612 (Homework 2)}
\author{Hidenori Shinohara}
\maketitle

\begin{exer}{(Exercise 1)}
  Fix $G$ and let $\alpha: H \rightarrow H'$ be given.
  Let $0 \rightarrow F_1 \xrightarrow{f_1} F_0 \xrightarrow{f_0} H \rightarrow 0, 0 \rightarrow G_1 \xrightarrow{g_1} G_0 \xrightarrow{g_0} H \rightarrow 0$ be free resolutions.
  By Lemma 3.1(a), we obtain two homomorphisms $\alpha_1: F_1 \rightarrow G_1, \alpha_0: F_0 \rightarrow G_0$ which commutes with $f_i, g_i, \alpha$.
  Then we obtain two chain complexes

  \begin{align*}
    &0 \leftarrow \Hom(F_1, G) \xleftarrow{f_1^{\ast}} \Hom(F_0, G) \xleftarrow{f_0^{\ast}} \Hom(H, G) \leftarrow 0 \\
    &0 \leftarrow \Hom(F_1, G') \xleftarrow{f_1^{\ast}} \Hom(F_0, G') \xleftarrow{f_0^{\ast}} \Hom(H, G') \leftarrow 0.
  \end{align*}

  with induced maps $\alpha_1^{\ast}, \alpha_0^{\ast}, \alpha^{\ast}$ forming a chain map from the chain complex on the bottom to the one on the top.
  Then $\alpha_1^{\ast}$ induces a map from $\Ext(H', G) \rightarrow \Ext(H, G)$.

  Fix $H$ and let $f: G \rightarrow G'$ be given.
  Let $0 \rightarrow F_1 \xrightarrow{f_1} F_0 \xrightarrow{f_0} H \rightarrow 0$ be a free resolution of $H$.
  We obtain two cochain complexes where $f_{\ast}$ is a chain map from the top one to the bottom one.

  \begin{align*}
    &0 \leftarrow \Hom(F_1, G) \xleftarrow{f_1^{\ast}} \Hom(F_0, G) \xleftarrow{f_0^{\ast}} \Hom(H, G) \leftarrow 0 \\
    &0 \leftarrow \Hom(F_1, G') \xleftarrow{f_1^{\ast}} \Hom(F_0, G') \xleftarrow{f_0^{\ast}} \Hom(H, G') \leftarrow 0.
  \end{align*}

  $f_{\ast}$ indeed makes the diagram commute because for any $\sigma \in \Hom(H, G)$,
  \begin{align*} 
    f_{\ast}(f_0^{\ast}(\sigma))
      &= f_{\ast}(\sigma \circ f_0) \\
      &= f \circ (\sigma \circ f_0) \\
      &= (f \circ \sigma) \circ f_0 \\
      &= f_0^{\ast}(f \circ \sigma) \\
      &= f_0^{\ast}(f_{\ast}(\sigma)).
  \end{align*}
  Similarly, $f_{\ast}(f_1^{\ast}(\sigma)) = f_1^{\ast}(f_{\ast}(\sigma))$ for every $\sigma \in \Hom(F_0, G)$.
  Since a chain map induces a homomorphism on cohomology groups, $f$ induces a map from $\Ext(H, G) \rightarrow \Ext(H, G')$.
\end{exer}

\begin{exer}{(Exercise 1.2)}
  \begin{center}
    \begin{tikzcd}[cells={nodes={minimum height=2em}}]
      0 \arrow[r] & F_1 \arrow[r,"f_1"] \arrow[d,"\cdot n"] &  F_0 \arrow[r,"f_0"] \arrow[d,"\cdot n"] &  H \arrow[r] \arrow[d,"\cdot n"] & 0 \\
      0 \arrow[r] & F_1 \arrow[r,"f_1"]                     &  F_0 \arrow[r,"f_0"]                     &  H \arrow[r]                     & 0 \\
    \end{tikzcd}
  \end{center}
  turn into two chain complexes with a chain map
  \begin{center}
    \begin{tikzcd}[cells={nodes={minimum height=2em}}]
      0 & \Hom(F_1, G)                              \arrow[l]& \Hom(F_0, G) \arrow[l,"f_1^{\ast}"]                              &  \Hom(H, G) \arrow[l,"f_0^{\ast}"]                             & 0 \arrow[l] \\
      0 & \Hom(F_1, G) \arrow[u,"(\cdot n)^{\ast}"] \arrow[l]& \Hom(F_0, G) \arrow[l,"f_1^{\ast}"] \arrow[u,"(\cdot n)^{\ast}"] &  \Hom(H, G) \arrow[l,"f_0^{\ast}"] \arrow[u,"(\cdot n)^{\ast}"] & 0 \arrow[l].
    \end{tikzcd}
  \end{center}
  This diagram commutes because a group homomorphism for abelian groups commute with multiplication by $n$.
  Therefore, $(\cdot n)^{\ast}$ induces a homomorphism on $\Ext(H, G) = \Hom(F_1, G) / \im(f_1^{\ast})$.
  Moreover, $\forall \phi + \im(f_1^{\ast}) \in \Ext(H, G)$,
  \begin{align*}
    (\cdot n)^{\ast}(\phi + \im(f_1^{\ast}))
      &= \phi \circ (\cdot n) + \im(f_1^{\ast})
  \end{align*}
  where $(\phi \circ (\cdot n))(x) = \phi(n(x)) = n(\phi(x)) = (n\phi)(x)$ for all $x \in F_1$.
  Therefore, the map induced by $(\cdot n)^{\ast}$ is simply multiplication by $n$.

  \begin{center}
    \begin{tikzcd}[cells={nodes={minimum height=2em}}]
      0 & \Hom(F_1, G) \arrow[d,"(\cdot n)_{\ast}"] \arrow[l]& \Hom(F_0, G) \arrow[l,"f_1^{\ast}"] \arrow[d,"(\cdot n)_{\ast}"] &  \Hom(H, G) \arrow[l,"f_0^{\ast}"] \arrow[d,"(\cdot n)_{\ast}"] & 0 \arrow[l] \\
      0 & \Hom(F_1, G)                              \arrow[l]& \Hom(F_0, G) \arrow[l,"f_1^{\ast}"]                              &  \Hom(H, G) \arrow[l,"f_0^{\ast}"]                             & 0 \arrow[l].
    \end{tikzcd}
  \end{center}
  For every $\phi \in \Hom(H, G)$ and $x \in F_0$,

  \begin{align*}
    ((\cdot n)_{\ast}(f_0^{\ast}(\phi)))(x)
      &= ((\cdot n)_{\ast}(\phi \circ f_0))(x) \\
      &= n((\phi \circ f_0)(x)) \\
      &= n(\phi(f_0(x))) \\
      &= ((\cdot n)_{\ast}\phi)(f_0(x)) \\
      &= f_0^{\ast}((\cdot n)_{\ast}\phi)(x).
  \end{align*}

  Similarly, $(\cdot n)_{\ast}$ commutes with $f_1^{\ast}$, so $(\cdot n)_{\ast}$ is a chain map.
  For any $\phi + \im(f_1^{\ast}) \in \Ext(H, G)$, $(\cdot n)_{\ast}(\phi + \im(f_1^{\ast})) = n\phi + \im(f_1^{\ast})$, so it is multiplication by $n$.
\end{exer}

\begin{exer}{(Exercise 3.1.3)}
  $\cdots \xrightarrow{d_2} \mathbb{Z}_4 \xrightarrow{d_1} \mathbb{Z}_4 \xrightarrow{d_0} \mathbb{Z}_2 \rightarrow 0$ is a free resolution where $d_0: a \mapsto a$ and $d_i: a \mapsto 2a$ because $\ker(d_0) = \im(d_i) = \ker(d_i) = \{ 0 , 2 \}$ for each $i \geq 1$.
  Apply $\Hom(-, \mathbb{Z}_2)$ and replace $\mathbb{Z}_2^{\ast}$ with 0.
  For any $\phi \in \Hom(\mathbb{Z}_4, \mathbb{Z}_2)$ and $x \in \mathbb{Z}_4$, $((\cdot 2)^{\ast}(\phi))(x) = (\phi \circ (\cdot 2))(x) = \phi(2x) = \phi(0) = 0$.
  Thus $(\cdot 2)^{\ast}(\phi) = 0$.
  In other words, $d_i^{\ast} = 0$ for all $i \geq 1$, so $\Ext^n_{\mathbb{Z}_4}(\mathbb{Z}_2, \mathbb{Z}_2) = \Hom(\mathbb{Z}_4, \mathbb{Z}_2)$ which is nontrivial because $1 \mapsto 1$ is a nontrivial group homomorphism.
\end{exer}

\begin{exer}{(Exercise 3.1.6(a))}
  The chain complex we obtain is isomorphic to $0 \rightarrow \mathbb{Z}^2 \xrightarrow{\alpha} \mathbb{Z}^3 \xrightarrow{0} \mathbb{Z} \rightarrow 0$ where $\alpha(a, b) = (a + b)(1, 1, -1)$.
  If we apply $\Hom(-, \mathbb{Z})$, we obtain
  \begin{itemize}
    \item
      $H^0(T; \mathbb{Z}) = \Hom(\mathbb{Z}, \mathbb{Z}) = \mathbb{Z}$.
    \item
      $\alpha^{\ast}(\phi) = 0$ if and only if $\phi(1, 1, -1) = 0$.
      $(a, b, c) \mapsto a - b$ and $(a, b, c) \mapsto a + c$ form a basis for the subspace consisting of such homomorphisms.
      $H^1(T; \mathbb{Z}) = \ker(\alpha^{\ast}) = \mathbb{Z} \oplus \mathbb{Z}$.
    \item
      $H^2(T; \mathbb{Z}) = \Hom(\mathbb{Z}^2, \mathbb{Z}) / \im(\alpha^{\ast}) = \mathbb{Z}$ because $(a, b) \mapsto a$ and $(a, b) \mapsto a + b$ form a basis for $\Hom(\mathbb{Z}^2, \mathbb{Z})$ and $\im(\alpha^{\ast})$ is spanned by $(a, b) \mapsto a + b$.
  \end{itemize}
  If we apply $\Hom(-, \mathbb{Z}_2)$, we obtain
  \begin{itemize}
    \item
      $H^0(T; \mathbb{Z}_2) = \Hom(\mathbb{Z}_2, \mathbb{Z}_2) = \mathbb{Z}_2$.
    \item
      $\alpha^{\ast}(\phi) = 0$ if and only if $\phi(1, 1, 1) = 0$.
      $(a, b, c) \mapsto a + b$ and $(a, b, c) \mapsto a + c$ form a basis for the subspace consisting of such homomorphisms.
      $H^1(T; \mathbb{Z}_2) = \ker(\alpha^{\ast}) = \mathbb{Z}_2 \oplus \mathbb{Z}_2$.
    \item
      $H^2(T; \mathbb{Z}_2) = \Hom(\mathbb{Z}_2^2, \mathbb{Z}_2) / \im(\alpha^{\ast}) = \mathbb{Z}_2$ because $(a, b) \mapsto a$ and $(a, b) \mapsto a + b$ form a basis for $\Hom(\mathbb{Z}_2^2, \mathbb{Z}_2)$ and $\im(\alpha^{\ast})$ is spanned by $(a, b) \mapsto a + b$.
  \end{itemize}
\end{exer}

\begin{exer}{(Exercise 3.1.6(b), projective plane)}
  We obtain a chain complex $0 \rightarrow \mathbb{Z}^2 \xrightarrow{\alpha} \mathbb{Z}^3 \xrightarrow{\beta} \mathbb{Z}^2 \rightarrow 0$ where $\alpha(a, b) = (b - a, a - b, a + b)$ and $\beta(a, b, c) = (a + b, -a - b)$.
  By applying $\Hom(-, \mathbb{Z})$, we obtain a cochain complex.
  Each $\Hom(\mathbb{Z}^k, \mathbb{Z})$ has a basis $\{ \pi_1, \pi_2, \cdots, \pi_k \}$ where $\pi_i$ is a projection on the $i$th coordinate.
  Then $(\beta^{\ast}(\pi_1))(a, b, c) = a + b, (\beta^{\ast}(\pi_2))(a, b, c) = -a - b$.
  Thus $\ker(\beta^{\ast}) = \ev{ \pi_1 + \pi_2 }$ and $\im(\beta^{\ast}) = \ev{\pi_1 + \pi_2}$.
  The kernel and image of $\alpha$ can be calculated similarly.
  \begin{itemize}
    \item
      $H^0 = \ker(\beta^{\ast}) = \mathbb{Z}$.
    \item
      $H^1 = \ker(\alpha^{\ast}) / \im(\beta^{\ast}) = \ev{\pi_1 + \pi_2} / \ev{\pi_1 + \pi_2} = 0$.
    \item
      $H_2 = \ker(0) / \im(\alpha^{\ast}) = \ev{\pi_1, \pi_2} / \ev{\pi_1 - \pi_2, \pi_1 - \pi_2} = \ev{\pi_1 + \pi_2, \pi_1 \mid \pi_1 + \pi_2, 2\pi_1} = \mathbb{Z}_2$.
  \end{itemize}

  Similarly, we apply $\Hom(-\mathbb{Z}_2)$.
  Each $\Hom(\mathbb{Z}^k, \mathbb{Z}_2)$ has a basis $\{ \pi_1, \pi_2, \cdots, \pi_k \}$ where $\pi_i$ is a projection on the $i$th coordinate.
  The calculation of the kernels and images are almost identical as above with the only exception $\ker(\alpha^{\ast})$.
  This is because $\alpha^{\ast}(\pi_i) : (a, b) \mapsto a + b$ for each $i = 1, 2, 3$, so the kernel is $\ev{ \pi_1 + \pi_2, \pi_1 + \pi_3 }$.
  \begin{itemize}
    \item
      $H^0 = \ker(\beta^{\ast}) = \ev{\pi_1 + \pi_2} = \mathbb{Z}_2$.
    \item
      $H^1 = \ker(\alpha^{\ast}) / \im(\beta^{\ast}) = \ev{\pi_1 + \pi_2, \pi_1 + \pi_3} / \ev{\pi_1 + \pi_2} = \mathbb{Z}_2$.
    \item
      $H_2 = \ker(0) / \im(\alpha^{\ast}) = \ev{\pi_1, \pi_2} / \ev{\pi_1 + \pi_2, \pi_1 + \pi_2} = \ev{\pi_1} = \mathbb{Z}_2$.
  \end{itemize}
\end{exer}

\begin{exer}{(Exercise 3.1.6(b), klein bottle)}
  The chain complex we obtain is $0 \rightarrow \mathbb{Z}^2 \xrightarrow{\alpha} \mathbb{Z}^3 \xrightarrow{0} \mathbb{Z} \rightarrow 0$ with $\alpha(a, b) = (a + b, a - b, b - a)$.
  Again, we will use the projection map $\pi_i$ of the $i$th coordinate to form bases of the dual spaces.
  $\ker 0^{\ast} = \mathbb{Z}, \im 0^{\ast} = 0$.
  $\ker(\alpha^{\ast}) = \ev{\pi_2 + \pi_3}$ and $\im(\alpha^{\ast}) = \ev{\pi_1 + \pi_2, \pi_1 - \pi_2}$ because
  \begin{align*}
    (\alpha^{\ast}(\pi_i))(a, b) &= \begin{cases}
      a + b & (i = 1) \\
      a - b & (i = 2) \\
      b - a & (i = 3).
    \end{cases}
  \end{align*}

  Thus $H_0 = \mathbb{Z}$, $H_1 = \ev{\pi_2 + \pi_3} / 0 = \mathbb{Z}$ and $H_2 = \ev{\pi_1, \pi_2 \mid \pi_1 + \pi_2, \pi_1 - \pi_2} = \mathbb{Z} / 2$.

  $\ker 0^{\ast} = \mathbb{Z}_2, \im 0^{\ast} = 0$.
  $\ker(\alpha^{\ast}) = \ev{\pi_1 + \pi_2, \pi_1 + \pi_3}$ and $\im(\alpha^{\ast}) = \ev{\pi_1 + \pi_2}$.

  Thus $H_0 = \mathbb{Z}_2$, $H_1 = \ev{\pi_1 + \pi_2, \pi_1 + \pi_3} / 0 = \mathbb{Z}_2 \oplus \mathbb{Z}_2$ and $H_2 = \ev{\pi_1, \pi_2 \mid \pi_1 + \pi_2} = \mathbb{Z}_2$.
\end{exer}

\end{document}


