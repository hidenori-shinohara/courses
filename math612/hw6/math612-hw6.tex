\documentclass[12pt, psamsfonts]{amsart}

%-------Packages---------
\usepackage{amssymb,amsfonts}
\usepackage{semantic}
\usepackage{fullpage}
\usepackage{tikz-cd}
\usepackage{todonotes}
\usepackage{physics}
\usepackage[all,arc]{xy}
\usepackage{enumerate}
\usepackage{enumitem}
\usepackage{mathrsfs}
\usepackage{theoremref}
\usepackage{graphicx}
\usepackage[bookmarks]{hyperref}


%--------Theorem Environments--------
%theoremstyle{plain} --- default
\newtheorem{thm}{Theorem}[section]
\newtheorem{cor}[thm]{Corollary}
\newtheorem{prop}[thm]{Proposition}
\newtheorem{lem}[thm]{Lemma}
\newtheorem{conj}[thm]{Conjecture}
\newtheorem{quest}[thm]{Question}

\theoremstyle{definition}
\newtheorem{defn}[thm]{Definition}
\newtheorem{defns}[thm]{Definitions}
\newtheorem{con}[thm]{Construction}
\newtheorem{exmp}[thm]{Example}
\newtheorem{exmps}[thm]{Examples}
\newtheorem{notn}[thm]{Notation}
\newtheorem{notns}[thm]{Notations}
\newtheorem{addm}[thm]{Addendum}
\newtheorem*{exer}{Exercise}

\theoremstyle{remark}
\newtheorem{rem}[thm]{Remark}
\newtheorem{rems}[thm]{Remarks}
\newtheorem{warn}[thm]{Warning}
\newtheorem{sch}[thm]{Scholium}

\DeclareMathOperator{\Hom}{Hom}
\DeclareMathOperator{\Id}{Id}
\DeclareMathOperator{\End}{End}
\DeclareMathOperator{\ord}{ord}
\DeclareMathOperator{\Aut}{Aut}
\DeclareMathOperator{\Gal}{Gal}

\makeatletter
\let\c@equation\c@thm
\makeatother
\numberwithin{equation}{section}

\bibliographystyle{plain}

\begin{document}

\title{Math 612 Homework 6}
\author{Hidenori Shinohara}
\maketitle

% At long last, here is the next HW assignment. Because of our extra meeting this week, it will be due next Friday, April 3, at 5pm.
% 
% Hatcher, section 3.3: 6, 8, 10, 11, 12, 15, 17
% 
\begin{exer}{(Exercise 3.3.8)}
  Let $y \in B$ and $x_i$ denote the point in $B_i$ such that $f(x_i) = y$ for each $i$.
  Let $\mu_i$ denote the local orientation at $x_i$ induced by the orientation of $M$.
  For each $i$, we have the following diagrams using excisions, exact sequences of pairs and maps induced by $f$, inclusions and projections:
  \begin{center}
    \begin{tikzcd}[cells={nodes={minimum height=2em}}]
                      & H_n(B_i, B_i - x_i) \arrow[dl] \arrow[r] \arrow[d] & H_n(B, B - y) \arrow[d]\\
      H_n(M, M - x_i) & H_n(M, M - f^{-1}(y)) \arrow[l] \arrow[r]          & H_n(N, N - y) \\
                      & H_n(M) \arrow[ul, "\cong"] \arrow[r] \arrow[u, "j"]              & H_n(N) \arrow[u] \\
    \end{tikzcd}
  \end{center}
  using the idea for the proof of Proposition 2.30.

  $H_n(B_i, B_i - x_i) \rightarrow H_n(M, M - x_i)$, $H_n(B, B - y) \rightarrow H_n(N, N - y)$ and $H_n(N) \rightarrow H_n(N, N - y)$ are all isomorphisms.
  Since this diagram commutes for each $i$, $H_n(M, M - f^{-1}(y)) = \oplus_{i} H_n(B_i, B_i - x_i) = \oplus_{i} \mathbb{Z}$.
  Since $1 = [M]$ gets mapped to 1 in $H_n(M, M - x_i)$, $j([M]) = \sum_i k_i(\mu_i)$ where $k_i$ is the map $H_n(B_i, B_i - x_i) \rightarrow H_n(M, M - f^{-1}(y))$.
  Furthermore, $f_{\ast}(\mu_i) = \epsilon_i$, so $f_{\ast}(k_i(\mu_i)) = \epsilon_i$.
  Therefore, $f_{\ast}([M]) = f_{\ast}(\sum k_i(\mu_i)) = \sum f_{\ast}(k_i(\mu_i)) = \sum \epsilon_i$.
\end{exer}

\end{document}


