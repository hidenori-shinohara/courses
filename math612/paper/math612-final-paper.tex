\documentclass[12pt, psamsfonts]{amsart}

%-------Packages---------
\usepackage{amssymb,amsfonts}
\usepackage{semantic}
\usepackage{fullpage}
\usepackage{tikz-cd}
\usepackage{todonotes}
\usepackage{physics}
\usepackage[all,arc]{xy}
\usepackage{enumerate}
\usepackage{enumitem}
\usepackage{mathrsfs}
\usepackage{theoremref}
\usepackage{graphicx}
\usepackage[bookmarks]{hyperref}

%--------Theorem Environments--------
%theoremstyle{plain} --- default
\newtheorem{thm}{Theorem}[section]
\newtheorem{cor}[thm]{Corollary}
\newtheorem{prop}[thm]{Proposition}
\newtheorem{lem}[thm]{Lemma}
\newtheorem{conj}[thm]{Conjecture}
\newtheorem{quest}[thm]{Question}

\theoremstyle{definition}
\newtheorem{defn}[thm]{Definition}
\newtheorem{defns}[thm]{Definitions}
\newtheorem{con}[thm]{Construction}
\newtheorem{exmp}[thm]{Example}
\newtheorem{exmps}[thm]{Examples}
\newtheorem{notn}[thm]{Notation}
\newtheorem{notns}[thm]{Notations}
\newtheorem{addm}[thm]{Addendum}
\newtheorem*{exer}{Exercise}

\theoremstyle{remark}
\newtheorem{rem}[thm]{Remark}
\newtheorem{rems}[thm]{Remarks}
\newtheorem{warn}[thm]{Warning}
\newtheorem{sch}[thm]{Scholium}

\DeclareMathOperator{\Hom}{Hom}
\DeclareMathOperator{\Id}{Id}
\DeclareMathOperator{\End}{End}
\DeclareMathOperator{\ord}{ord}
\DeclareMathOperator{\Aut}{Aut}
\DeclareMathOperator{\Gal}{Gal}

\makeatletter
\let\c@equation\c@thm
\makeatother
\numberwithin{equation}{section}

\bibliographystyle{plain}

\begin{document}

\title{Math 612 Final Project}
\author{Hidenori Shinohara}
\maketitle

\begin{abstract}
  This is based on the book \textit{4-Manifolds and Kirby Calculus} by Robert E. Gompf and Andras I. Stipsicz.
\end{abstract}
\tableofcontents

\section{Manifolds}

\begin{defn}
  A topological manifold is a separable Hausdorff space such that every point has a neighborhood which is homeomorphic to an open subset of $\mathbb{R}^n_{+}$.
  Each pair $(U_{\alpha}, \phi_{\alpha})$ containing a neighborhood and a homeomorphism is called a chart, and a collection of charts covering the manifold is called an atlas of the manifold.
\end{defn}

\begin{defn}
  A topological manifold is called a $C^r$-manifold if, for every pair of charts in the given atlas, the transition function $\phi_{\alpha} \circ \phi_{\beta}^{-1}$ is $C^r$.
\end{defn}

This definition makes sense because $\phi_{\alpha} \circ \phi_{\beta}^{-1}$ maps $U_{\beta}$ into $U_{\alpha}$, both of which are open subsets of $\mathbb{R}^n_{+}$, thus the usual calculus definition of $C^r$ is applied.
More rigorously, a $C^r$-manifold is $(X, \mathcal{T}, \mathcal{A})$ where $X$ is the set, $\mathcal{T}$ is the set of open subsets of $X$, and $\mathcal{A}$ is the atlas of $X$.
However, just like we normally say a topological space $X$ instead of $(X, \mathcal{T})$, we normally just say a $C^r$-manifold $X$ without specifying the atlas.

\begin{defn}
  Let $X, X'$ be $C^r$-manifolds.
  Then a map $f: X \rightarrow X'$ is called a $C^r$-map if $\phi_{\alpha} \circ f \circ \phi_{\beta}^{-1}$ is $C^r$ for $\alpha, \beta$.
  Moreover, $f$ is called a $C^r$-diffeomorphism if $f$ is bijective and both $f$ and $f^{-1}$ are $C^r$-maps.
\end{defn}

Again, in this definition, the usual calculus definition of $C^r$ is used for $\phi_{\alpha} \circ f \circ \phi_{\beta}^{-1}$.

\begin{defn}
  Let $X$ be a topological manifold.
  Let $A, A'$ be two atlases of $X$ such that $(X, A)$ and $(X, A')$ are both $C^r$ manifolds.
  The two structures are called isotopic if the ``identity" map $(X, A) \mapsto (X, A')$ is isotopic to a $C^r$-diffeomorphism between $(X, A)$ and $(X, A')$.
\end{defn}

We will usually consider structures up to isotopy.

\begin{exmp}
  TODO
  Examples of isotopic structures.
\end{exmp}

\end{document}


