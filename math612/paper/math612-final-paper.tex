\documentclass[11pt, psamsfonts]{amsart}

%-------Packages---------
\usepackage{amssymb,amsfonts}
\usepackage{semantic}
\usepackage{fullpage}
\usepackage{tikz-cd}
\usepackage{todonotes}
\usepackage{physics}
\usepackage[all,arc]{xy}
\usepackage{enumerate}
\usepackage{enumitem}
\usepackage{mathrsfs}
\usepackage{theoremref}
\usepackage{graphicx}
\usepackage[bookmarks]{hyperref}

%--------Theorem Environments--------
%theoremstyle{plain} --- default
\newtheorem{thm}{Theorem}[section]
\newtheorem{cor}[thm]{Corollary}
\newtheorem{prop}[thm]{Proposition}
\newtheorem{lem}[thm]{Lemma}
\newtheorem{conj}[thm]{Conjecture}
\newtheorem{quest}[thm]{Question}

\theoremstyle{definition}
\newtheorem{defn}[thm]{Definition}
\newtheorem{defns}[thm]{Definitions}
\newtheorem{con}[thm]{Construction}
\newtheorem{exmp}[thm]{Example}
\newtheorem{exmps}[thm]{Examples}
\newtheorem{notn}[thm]{Notation}
\newtheorem{notns}[thm]{Notations}
\newtheorem{addm}[thm]{Addendum}
\newtheorem*{exer}{Exercise}

\theoremstyle{remark}
\newtheorem{rem}[thm]{Remark}
\newtheorem{rems}[thm]{Remarks}
\newtheorem{warn}[thm]{Warning}
\newtheorem{sch}[thm]{Scholium}

\DeclareMathOperator{\Hom}{Hom}
\DeclareMathOperator{\Id}{Id}
\DeclareMathOperator{\End}{End}
\DeclareMathOperator{\ord}{ord}
\DeclareMathOperator{\Aut}{Aut}
\DeclareMathOperator{\Gal}{Gal}

\makeatletter
\let\c@equation\c@thm
\makeatother
\numberwithin{equation}{section}

\bibliographystyle{plain}

\begin{document}

\title{Math 612 Final Project}
\author{Hidenori Shinohara}
\maketitle

\begin{abstract}
  This is based on the book \textit{4-Manifolds and Kirby Calculus} by Robert E. Gompf and Andras I. Stipsicz.
\end{abstract}
\tableofcontents

\section{Manifolds}

\begin{defn}
  A topological manifold is a separable Hausdorff space such that every point has a neighborhood which is homeomorphic to an open subset of $\mathbb{R}^n_{+}$.
  Each pair $(U_{\alpha}, \phi_{\alpha})$ containing a neighborhood and a homeomorphism is called a chart, and a collection of charts covering the manifold is called an atlas of the manifold.
\end{defn}

\begin{defn}
  A topological manifold is called a $C^r$-manifold if, for every pair of charts in the given atlas, the transition function $\phi_{\alpha} \circ \phi_{\beta}^{-1}$ is $C^r$.
\end{defn}

This definition makes sense because $\phi_{\alpha} \circ \phi_{\beta}^{-1}$ maps $U_{\beta}$ into $U_{\alpha}$, both of which are open subsets of $\mathbb{R}^n_{+}$, thus the usual calculus definition of $C^r$ is applied.
More rigorously, a $C^r$-manifold is $(X, \mathcal{T}, \mathcal{A})$ where $X$ is the set, $\mathcal{T}$ is the set of open subsets of $X$, and $\mathcal{A}$ is the atlas of $X$.
However, just like we normally say a topological space $X$ instead of $(X, \mathcal{T})$, we normally just say a $C^r$-manifold $X$ without specifying the atlas.

\begin{defn}
  Let $X, X'$ be $C^r$-manifolds.
  Then a map $f: X \rightarrow X'$ is called a $C^r$-map if $\phi_{\alpha} \circ f \circ \phi_{\beta}^{-1}$ is $C^r$ for $\alpha, \beta$.
  Moreover, $f$ is called a $C^r$-diffeomorphism if $f$ is bijective and both $f$ and $f^{-1}$ are $C^r$-maps.
\end{defn}

Again, in this definition, the usual calculus definition of $C^r$ is used for $\phi_{\alpha} \circ f \circ \phi_{\beta}^{-1}$.

\begin{defn}
  Let $X$ be a topological manifold.
  Let $A, A'$ be two atlases of $X$ such that $(X, A)$ and $(X, A')$ are both $C^r$ manifolds.
  The two structures are called isotopic if the ``identity" map $(X, A) \mapsto (X, A')$ is isotopic to a $C^r$-diffeomorphism between $(X, A)$ and $(X, A')$.
\end{defn}

We will usually consider structures up to isotopy.

\begin{exmp}
  TODO
  Examples of isotopic structures.
\end{exmp}

\begin{defn}
  A compact manifold with no boundary is said to be closed.
\end{defn}

\begin{defn}
  A space $X$ is said to be a singular manifold if the complement of a finite subset of $X$ is a smooth manifold.
\end{defn}


\begin{defn}
  Let $X$ be a smooth manifold.
  A derivation at $p \in X$ is a linear map $v: C^{\infty}(X) \rightarrow \mathbb{R}$ such that

  \begin{align*}
    v(fg) = f(p)vg + g(p)vf
  \end{align*}
  for all $f, g \in C^{\infty}(X)$.

  This corresponds to ``arrows that are tangent to $X$ and whose basepoints are attached to $X$ at $p$" even though it may not be easy to see that from this definition.
\end{defn}

\begin{defn}
  The tangent space $T_pX$ to $X$ at $p$ is the vector space of all derivations of $C^{\infty}(X)$ at $p$.
\end{defn}


\begin{defn}
  The tangent bundle $TX$ is the disjoint union $\coprod_{p \in X} T_pX$.
\end{defn}

Note that given the projection $\pi: TX \rightarrow X$, $\pi^{-1}(p)$ is a vector space isomorphic to $\mathbb{R}^n$ where $X$ is an $n$-dimensional manifold.

Next, we will define an orientation of a manifold using a tangent bundle.

\begin{defn}
  Two ordered bases $(E_1, \cdots, E_n)$ and $(\tilde{E}_1, \cdots, \tilde{E}_n)$ for $V$ are consistently oriented if the transition matrix $(B^j_i)$, defined by
  \begin{align*}
    E_i = B^j_i \tilde{E}_j
  \end{align*}

  has positive determinant.
\end{defn}

Since the transition matrix always has nonzero determinant, by fixing an ordered basis, we can split all ordered bases into two disjoint sets, one of which contains consistently oriented bases.
This leads to the following definition:

\begin{defn}
  A vector space with a choice of orientation is an oriented vector space.
\end{defn}

\begin{defn}
  The orientation $[e_1, \cdots, e_n]$ of $\mathbb{R}^n$ determined by the standard basis is called the standard orientation.
\end{defn}

This definition can be applied to a tangent space because, even though the definition of a tangent space may seem complicated, it is a vector space.

% \begin{defn}
%   A pointwise orientation on $M$ is defined to be a choice of orientation of each tangent space.
% \end{defn}
% 

% \begin{defn}
%   A local frame of $TX$ is $(E_1, \cdots, E_n)$ such that $(E_1\vert_p, \cdots, E_n\vert_p)$ is a basis of $T_pX$.
% \end{defn}
% 
% \begin{defn}
%   If $(E_i)$ is a local frame for $TX$, we say that $(E_i)$ is oriented if $(E_1\vert_p, \cdots, E_n\vert_p)$ is a positively oriented basis for $T_pX$ for each $p \in X$.
% \end{defn}

\begin{defn}
  $M, N$ are smooth manifolds with or without boundary, and $F: M \rightarrow N$ is a smooth map.
  The differential of $F$ at $p$ is the linear map $dF_p: T_pM \rightarrow T_{F(p)}N$ defined by
  \begin{align*}
    dF_p(v) := f \mapsto v(f \circ F)
  \end{align*}
  Equivalently, $\forall v \in T_pM, \forall f \in C^{\infty}(N), dF_p(v)(f) = v(f \circ F)$.
  This corresponds to ``the directional derivative of $F$ at $p$ in the direction of the arrow $v$."
\end{defn}

\begin{defn}
  A consistent choice of orientation of the tangent space at each point of $X$ is called an orientation of $X$.
  A choice is consistent if and only if for each point in $X$, there is a chart $(V, \phi)$ mapping into $\mathbb{R}^n$ with its standard orientation, such that $d\phi_p$ preserves orientation at every pint $p$ of $V$.
\end{defn}

\begin{defn}
  A manifold is orientable if it admits an orientation and it is said to be oriented if an orientation of $X$ is fixed.
\end{defn}

\begin{thm}
  An orientation as defined above specified a class $[X]$ in $H_n(X, \partial X; \mathbb{Z}$, called the fundamental class of the oriented manifold.
\end{thm}

\begin{defn}
  A complex function is said to be holomorphic if it is differentiable at every point.
\end{defn}

\begin{defn}
  An atlas $\{ (U_{\alpha}, \phi_{\alpha}) \vert \alpha \in A_{\mathbb{C}} \}$ on a real $2n$-dimensional manifold $X$ is a complex structure if each $\phi_{\alpha}$ is a homeomorphism between $U_{\alhpa}$ and an open subset of $\mathbb{C}^n$ which is identified with $\mathbb{R}^{2n}$ and the transition functions $\phi_{\beta} \circ \phi_{\alpha}^{-1}$ is holomorphic.
\end{defn}

\end{document}


