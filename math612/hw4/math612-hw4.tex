\documentclass[12pt, psamsfonts]{amsart}

%-------Packages---------
\usepackage{amssymb,amsfonts}
\usepackage{semantic}
\usepackage{fullpage}
\usepackage{tikz-cd}
\usepackage{todonotes}
\usepackage{physics}
\usepackage[all,arc]{xy}
\usepackage{enumerate}
\usepackage{enumitem}
\usepackage{mathrsfs}
\usepackage{theoremref}
\usepackage{graphicx}
\usepackage[bookmarks]{hyperref}

%--------Theorem Environments--------
%theoremstyle{plain} --- default
\newtheorem{thm}{Theorem}[section]
\newtheorem{cor}[thm]{Corollary}
\newtheorem{prop}[thm]{Proposition}
\newtheorem{lem}[thm]{Lemma}
\newtheorem{conj}[thm]{Conjecture}
\newtheorem{quest}[thm]{Question}

\theoremstyle{definition}
\newtheorem{defn}[thm]{Definition}
\newtheorem{defns}[thm]{Definitions}
\newtheorem{con}[thm]{Construction}
\newtheorem{exmp}[thm]{Example}
\newtheorem{exmps}[thm]{Examples}
\newtheorem{notn}[thm]{Notation}
\newtheorem{notns}[thm]{Notations}
\newtheorem{addm}[thm]{Addendum}
\newtheorem*{exer}{Exercise}

\theoremstyle{remark}
\newtheorem{rem}[thm]{Remark}
\newtheorem{rems}[thm]{Remarks}
\newtheorem{warn}[thm]{Warning}
\newtheorem{sch}[thm]{Scholium}

\DeclareMathOperator{\Hom}{Hom}
\DeclareMathOperator{\Id}{Id}
\DeclareMathOperator{\End}{End}
\DeclareMathOperator{\ord}{ord}
\DeclareMathOperator{\Aut}{Aut}
\DeclareMathOperator{\Gal}{Gal}

\makeatletter
\let\c@equation\c@thm
\makeatother
\numberwithin{equation}{section}

\bibliographystyle{plain}

\begin{document}

\title{Math 612(Homework 4)}
\author{Hidenori Shinohara}
\maketitle

\begin{exer}{(8)}
  By using cellular cohomology, we obtain 
  \begin{align*}
    H^i(X; \mathbb{Z}) = H^i(Y; \mathbb{Z}) &= \begin{cases}
      \mathbb{Z} & (i = 0, 4), \\
      \mathbb{Z}_p & (i = 3),
    \end{cases} \\
    H^i(X; \mathbb{Z}_p) = H^i(Y; \mathbb{Z}_p) &= \begin{cases}
      \mathbb{Z}_p & (i = 0, 2, 3, 4),
    \end{cases} \\
  \end{align*}
  Therefore, we cannot distinguish $X$ from $Y$ by looking at the cohomology groups.
  When using the coefficient $\mathbb{Z}$, cup products are simply 0 because nontrivial cohomology groups are of order 3 and 4.
  Thus we cannot distinguish $X$ from $Y$ by looking at the cohomology rings of $X$ and $Y$.
  Since $H^i(Y; \mathbb{Z}_p) = H^i(S^4; \mathbb{Z}_p) \oplus H^i(M(\mathbb{Z}_p, 2); \mathbb{Z}_p)$ and the cup product of elements from different ``components" in a wedge sum is 0, cup products in $H^{\ast}(Y; \mathbb{Z}_p)$ are all 0.
  On the other hand, the cup product $\alpha \smile \alpha$ where $\alpha$ is a generator of $H^2(\mathbb{C}P^2; \mathbb{Z}_p)$ is nontrivial because $\alpha \smile \alpha$ is a generator of $H^4(\mathbb{C}P^2; \mathbb{Z}_p)$.
\end{exer}

\end{document}


