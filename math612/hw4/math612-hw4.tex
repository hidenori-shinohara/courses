\documentclass[12pt, psamsfonts]{amsart}

%-------Packages---------
\usepackage{amssymb,amsfonts}
\usepackage{semantic}
\usepackage{fullpage}
\usepackage{tikz-cd}
\usepackage{todonotes}
\usepackage{physics}
\usepackage[all,arc]{xy}
\usepackage{enumerate}
\usepackage{enumitem}
\usepackage{mathrsfs}
\usepackage{theoremref}
\usepackage{graphicx}
\usepackage[bookmarks]{hyperref}

%--------Theorem Environments--------
%theoremstyle{plain} --- default
\newtheorem{thm}{Theorem}[section]
\newtheorem{cor}[thm]{Corollary}
\newtheorem{prop}[thm]{Proposition}
\newtheorem{lem}[thm]{Lemma}
\newtheorem{conj}[thm]{Conjecture}
\newtheorem{quest}[thm]{Question}

\theoremstyle{definition}
\newtheorem{defn}[thm]{Definition}
\newtheorem{defns}[thm]{Definitions}
\newtheorem{con}[thm]{Construction}
\newtheorem{exmp}[thm]{Example}
\newtheorem{exmps}[thm]{Examples}
\newtheorem{notn}[thm]{Notation}
\newtheorem{notns}[thm]{Notations}
\newtheorem{addm}[thm]{Addendum}
\newtheorem*{exer}{Exercise}

\theoremstyle{remark}
\newtheorem{rem}[thm]{Remark}
\newtheorem{rems}[thm]{Remarks}
\newtheorem{warn}[thm]{Warning}
\newtheorem{sch}[thm]{Scholium}

\DeclareMathOperator{\Hom}{Hom}
\DeclareMathOperator{\Id}{Id}
\DeclareMathOperator{\End}{End}
\DeclareMathOperator{\ord}{ord}
\DeclareMathOperator{\Aut}{Aut}
\DeclareMathOperator{\Gal}{Gal}
\DeclareMathOperator{\RP}{\mathbb{R}P}

\makeatletter
\let\c@equation\c@thm
\makeatother
\numberwithin{equation}{section}

\bibliographystyle{plain}

\begin{document}

\title{Math 612(Homework 4)}
\author{Hidenori Shinohara}
\maketitle

\begin{exer}{(8)}
  By using cellular cohomology, we obtain 
  \begin{align*}
    H^i(X; \mathbb{Z}) = H^i(Y; \mathbb{Z}) &= \begin{cases}
      \mathbb{Z} & (i = 0, 4), \\
      \mathbb{Z}_p & (i = 3),
    \end{cases} \\
    H^i(X; \mathbb{Z}_p) = H^i(Y; \mathbb{Z}_p) &= \begin{cases}
      \mathbb{Z}_p & (i = 0, 2, 3, 4),
    \end{cases} \\
  \end{align*}
  Therefore, we cannot distinguish $X$ from $Y$ by looking at the cohomology groups.
  When using the coefficient $\mathbb{Z}$, cup products are simply 0 because nontrivial cohomology groups are of order 3 and 4.
  Thus we cannot distinguish $X$ from $Y$ by looking at the cohomology rings of $X$ and $Y$.
  Since $H^i(Y; \mathbb{Z}_p) = H^i(S^4; \mathbb{Z}_p) \oplus H^i(M(\mathbb{Z}_p, 2); \mathbb{Z}_p)$ and the cup product of elements from different ``components" in a wedge sum is 0, cup products in $H^{\ast}(Y; \mathbb{Z}_p)$ are all 0.
  On the other hand, the cup product $\alpha \smile \alpha$ where $\alpha$ is a generator of $H^2(\mathbb{C}P^2; \mathbb{Z}_p)$ is nontrivial because $\alpha \smile \alpha$ is a generator of $H^4(\mathbb{C}P^2; \mathbb{Z}_p)$.
\end{exer}

\begin{exer}{(5)}
  Consider the canonical map $\mathbb{Z}_{2k} \rightarrow \mathbb{Z}_2$.
  It induces a chain map between the cellular chain complexes of $\RP^{\infty}$ over $\mathbb{Z}_{2k}$ and $\mathbb{Z}_2$.
  Moreover, they induce homomorphisms $\phi: H^i(\RP^{\infty};\mathbb{Z}_{2}) \rightarrow H^i(\RP^{\infty};\mathbb{Z}_{2k})$.
  By cellular cohomology, $H^0(\RP^{\infty}; \mathbb{Z}_{2k}) = \mathbb{Z}_{2k}$ and $H^i(\RP^{\infty}; \mathbb{Z}_{2k}) = \mathbb{Z}_2$ for $i \geq 1$.
  Let $\gamma$ denote a generator of $H^1(\RP^{\infty};\mathbb{Z}_2)$.
  Then $\phi(\gamma)$ must be a generator of $H^1(\RP^{\infty};\mathbb{Z}_{2k})$ because $\phi$ is induced by the map $1 \mapsto 1$.
  Let $\alpha = \phi(\gamma)$.
  $H^1(\RP^{\infty}; \mathbb{Z}_{2k}) = \mathbb{Z}_2$, so we obtain the relation $2\alpha$.

  Let $\beta$ be a generator of $H^2(\RP^{\infty};\mathbb{Z}_{2k})$.
  Since $H^2(\RP^{\infty};\mathbb{Z}_{2k}) = \mathbb{Z}_2$, we obtain the relation $2\beta$.
  \todo[inline,caption={}]{
    How do I obtain the relation $\alpha^2 - k\beta$?
    More specifically, is $\phi: H^2 \rightarrow H^2$ an isomorphism or the zero map?
  }
\end{exer}

\begin{exer}{(10)}
  Let $X = Y = \mathbb{Z}$ with the discrete topology.
  Then the only nontrivial cohomology groups are $H^0(X; \mathbb{Z}) = H^0(Y; \mathbb{Z}) = \mathbb{Z}$.
  Therefore, it suffices to check the cross product map $H^0(X; \mathbb{Z}) \otimes H^0(Y; \mathbb{Z}) \rightarrow H^0(X \times Y; \mathbb{Z})$.
  Every element in $H^0(\mathbb{Z}; \mathbb{Z})$ simply represents a map $\mathbb{Z} \rightarrow \mathbb{Z}$.
  Then for each $f \in H^0(X; \mathbb{Z}), g \in H^0(Y; \mathbb{Z})$, $f \times g: (a, b) \mapsto f(a)g(b)$.
  We claim that this is not surjective.

  Let $\delta$ be the map such that $\delta(i, j) = \delta_{i, j}$.
  Then clearly, $\delta \in H^0(X \times Y; \mathbb{Z})$.
  Suppose that there exists $\sum_{i=1}^{n} a^i \otimes b^i$ that gets mapped to $\delta$.
  Let $a_i, b_i \in \mathbb{Z}^n$ (with subscripts instead of superscripts) denote the vectors $a_i = \ev{a^1(i), \cdots, a^n(i)}, b_i = \ev{b^1(i), \cdots, b^n(i)}$.
  Then for each $i \in \mathbb{Z}$, the inner product $\ev{a_i, b_i} = \delta_{i, j}$.
  We claim that the set $\{ a_i \mid i \in \mathbb{Z} \}$ is linearly independent over $\mathbb{R}$.
  For simplicity, let $c_1, \cdots, c_m \in \mathbb{R}$ be given such that $\sum_{i=1}^{m} c_ia_i = 0$.
  (In general, indices could be taken over any finite subset of $\mathbb{Z}$.)
  This implies $\sum_{i=1}^{m} c_i\delta_{i, j} = 0$ by taking the inner product with $b_j$ for each $j$.
  Therefore, we obtain a linearly independent set of infinitely many vectors in $\mathbb{R}^n$.
  This is clearly impossible, so the cross product map cannot be surjective.
\end{exer}


\end{document}


