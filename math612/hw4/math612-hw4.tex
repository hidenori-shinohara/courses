\documentclass[12pt, psamsfonts]{amsart}

%-------Packages---------
\usepackage{amssymb,amsfonts}
\usepackage{semantic}
\usepackage{fullpage}
\usepackage{tikz-cd}
\usepackage{todonotes}
\usepackage{physics}
\usepackage[all,arc]{xy}
\usepackage{enumerate}
\usepackage{enumitem}
\usepackage{mathrsfs}
\usepackage{theoremref}
\usepackage{graphicx}
\usepackage[bookmarks]{hyperref}

%--------Theorem Environments--------
%theoremstyle{plain} --- default
\newtheorem{thm}{Theorem}[section]
\newtheorem{cor}[thm]{Corollary}
\newtheorem{prop}[thm]{Proposition}
\newtheorem{lem}[thm]{Lemma}
\newtheorem{conj}[thm]{Conjecture}
\newtheorem{quest}[thm]{Question}

\theoremstyle{definition}
\newtheorem{defn}[thm]{Definition}
\newtheorem{defns}[thm]{Definitions}
\newtheorem{con}[thm]{Construction}
\newtheorem{exmp}[thm]{Example}
\newtheorem{exmps}[thm]{Examples}
\newtheorem{notn}[thm]{Notation}
\newtheorem{notns}[thm]{Notations}
\newtheorem{addm}[thm]{Addendum}
\newtheorem*{exer}{Exercise}

\theoremstyle{remark}
\newtheorem{rem}[thm]{Remark}
\newtheorem{rems}[thm]{Remarks}
\newtheorem{warn}[thm]{Warning}
\newtheorem{sch}[thm]{Scholium}

\DeclareMathOperator{\Hom}{Hom}
\DeclareMathOperator{\Id}{Id}
\DeclareMathOperator{\End}{End}
\DeclareMathOperator{\ord}{ord}
\DeclareMathOperator{\Aut}{Aut}
\DeclareMathOperator{\Gal}{Gal}
\DeclareMathOperator{\RP}{\mathbb{R}P}

\makeatletter
\let\c@equation\c@thm
\makeatother
\numberwithin{equation}{section}

\bibliographystyle{plain}

\begin{document}

\title{Math 612(Homework 4)}
\author{Hidenori Shinohara}
\maketitle

\begin{exer}{(8)}
  By using cellular cohomology, we obtain 
  \begin{align*}
    H^i(X; \mathbb{Z}) = H^i(Y; \mathbb{Z}) &= \begin{cases}
      \mathbb{Z} & (i = 0, 4), \\
      \mathbb{Z}_p & (i = 3),
    \end{cases} \\
    H^i(X; \mathbb{Z}_p) = H^i(Y; \mathbb{Z}_p) &= \begin{cases}
      \mathbb{Z}_p & (i = 0, 2, 3, 4),
    \end{cases} \\
  \end{align*}
  Therefore, we cannot distinguish $X$ from $Y$ by looking at the cohomology groups.
  When using the coefficient $\mathbb{Z}$, cup products are simply 0 because nontrivial cohomology groups are of order 3 and 4.
  Thus we cannot distinguish $X$ from $Y$ by looking at the cohomology rings of $X$ and $Y$.
  Since $H^i(Y; \mathbb{Z}_p) = H^i(S^4; \mathbb{Z}_p) \oplus H^i(M(\mathbb{Z}_p, 2); \mathbb{Z}_p)$ and the cup product of elements from different ``components" in a wedge sum is 0, cup products in $H^{\ast}(Y; \mathbb{Z}_p)$ are all 0.
  On the other hand, the cup product $\alpha \smile \alpha$ where $\alpha$ is a generator of $H^2(\mathbb{C}P^2; \mathbb{Z}_p)$ is nontrivial because $\alpha \smile \alpha$ is a generator of $H^4(\mathbb{C}P^2; \mathbb{Z}_p)$.
\end{exer}

\begin{exer}{(5)}
  Consider the canonical map $\mathbb{Z}_{2k} \rightarrow \mathbb{Z}_2$.
  It induces homomorphisms $\phi: H^i(\RP^{\infty};\mathbb{Z}_{2k}) \rightarrow H^i(\RP^{\infty};\mathbb{Z}_{2})$.
  By cellular cohomology, $H^0(\RP^{\infty}; \mathbb{Z}_{2k}) = \mathbb{Z}_{2k}$ and $H^i(\RP^{\infty}; \mathbb{Z}_{2k}) = \mathbb{Z}_2$ for $i \geq 1$.
  Let $\alpha$ denote a generator of $H^1(\RP^{\infty};\mathbb{Z}_{2k})$, which equals the coset represented by $k$, and let $\beta$ denote a generator of $H^2(\RP^{\infty}; \mathbb{Z}_{2k})$, which equals the coset represented by $1$, and let $\gamma$ denote a generator of $H^1(\RP^{\infty}; \mathbb{Z}_2)$.
  Then $2\alpha = 2\beta = 0$.
  Then $\phi$ on the even dimensions are all isomorphisms because $1 \mapsto 1$.

  Suppose $k$ is even.
  Then $\phi(\alpha) = 0$ because $k$ is even.
  Moreover, $\phi(\alpha^2) = (\phi(\alpha))^2 = 0$.
  Since $\phi$ is an isomorphism on the even dimensions, $\alpha^2 = 0$.
  Thus $\alpha - k\beta = 0$.

  Suppose $k$ is odd.
  Then the $\phi$ are isomorphisms on the odd dimensions as well because $\overline{k} \mapsto 1$.
  Then $\phi(\beta) = \gamma^2 = \phi(\alpha)^2$, so $\alpha^2 = \beta$.
  Thus $\alpha - k\beta = 0$.

  Therefore, we obtained the relations $2\alpha, 2\beta, \alpha^2 - k\beta$.
\end{exer}

\begin{exer}{(9)}
  The quotient map $\mathbb{Z} \rightarrow \mathbb{Z}_p$ induces a map $f: H^{\ast}(X) \rightarrow H^{\ast}(X; \mathbb{Z}_p)$.
  Then we have a map $H^{\ast}(X) \times \mathbb{Z}_p \rightarrow H^{\ast}(X; \mathbb{Z}_p)$ defined by $(\alpha, a) \mapsto af(\alpha)$.
  Since this is bilinear, we obtain a map $\phi: H^{\ast}(X) \otimes \mathbb{Z}_p \rightarrow H^{\ast}(X; \mathbb{Z}_p)$.
  Suppose $\phi(\alpha \otimes a) = 0$.
  Without loss of generality, we assume $a = 1$.
  Then $f(\alpha)(\sigma) = 0$ for any $\sigma$.
  In other words, $\alpha(\sigma) \in p\mathbb{Z}$ for any $\sigma$.
  This implies the existence of $\beta \in H^{\ast}(X)$ such that $\alpha = p\beta$.
  Then $\alpha \otimes 1 = \beta \otimes 0 = 0$.
  Thus the kernel is 0, so $\phi$ is injective.

  \todo[inline,caption={}]{
    Prove this!
  }
\end{exer}

\begin{exer}{(10)}
  Let $X = Y = \mathbb{Z}$ with the discrete topology.
  Then the only nontrivial cohomology groups are $H^0(X; \mathbb{Z}) = H^0(Y; \mathbb{Z}) = \mathbb{Z}$.
  Therefore, it suffices to check the cross product map $H^0(X; \mathbb{Z}) \otimes H^0(Y; \mathbb{Z}) \rightarrow H^0(X \times Y; \mathbb{Z})$.
  Every element in $H^0(\mathbb{Z}; \mathbb{Z})$ simply represents a map $\mathbb{Z} \rightarrow \mathbb{Z}$.
  Then for each $f \in H^0(X; \mathbb{Z}), g \in H^0(Y; \mathbb{Z})$, $f \times g: (a, b) \mapsto f(a)g(b)$.
  We claim that this is not surjective.

  Let $\delta$ be the map such that $\delta(i, j) = \delta_{i, j}$.
  Then clearly, $\delta \in H^0(X \times Y; \mathbb{Z})$.
  Suppose that there exists $\sum_{i=1}^{n} a^i \otimes b^i$ that gets mapped to $\delta$.
  Let $a_i, b_i \in \mathbb{Z}^n$ (with subscripts instead of superscripts) denote the vectors $a_i = \ev{a^1(i), \cdots, a^n(i)}, b_i = \ev{b^1(i), \cdots, b^n(i)}$.
  Then for each $i \in \mathbb{Z}$, the inner product $\ev{a_i, b_i} = \delta_{i, j}$.
  We claim that the set $\{ a_i \mid i \in \mathbb{Z} \}$ is linearly independent over $\mathbb{R}$.
  For simplicity, let $c_1, \cdots, c_m \in \mathbb{R}$ be given such that $\sum_{i=1}^{m} c_ia_i = 0$.
  (In general, indices could be taken over any finite subset of $\mathbb{Z}$.)
  This implies $\sum_{i=1}^{m} c_i\delta_{i, j} = 0$ by taking the inner product with $b_j$ for each $j$.
  Therefore, we obtain a linearly independent set of infinitely many vectors in $\mathbb{R}^n$.
  This is clearly impossible, so the cross product map cannot be surjective.
\end{exer}

\begin{exer}{(11)}
  Let $f: S^{k + l} \rightarrow S^k \times S^l$.
  By the Kunneth formula, $H^{\ast}(S^k \times S^l) \cong H^{\ast}(S^k) \otimes H^{\ast}(S^l)$.
  Let $\alpha \in H^{\ast}(S^k \times S^l)$.
  By the isomorphism, $\alpha$ corresponds to some $\beta \in H^k(S^k)$ and $\gamma \in H^l(S^l)$ where $\alpha = \beta \times \gamma$.
  Then $f^{\ast}(\alpha) = f^{\ast}p_1^{\ast}\beta \smile f^{\ast}p_2^{\ast}\gamma$.
  Since $H^k(S^{k + l}) = 0$, $f^{\ast}p_1^{\ast} = 0$.
  Therefore, $f^{\ast}(\alpha) = 0$.
  In other words, $f^{\ast}$ is the zero map.

  Since each cohomology group of $S^{k + l}$ is free, the UCT implies $H^{k + l}(S^{k + l}) \cong \Hom(H_{k + l}(S^{k + l}), \mathbb{Z})$.
  Similarly, $H^{k + l}(S^k \times S^l) \cong \Hom(H_{k + l}(S^k \times S^l), \mathbb{Z})$.

  Then $f^{\ast}$ can be seen as a homomorphism from $\Hom(H_{k + l}(S^k \times S^l), \mathbb{Z})$ to $\Hom(H_{k + l}(S^{k + l}), \mathbb{Z})$.
  In other words, $f^{\ast}$ and $f_{\ast}$ are the dual of each other.
  Therefore, $f^{\ast} = 0$ implies $f_{\ast} = 0$.
\end{exer}


\end{document}


