\documentclass[12pt, psamsfonts]{amsart}

%-------Packages---------
\usepackage{amssymb,amsfonts}
\usepackage{semantic}
\usepackage{fullpage}
\usepackage{tikz-cd}
\usepackage{todonotes}
\usepackage{physics}
\usepackage[all,arc]{xy}
\usepackage{enumerate}
\usepackage{enumitem}
\usepackage{mathrsfs}
\usepackage{theoremref}
\usepackage{graphicx}
\usepackage[bookmarks]{hyperref}

%--------Theorem Environments--------
%theoremstyle{plain} --- default
\newtheorem{thm}{Theorem}[section]
\newtheorem{cor}[thm]{Corollary}
\newtheorem{prop}[thm]{Proposition}
\newtheorem{lem}[thm]{Lemma}
\newtheorem{conj}[thm]{Conjecture}
\newtheorem{quest}[thm]{Question}

\theoremstyle{definition}
\newtheorem{defn}[thm]{Definition}
\newtheorem{defns}[thm]{Definitions}
\newtheorem{con}[thm]{Construction}
\newtheorem{exmp}[thm]{Example}
\newtheorem{exmps}[thm]{Examples}
\newtheorem{notn}[thm]{Notation}
\newtheorem{notns}[thm]{Notations}
\newtheorem{addm}[thm]{Addendum}
\newtheorem*{exer}{Exercise}

\theoremstyle{remark}
\newtheorem{rem}[thm]{Remark}
\newtheorem{rems}[thm]{Remarks}
\newtheorem{warn}[thm]{Warning}
\newtheorem{sch}[thm]{Scholium}

\DeclareMathOperator{\Hom}{Hom}
\DeclareMathOperator{\Id}{Id}
\DeclareMathOperator{\End}{End}
\DeclareMathOperator{\ord}{ord}
\DeclareMathOperator{\Aut}{Aut}
\DeclareMathOperator{\Gal}{Gal}

\makeatletter
\let\c@equation\c@thm
\makeatother
\numberwithin{equation}{section}

\bibliographystyle{plain}

\begin{document}

\title{Math 612(Homework 5)}
\author{Hidenori Shinohara}
\maketitle

\begin{exer}{(2.2.7)}
  Let $f(x_1, \cdots, x_n) = (-x_1, x_2, x_3, \cdots, x_n)$.
  Then
  \begin{center}
    \begin{tikzcd}[cells={nodes={minimum height=2em}}]
      \mathbb{R}^n \setminus \{ 0 \} \arrow[r, "f"] \arrow[d, "r"]   & \mathbb{R}^n \setminus \{ 0 \} \arrow[d, "r"] \\
      S^{n - 1} \arrow[r, "\text{reflection}"] & S^{n - 1}
    \end{tikzcd}
  \end{center}
  where $r$ is the obvious deformation retraction.
  By (e) on P.134, the reflection map induces -1 on $H^{n - 1}(S^{n - 1})$.
  By naturality, $f_{\ast}$ is -1.

  Similarly, let $f(x_1, \cdots, x_n) = (cx_1, x_2, x_3, \cdots, x_n)$ with $c > 0$.
  Then
  \begin{center}
    \begin{tikzcd}[cells={nodes={minimum height=2em}}]
      \mathbb{R}^n \setminus \{ 0 \} \arrow[r, "f"] \arrow[d, "r"]   & \mathbb{R}^n \setminus \{ 0 \} \arrow[d, "r"] \\
      S^{n - 1} \arrow[r, "g"] & S^{n - 1}
    \end{tikzcd}
  \end{center}
  where $r$ is the obvious deformation retraction.
  Then $g$ is a function that is homotopy equivalent to the identity map on $S^{n - 1}$.
  By (e) on P.134, $g$ induces the identity map on $H^{n - 1}(S^{n - 1})$.
  By naturality, $f_{\ast}$ is 1.

  Using the exact same argument, $(x_1, \cdots, x_i, \cdots, x_j, \cdots, x_n) \mapsto (x_1, \cdots, x_j, \cdots, x_i, \cdots, x_n)$ induces -1 because a reflection is -1 and $(x_1, \cdots, x_i, \cdots, x_j, \cdots, x_n) \mapsto (x_1, \cdots, x_i, \cdots, x_j + x_i, \cdots, x_n)$ induces 1 because homotopy equivalent maps induce the same map.
  Therefore, we have shown that elementary matrices induce 1 or -1 based on the sign of their determinants.
  Any invertible linear operation can be written as a product of elementary matrices and since $(fg)_{\ast} = f_{\ast}g_{\ast}$ the given invertible linear operation induces 1 or -1 based on the sign of their determinants.
\end{exer}

3.3 (p. 257): 1, 2, 3. (We will talk a lot about the concept of orientability in class the Monday after break, but feel to start reading up.

And also the following: Show that there exists a homeomorphism $f: CP^n \to CP^n$ whose induced map on $H^{2n}(CP^n;Z)$ is multiplication by -1 iff n is odd.


\end{document}


