\documentclass[12pt, psamsfonts]{amsart}

%-------Packages---------
\usepackage{amssymb,amsfonts}
\usepackage{semantic}
\usepackage{fullpage}
\usepackage{tikz-cd}
\usepackage{todonotes}
\usepackage{physics}
\usepackage[all,arc]{xy}
\usepackage{enumerate}
\usepackage{enumitem}
\usepackage{mathrsfs}
\usepackage{theoremref}
\usepackage{graphicx}
\usepackage[bookmarks]{hyperref}

%--------Theorem Environments--------
%theoremstyle{plain} --- default
\newtheorem{thm}{Theorem}[section]
\newtheorem{cor}[thm]{Corollary}
\newtheorem{prop}[thm]{Proposition}
\newtheorem{lem}[thm]{Lemma}
\newtheorem{conj}[thm]{Conjecture}
\newtheorem{quest}[thm]{Question}

\theoremstyle{definition}
\newtheorem{defn}[thm]{Definition}
\newtheorem{defns}[thm]{Definitions}
\newtheorem{con}[thm]{Construction}
\newtheorem{exmp}[thm]{Example}
\newtheorem{exmps}[thm]{Examples}
\newtheorem{notn}[thm]{Notation}
\newtheorem{notns}[thm]{Notations}
\newtheorem{addm}[thm]{Addendum}
\newtheorem*{exer}{Exercise}

\theoremstyle{remark}
\newtheorem{rem}[thm]{Remark}
\newtheorem{rems}[thm]{Remarks}
\newtheorem{warn}[thm]{Warning}
\newtheorem{sch}[thm]{Scholium}

\DeclareMathOperator{\Hom}{Hom}
\DeclareMathOperator{\Id}{Id}
\DeclareMathOperator{\End}{End}
\DeclareMathOperator{\ord}{ord}
\DeclareMathOperator{\Aut}{Aut}
\DeclareMathOperator{\Gal}{Gal}
\DeclareMathOperator{\RP}{\mathbb{R}P}

\makeatletter
\let\c@equation\c@thm
\makeatother
\numberwithin{equation}{section}

\bibliographystyle{plain}

\begin{document}

\title{Math 612 (Homework 1)}
\author{Hidenori Shinohara}
\maketitle

\begin{exer}{(Exercise 1(a))}
  The case of $G = \mathbb{Z}$ is discussed in Example 2.42.
  \begin{align*}
    H_k(\RP^n; \mathbb{Z}) &= \begin{cases}
      \mathbb{Z} & \text{for $k = 0$ and for $k = n$ odd} \\
      \mathbb{Z}_2 & \text{for $k$ odd, $0 < k < n$} \\
      0 & \text{otherwise}.
    \end{cases}
  \end{align*}
  Suppose $n$ is even.
  For any field $F$, we obtain the cellular chain complex

  \begin{align*}
    0 \rightarrow F \xrightarrow{2} F \xrightarrow{0} \cdots \xrightarrow{2} F \xrightarrow{0} F \rightarrow 0.
  \end{align*}

  If the characteristic is 2, then all maps are 0.
  Therefore, $H_k(\RP^n; F) = F$ if $k \leq n$ and $H_k(\RP^n; F) = 0$ otherwise.
  If the characteristic is not 2, then $H_0(\RP^n; F) = F$ and all other homology groups are 0.
  If $n$ is odd, we obtain
  \begin{align*}
    0 \rightarrow F \xrightarrow{0} F \xrightarrow{2} \cdots \xrightarrow{2} F \xrightarrow{0} F \rightarrow 0.
  \end{align*}
  If the characteristic is 2, $H_k(\RP^n; F) = F$ if $k \leq n$ and $H_k(\RP^n; F) = 0$ otherwise.
 Otherwise, $H_0(\RP^n; F) = H_n(\RP^n; F) = F$ and all other homology groups are 0.
\end{exer}

\begin{exer}{(Exercise 1(b))}
  As discussed in Example 2.37, $H_2(N_g; \mathbb{Z}) = 0, H_1(N_g; \mathbb{Z}) = \mathbb{Z}^{g - 1} \oplus \mathbb{Z}_2$, and $H_0(N_g; \mathbb{Z}) = \mathbb{Z}$.
  For a field $F$, the cellular chain complex is

  \begin{align*}
    0 \rightarrow F \xrightarrow{d_2} F^g \xrightarrow{d_1} F \rightarrow 0.
  \end{align*}

  As discussed in Example 2.37, $d_2(1) = (2, 2, \cdots, 2)$ and $d_1 = 0$.
  If the characteristic of $F$ is 2, then $H_2(X; F) = H_0(X; F) = F$ and $H_1(X; F) = F^g$.
  Otherwise, then $H_2(X; F) = 0, H_1(X; F) = F^{g - 1}$ and $H_0(X; F) = F$.
\end{exer}

\begin{exer}{(Exercise 1(c))}
  For a $Z$-module $R$, we have
  \begin{align*}
    0 \rightarrow R \xrightarrow{0} R \xrightarrow{a} R \xrightarrow{0} R \xrightarrow 0.
  \end{align*}

  When $R = \mathbb{Z}$, we obtain
  \begin{align*}
    H_k(X; \mathbb{Z}) &= \begin{cases}
      \mathbb{Z} & \text{for $k = 0, 2n - 1$} \\
      \mathbb{Z}_m & \text{for $k$ odd, $0 < k < 2n - 1$} \\
      0 & \text{otherwise}.
    \end{cases}
  \end{align*}

  When $R$ is a field with characteristic that divides $a$, $H_i(X; R) = R$ if $i = 0, 1, 2, 3$.
  If $R$ is a field with characteristic that does not divide $a$, $H_3(X; R) = H_0(X; R) = R$ and all other cohomology groups are 0.
\end{exer}

\end{document}


