\documentclass[12pt, psamsfonts]{amsart}

%-------Packages---------
\usepackage{amssymb,amsfonts}
\usepackage{semantic}
\usepackage{fullpage}
\usepackage{tikz-cd}
\usepackage{todonotes}
\usepackage{physics}
\usepackage[all,arc]{xy}
\usepackage{enumerate}
\usepackage{enumitem}
\usepackage{mathrsfs}
\usepackage{theoremref}
\usepackage{graphicx}
\usepackage[bookmarks]{hyperref}

%--------Theorem Environments--------
%theoremstyle{plain} --- default
\newtheorem{thm}{Theorem}[section]
\newtheorem{cor}[thm]{Corollary}
\newtheorem{prop}[thm]{Proposition}
\newtheorem{lem}[thm]{Lemma}
\newtheorem{conj}[thm]{Conjecture}
\newtheorem{quest}[thm]{Question}

\theoremstyle{definition}
\newtheorem{defn}[thm]{Definition}
\newtheorem{defns}[thm]{Definitions}
\newtheorem{con}[thm]{Construction}
\newtheorem{exmp}[thm]{Example}
\newtheorem{exmps}[thm]{Examples}
\newtheorem{notn}[thm]{Notation}
\newtheorem{notns}[thm]{Notations}
\newtheorem{addm}[thm]{Addendum}
\newtheorem*{exer}{Exercise}

\theoremstyle{remark}
\newtheorem{rem}[thm]{Remark}
\newtheorem{rems}[thm]{Remarks}
\newtheorem{warn}[thm]{Warning}
\newtheorem{sch}[thm]{Scholium}

\DeclareMathOperator{\Hom}{Hom}
\DeclareMathOperator{\Id}{Id}
\DeclareMathOperator{\End}{End}
\DeclareMathOperator{\ord}{ord}
\DeclareMathOperator{\Aut}{Aut}
\DeclareMathOperator{\Gal}{Gal}
\DeclareMathOperator{\Tor}{Tor}
\DeclareMathOperator{\RP}{\mathbb{R}P}

\makeatletter
\let\c@equation\c@thm
\makeatother
\numberwithin{equation}{section}

\bibliographystyle{plain}

\begin{document}

\title{Math 612 (Homework 1)}
\author{Hidenori Shinohara}
\maketitle

\begin{exer}{(Exercise 1(a))}
  The case of $G = \mathbb{Z}$ is discussed in Example 2.42.
  \begin{align*}
    H_k(\RP^n; \mathbb{Z}) &= \begin{cases}
      \mathbb{Z} & \text{for $k = 0$ and for $k = n$ odd} \\
      \mathbb{Z}_2 & \text{for $k$ odd, $0 < k < n$} \\
      0 & \text{otherwise}.
    \end{cases}
  \end{align*}
  Suppose $n$ is even.
  For any abelian group $G$, we obtain the cellular chain complex

  \begin{align*}
    0 \rightarrow G \xrightarrow{2} G \xrightarrow{0} \cdots \xrightarrow{2} G \xrightarrow{0} G \rightarrow 0.
  \end{align*}

  If $n$ is odd, we obtain
  \begin{align*}
    0 \rightarrow G \xrightarrow{0} G \xrightarrow{2} \cdots \xrightarrow{2} G \xrightarrow{0} G \rightarrow 0.
  \end{align*}

  \begin{itemize}
    \item
      Suppose $k$ is even and $2 \leq k \leq n$.
      The homology at $\xrightarrow{0} G \xrightarrow{2}$ is
      \begin{itemize}
        \item
          $0$ if $G = \mathbb{Q}, \mathbb{Z}/p^l\mathbb{Z}$ with $p \ne 2$.
        \item
          $\mathbb{Z} / 2\mathbb{Z}$ if $G = \mathbb{Z}/2^l$.
      \end{itemize}
    \item
      Suppose $k$ is odd and $1 \leq k \leq n - 1$.
      The homology at $\xrightarrow{2} G \xrightarrow{0}$ is
      \begin{itemize}
        \item
          $G / 2G \cong 0$ if $G = \mathbb{Q}, \mathbb{Z}/p^l\mathbb{Z}$ with $p \ne 2$ because multiplication by 2 is an isomorphism.
        \item
          $\mathbb{Z} / 2\mathbb{Z}$ if $G = \mathbb{Z}/2^l$.
      \end{itemize}
    \item
      Suppose $k = n$ and $n$ is odd, or $k = 0$.
      The homology at $\xrightarrow{0} G \xrightarrow{0}$ is $G$.
  \end{itemize}

  When $G = \mathbb{Q}$, the universal coefficient theorem gives an isomorphism $H_k(X) \otimes Q \cong H_k(X; \mathbb{Q})$ since $Q$ is torsion free.
  $\mathbb{Z} \otimes \mathbb{Q} \cong \mathbb{Q}$ and $\mathbb{Z} / 2 \otimes \mathbb{Q} = 0$ because 2 is invertible in $\mathbb{Q}$.
  This agrees with the results above.

  When $G = \mathbb{Z} / 2^l$, we have $0 \rightarrow H_k(X) \otimes G \rightarrow H_k(X; G) \rightarrow \Tor(H_{k - 1}(C), G) \rightarrow 0$.
  If $k = n$ and $k$ is odd, $H_k(X) = \mathbb{Z}$, so $\mathbb{Z} / 2^l \cong H_k(X; \mathbb{Z} / 2^l)$.
  If $k - 1 = n$ and $k - 1$ is odd, we obtain $0 \rightarrow 0 \rightarrow H_k(X; \mathbb{Z} / 2^l) \rightarrow \Tor(\mathbb{Z}, \mathbb{Z} / 2^l) \rightarrow 0$, so $H_k(X; \mathbb{Z} / 2^l) = 0$.
  If $k$ is odd and $0 < k < n$, $0 \rightarrow \mathbb{Z} / 2 \otimes \mathbb{Z} / 2^l \rightarrow H_k(X; \mathbb{Z} / 2^l) \rightarrow \Tor(H_{k - 1}(X), \mathbb{Z} / 2^l) \rightarrow 0$.
  The $\Tor$ is 0 because if $k = 0$, $H_{k - 1}(X) = \mathbb{Z}$ and $H_{k - 1}(X) = 0$ otherwise.
  Thus $H_k(X; \mathbb{Z} / 2^l) = \mathbb{Z} / 2 \otimes \mathbb{Z} / 2^l = \mathbb{Z} / 2$.
  In any other cases, the universal coefficient theorem gives the SES $0 \rightarrow 0 \rightarrow H_n(X; G) \rightarrow 0 \rightarrow 0$.
  This agrees with the results above.

  Suppose $G = \mathbb{Z} / p^l$.
  Then the case that $k = n$ and $k$ is odd and the case that $k - 1 = n$ and $k$ is odd can be handled in the same way as above.
  Suppose $k$ is odd and $0 < k < n$.
  Then $\mathbb{Z} / 2 \otimes \mathbb{Z} / p^l = 0$.
  Moreover, $\Tor(H_{k - 1}(X), \mathbb{Z}) = 0$ as discussed above.
  Thus $H_{k}(X) = 0$.
  In any other cases, the universal coefficient theorem gives the SES $0 \rightarrow 0 \rightarrow H_n(X; G) \rightarrow 0 \rightarrow 0$.
  This agrees with the results above.
\end{exer}


\begin{exer}{(Exercise 1(b))}
  As discussed in Example 2.37, $H_2(N_g; \mathbb{Z}) = 0, H_1(N_g; \mathbb{Z}) = \mathbb{Z}^{g - 1} \oplus \mathbb{Z}_2$, and $H_0(N_g; \mathbb{Z}) = \mathbb{Z}$.
  For an abelian group $G$, the cellular chain complex is

  \begin{align*}
    0 \rightarrow G \xrightarrow{d_2} G^g \xrightarrow{d_1} G \rightarrow 0.
  \end{align*}

  As discussed in Example 2.37, $d_2(1) = (2, 2, \cdots, 2)$ and $d_1 = 0$.
  If $G = \mathbb{Z} / p^l$ with $p \ne 2$ or $G = \mathbb{Q}$, then $H_2(X; G) = 0, H_1(X; G) = G^g / \ev{(1, \cdots, 1)} = G^{g - 1}$ and $H_0(X; G) = G$ because $2^{-1}$ exists.
  Suppose $G = \mathbb{Z} / 2^l$.
  Then $H_2(X; G) = \mathbb{Z} / 2$ because the kernel is an index-2 subgroup.
  $H_1(X; G) = G^g / \ev{(2a, \cdots, 2a)} = G^{g - 1} \otimes \mathbb{Z} / 2$, and $H_0(X; G) = G$.

  We will verify the results using the universal coefficient theorem.

  Suppose $G = \mathbb{Q}$.
  Then $\Tor(H_{n - 1}(C), G) = 0$ for any $n$.
  Thus $H_0(X; G) = \mathbb{Z} \otimes \mathbb{Q} = \mathbb{Q}$ and $H_1(X; G) = (\mathbb{Z}^{g - 1} \oplus \mathbb{Z}_2) \otimes \mathbb{Q} = (\mathbb{Z} \otimes \mathbb{Q})^{g - 1} \oplus (\mathbb{Z}_2 \otimes \mathbb{Q}) = \mathbb{Q}^{g - 1}$.

  Suppose $G = \mathbb{Z} / p^l$ with $p \ne 2$.
  When $n = 1$, $H_{n - 1}(C) = \mathbb{Z}$, so $\Tor(H_{n - 1}(C), G) = 0$.
  Thus $H_1(C; G) = (\mathbb{Z}^{g - 1} \oplus \mathbb{Z}_2) \otimes \mathbb{Z} / p^l = (\mathbb{Z} / p^l)^{g - 1}$.
  When $n = 2$, $H_n(C) = 0$ and $\Tor(H_{n - 1}(C), \mathbb{Z} / p^l) = 0$ because multiplication by $p^l$ does not kill any element in $\mathbb{Z}^{g - 1} \oplus \mathbb{Z} / 2$.

  Suppose $G = \mathbb{Z} / 2^l$.
  When $n = 1$, $\Tor(H_{n - 1}(C), G) = \Tor(\mathbb{Z}, G) = 0$.
  Thus $H_n(C; G) = H_n(C) \otimes G = (\mathbb{Z} / 2^l)^{g - 1} \oplus \mathbb{Z} / 2$.
  When $n = 2$, $H_n(C) = 0$ and $\Tor(H_{n - 1}(C), \mathbb{Z} / 2^l) = \ker((\mathbb{Z}^{g - 1} \oplus \mathbb{Z}_2) \xrightarrow{2^l} (\mathbb{Z}^{g - 1} \oplus \mathbb{Z}_2)) = \mathbb{Z} / 2$.
  Thus $H_2(C; G) = \mathbb{Z} / 2$.
\end{exer}

\begin{exer}{(Exercise 1(c))}
  For a $Z$-module $R$, we have
  \begin{align*}
    0 \rightarrow R \xrightarrow{0} R \xrightarrow{a} R \xrightarrow{0} R \rightarrow 0.
  \end{align*}

  Clearly, $H_k(X; R) = 0$ for $k \geq 4$ for any $R$.
  \begin{itemize}
    \item
      When $k = 0, 3$, $H_k(X; R) = R / 0 = R$.
    \item
      $H_2(X; R) = \ker(R \xrightarrow{a} R)$.
      When $R = \mathbb{Z}, \mathbb{Q}$, the kernel is 0.
      When $R = \mathbb{Z} / p^k$, the kernel is isomorphic to $\mathbb{Z} / \gcd(p^k, a)$.
    \item
      $H_1(X; R) = R / aR$.
      Thus $H_1(X; \mathbb{Q}) = 0$.
      $H_1(X; \mathbb{Z}) = \mathbb{Z} / a\mathbb{Z}$.
      When $R = \mathbb{Z} / p^k$, we obtain $(\mathbb{Z} / p^k) / a(\mathbb{Z} / p^k) = \mathbb{Z} / \gcd(p^k, a)$.
  \end{itemize}
  \todo[inline,caption={}]{
    Use the UCT to verify the results.
  }
\end{exer}

\begin{exer}{(Exercise 3(a))}
  $e_i \mapsto t^ix$ is an isomorphism between $C_1^{CW}(X)$ and a free module generated over $\mathbb{Z}[t, t^{-1}]$ where $x$ is the only element in a basis.
  Similarly, $f_i \mapsto t^iy$ gives an isomorphism.
  With this identification, the boundary map $f_i \mapsto -e_i + 2e_{i + 1}$ becomes $(\sum a_it^{b_i}) x \mapsto (\sum a_i(-t^{b_i} + 2t^{b_i + 1}))x$ which is clearly $\mathbb{Z}[t, t^{-1}]$-linear.
  Moreover, the property that $d^2 = 0$ is clearly preserved after the identification, so the homology groups, which are just the kernels modulo the images, must be $\mathbb{Z}[t, t^{-1}]$-modules.
\end{exer}

\begin{exer}{(Exercise 3(b))}
  Since $d_2: 2f_0 \mapsto -e_0 + 2e_1$, $x \mapsto -x + 2tx = (2t - 1)x$ after the identification described above.
  Then for all $\alpha \in \mathbb{Z}[t, t^{-1}]$, $d_2(\alpha x) = 0 \implies (2t - 1)\alpha = 0 \implies \alpha = 0$.
  Thus $H_2(X) = 0$.

  $d_1 = 0$ because there is only one 0-cell.
  Thus $H_1(X) = \mathbb{Z}[t, t^{-1}] / (2t - 1)$.
  This is isomorphic to $\mathbb{Z}[1/2]$ because the kernel of the homomorphism $\phi: \mathbb{Z}[t, t^{-1}] \mapsto \mathbb{Z}[1/2]$ defined by $t \mapsto 1/2$ is $(2t - 1)$.
\end{exer}

\begin{exer}{(Exercise 3(c))}
  We will use the universal coefficient theorem with the values we have calculated: $H_2(X) = 0, H_1(X) = \mathbb{Z}[1/2], H_0(X) = \mathbb{Z}$.
  \begin{itemize}
    \item
      $\mathbb{Q}$.
      The UCT states $H_k(X, \mathbb{Q}) = (H_k(X) \otimes \mathbb{Q}) \oplus \Tor(H_{k - 1}(X), \mathbb{Q})$.
      $\Tor(H_{k - 1}(X), \mathbb{Q}) = 0$ because $\mathbb{Q}$ is torsion-free.
      \begin{align*}
        H_k(X, \mathbb{Q}) = \begin{cases}
          0 & (k = 2) \\
          \mathbb{Q} & (k = 0, 1).
        \end{cases}
      \end{align*}
    \item
      $\mathbb{Z}/p^k$ with $p \ne 2$.
      The UCT states $H_k(X, \mathbb{Z}/p^k) = (H_k(X) \otimes \mathbb{Z}/p^k) \oplus \Tor(H_{k - 1}(X), \mathbb{Z}/p^k)$.
      Since $\Tor(H_{k - 1}(X), \mathbb{Z}/p^k) = \ker(\mathbb{Z}[1/2] \xrightarrow{p^k} \mathbb{Z}[1/2]) = 0$, it suffices to consider the tensor product.
      \begin{align*}
        H_k(X, \mathbb{Z}/p^k) = \begin{cases}
          0 & (k = 2) \\
          \mathbb{Z}/p^k & (k = 0, 1).
        \end{cases}
      \end{align*}
    \item
      $\mathbb{Z}/2^k$.
      The UCT states $H_k(X, \mathbb{Z}/2^k) = (H_k(X) \otimes \mathbb{Z}/2^k) \oplus \Tor(H_{k - 1}(X), \mathbb{Z}/2^k)$.
      Again, $\Tor(H_{k - 1}(X), \mathbb{Z}/2^k) = \ker(\mathbb{Z}[1/2] \xrightarrow{2^k} \mathbb{Z}[1/2]) = 0$.
      \begin{align*}
        H_k(X, \mathbb{Z}/2^k) = \begin{cases}
          0 & (k = 2, k = 1) \\
          \mathbb{Z}/2^k & (k = 0).
        \end{cases}
      \end{align*}

      When $k = 1$, $\mathbb{Z}[1/2] \otimes \mathbb{Z}/2^k = 0$ because $a \otimes b = a/2^k \otimes 2^kb = 0$.
  \end{itemize}
\end{exer}

\end{document}


