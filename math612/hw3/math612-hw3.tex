\documentclass[12pt, psamsfonts]{amsart}

%-------Packages---------
\usepackage{amssymb,amsfonts}
\usepackage{semantic}
\usepackage{fullpage}
\usepackage{tikz-cd}
\usepackage{todonotes}
\usepackage{physics}
\usepackage[all,arc]{xy}
\usepackage{enumerate}
\usepackage{enumitem}
\usepackage{mathrsfs}
\usepackage{theoremref}
\usepackage{graphicx}
\usepackage[bookmarks]{hyperref}

%--------Theorem Environments--------
%theoremstyle{plain} --- default
\newtheorem{thm}{Theorem}[section]
\newtheorem{cor}[thm]{Corollary}
\newtheorem{prop}[thm]{Proposition}
\newtheorem{lem}[thm]{Lemma}
\newtheorem{conj}[thm]{Conjecture}
\newtheorem{quest}[thm]{Question}

\theoremstyle{definition}
\newtheorem{defn}[thm]{Definition}
\newtheorem{defns}[thm]{Definitions}
\newtheorem{con}[thm]{Construction}
\newtheorem{exmp}[thm]{Example}
\newtheorem{exmps}[thm]{Examples}
\newtheorem{notn}[thm]{Notation}
\newtheorem{notns}[thm]{Notations}
\newtheorem{addm}[thm]{Addendum}
\newtheorem*{exer}{Exercise}

\theoremstyle{remark}
\newtheorem{rem}[thm]{Remark}
\newtheorem{rems}[thm]{Remarks}
\newtheorem{warn}[thm]{Warning}
\newtheorem{sch}[thm]{Scholium}

\DeclareMathOperator{\Hom}{Hom}
\DeclareMathOperator{\Id}{Id}
\DeclareMathOperator{\End}{End}
\DeclareMathOperator{\ord}{ord}
\DeclareMathOperator{\Aut}{Aut}
\DeclareMathOperator{\Gal}{Gal}

\makeatletter
\let\c@equation\c@thm
\makeatother
\numberwithin{equation}{section}

\bibliographystyle{plain}

\begin{document}

\title{Math 612 (Homework 3)}
\author{Hidenori Shinohara}
\maketitle

\begin{exer}{(3.1.11)}
  Since $M(\mathbb{Z}_m, n)$ consists of $e^0, e^n, e^{n + 1}$, we obtain
  \begin{align*}
    \tilde{H}_i(X) &= \begin{cases}
      \mathbb{Z} / m\mathbb{Z} & (i = n) \\
      0 & (i \ne n).
    \end{cases}
  \end{align*}

  From previous homework,
  \begin{align*}
    \tilde{H}_i(X / S^n) = \tilde{H}_i(S^{n + 1}) &= \begin{cases}
      \mathbb{Z} & (i = n + 1) \\
      0 & (i \ne n + 1).
    \end{cases}
  \end{align*}

  Thus the map on $\tilde{H}_i(-; \mathbb{Z})$ is the zero map for each $i$.
  On the other hand, the long exact sequence of a pair gives us $\tilde{H}^{n + 1}(X, S^n; \mathbb{Z}) \xrightarrow{q^{\ast}} \tilde{H}^{n + 1}(X; \mathbb{Z}) \rightarrow \tilde{H}^{n + 1}(S^n; \mathbb{Z})$ where $\tilde{H}^{n + 1}(S^n; \mathbb{Z}) = 0$, so $q^{\ast}$ is surjective.
  Therefore, it is nontrivial because $\tilde{H}^{n + 1}(X; \mathbb{Z}) \ne 0$.

  \todo[inline,caption={}]{
    Natural?
  }

  The long exact sequence of a pair gives us $\tilde{H}_n(S^n; \mathbb{Z}) \rightarrow \tilde{H}_n(X; \mathbb{Z}) \rightarrow \tilde{H}_n(X, S^n; \mathbb{Z}) = \tilde{H}_n(S^{n + 1}; \mathbb{Z}) = 0$ which implies the surjectivity of the induced map.
  Since $\tilde{H}_n(X; \mathbb{Z}) \ne 0$, the induced map is nonzero.
\end{exer}

\begin{exer}{(3.1.13)}
\end{exer}

\begin{exer}{(3.2.1)}
\end{exer}

\begin{exer}{(3.2.2)}
\end{exer}

\begin{exer}{(3.2.3)}
\end{exer}

\begin{exer}{(3.2.6)}
\end{exer}

\begin{exer}{(3.2.7)}
\end{exer}

\end{document}


