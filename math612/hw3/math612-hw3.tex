\documentclass[12pt, psamsfonts]{amsart}

%-------Packages---------
\usepackage{amssymb,amsfonts}
\usepackage{semantic}
\usepackage{fullpage}
\usepackage{tikz-cd}
\usepackage{todonotes}
\usepackage{physics}
\usepackage[all,arc]{xy}
\usepackage{enumerate}
\usepackage{enumitem}
\usepackage{mathrsfs}
\usepackage{theoremref}
\usepackage{graphicx}
\usepackage[bookmarks]{hyperref}

%--------Theorem Environments--------
%theoremstyle{plain} --- default
\newtheorem{thm}{Theorem}[section]
\newtheorem{cor}[thm]{Corollary}
\newtheorem{prop}[thm]{Proposition}
\newtheorem{lem}[thm]{Lemma}
\newtheorem{conj}[thm]{Conjecture}
\newtheorem{quest}[thm]{Question}

\theoremstyle{definition}
\newtheorem{defn}[thm]{Definition}
\newtheorem{defns}[thm]{Definitions}
\newtheorem{con}[thm]{Construction}
\newtheorem{exmp}[thm]{Example}
\newtheorem{exmps}[thm]{Examples}
\newtheorem{notn}[thm]{Notation}
\newtheorem{notns}[thm]{Notations}
\newtheorem{addm}[thm]{Addendum}
\newtheorem*{exer}{Exercise}

\theoremstyle{remark}
\newtheorem{rem}[thm]{Remark}
\newtheorem{rems}[thm]{Remarks}
\newtheorem{warn}[thm]{Warning}
\newtheorem{sch}[thm]{Scholium}

\DeclareMathOperator{\Hom}{Hom}
\DeclareMathOperator{\Id}{Id}
\DeclareMathOperator{\End}{End}
\DeclareMathOperator{\ord}{ord}
\DeclareMathOperator{\Aut}{Aut}
\DeclareMathOperator{\Gal}{Gal}
\DeclareMathOperator{\Ext}{Ext}
\DeclareMathOperator{\RP}{\mathbb{R}P}

\makeatletter
\let\c@equation\c@thm
\makeatother
\numberwithin{equation}{section}

\bibliographystyle{plain}

\begin{document}

\title{Math 612 (Homework 3)}
\author{Hidenori Shinohara}
\maketitle

\begin{exer}{(3.1.11)}
  Using the cellular homology, we obtain
  \begin{align*}
    \tilde{H}_i(X) &= \begin{cases}
      \mathbb{Z} / m\mathbb{Z} & (i = n) \\
      0 & (i \ne n).
    \end{cases} \\
    \tilde{H}^i(X) &= \begin{cases}
      \mathbb{Z} / m\mathbb{Z} & (i = n + 1) \\
      0 & (i \ne n + 1).
    \end{cases}
  \end{align*}

  From previous homework,
  \begin{align*}
    \tilde{H}^i(X / S^n) = \tilde{H}_i(S^{n + 1}) &= \begin{cases}
      \mathbb{Z} & (i = n + 1) \\
      0 & (i \ne n + 1).
    \end{cases}
  \end{align*}

  Thus the map on $\tilde{H}_i(-; \mathbb{Z})$ is the zero map for each $i$.
  On the other hand, the long exact sequence of a pair gives us $\tilde{H}^{n + 1}(X, S^n; \mathbb{Z}) \xrightarrow{q^{\ast}} \tilde{H}^{n + 1}(X; \mathbb{Z}) \rightarrow \tilde{H}^{n + 1}(S^n; \mathbb{Z})$ where $\tilde{H}^{n + 1}(S^n; \mathbb{Z}) = 0$, so $q^{\ast}$ is surjective.
  Therefore, it is nontrivial because $\tilde{H}^{n + 1}(X; \mathbb{Z}) \ne 0$.

  \begin{center}
    \begin{tikzcd}[cells={nodes={minimum height=2em}}]
      0 \arrow[r] & \Ext(H_n(X); \mathbb{Z})     \arrow[r] \arrow[d] &  H^{n + 1}(X; \mathbb{Z})       \arrow[r] \arrow[d] &  \Hom(H_{n + 1}(X); \mathbb{Z}) \arrow[r] \arrow[d] & 0 \\
      0 \arrow[r] & \Ext(H_n(X/S^n); \mathbb{Z}) \arrow[r]           &  H^{n + 1}(X / S^n; \mathbb{Z}) \arrow[r]           &  \Hom(H_{n + 1}(X / S^n); \mathbb{Z}) \arrow[r]    & 0
    \end{tikzcd}
  \end{center}

  is

  \begin{center}
    \begin{tikzcd}[cells={nodes={minimum height=2em}}]
      0 \arrow[r] & \mathbb{Z}_m \arrow[r] \arrow[d] &  \mathbb{Z}_m  \arrow[r] \arrow[d] &  0 \arrow[r] \arrow[d] & 0 \\
      0 \arrow[r] & 0            \arrow[r]           &  \mathbb{Z}    \arrow[r]           &  \mathbb{Z} \arrow[r]  & 0.
    \end{tikzcd}
  \end{center}

  This splitting is not natural because the middle term in the first sequence is isomorphic to $\mathbb{Z}_m \oplus 0$ and the second one is $0 \oplus \mathbb{Z}$.


  The long exact sequence of a pair gives us $\tilde{H}_n(S^n; \mathbb{Z}) \rightarrow \tilde{H}_n(X; \mathbb{Z}) \rightarrow \tilde{H}_n(X, S^n; \mathbb{Z}) = \tilde{H}_n(S^{n + 1}; \mathbb{Z}) = 0$ which implies the surjectivity of the induced map.
  Since $\tilde{H}_n(X; \mathbb{Z}) \ne 0$, the induced map is nonzero.

  The map induced on $\tilde{H}^i(-;\mathbb{Z})$ is the zero map for any $i$ because at least one of $\tilde{H}^i(S^n; \mathbb{Z})$ or $\tilde{H}^i(X; \mathbb{Z})$ is 0 for each $i$.
\end{exer}

\begin{exer}{(3.1.13)}
  Let $\Phi: \ev{X, Y} \rightarrow \Hom(H_1(X), H_1(Y))$ denote the map in the problem statement.
  \begin{itemize}
    \item
      $\Phi$ is well-defined because homotopy equivalent maps induce the same homomorphisms on homology classes.
    \item
      Let $f, g \in \ev{X, Y}$ be given such that $f_{\ast} = g_{\ast}$.
      Let $q: \pi_1(X) \rightarrow H_1(X)$ be the canonical quotient map as $H_1(X)$ is the abelianization of $\pi_1(X)$.
      Since $\pi_1(Y) = G$ is abelian, $\pi_1(Y) = H_1(Y)$.
      This implies that $f_{\ast} \circ q, g_{\ast} \circ q$ are both homomorphisms from $\pi_1(X)$ to $\pi_1(Y)$.
      By Proposition 1B.9, such homomorphisms must be induced by a map $(X, x_0) \rightarrow (Y, y_0)$ that is unique up to homotopy fixing the base point.
      In other words, $f = g$ in $\ev{X, Y}$.
    \item
      For any $\phi \in \Hom(H_1(X), H_1(Y))$, we obtain $\phi \circ q \in \Hom(\pi_1(X), \pi_1(Y))$.
      By Proposition 1B.9, there exists a map $f \in \ev{X, Y}$ that induces $\phi \circ q$.
      Then $f_{\ast}: H_1(X) \rightarrow H_1(Y)$ equals $\phi$ since each equivalence class in $H_1$ and $\pi_1$ denotes a path in the corresponding space and the induced map by $f$ simply maps a path into another path in the other space while respecting the equivalence class the path is in.
  \end{itemize}
\end{exer}

\begin{exer}{(3.2.1)}
  $H^0(M_g) = H^2(M_g) = \mathbb{Z}$ and $H^1(M_g) = \mathbb{Z}^{2g}$.
  Thus the only nontrivial cup products are elements among $H^1(M_g)$.
  Let $a_1, \cdots, a_g, b_1, \cdots, b_g$ be generators of $H^1(M_g)$.
  Let $q$ be the quotient map $M_g \rightarrow \vee_g M_1$.
  Then $q^{\ast}: H^1(\vee_g M_1) \rightarrow H^1(M_g)$.
  Since $H^1(\vee_g M_1) = \oplus_g H^1(M_1)$, let $A_i, B_i$ denote generators of the $i$th $H^1(M_1)$ such that $q^{\ast}(A_i) = a_i$ and $q^{\ast}(B_i) = b_i$.
  $H^2(\vee_g M_1) = \oplus_g H^2(M_1)$, and let $c_i$ denote a generator of the $i$th $H^2(M_1)$ such that $\{ C_1, \cdots, C_g \}$ generate $H^2(M_g)$ and $q^{\ast}(C_i) = c_i$.
  Since cup products are natural, they commute with $q^{\ast}$.
  \begin{itemize}
    \item
      $a_i \smile a_i = q^{\ast}(A_i) \smile q^{\ast}(A_i) = q^{\ast}(A_i \smile A_i) = q^{\ast}(0) = 0$.
    \item
      $b_i \smile b_i = q^{\ast}(B_i) \smile q^{\ast}(B_i) = q^{\ast}(B_i \smile B_i) = q^{\ast}(0) = 0$.
    \item
      $a_i \smile b_i = q^{\ast}(A_i) \smile q^{\ast}(B_i) = q^{\ast}(A_i \smile B_i) = q^{\ast}(C_i) = c_i$.
    \item
      All other cases are 0 because the cup product of elements from different ``components" when dealing with a wedge sum of spaces is 0 as discussed in class.
  \end{itemize}
\end{exer}

\begin{exer}{(3.2.2)}
  Suppose $X$ is the union of contractible open sets $A_1, \cdots, A_n$.
  Since each $A_i$ is contractible, $H^k(X, A_i; R) = H^k(X; R)$ for all $k \geq 1$.

  \begin{center}
    \begin{tikzcd}[cells={nodes={minimum height=2em}}]
      H^{k_1}(X, A_1; R) \times \cdots \times H^{k_n}(X, A_n; R) \arrow[r]      \arrow[d, "\cong"] &  H^{k_1 + \cdots + k_n}(X, A_1 \cup \cdots \cup A_n; R) \arrow[d] \\
      H^{k_1}(X; R) \times \cdots \times H^{k_n}(X; R)           \arrow[r, "f"]                    &  H^{k_1 + \cdots + k_n}(X; R).
    \end{tikzcd}
  \end{center}

  This diagram commutes by the naturality of a cup product.
  $H^{k_1 + \cdots + k_n}(X, \bigcup_i A_i; R) = H^{k_1 + \cdots + k_n}(X, X; R) = 0$ for all $k + l \ge 1$.
  By the commutativity of this diagram, the function $f$ must be 0.
\end{exer}

\begin{exer}{(3.2.3(a))}
  Suppose otherwise.
  Let $f: \RP^n \rightarrow \RP^m$ be such a function.
  Then $f$ induces a map on $f^{\ast}: H^{\ast}(\RP^m) \rightarrow H^{\ast}(\RP^n)$.
  In other words, $f^{\ast}: \mathbb{Z}_m[\alpha]/(\alpha^{m + 1}) \rightarrow \mathbb{Z}_n[\beta]/(\beta^{n + 1})$ where $\alpha, \beta$ are generators of $H^1(\RP^m)$ and $H^1(\RP^n)$.
  $H^1(\RP^m; \mathbb{Z}_2) = \mathbb{Z}_2 = \{ 0, \alpha \}$ and $H^1(\RP^n; \mathbb{Z}_2) = \mathbb{Z}_2 = \{ 0, \beta \}$.
  Since $f$ induces a nontrivial map $H^1(\RP^m; \mathbb{Z}_2) \rightarrow H^1(\RP^n; \mathbb{Z}_2)$, $f^{\ast}(\alpha) = \beta$.
  However, $f^{\ast}(0) = f^{\ast}(\alpha^{m + 1}) = (f^{\ast}(\alpha))^{m + 1} = \beta^{m + 1} \ne 0$ because $m < n$.
  This is a contradiction, so such a function does not exist.

  $H^1(\mathbb{C}P^n; \mathbb{Z}_2) = 0$ for any $n$, so there exists no such nontrivial map.
  The case for $H^2(\mathbb{C}P^n)$ can be argued the same way as above because $H^2(\mathbb{C}P^n;\mathbb{Z}_2) = \mathbb{Z}_2[\alpha]/(\alpha^{n + 1})$ where $\alpha$ is a generator of $H^2(\mathbb{C}P^n)$.
\end{exer}

\begin{exer}{(3.2.3(b))}
  Suppose $n \geq 2$ because if $n = 1$, then this can be shown using the intermediate value theorem.
  \begin{center}
    \begin{tikzcd}[cells={nodes={minimum height=2em}}]
      S^n \arrow[d, "p"] \arrow[r, "g"] & S^{n - 1} \arrow[d, "p"]\\
      \RP^n \arrow[r, "g"] & \RP^{n - 1}.
    \end{tikzcd}
  \end{center}

  Let $p$ denote covering maps.
  Let $\gamma$ be a nontrivial loop in $\RP^n$.
  Let $a, -a$ denote the end points of the lift $\tilde{\gamma}$.
  $g(-a) = -g(a)$, so $g$ sends $\tilde{\gamma}$ to a path from $g(a)$ to $g(-a)$.
  Finally, $p$ pushes it down to a nontrivial loop in $\RP^{n - 1}$.
  By the commutativity of the diagram, $g(\gamma)$ is a nontrivial path in $\RP^{n - 1}$.
  Therefore, $f$ induces a nontrivial map from $\pi_1(\RP^n) (\cong \mathbb{Z}_2)$ to $\pi_1(\RP^{n - 1}) (\cong \mathbb{Z}_2)$.
  Thus $f$ induces an isomorphism.
  Since the fundamental groups are abelian, the fundamental groups are isomorphic to the first homology groups.
  By the UCT, $H^1(\RP^n; \mathbb{Z}_2) = \Hom(H_1(\RP^n), \mathbb{Z}_2) = \mathbb{Z}_2$ and $H^1(\RP^{n - 1}; \mathbb{Z}_2) = \Hom(H_1(\RP^{n - 1}), \mathbb{Z}_2) = \mathbb{Z}_2$.
  Then $f$ induces an isomorphism from $H^1(\RP^n; \mathbb{Z}_2)$ into $H^1(\RP^{n - 1}; \mathbb{Z}_2)$.
  This is a contradiction as shown in 3(a).
\end{exer}

\begin{exer}{(3.2.6)}
  For simplicity, we will abuse a notation and let $g$ be the quotient of the map $(z_0, \cdots, z_n) \mapsto (z_0^d, \cdots, z_n^d)$ for any $n$.
  We will first consider the case when $n = 1$.
  Then $\mathbb{C}P^1$ is homeomorphic to $S^2$, so $g^{\ast}: H^2(\mathbb{C}P^1; \mathbb{Z}) \rightarrow H^2(\mathbb{C}P^1; \mathbb{Z})$ is simply multiplication by $d$ since $H^2(\mathbb{C}P^1; \mathbb{Z}) = \mathbb{Z}$.
  Consider the inclusion $i: \mathbb{C}P^1 \rightarrow \mathbb{C}P^n$.
  Then we obtain the following commutative diagram:
  \begin{center}
    \begin{tikzcd}[cells={nodes={minimum height=2em}}]
      H^2(\mathbb{C}P^1; \mathbb{Z})                                   & \arrow[l, "i^{\ast}"] H^2(\mathbb{C}P^n; \mathbb{Z}) \\
      H^2(\mathbb{C}P^1; \mathbb{Z}) \arrow[u, "g^{\ast} = (\cdot d)"] & \arrow[l, "i^{\ast}"] H^2(\mathbb{C}P^n; \mathbb{Z}) \arrow[u, "g^{\ast}"].
    \end{tikzcd}
  \end{center}
  Let $\alpha, \beta$ denote generators of $H^2(\mathbb{C}P^1; \mathbb{Z}), H^2(\mathbb{C}P^n; \mathbb{Z})$.
  Then $i^{\ast}(\beta) = \alpha$.
  Since the diagram commutes, this shows that $g^{\ast}(\beta) = d\beta$.
  Therefore, $g^{\ast}(\beta^k) = (g^{\ast}(\beta))^k = (d\beta)^k = d^k\beta^k$ for any $\beta^k \in H^{\ast}(\mathbb{C}P^n; \mathbb{Z})$.
\end{exer}

\begin{exer}{(3.2.7)}
  Let $f:\RP^3 \rightarrow \RP^2 \vee S^3$ be a homotopy equivalence.
  Then it induces isomorphisms.
  \begin{center}
    \begin{tikzcd}[cells={nodes={minimum height=2em}}]
      H^1(\RP^3; \mathbb{Z}_2) \arrow[d, "f^{\ast}"] & \times & H^2(\RP^3; \mathbb{Z}_2) \arrow[r] \arrow[d, "f^{\ast}"] & H^3(\RP^3; \mathbb{Z}_2) \arrow[d, "f^{\ast}"] \\
      H^1(\RP^2 \vee S^3; \mathbb{Z}_2)              & \times & H^2(\RP^2 \vee S^3; \mathbb{Z}_2) \arrow[r]              & H^3(\RP^2 \vee S^3; \mathbb{Z}_2).
    \end{tikzcd}
  \end{center}

  The cohomology groups of a wedge sum is the direct sum of cohomology groups of the two spaces.
  By rewriting the diagram above with generators, we obtain
  \begin{center}
    \begin{tikzcd}[cells={nodes={minimum height=2em}}]
      \{ 0, \alpha \} \arrow[d, "f^{\ast}"] & \times & \{ 0, \alpha^2 \} \arrow[r] \arrow[d, "f^{\ast}"] & \{ 0, \alpha^3 \} \arrow[d, "f^{\ast}"]\\
      \{ 0, \beta \} \oplus \{ 0, \gamma \} & \times & \{ 0, \beta^2 \} \oplus 0 \arrow[r]               & 0 \oplus \{ 0, \gamma^2 \}.
    \end{tikzcd}
  \end{center}
  This implies $f^{\ast}$ sends $\alpha^2$ to $(\beta^2, 0)$ and $\alpha^3$ to $(0, \gamma^2)$.
  However, this implies $(0, 0) = (f^{\ast}(\alpha^2))^3 = (f^{\ast}(\alpha^3))^2 = (0, \gamma^4) = (0, \gamma)$.
  This is a contradiction because $0 \ne \gamma$.
\end{exer}

\end{document}


