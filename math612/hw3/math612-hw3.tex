\documentclass[12pt, psamsfonts]{amsart}

%-------Packages---------
\usepackage{amssymb,amsfonts}
\usepackage{semantic}
\usepackage{fullpage}
\usepackage{tikz-cd}
\usepackage{todonotes}
\usepackage{physics}
\usepackage[all,arc]{xy}
\usepackage{enumerate}
\usepackage{enumitem}
\usepackage{mathrsfs}
\usepackage{theoremref}
\usepackage{graphicx}
\usepackage[bookmarks]{hyperref}

%--------Theorem Environments--------
%theoremstyle{plain} --- default
\newtheorem{thm}{Theorem}[section]
\newtheorem{cor}[thm]{Corollary}
\newtheorem{prop}[thm]{Proposition}
\newtheorem{lem}[thm]{Lemma}
\newtheorem{conj}[thm]{Conjecture}
\newtheorem{quest}[thm]{Question}

\theoremstyle{definition}
\newtheorem{defn}[thm]{Definition}
\newtheorem{defns}[thm]{Definitions}
\newtheorem{con}[thm]{Construction}
\newtheorem{exmp}[thm]{Example}
\newtheorem{exmps}[thm]{Examples}
\newtheorem{notn}[thm]{Notation}
\newtheorem{notns}[thm]{Notations}
\newtheorem{addm}[thm]{Addendum}
\newtheorem*{exer}{Exercise}

\theoremstyle{remark}
\newtheorem{rem}[thm]{Remark}
\newtheorem{rems}[thm]{Remarks}
\newtheorem{warn}[thm]{Warning}
\newtheorem{sch}[thm]{Scholium}

\DeclareMathOperator{\Hom}{Hom}
\DeclareMathOperator{\Id}{Id}
\DeclareMathOperator{\End}{End}
\DeclareMathOperator{\ord}{ord}
\DeclareMathOperator{\Aut}{Aut}
\DeclareMathOperator{\Gal}{Gal}
\DeclareMathOperator{\Ext}{Ext}
\DeclareMathOperator{\RP}{\mathbb{R}P}

\makeatletter
\let\c@equation\c@thm
\makeatother
\numberwithin{equation}{section}

\bibliographystyle{plain}

\begin{document}

\title{Math 612 (Homework 3)}
\author{Hidenori Shinohara}
\maketitle

\begin{exer}{(3.1.11)}
  Using the cellular homology, we obtain
  \begin{align*}
    \tilde{H}_i(X) &= \begin{cases}
      \mathbb{Z} / m\mathbb{Z} & (i = n) \\
      0 & (i \ne n).
    \end{cases} \\
    \tilde{H}^i(X) &= \begin{cases}
      \mathbb{Z} / m\mathbb{Z} & (i = n + 1) \\
      0 & (i \ne n + 1).
    \end{cases}
  \end{align*}

  From previous homework,
  \begin{align*}
    \tilde{H}^i(X / S^n) = \tilde{H}_i(S^{n + 1}) &= \begin{cases}
      \mathbb{Z} & (i = n + 1) \\
      0 & (i \ne n + 1).
    \end{cases}
  \end{align*}

  Thus the map on $\tilde{H}_i(-; \mathbb{Z})$ is the zero map for each $i$.
  On the other hand, the long exact sequence of a pair gives us $\tilde{H}^{n + 1}(X, S^n; \mathbb{Z}) \xrightarrow{q^{\ast}} \tilde{H}^{n + 1}(X; \mathbb{Z}) \rightarrow \tilde{H}^{n + 1}(S^n; \mathbb{Z})$ where $\tilde{H}^{n + 1}(S^n; \mathbb{Z}) = 0$, so $q^{\ast}$ is surjective.
  Therefore, it is nontrivial because $\tilde{H}^{n + 1}(X; \mathbb{Z}) \ne 0$.

  \begin{center}
    \begin{tikzcd}[cells={nodes={minimum height=2em}}]
      0 \arrow[r] & \Ext(H_n(X); \mathbb{Z})     \arrow[r] \arrow[d] &  H^{n + 1}(X; \mathbb{Z})       \arrow[r] \arrow[d] &  \Hom(H_{n + 1}(X); \mathbb{Z}) \arrow[r] \arrow[d] & 0 \\
      0 \arrow[r] & \Ext(H_n(X/S^n); \mathbb{Z}) \arrow[r]           &  H^{n + 1}(X / S^n; \mathbb{Z}) \arrow[r]           &  \Hom(H_{n + 1}(X / S^n); \mathbb{Z}) \arrow[r]    & 0
    \end{tikzcd}
  \end{center}

  is

  \begin{center}
    \begin{tikzcd}[cells={nodes={minimum height=2em}}]
      0 \arrow[r] & \mathbb{Z}_m \arrow[r] \arrow[d] &  \mathbb{Z}_m  \arrow[r] \arrow[d] &  0 \arrow[r] \arrow[d] & 0 \\
      0 \arrow[r] & 0            \arrow[r]           &  \mathbb{Z}    \arrow[r]           &  \mathbb{Z} \arrow[r]  & 0.
    \end{tikzcd}
  \end{center}

  This splitting is not natural because the middle term in the first sequence is isomorphic to $\mathbb{Z}_m \oplus 0$ and the second one is $0 \oplus \mathbb{Z}$.


  The long exact sequence of a pair gives us $\tilde{H}_n(S^n; \mathbb{Z}) \rightarrow \tilde{H}_n(X; \mathbb{Z}) \rightarrow \tilde{H}_n(X, S^n; \mathbb{Z}) = \tilde{H}_n(S^{n + 1}; \mathbb{Z}) = 0$ which implies the surjectivity of the induced map.
  Since $\tilde{H}_n(X; \mathbb{Z}) \ne 0$, the induced map is nonzero.

  The map induced on $\tilde{H}^i(-;\mathbb{Z})$ is the zero map for any $i$ because at least one of $\tilde{H}^i(S^n; \mathbb{Z})$ or $\tilde{H}^i(X; \mathbb{Z})$ is 0 for each $i$.
\end{exer}

\begin{exer}{(3.1.13)}
\end{exer}

\begin{exer}{(3.2.1)}
\end{exer}

\begin{exer}{(3.2.2)}
  Suppose $X$ is the union of contractible open sets $A_1, \cdots, A_n$.
  Since each $A_i$ is contractible, $H^k(X, A_i; R) = H^k(X; R)$ for all $k \geq 1$.

  \begin{center}
    \begin{tikzcd}[cells={nodes={minimum height=2em}}]
      H^{k_1}(X, A_1; R) \times \cdots \times H^{k_n}(X, A_n; R) \arrow[r]      \arrow[d, "\cong"] &  H^{k_1 + \cdots + k_n}(X, A_1 \cup \cdots \cup A_n; R) \arrow[d] \\
      H^{k_1}(X; R) \times \cdots \times H^{k_n}(X; R)           \arrow[r, "f"]                    &  H^{k_1 + \cdots + k_n}(X; R).
    \end{tikzcd}
  \end{center}

  This diagram commutes by the naturality of a cup product.
  $H^{k_1 + \cdots + k_n}(X, \bigcup_i A_i; R) = H^{k_1 + \cdots + k_n}(X, X; R) = 0$ for all $k + l \ge 1$.
  By the commutativity of this diagram, the function $f$ must be 0.
\end{exer}

\begin{exer}{(3.2.3(a))}
  Suppose otherwise.
  Let $f: \RP^n \rightarrow \RP^m$ be such a function.
  Then $f$ induces a map on $f^{\ast}: H^{\ast}(\RP^m) \rightarrow H^{\ast}(\RP^n)$.
  In other words, $f^{\ast}: \mathbb{Z}_m[\alpha]/(\alpha^{m + 1}) \rightarrow \mathbb{Z}_n[\beta]/(\beta^{n + 1})$ where $\alpha, \beta$ are generators of $H^1(\RP^m)$ and $H^1(\RP^n)$.
  $H^1(\RP^m; \mathbb{Z}_2) = \mathbb{Z}_2 = \{ 0, \alpha \}$ and $H^1(\RP^n; \mathbb{Z}_2) = \mathbb{Z}_2 = \{ 0, \beta \}$.
  Since $f$ induces a nontrivial map $H^1(\RP^m; \mathbb{Z}_2) \rightarrow H^1(\RP^n; \mathbb{Z}_2)$, $f^{\ast}(\alpha) = \beta$.
  However, $f^{\ast}(0) = f^{\ast}(\alpha^{m + 1}) = (f^{\ast}(\alpha))^{m + 1} = \beta^{m + 1} \ne 0$ because $m < n$.
  This is a contradiction, so such a function does not exist.

  $H^1(\mathbb{C}P^n; \mathbb{Z}_2) = 0$ for any $n$, so there exists no such nontrivial map.
  The case for $H^2(\mathbb{C}P^n)$ can be argued the same way as above because $H^2(\mathbb{C}P^n;\mathbb{Z}_2) = \mathbb{Z}_2[\alpha]/(\alpha^{n + 1})$ where $\alpha$ is a generator of $H^2(\mathbb{C}P^n)$.
\end{exer}

\begin{exer}{(3.2.3(b))}
\end{exer}

\begin{exer}{(3.2.6)}
\end{exer}

\begin{exer}{(3.2.7)}
  Let $f:\RP^3 \rightarrow \RP^2 \vee S^3$ be a homotopy equivalence.
  Then it induces isomorphisms.
  \begin{center}
    \begin{tikzcd}[cells={nodes={minimum height=2em}}]
      H^1(\RP^3; \mathbb{Z}_2) \arrow[d, "f^{\ast}"] & \times & H^2(\RP^3; \mathbb{Z}_2) \arrow[r] \arrow[d, "f^{\ast}"] & H^3(\RP^3; \mathbb{Z}_2) \arrow[d, "f^{\ast}"] \\
      H^1(\RP^2 \vee S^3; \mathbb{Z}_2)              & \times & H^2(\RP^2 \vee S^3; \mathbb{Z}_2) \arrow[r]              & H^3(\RP^2 \vee S^3; \mathbb{Z}_2).
    \end{tikzcd}
  \end{center}

  The cohomology groups of a wedge sum is the direct sum of cohomology groups of the two spaces.
  By rewriting the diagram above with generators, we obtain
  \begin{center}
    \begin{tikzcd}[cells={nodes={minimum height=2em}}]
      \{ 0, \alpha \} \arrow[d, "f^{\ast}"] & \times & \{ 0, \alpha^2 \} \arrow[r] \arrow[d, "f^{\ast}"] & \{ 0, \alpha^3 \} \arrow[d, "f^{\ast}"]\\
      \{ 0, \beta \} \oplus \{ 0, \gamma \} & \times & \{ 0, \beta^2 \} \oplus 0 \arrow[r]               & 0 \oplus \{ 0, \gamma^2 \}.
    \end{tikzcd}
  \end{center}
  This implies $f^{\ast}$ sends $\alpha^2$ to $(\beta^2, 0)$ and $\alpha^3$ to $(0, \gamma^2)$.
  However, this implies $(0, 0) = (f^{\ast}(\alpha^2))^3 = (f^{\ast}(\alpha^3))^2 = (0, \gamma^4) = (0, \gamma)$.
  This is a contradiction because $0 \ne \gamma$.
\end{exer}

\end{document}


