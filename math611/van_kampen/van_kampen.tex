\documentclass[12pt, psamsfonts]{amsart}

%-------Packages---------
\usepackage{amssymb,amsfonts}
\usepackage[all,arc]{xy}
\usepackage{enumerate}
\usepackage{mathrsfs}
\usepackage{tikz-cd}
\usepackage{theoremref}
\usepackage{graphicx}
\usepackage[bookmarks]{hyperref}

%--------Theorem Environments--------
%theoremstyle{plain} --- default
\newtheorem{thm}{Theorem}[section]
\newtheorem{cor}[thm]{Corollary}
\newtheorem{prop}[thm]{Proposition}
\newtheorem{lem}[thm]{Lemma}
\newtheorem{conj}[thm]{Conjecture}
\newtheorem{quest}[thm]{Question}

\theoremstyle{definition}
\newtheorem{defn}[thm]{Definition}
\newtheorem{defns}[thm]{Definitions}
\newtheorem{con}[thm]{Construction}
\newtheorem{exmp}[thm]{Example}
\newtheorem{exmps}[thm]{Examples}
\newtheorem{notn}[thm]{Notation}
\newtheorem{notns}[thm]{Notations}
\newtheorem{addm}[thm]{Addendum}
\newtheorem*{exer}{Exercise}

\theoremstyle{remark}
\newtheorem{rem}[thm]{Remark}
\newtheorem{rems}[thm]{Remarks}
\newtheorem{warn}[thm]{Warning}
\newtheorem{sch}[thm]{Scholium}

\DeclareMathOperator{\Hom}{Hom}
\DeclareMathOperator{\Id}{Id}

\makeatletter
\let\c@equation\c@thm
\makeatother
\numberwithin{equation}{section}

\bibliographystyle{plain}

\begin{document}

\title{MyTitle}
\author{Hidenori Shinohara}
\maketitle

\begin{thm}[Universal property of free products]
  Let $G = G_1 * G_2$ and $i_1: G_{1} \rightarrow G, i_2: G_{2} \rightarrow G$ be given.
  For any group $H$ and homomorphisms $\phi_1: G_1 \rightarrow H, \phi_2: G_2 \rightarrow H$, there exists a unique homomorphism $\phi: G \rightarrow H$ such that $\phi \circ i_1 = \phi_1$ and $\phi \circ i_2 = \phi_2$.

\[\begin{tikzcd}[column sep=tiny]
& G_1 \ar[dr, "i_1"] \ar[drr, "\phi_1", bend left=20] & &[1.5em] \\
& & G \ar[r,  dashed, "\phi"] & H \\
& G_2 \ar[ur, "i_2"]\ar[urr, "\phi_2"', bend right=20] & &
\end{tikzcd}\]

\end{thm}

\begin{thm}
Let $U, V \subset X$ be open.
Suppose:

\begin{itemize}
  \item
    $X = U \cup V$.
  \item
    $U, V, U \cap V$ are all path connected.
  \item
    $x_0 \in U \cap V$.
\end{itemize}

By the universal property, there exists a homomorphism $\Phi: \pi_1(U, x_0) \times \pi_1(V, x_0) \rightarrow \pi_1(X, x_0)$.

\begin{tikzcd}[column sep=tiny]
  & \pi_1(U, x_0) \ar[dr, "j_U"] \ar[drr, "j_1", bend left=20] & &[1.5em] \\
\pi_1(U \cap V, x_0) \ar[ur, "i_{UV}"] \ar[dr, "i_{VU}"'] & & \pi_1(U, x_0) * \pi_1(V, x_0) \ar[r, dashed, "\Phi"] & \pi_1(X, x_0) \\
  & \pi_1(V, x_0) \ar[ur, "j_V"]\ar[urr, "j_2"', bend right=20] & &
\end{tikzcd}

Then $\Phi$ is surjective and $\ker \Phi$ is the normal subgroup generated by $\{ i_{UV}(g)i_{VU}(g)^{-1} \mid g \in \pi_1(U \cap V, x_0) \}$.

Therefore, we can calculate $(\pi_1(U, x_0) * \pi_1(V, x_0)) / \ker \Phi$ to find a group isomorphic to $\pi_1(X, x_0)$.
\end{thm}

\end{document}


