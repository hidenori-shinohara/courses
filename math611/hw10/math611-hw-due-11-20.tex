\documentclass[psamsfonts]{amsart}

%-------Packages---------
\usepackage{amssymb,amsfonts}
\usepackage{fullpage}
\usepackage{tikz-cd}
\usepackage{todonotes}
\usepackage{physics}
\usepackage[all,arc]{xy}
\usepackage{enumerate}
\usepackage{enumitem}
\usepackage{mathrsfs}
\usepackage{theoremref}
\usepackage{graphicx}
\usepackage[bookmarks]{hyperref}

%--------Theorem Environments--------
%theoremstyle{plain} --- default
\newtheorem{thm}{Theorem}[section]
\newtheorem{cor}[thm]{Corollary}
\newtheorem{prop}[thm]{Proposition}
\newtheorem{lem}[thm]{Lemma}
\newtheorem{conj}[thm]{Conjecture}
\newtheorem{quest}[thm]{Question}

\theoremstyle{definition}
\newtheorem{defn}[thm]{Definition}
\newtheorem{defns}[thm]{Definitions}
\newtheorem{con}[thm]{Construction}
\newtheorem{exmp}[thm]{Example}
\newtheorem{exmps}[thm]{Examples}
\newtheorem{notn}[thm]{Notation}
\newtheorem{notns}[thm]{Notations}
\newtheorem{addm}[thm]{Addendum}
\newtheorem*{exer}{Exercise}

\theoremstyle{remark}
\newtheorem{rem}[thm]{Remark}
\newtheorem{rems}[thm]{Remarks}
\newtheorem{warn}[thm]{Warning}
\newtheorem{sch}[thm]{Scholium}

\DeclareMathOperator{\Hom}{Hom}
\DeclareMathOperator{\Id}{Id}
\DeclareMathOperator{\End}{End}
\DeclareMathOperator{\ord}{ord}
\DeclareMathOperator{\Aut}{Aut}

\makeatletter
\let\c@equation\c@thm
\makeatother
\numberwithin{equation}{section}

\bibliographystyle{plain}

\begin{document}

\title{Math 611 (Due 11/20)}
\author{Hidenori Shinohara}
\maketitle


\begin{exer}{(Problem 1)}
  \todo[inline,caption={}]{
  }
\end{exer}

\begin{exer}{(Problem 28 (a))}
  Let $A, B$ be the Mobius strip and a torus with a small neighborhood around them so the strip and torus are contained in $A$ and $B$.
  For any $n \geq 3$, the exact sequence $H_n(A \cap B) \rightarrow H_n(A) \oplus H_n(B) \rightarrow H_n(X) \rightarrow H_{n - 1}(A \cap B)$ implies that $H_n(X) \cong H_n(A) \oplus H_n(B) = 0 \oplus 0 = 0$ because the intersection $A \cap B$ is homotopic to $S^1$, so $H_n(A \cap B) = H_{n - 1}(A \cap B) = 0$.
  $H_0(X) = \mathbb{Z]$ because $X$ has only one path component.

  We will examine the LES

  \begin{align*}
    \tilde{H}_2(A \cap B) \rightarrow \tilde{H}_2(A) \oplus \tilde{H}_2(B) \xrightarrow{f_1} \tilde{H}_2(X) 
    \xrightarrow{f_2} \tilde{H}_1(A \cap B) \xrightarrow{f_3} \tilde{H}_1(A) \oplus \tilde{H}_1(B) \xrightarrow{f_4} \tilde{H}_1(X) 
    \rightarrow \tilde{H}_0(A \cap B).
  \end{align*}

  \begin{itemize}
    \item
      Sine $\tilde{H}_2(A \cap B) = 0$, so $f_1$ is injective.
    \item
      $\tilde{H}_1(A \cap B) = \mathbb{Z}$, and $f_3(1) = (2, (1, 0))$ because the intersection goes around the mobius strip twice while it only goes around the torus once.
      Then $f_3$ is injective, so $\Im(f_2) = \ker(f_3) = 0$.
      This implies that $\im(f_1) = \ker(f_2) = H_2$, so $f_1$ is surjective.
  \end{itemize}

  Therefore, $f_1$ is bijective, so $H_2(X) = \tilde{H}_2(X) = \tilde{H}_2(A) \oplus \tilde{H}_2(B) = 0 \oplus \mathbb{Z} = \mathbb{Z}$.

  Finally, $f_4$'s surjectivity implies that 
  \begin{align*}
    \tilde{H}_1(X)
      &\cong \tilde{H}_1(A) \oplus \tilde{H}_1(B) / \ker(f_4) \\
      &= \mathbb{Z} \oplus \mathbb{Z}^2 / \ev{(2, (1, 0))} \\
      &\cong \ev{a, b, c} / \ev{2a + b} \\
      &\cong \ev{a, b, c \mid 2a + b} \\
      &\cong \ev{a, -2a, c} \\
      &\cong \ev{a, c} = \mathbb{Z} \oplus \mathbb{Z}.
  \end{align*}
  Thus $H_1(X) = \mathbb{Z} \oplus \mathbb{Z}$.
\end{exer}

\begin{exer}{(Problem 28 (b))}
  Let $A, B$ be the Mobius strip and $\mathbb{R}P^2$ with a small neighborhood around them so the strip and $\mathbb{R}P^2$ are contained in $A$ and $B$.
  For any $n \geq 3$, the exact sequence $H_n(A \cap B) \rightarrow H_n(A) \oplus H_n(B) \rightarrow H_n(X) \rightarrow H_{n - 1}(A \cap B)$ implies that $H_n(X) \cong H_n(A) \oplus H_n(B) = 0 \oplus 0 = 0$ because the intersection $A \cap B$ is homotopic to $S^1$, so $H_n(A \cap B) = H_{n - 1}(A \cap B) = 0$.
  Since $X = A \cup B$ has one path component, $H_0(X) = \mathbb{Z}$.
  We will consider the LES

  \begin{align*}
    \tilde{H}_2(A) \oplus \tilde{H}_2(B) \xrightarrow{f_1} \tilde{H}_2(X) 
    \xrightarrow{f_2} \tilde{H}_1(A \cap B) \xrightarrow{f_3} \tilde{H}_1(A) \oplus \tilde{H}_1(B) \xrightarrow{f_4} \tilde{H}_1(X) 
    \rightarrow \tilde{H}_0(A \cap B).
  \end{align*}

  $\tilde{H}_1(A \cap B) = \mathbb{Z}$, and $f_3$ maps $1$ to $(2, 1)$ because the generator wraps around the Mobius strip twice and the $\mathbb{R}P^2$ once.
  Then $f_3$ is injective, so $f_2$ is the zero map.
  In other words, $\ker(f_2) = \tilde{H}_2(X)$, so $f_1$ is surjective.
  Since $\tilde{H}_2(A) \oplus \tilde{H}_2(B) = 0$, $\tilde{H}_2(X) = 0$.
  Thus $H_2(X) = 0$.

  By the first isomorphism theorem and exactness,

  \begin{align*}
    \tilde{H}_1(X)
      &= \tilde{H}_1(A) \oplus \tilde{H}_1(B) / \ker(f_4) \\
      &= (\mathbb{Z} \oplus \mathbb{Z}/2\mathbb{Z}) / \ev{(2, 1)} \\
      &\cong \ev{a, b \mid 2b}  / \ev{2a + b} \\
      &= \ev{a, b \mid 2b, 2a + b} \\
      &= \ev{a, -2a \mid 2(-2a)} \\
      &= \ev{a \mid 4a} \\
      &= \mathbb{Z}_4.
  \end{align*}
  Therefore, $H_1(X) = \mathbb{Z}_4$.
\end{exer}

\begin{exer}{(Problem 29)}
  \todo[inline,caption={}]{
  }
\end{exer}

\end{document}


