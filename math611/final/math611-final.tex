\documentclass[12pt, psamsfonts]{amsart}

%-------Packages---------
\usepackage{amssymb,amsfonts}
\usepackage{semantic}
\usepackage{fullpage}
\usepackage{tikz-cd}
\usepackage{todonotes}
\usepackage{physics}
\usepackage[all,arc]{xy}
\usepackage{enumerate}
\usepackage{enumitem}
\usepackage{mathrsfs}
\usepackage{theoremref}
\usepackage{graphicx}
\usepackage[bookmarks]{hyperref}

%--------Theorem Environments--------
%theoremstyle{plain} --- default
\newtheorem{thm}{Theorem}[section]
\newtheorem{cor}[thm]{Corollary}
\newtheorem{prop}[thm]{Proposition}
\newtheorem{lem}[thm]{Lemma}
\newtheorem{conj}[thm]{Conjecture}
\newtheorem{quest}[thm]{Question}

\theoremstyle{definition}
\newtheorem{defn}[thm]{Definition}
\newtheorem{defns}[thm]{Definitions}
\newtheorem{con}[thm]{Construction}
\newtheorem{exmp}[thm]{Example}
\newtheorem{exmps}[thm]{Examples}
\newtheorem{notn}[thm]{Notation}
\newtheorem{notns}[thm]{Notations}
\newtheorem{addm}[thm]{Addendum}
\newtheorem*{exer}{Exercise}

\theoremstyle{remark}
\newtheorem{rem}[thm]{Remark}
\newtheorem{rems}[thm]{Remarks}
\newtheorem{warn}[thm]{Warning}
\newtheorem{sch}[thm]{Scholium}

\DeclareMathOperator{\Hom}{Hom}
\DeclareMathOperator{\Id}{Id}
\DeclareMathOperator{\End}{End}
\DeclareMathOperator{\ord}{ord}
\DeclareMathOperator{\Aut}{Aut}
\DeclareMathOperator{\Gal}{Gal}

\makeatletter
\let\c@equation\c@thm
\makeatother
\numberwithin{equation}{section}

\bibliographystyle{plain}

\begin{document}

\title{Math 611 Final}
\author{Hidenori Shinohara}
\maketitle

\begin{exer}{(Problem 2)}
 \begin{figure}
   \includegraphics[width=.5\linewidth]{k33.jpeg}
   \caption{$K_{3, 3}$}
   \label{fig:k33}
 \end{figure}
 Figure \ref{fig:k33} shows how $K_{3, 3}$ is homotopy equivalent to $S_1 \vee S_1 \vee S_1 \vee S_1$.
 Thus the Van Kampen theorem implies that the fundamental group is the free group generated by 4 elements $\ev{ a, b, c, d }$ where each generator corresponds to each $S_1$.
\end{exer}

\begin{exer}{(Problem 5)}
  Let $X = S^1 \times S^2$ and $Y = S^1 \vee S^2 \vee S^3$.
  \begin{align*}
    \pi_1(S^1 \times S^2)
      &= \pi_1(S^1) \times \pi_1(S^2) & \text{(Proposition 1.12)} \\
      &= \mathbb{Z} \times 0 \\
      &= \mathbb{Z}. \\
    \pi_1(S^1 \vee S^2 \vee S^3)
      &= \pi_1(S^1) * \pi_1(S^2) * \pi_1(S^3) & \text{(Van Kampen)} \\
      &= \mathbb{Z} * 0 * 0 \\
      &= \mathbb{Z}.
  \end{align*}

  $X$ and $Y$ are both path connected, so $H_0(X) = H_0(Y) = \mathbb{Z}$.

  We will consider two subspaces of $X$ the union of whose interiors equals $X$.
  Identify each point of $X = S^1 \times S^2$ by a pair of coordinates $(\theta, (x, y, z))$ where $\theta$ is the angle in $S^1$ and $(x, y, z)$ satisfies $x^2 + y^2 + z^2 = 1$.
  Let $A = \{ (\theta, (x, y, z)) \mid -\epsilon \leq \theta \leq \pi + \epsilon \}, B = \{ (\theta, (x, y, z)) \mid \pi - \epsilon \leq \theta \leq 2\pi + \epsilon \}$ where $\epsilon > 0$ is a small number.
  Then each $A$ and $B$ deformation retracts to a space homeomorphic to $S^2$.
  $A \cap B$ consists of two path components, each of which deformation retracts to a space homeomorphic to $S^2$.
  Moreover, it is clear that $\int(A) \cup \int(B) = X$.
  We will consider the Mayer-Vietoris sequence formed by $A, B \subset X$.
  \todo[inline,caption={}]{
    Do the Mayer Vietoris stuff.
  }



  By Corollary 2.25, $\tilde{H}_n(S^1 \vee S^2 \vee S^3) = \tilde{H}_n(S^1) \otimes \tilde{H}_n(S^2) \otimes \tilde{H}_n(S^3)$.

  Therefore,
  \begin{align*}
    \tilde{H}_n(Y) = \begin{cases}
      \mathbb{Z} & (n = 1, 2, 3) \\
      0 & (n = 0, n \geq 4).
    \end{cases}
  \end{align*}
  For $n \geq 1$, $\tilde{H}_n(Y) = H_n(Y)$, so $H_0(Y) = H_1(Y) = H_2(Y) = H_3(Y) = \mathbb{Z}$ and $H_n(Y) = 0$ for all $n \geq 4$.

\end{exer}

\end{document}


