\documentclass[12pt, psamsfonts]{amsart}

%-------Packages---------
\usepackage{amssymb,amsfonts}
\usepackage{fullpage}
\usepackage{tikz-cd}
\usepackage{todonotes}
\usepackage{physics}
\usepackage[all,arc]{xy}
\usepackage{enumerate}
\usepackage{enumitem}
\usepackage{mathrsfs}
\usepackage{theoremref}
\usepackage{graphicx}
\usepackage[bookmarks]{hyperref}

%--------Theorem Environments--------
%theoremstyle{plain} --- default
\newtheorem{thm}{Theorem}[section]
\newtheorem{cor}[thm]{Corollary}
\newtheorem{prop}[thm]{Proposition}
\newtheorem{lem}[thm]{Lemma}
\newtheorem{conj}[thm]{Conjecture}
\newtheorem{quest}[thm]{Question}

\theoremstyle{definition}
\newtheorem{defn}[thm]{Definition}
\newtheorem{defns}[thm]{Definitions}
\newtheorem{con}[thm]{Construction}
\newtheorem{exmp}[thm]{Example}
\newtheorem{exmps}[thm]{Examples}
\newtheorem{notn}[thm]{Notation}
\newtheorem{notns}[thm]{Notations}
\newtheorem{addm}[thm]{Addendum}
\newtheorem*{exer}{Exercise}

\theoremstyle{remark}
\newtheorem{rem}[thm]{Remark}
\newtheorem{rems}[thm]{Remarks}
\newtheorem{warn}[thm]{Warning}
\newtheorem{sch}[thm]{Scholium}

\DeclareMathOperator{\Hom}{Hom}
\DeclareMathOperator{\Id}{Id}
\DeclareMathOperator{\End}{End}
\DeclareMathOperator{\ord}{ord}
\DeclareMathOperator{\Aut}{Aut}

\makeatletter
\let\c@equation\c@thm
\makeatother
\numberwithin{equation}{section}

\bibliographystyle{plain}

\begin{document}

\title{Math 611 (Due 11/6)}
\author{Hidenori Shinohara}
\maketitle

\section{Simplicial and Singular Homology}

\begin{exer}{(Problem 14)}
  Determine whether there exists a short exact sequence $0 \rightarrow \mathbb{Z}_4 \rightarrow \mathbb{Z}_8 \oplus \mathbb{Z}_2 \rightarrow \mathbb{Z}_4 \rightarrow 0$.
  More generally, determine which abelian groups $A$ fit into a short exact sequence $0 \rightarrow \mathbb{Z}_{p^m} \rightarrow A \rightarrow \mathbb{Z}_{p^n} \rightarrow 0$ with $p$ prime.
  What about the case of short exact sequences $0 \rightarrow A \rightarrow \mathbb{Z}_n \rightarrow 0$?
\end{exer}

\begin{proof}
  Let $\phi_1: \mathbb{Z}_4 \rightarrow \mathbb{Z}_8 \oplus \mathbb{Z}_2, \phi_2: \mathbb{Z}_8 \oplus \mathbb{Z}_2 \rightarrow \mathbb{Z}_4$ be defined such that $\phi_1(a) = (2a, a)$ and $\phi_2(a, b) = 2b - a$.
  \todo[inline,caption={}]{
    Solve this!
  }
\end{proof}

\end{document}


