\documentclass[12pt, psamsfonts]{amsart}

%-------Packages---------
\usepackage{amssymb,amsfonts}
\usepackage{fullpage}
\usepackage{tikz-cd}
\usepackage{todonotes}
\usepackage{physics}
\usepackage[all,arc]{xy}
\usepackage{enumerate}
\usepackage{enumitem}
\usepackage{mathrsfs}
\usepackage{theoremref}
\usepackage{graphicx}
\usepackage[bookmarks]{hyperref}

%--------Theorem Environments--------
%theoremstyle{plain} --- default
\newtheorem{thm}{Theorem}[section]
\newtheorem{cor}[thm]{Corollary}
\newtheorem{prop}[thm]{Proposition}
\newtheorem{lem}[thm]{Lemma}
\newtheorem{conj}[thm]{Conjecture}
\newtheorem{quest}[thm]{Question}

\theoremstyle{definition}
\newtheorem{defn}[thm]{Definition}
\newtheorem{defns}[thm]{Definitions}
\newtheorem{con}[thm]{Construction}
\newtheorem{exmp}[thm]{Example}
\newtheorem{exmps}[thm]{Examples}
\newtheorem{notn}[thm]{Notation}
\newtheorem{notns}[thm]{Notations}
\newtheorem{addm}[thm]{Addendum}
\newtheorem*{exer}{Exercise}

\theoremstyle{remark}
\newtheorem{rem}[thm]{Remark}
\newtheorem{rems}[thm]{Remarks}
\newtheorem{warn}[thm]{Warning}
\newtheorem{sch}[thm]{Scholium}

\DeclareMathOperator{\Hom}{Hom}
\DeclareMathOperator{\Id}{Id}
\DeclareMathOperator{\End}{End}
\DeclareMathOperator{\ord}{ord}
\DeclareMathOperator{\Aut}{Aut}

\makeatletter
\let\c@equation\c@thm
\makeatother
\numberwithin{equation}{section}

\bibliographystyle{plain}

\begin{document}

\title{Math 611 (Due 11/6)}
\author{Hidenori Shinohara}
\maketitle

\section{Simplicial and Singular Homology}

\begin{exer}{(Problem 14)}
  Determine whether there exists a short exact sequence $0 \rightarrow \mathbb{Z}_4 \rightarrow \mathbb{Z}_8 \oplus \mathbb{Z}_2 \rightarrow \mathbb{Z}_4 \rightarrow 0$.
  More generally, determine which abelian groups $A$ fit into a short exact sequence $0 \rightarrow \mathbb{Z}_{p^m} \rightarrow A \rightarrow \mathbb{Z}_{p^n} \rightarrow 0$ with $p$ prime.
  What about the case of short exact sequences $0 \rightarrow A \rightarrow \mathbb{Z}_n \rightarrow 0$?
\end{exer}

\begin{proof}
  Let $\phi_1: \mathbb{Z}_4 \rightarrow \mathbb{Z}_8 \oplus \mathbb{Z}_2, \phi_2: \mathbb{Z}_8 \oplus \mathbb{Z}_2 \rightarrow \mathbb{Z}_4$ be defined such that $\phi_1(a) = (2a, a)$ and $\phi_2(a, b) = 2b - a$.
  Then $\ker(\phi_1) = 0, \Im(\phi_1) = \ker(\phi_2) = \{ (2k, k) \mid 0 \leq k \leq 3 \}$ and $\Im(\phi_2) = \mathbb{Z}_4$.
  Thus this is indeed an exact sequence.

  \todo[inline,caption={}]{
    Finish this!
  }
\end{proof}

\begin{exer}{(Problem 15)}
  For an exact sequence $A \rightarrow B \rightarrow C \rightarrow D \rightarrow E$ show that $C = 0$ if and only if the map $A \rightarrow B$ is surjective and $D \rightarrow E$ is injective.
  Hence, for a pair of spaces $(X, A)$, the inclusion $A \rightarrow X$ induces isomorphisms on all homology groups if and only if $H_n(X, A) = 0$ for all $n$.
\end{exer}

\begin{proof}
  Suppose $C = 0$.
  $\Im(\phi_{AB}) = \ker(\phi_{BC}) = B$, so $\phi_{AB}$ is surjective.
  $\ker(\phi_{DE}) = \Im(\phi_{CD}) = \{ 0 \}$, so $\phi_{DE}$ is injective.

  On the other hand, suppose $\phi_{AB}$ is surjective and $\phi_{DE}$ is injective.
  $\Im(\phi_{CD}) = \ker(\phi_{DE}) = \{ 0 \}$, so $\phi_{CD}$ is the zero map.
  Therefore, $\ker(\phi_{CD}) = C$.
  $\ker(\phi_{BC}) = \Im(\phi_{AB}) = B$, so $\phi_{BC}$ is the zero map.
  Therefore, $\Im(\phi_{BC}) = 0$.
  Hence, $C = \ker(\phi_{CD}) = \Im(\phi_{BC}) = 0$.

  For each $n$, we have an exact sequence $0 \rightarrow C_n(A) \xrightarrow{i} C_n(X) \xrightarrow{j} C_n(X)/C_n(A) \rightarrow 0$ where $i$ is induced by the inclusion map $A \rightarrow X$ and $j$ is the canonical quotient map.
  Moreover, the diagram formed by the exact sequence for each $n$ joined by $\partial$ is commutative by the definition of $\partial$.
  \todo[inline,caption={}]{
    Apply Theorem 2.16!
  }
\end{proof}

\begin{exer}{(Problem 16)}
  \begin{itemize}
    \item
      Show that $H_0(X, A) = 0$ if and only if $A$ meets each path-component of $X$.
    \item
      \todo[inline,caption={}]{
        Do Part (b).
      }
  \end{itemize}
\end{exer}

\begin{proof}
  $ $
  \begin{itemize}
    \item
      Let $\gamma_x + C_0(A) \in C_0(X) / C_0(A)$.
      Since $A$ meets each path-component of $X$, there exists a path $\gamma: I \rightarrow X$ that joins a point $a \in A$ and the image of $\gamma_x$.
      Then $\gamma$ can be seen as an element of $C_1(X)$ since $\gamma$ maps a 1-simplex into $X$.
      Moreover, $\partial\gamma = \gamma_x - \gamma_a$ where $\gamma_a \in C_0(A)$ with $\Im(\gamma_a) = a$.
      Therefore, $\partial(\gamma + C_1(A)) = \gamma_x + C_0(A)$, so $\gamma_x + C_0(A) \in \Im(\partial)$.
      Hence, $H_0(X, A) = \ker(\partial_0)/\Im(\partial_1) = (C_0(X)/C_0(A)) / (C_0(X)/C_1(A)) = 0$.

      On other hand, suppose that $A$ does not meet each path component of $X$.
      Let $x \in X$ be a point in a path component that $A$ does not intersect.
      Let $\gamma_x: \Delta^0 \rightarrow X$ such that $\Im(\gamma_x) = \{ x \}$.
      Then $\gamma_x \in \ker(\partial_0) = C_0(X, A)$.
      Let $\gamma + C_1(A) \in C_1(X, A)$.
      Then $\partial_1(\gamma + C_1(A)) = \partial_1(\gamma) + C_0(A)$.
      Let $\gamma_{x_1}, \gamma_{x_2} \in C_0(X)$ such that $\partial_1(\gamma) = \gamma_{x_1} - \gamma_{x_2}$.
      $\gamma_{x_1} - \gamma_{x_2} + C_0(A) \ne \gamma_x + C_0(A)$ if and only if $\gamma_{x_1} - \gamma_{x_2} - \gamma_x \in C_0(A)$.

      \begin{itemize}
        \item
          If $\gamma$ lies in the same path component as $x$, then so do $x_1$ and $x_2$.
          Suppose $x = x_1$.
          Since $-\gamma_{x_2} \notin C_0(A)$, $\gamma_{x_1} - \gamma_{x_2} + C_0(A) \ne \gamma_x + C_0(A)$.
          The case when $x \ne x_1$ and $x = x_2$ and the case when $x \ne x_1$ and $x \ne x_2$ can be proven in a similar way.
        \item
          If $\gamma$ lies in a different path component, then $\gamma_x \ne \gamma_{x_1}$ and $\gamma_x \ne \gamma_{x_2}$.
          Therefore, $\gamma_{x_1} - \gamma_{x_2} + C_0(A) \ne \gamma_x + C_0(A)$.
      \end{itemize}
      Therefore, $\gamma_x \notin \Im(\partial_1)$.
      Thus $H_0(X, A) = C_0(X, A) / \Im(\partial_1)$ is not 0.
    \item
      \todo[inline,caption={}]{
        Do part (b).
      }
  \end{itemize}
\end{proof}

\begin{exer}{(Problem 17)}
  $ $
  \begin{itemize}
    \item
      Compute the homology groups $H_n(X, A)$ when $X$ is $S^2$ or $S^1 \times S^1$ and $A$ is a finite set of points in $X$.
    \item
      Compute the groups $H_n(X, A)$ and $H_n(X, B)$ for $X$ a closed orientable surface of genus two with $A$ and $B$ the circles shown.
  \end{itemize}
\end{exer}

\begin{proof}
  $ $
  \begin{itemize}
    \item
      Since a finite set of points in $S^2$ is a nonempty closed subspace that is a deformation retract of some neighborhood in $S^2$, we can apply Theorem 2.13.
      Thus $\tilde{H}_2(A) \rightarrow \tilde{H}_2(S^2) \xrightarrow{\phi} \tilde{H_2}(S^2, A) \xrightarrow{\psi} \tilde{H_1}{A} \rightarrow \tilde{H_1}(S^2)$ is an exact sequence.
      Then $\tilde{H_2}(A) = \tilde{H_1}(A) = \tilde{H_1}(S^2) = 0$ and $\tilde{H_2}(S^2) = \mathbb{Z}$.
      Then $\tilde{H_2}(S^2, A) = \tilde{H_2}(S^2)/\ker(\phi) = \tilde{H_2}(S^2) = \mathbb{Z}$.

      $\tilde{H}_0(A) \rightarrow \tilde{H}_0(S^2) \rightarrow \tilde{H_0}(S^2, A) \rightarrow 0$ is an exact sequence.
      Since $\tilde{H}_0(A) = \tilde{H}_0(S^2) = 0$, $\tilde{H_0}(S^2, A) = 0$.

      For any $n \geq 3$, $\tilde{H}_n(A) \rightarrow \tilde{H}_n(S^2) \rightarrow \tilde{H_n}(S^2, A) \rightarrow \tilde{H}_{n - 1}(A)$ is an exact sequence.
      Since $\tilde{H}_n(A) = \tilde{H}_n(S^2) = \tilde{H}_{n - 1}(A) = 0$, $\tilde{H}_n(S^2, A) = 0$.

      Finally, $\tilde{H_1}(A) \rightarrow \tilde{H}_1(X) \xrightarrow{\phi} \tilde{H}_1(X?A) \xrightarrow{\psi} \tilde{H_0}(A) \rightarrow \tilde{H_0}(X)$.
      We have $\tilde{H}_1(X / A) / \ker(\psi) = \Im(\psi)$.
      Since $\psi$ is surjective and $\ker(\psi) = \Im(\phi) = 0$, $\tilde{H}_1(X / A) = \tilde{H}_0(A)$.

      \todo[inline,caption={}]{
        $S^1 \times S^1$
      }
    \item
      \todo[inline,caption={}]{
        Finish this!
      }
  \end{itemize}
\end{proof}

\end{document}


