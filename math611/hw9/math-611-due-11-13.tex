\documentclass[12pt, psamsfonts]{amsart}

%-------Packages---------
\usepackage{amssymb,amsfonts}
\usepackage{fullpage}
\usepackage{tikz-cd}
\usepackage{todonotes}
\usepackage{physics}
\usepackage[all,arc]{xy}
\usepackage{enumerate}
\usepackage{enumitem}
\usepackage{mathrsfs}
\usepackage{theoremref}
\usepackage{graphicx}
\usepackage[bookmarks]{hyperref}

%--------Theorem Environments--------
%theoremstyle{plain} --- default
\newtheorem{thm}{Theorem}[section]
\newtheorem{cor}[thm]{Corollary}
\newtheorem{prop}[thm]{Proposition}
\newtheorem{lem}[thm]{Lemma}
\newtheorem{conj}[thm]{Conjecture}
\newtheorem{quest}[thm]{Question}

\theoremstyle{definition}
\newtheorem{defn}[thm]{Definition}
\newtheorem{defns}[thm]{Definitions}
\newtheorem{con}[thm]{Construction}
\newtheorem{exmp}[thm]{Example}
\newtheorem{exmps}[thm]{Examples}
\newtheorem{notn}[thm]{Notation}
\newtheorem{notns}[thm]{Notations}
\newtheorem{addm}[thm]{Addendum}
\newtheorem*{exer}{Exercise}

\theoremstyle{remark}
\newtheorem{rem}[thm]{Remark}
\newtheorem{rems}[thm]{Remarks}
\newtheorem{warn}[thm]{Warning}
\newtheorem{sch}[thm]{Scholium}

\DeclareMathOperator{\Hom}{Hom}
\DeclareMathOperator{\Id}{Id}
\DeclareMathOperator{\End}{End}
\DeclareMathOperator{\ord}{ord}
\DeclareMathOperator{\Aut}{Aut}

\makeatletter
\let\c@equation\c@thm
\makeatother
\numberwithin{equation}{section}

\bibliographystyle{plain}

\begin{document}

\title{Math 611 (Due 11/13)}
\author{Hidenori Shinohara}
\maketitle

\section{Simplicial and Singular Homology}

\begin{exer}{(Problem 20)}
  Show that $H(X) = H(SX)$ for all $n$, where $SX$ is the suspension of $X$.
  More generally, thinking of $SX$ as the union of two cones $CX$ with their bases identified, compute the reduced homology groups of the union of any finite number of cones $CX$ with their bases identified.
\end{exer}

\begin{proof}
  It suffices to only prove the second part.
  Let $(CX)^k$ denote the union of $k$ cones with their bases identified.
  Specifically, $(CX)^2 = SX$.
  We claim that $\tilde{H}_n(X) \cong \tilde{H}_{n + 1}((CX)^{k + 1})$ for all $k \geq 1$.

  Let $i \geq 1$.
  We have $\tilde{H}_i(CX) \rightarrow \tilde{H}_i((CX)^{k + 1}) \rightarrow \tilde{H}_i((CX)^{k + 1}, CX) \rightarrow \tilde{H}_{i - 1}(CX)$.
  Since $CX$ is contractible by the deformation retract $f_t: (x, s) \mapsto (x, s(1 - t))$, $\tilde{H}_i(CX) = \tilde{H}_{i - 1}(CX) = 0$.
  By the exactness, $\tilde{H}_i((CX)^{k + 1}) = \tilde{H}_i((CX)^{k + 1}, CX)$.
  Let $x$ be the north pole of $CX$.
  We will consider the inclusion $((CX)^{k + 1} - x, CX - x) \rightarrow ((CX)^{k + 1}, CX)$.
  By the excision theorem, $\tilde{H}_i((CX)^{k + 1} - x, CX - x) = \tilde{H}_i((CX)^{k + 1}, CX)$.

  We will consider the exact sequence $\tilde{H}_i((CX)^{k + 1} - x) \rightarrow \tilde{H}_i((CX)^{k + 1} - x, CX - x) \rightarrow \tilde{H}_{i - 1}(CX - x) \rightarrow \tilde{H}_{i - 1}((CX)^{k + 1} - x)$.
  Since $(CX)^{k + 1} - x$ is contractible by the same argument as $CX$, $\tilde{H}_i((CX)^{k + 1} - x) = \tilde{H}_{i - 1}((CX)^{k + 1} - x) = 0$.
  By the exactness, we have $\tilde{H}((CX)^{k + 1} - x, CX - x) = \tilde{H}_{i - 1}(CX - x)$.
  Since $CX - x$ is homeomorphic to $X$, $\tilde{H}_{i - 1}(CX - x) = \tilde{H}_{i - 1}(X)$.
  Hence, $\tilde{H}_i((CX)^{k + 1}) = \tilde{H}_{i - 1}(X)$.
\end{proof}

\begin{exer}{(Problem 22)}
  Prove by induction on dimension the following facts about the homology of a finite-dimensional $CW$ complex $X$, using the observation that $X^n / X^{n-1}$ is a wedge sum of $n$ spheres:
  \begin{itemize}
    \item
      If $X$ has dimension $n$ then $H_i(X) = 0$ for $i > n$ and $H_n(X)$ is free.
    \item
      $H_n(X)$ is free with basis in bijective correspondence with the $n$ cells if there are no cells of dimension $n - 1$ or $n + 1$.
    \item
      If $X$ has $k$ $n$-cells, then $H_n(X)$ is generated by at most $k$ elements.
  \end{itemize}
\end{exer}

\begin{proof}
$ $
  \begin{itemize}
    \item
      $X^0$ is a set of points, so it is clear that $H_i(X) = 0$ for $i > 0$.
      Let $k \geq 0$.
      Suppose that $H_i(X) = 0$ for $i > k$.
      Let $n = k + 1$.
      Then we have an exact sequence $H_{i}(X^{n - 1}) \rightarrow H_{i}(X^n) \rightarrow H_{i}(X^n, X^{n - 1})$ for any $i > n$.
      Since $(X^n, X^{n - 1})$ is a good pair, $H_{n + 1}(X^n, X^{n - 1}) = H_{n + 1}(X^n / X^{n - 1}) = H_{n + 1}(\vee_{\alpha} S^n) = \oplus_{\alpha} 0 = 0$.
      By the inductive hypothesis, $H_i(X^{n - 1}) = 0$.
      Therefore, the exactness of $0 \rightarrow H_i(X^n) \rightarrow 0$ implies that $H_i(X^n) = 0$ for all $i > n$.

      Moreover, the exact sequence $0 = H_n(X^{n - 1}) \rightarrow H_n(X^n) \xrightarrow{\phi} H_n(X^n / X^{n - 1})$ shows that $\phi$ is injective.
      This means that $H_n(X^n / X^{n - 1}) = H_n(\wedge_{\alpha} S^n) = \oplus_{\alpha} \mathbb{Z}$ contains an isomorphic copy of $H_n(X^n) = H_n(X)$.
      Therefore, $H_n(X)$ must be free.
    \item
  \end{itemize}
\end{proof}

\begin{exer}{(Problem 27)}
  Let $f:(X, A) \rightarrow (Y, B)$ be a map such that both $f: X \rightarrow Y, f:A \rightarrow B$ are homotopy equivalences.

  \begin{itemize}
    \item
      Show that $f_* : H_n(X, A) \rightarrow  H_n(Y, B)$ is an isomorphism for all $n$.
    \item
      For the case of the inclusion $f:(D^n, S^{n - 1}) \rightarrow (D^n, D^n \setminus \{ 0 \})$, show that $f$ is not a homotopy equivalence of pairs -
      there is no $g: (D^n, D^n \setminus \{ 0 \}) \rightarrow (D^n, S^{n - 1})$ such that $fg$ and $gf$ are homotopic to the identity through maps of pairs.
  \end{itemize}
\end{exer}

\begin{proof}
  $ $
  \begin{itemize}
    \item
      For each $n \geq 1$, we have an exact sequence $H_n(A) \rightarrow H_n(X) \rightarrow H_n(X, A) \rightarrow H_{n - 1}(A) \rightarrow H_{n - 1}(X)$ and another one with $X, A$ replaced with $Y, B$.
      Moreover, they are connected by homomorphisms $f_*: H_n(A) \rightarrow H_n(B), f_*: H_n(X) \rightarrow H_n(Y), f_*:H_n(X, A) \rightarrow H_n(Y, B)$ such that the diagram commutes. (naturality)
      Since $f: X \rightarrow Y$ and $f: A \rightarrow B$ are both homotopy equivalences, $f_*: H_n(X) \rightarrow H_n(Y), f_*: H_n(A) \rightarrow H_n(B)$ are isomorphisms.
      By the Five lemma, $f_*: H_n(X, A) \rightarrow H_n(X, B)$ is an isomorphism.

      The exact sequence $H_1(A) \rightarrow H_1(X) \rightarrow H_1(X, A) \rightarrow 0$ can be extended to $H_1(A) \rightarrow H_1(X) \rightarrow H_1(X, A) \rightarrow 0 \rightarrow 0$ by appending 0 at the end.
      Using the same argument as above, $f_*: H_1(X, A) \rightarrow H_1(Y, B)$ is an isomorphism.
    \item
      Suppose $f: (D^n, S^{n - 1}) \rightarrow (D^n, D^n - \{ 0 \})$ is a homotopy equivalence.
      Then there exists a $g: (D^n, D^n - \{ 0 \}) \rightarrow (D^n, S^{n - 1})$ such that $f \circ g$ and $g \circ f$ are homotopic to the identity maps in corresponding domains.
      Since $g$ is continuous, $g(\overline{D^n - \{ 0 \}}) = \overline{g(D^n - \{ 0 \})} \subset \overline{S^{n - 1}} = S^{n - 1}$.
      Therefore, $g$ maps $D^n$ into $S^{n - 1}$.
      Since $f$ maps $S^{n - 1}$ into $D^n$, $g \circ f$ maps $S^{n - 1}$ into $S^{n - 1}$.
      We know this is homotopic to the identity map from the problem statement.
      Similarly, $f \circ g$ maps $D^n$ into $D^n$ and we know this is homotopic to the identity map from the problem statement.
      Therefore, this implies that $D^n$ and $S^{n - 1}$ are homotopy equivalent.
      However, this is false because $D^n$ is contractible but $S^{n - 1}$ is not.

      Hence, $f$ cannot be homotopy equivalent.
  \end{itemize}

\end{proof}

\end{document}


