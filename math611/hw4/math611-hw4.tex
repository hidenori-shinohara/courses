\documentclass[12pt, psamsfonts]{amsart}

%-------Packages---------
\usepackage{amssymb,amsfonts}
\usepackage{fullpage}
\usepackage{todonotes}
\usepackage{physics}
\usepackage[all,arc]{xy}
\usepackage{enumerate}
\usepackage{mathrsfs}
\usepackage{theoremref}
\usepackage{graphicx}
\usepackage[bookmarks]{hyperref}

%--------Theorem Environments--------
%theoremstyle{plain} --- default
\newtheorem{thm}{Theorem}[section]
\newtheorem{cor}[thm]{Corollary}
\newtheorem{prop}[thm]{Proposition}
\newtheorem{lem}[thm]{Lemma}
\newtheorem{conj}[thm]{Conjecture}
\newtheorem{quest}[thm]{Question}

\theoremstyle{definition}
\newtheorem{defn}[thm]{Definition}
\newtheorem{defns}[thm]{Definitions}
\newtheorem{con}[thm]{Construction}
\newtheorem{exmp}[thm]{Example}
\newtheorem{exmps}[thm]{Examples}
\newtheorem{notn}[thm]{Notation}
\newtheorem{notns}[thm]{Notations}
\newtheorem{addm}[thm]{Addendum}
\newtheorem*{exer}{Exercise}

\theoremstyle{remark}
\newtheorem{rem}[thm]{Remark}
\newtheorem{rems}[thm]{Remarks}
\newtheorem{warn}[thm]{Warning}
\newtheorem{sch}[thm]{Scholium}

\DeclareMathOperator{\Hom}{Hom}
\DeclareMathOperator{\Id}{Id}

\makeatletter
\let\c@equation\c@thm
\makeatother
\numberwithin{equation}{section}

\bibliographystyle{plain}

\begin{document}

\title{Math 611 Homework (Due 9/25)}
\author{Hidenori Shinohara}
\maketitle

\begin{exer}{(Problem 4, Chapter 1.3)}
  Construct a simply-connected covering space of the space $X \subset \mathbb{R}^3$ that is the union of a sphere and a diameter.
  Do the same when $X$ is the union of a sphere and a circle intersecting it in two points.
\end{exer}

\begin{proof}
  \begin{figure}
    \includegraphics[width=.5\linewidth]{problem4-1.jpeg}
    \caption{Problem 4 (Part 1)}
    \label{fig:prob4_1}
  \end{figure}
  We claim that the space described in Figure \ref{fig:prob4_1} is a covering space of $X$.
  \begin{itemize}
    \item
      The shape is an infinitely long chain of spheres and lines.
      The chain goes infinitely both ways (up and down).
      This space is clearly simply connected.
    \item
      We will map each sphere to the sphere of $X$.
      Each line will be mapped to the diameter up side down.
      Figure \ref{fig:prob4_1} shows how each part gets mapped.
    \item
      We claim that such a mapping is a covering map and thus this infinite chain is indeed a covering space.
      Let $x \in X$.
      \todo[inline]{
        Prove this.
      }
  \end{itemize}
  \todo[inline]{
    Second part.
  }
\end{proof}

\begin{exer}{(Problem 5, Chapter 1.3)}
  Let $X$ be the subspace of $\mathbb{R}^2$ consisting of the four sides of the square $[0, 1] \times [0, 1]$ together with the segments of the vertical lines $x = 1/2, 1/3, 1/4, \cdots$ inside the square.
  Show that for every covering space $X \rightarrow X$ there is some neighborhood of the left edge of $X$ that lifts homeomorphically to $\tilde{X}$.
  Deduce that $X$ has no simply-connected covering space.
\end{exer}

\begin{proof}
  \todo[inline]{
    Idea:
    Open cover of the left edge by evenly covered open sets.
    Find a finite subcover.
    By the tube lemma, there exists $U \times I$ that covers the left edge and a partition $0 = t_0 < \cdots < t_n = 1$ such that $U \times [t_i, t_{i + 1}]$ is contained in an evenly covered neighborhood.
    Inductively, show $U \times [0, t_i]$ is in an evenly covered neighborhood.
    Lift a loop in $X$ with a vertical line $x = 1/n$ for some large $n$.
    Then the element maps back to itself by $p$.
    In other words, $p_*(\pi_1(\tilde{X})) \ne 0$.
  }
\end{proof}

\begin{exer}{(Problem 7, Chapter 1.3)}
  Let $Y$ be the quasi-circle in the figure in the textbook.
  Collapsing the segment of $Y$ in the $y$-axis to a point gives a quotient map $f: Y \rightarrow S^1$.
  Show that $f$ does not lift to the covering space $\mathbb{R} \rightarrow S^1$, even though $\pi_1(Y) = 0$.
  Thus local path-connectedness of $Y$ is a necessary hypothesis in the lifting criterion.
\end{exer}

\begin{proof}
  \todo[inline]{
    The lifting criterion is Proposition 1.33.
    The only property that $Y$ is missing is the local path connectedness.
    But I'm not sure how to make use of it.
  }
\end{proof}

\begin{exer}{(Problem 8, Chapter 1.3)}
  Let $\tilde{X}$ and $\tilde{Y}$ be simply-connected covering spaces of the path-connected, locally path-connected spaces $X$ and $Y$.
  Show that if $X \simeq Y$ then $\tilde{X} \simeq \tilde{Y}$.
\end{exer}

\begin{proof}
  \todo[inline]{
    By Proposition 1.33, we can lift the two compositions as in Figure \ref{fig:problem8}
    This works because $\pi_1(\tilde{X}) = \pi_1(\tilde{Y}) = 0$.
  }
  \begin{figure}
    \includegraphics[width=.5\linewidth]{problem8.jpeg}
    \caption{delete this!}
    \label{fig:problem8}
  \end{figure}
\end{proof}


\end{document}


