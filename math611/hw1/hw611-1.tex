\documentclass[12pt, psamsfonts]{amsart}

\usepackage{bbm}

%-------Packages---------
\usepackage{amssymb,amsfonts}
\usepackage[all,arc]{xy}
\usepackage{enumerate}
\usepackage{mathrsfs}
\usepackage{theoremref}
\usepackage{graphicx}
\usepackage[bookmarks]{hyperref}

%--------Theorem Environments--------
%theoremstyle{plain} --- default
\newtheorem{thm}{Theorem}[section]
\newtheorem{cor}[thm]{Corollary}
\newtheorem{prop}[thm]{Proposition}
\newtheorem{lem}[thm]{Lemma}
\newtheorem{conj}[thm]{Conjecture}
\newtheorem{quest}[thm]{Question}

\theoremstyle{definition}
\newtheorem{defn}[thm]{Definition}
\newtheorem{defns}[thm]{Definitions}
\newtheorem{con}[thm]{Construction}
\newtheorem{exmp}[thm]{Example}
\newtheorem{exmps}[thm]{Examples}
\newtheorem{notn}[thm]{Notation}
\newtheorem{notns}[thm]{Notations}
\newtheorem{addm}[thm]{Addendum}
\newtheorem{exer}[thm]{Exercise}

\theoremstyle{remark}
\newtheorem{rem}[thm]{Remark}
\newtheorem{rems}[thm]{Remarks}
\newtheorem{warn}[thm]{Warning}
\newtheorem{sch}[thm]{Scholium}

\DeclareMathOperator{\Hom}{Hom}
\DeclareMathOperator{\Id}{Id}

\makeatletter
\let\c@equation\c@thm
\makeatother
\numberwithin{equation}{section}

\bibliographystyle{plain}

\begin{document}

\title{Math 611 Problem Set 1 (Due 9/4)}
\author{Hidenori Shinohara}
\maketitle

\begin{exer}{(Exercise 4, Chapter 0)}
  A deformation retraction in the weak sense of a space $X$ to a subspace $A$ is a homotopy $f_t: X \rightarrow X$ such that $f_0 = \Id, f_1(X) \subset A$, and $f_t(A) \subset A$ for all $t$.
  Show that if $X$ deformation retracts to $A$ in this weak sense, then the inclusion $A \rightarrow X$ is a homotopy equivalence.
\end{exer}

\begin{proof}
  Let $i: A \rightarrow X$ denote the inclusion.
  Let $F: X \times I \rightarrow X$ denote the associated map $(x, t) \rightarrow f_t(x)$.
  Then $F$ is a continuous function by the definition of a homotopy.

  Let $f: X \rightarrow A$ be defined by $f(x) = F(x, 1) = f_1(x)$.
  This definition makes sense because $f_1(X) \subset A$.
  We claim that $f_1 \circ i \simeq \Id_A$ and $i \circ f_1 \simeq \Id_X$.

  Consider $G: A \times I \rightarrow A$ such that $G(a, t) = F(a, t)$ for all $(a, t) \in A \times I$.
  This definition makes sense because $f_t(A) \subset A$ for all $t$.

  Then $G$ is a homotopy in $A$ between $f \circ i$ and $\Id_A$ because:
  \begin{itemize}
    \item
      $G$ is a restriction of $F$, so $G$ is continuous.
    \item
      $\forall a \in A, G(a, 0) = F(a, 0) = f_0(a) = \Id_X(a) = \Id_A(a)$.
    \item
      $\forall a \in A, G(a, 1) = F(a, 1) = f(a) = f(i(a)) = (f \circ i)(a)$.
  \end{itemize}
  Therefore, $f \circ i \simeq \Id_A$.

  $F$ is a homotopy between $f_0$ and $f_1$.
  \begin{itemize}
    \item
      We are given that $f_0 = \Id_X$.
    \item
      For any $x \in X$, $(i \circ f)(x) = i(f(x)) = f(x) = f_1(x)$, so $i \circ f = f_1$.
  \end{itemize}
  Therefore, $F$ is a homotopy between $\Id_X$ and $i \circ F$, so $i \circ f \simeq \Id_X$.

  In conclusion, $i$ is indeed a homotopy equivalence.
\end{proof}

\end{document}
