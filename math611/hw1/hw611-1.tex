\documentclass[12pt, psamsfonts]{amsart}

\usepackage{bbm}

%-------Packages---------
\usepackage{amssymb,amsfonts}
\usepackage{tikz}
\usetikzlibrary{matrix}
\usepackage[all,arc]{xy}
\usepackage{enumerate}
\usepackage{mathrsfs}
\usepackage{theoremref}
\usepackage{graphicx}
\usepackage[bookmarks]{hyperref}

%--------Theorem Environments--------
%theoremstyle{plain} --- default
\newtheorem{thm}{Theorem}[section]
\newtheorem{cor}[thm]{Corollary}
\newtheorem{prop}[thm]{Proposition}
\newtheorem{lem}[thm]{Lemma}
\newtheorem{conj}[thm]{Conjecture}
\newtheorem{quest}[thm]{Question}

\theoremstyle{definition}
\newtheorem{defn}[thm]{Definition}
\newtheorem{defns}[thm]{Definitions}
\newtheorem{con}[thm]{Construction}
\newtheorem{exmp}[thm]{Example}
\newtheorem{exmps}[thm]{Examples}
\newtheorem{notn}[thm]{Notation}
\newtheorem{notns}[thm]{Notations}
\newtheorem{addm}[thm]{Addendum}
\newtheorem{exer}[thm]{Exercise}

\theoremstyle{remark}
\newtheorem{rem}[thm]{Remark}
\newtheorem{rems}[thm]{Remarks}
\newtheorem{warn}[thm]{Warning}
\newtheorem{sch}[thm]{Scholium}

\DeclareMathOperator{\Hom}{Hom}
\DeclareMathOperator{\Id}{Id}

\makeatletter
\let\c@equation\c@thm
\makeatother
\numberwithin{equation}{section}

\bibliographystyle{plain}

\begin{document}

\title{Math 611 Problem Set 1 (Due 9/4)}
\author{Hidenori Shinohara}
\maketitle

\begin{exer}{(Exercise 4, Chapter 0)}
  A deformation retraction in the weak sense of a space $X$ to a subspace $A$ is a homotopy $f_t: X \rightarrow X$ such that $f_0 = \Id, f_1(X) \subset A$, and $f_t(A) \subset A$ for all $t$.
  Show that if $X$ deformation retracts to $A$ in this weak sense, then the inclusion $A \rightarrow X$ is a homotopy equivalence.
\end{exer}

\begin{proof}
  Let $i: A \rightarrow X$ denote the inclusion.
  Let $F: X \times I \rightarrow X$ denote the associated map $(x, t) \rightarrow f_t(x)$.
  Then $F$ is a continuous function by the definition of a homotopy.

  Let $f: X \rightarrow A$ be defined by $f(x) = F(x, 1) = f_1(x)$.
  This definition makes sense because $f_1(X) \subset A$.
  We claim that $f_1 \circ i \simeq \Id_A$ and $i \circ f_1 \simeq \Id_X$.

  Consider $G: A \times I \rightarrow A$ such that $G(a, t) = F(a, t)$ for all $(a, t) \in A \times I$.
  This definition makes sense because $f_t(A) \subset A$ for all $t$.

  Then $G$ is a homotopy in $A$ between $f \circ i$ and $\Id_A$ because:
  \begin{itemize}
    \item
      $G$ is a restriction of $F$, so $G$ is continuous.
    \item
      $\forall a \in A, G(a, 0) = F(a, 0) = f_0(a) = \Id_X(a) = \Id_A(a)$.
    \item
      $\forall a \in A, G(a, 1) = F(a, 1) = f(a) = f(i(a)) = (f \circ i)(a)$.
  \end{itemize}
  Therefore, $f \circ i \simeq \Id_A$.

  $F$ is a homotopy between $f_0$ and $f_1$.
  \begin{itemize}
    \item
      We are given that $f_0 = \Id_X$.
    \item
      For any $x \in X$, $(i \circ f)(x) = i(f(x)) = f(x) = f_1(x)$, so $i \circ f = f_1$.
  \end{itemize}
  Therefore, $F$ is a homotopy between $\Id_X$ and $i \circ F$, so $i \circ f \simeq \Id_X$.

  In conclusion, $i$ is indeed a homotopy equivalence.
\end{proof}


\begin{exer}{(Exercise 5, Chapter 0)}
  Show that if a space $X$ deformation retracts to a point $x \in X$, then for each neighborhood $U$ of $x$ in $X$ there exists a neighborhood $V \subset U$ of $x$ such that the inclusion map $V \rightarrow U$ is nullhomotopic.
\end{exer}

\begin{proof}
  Let $p \in X$ be a point to which $X$ deformation retracts.
  Since $X$ deformation retracts to $p$, there exists a map $F: X \times I \rightarrow X$ such that

  \begin{enumerate}
    \item
      $\forall x \in X, F(x, 0) = x$.
    \item
      $\forall x \in X, F(x, 1) = p$.
    \item
      $\forall t \in I, F(p, t) = p$.
    \item
      $F$ is continuous.
  \end{enumerate}
  Let $U$ be a neighborhood of $p$.
  Then $F^{-1}(U)$ is an open subset of the product space $X \times I$.
  By the 3rd property of $F$ mentioned above, the slice $\{ p \} \times I$ is a subset of $F^{-1}(U)$.
  Since $I$ is compact, there must be a open subset $V$ of $X$ such that $\{ p \} \times I \subset V \times I \subset F^{-1}(U)$ by the tube lemma.

  We claim that this $V$ is a desired subset.

  \begin{itemize}
    \item
      $V$ is an open subset of $X$.
    \item
      Since $\{ p \} \times I \subset V \times I$, $p \in V$.
    \item
      Since $V \times I \subset F^{-1}(U)$, $F(V \times I) \subset U$.
      This implies that $\forall v \in V$, $F(v, 0) = v \in U$.
      Therefore, $V \subset U$.
    \item
      We claim that the inclusion map $i: V \rightarrow U$ is nullhomotopic.
      Let $e_p:V \rightarrow U$ denote the constant map at $p$, $G: V \times I \rightarrow U$ be defined by $G(x, t) = F(x, t)$ for all $x \in V, t \in I$.
      \begin{itemize}
        \item
          $G$ indeed maps $V \times I$ into $U$ because $F(V \times I) \subset U$.
          Therefore, $G$ is well-defined.
        \item
          Since $G$ is the restriction of $F$ to $V \times I$ and $F$ is continuous, $G$ is continuous.
        \item
          $\forall x \in V, G(x, 0) = F(x, 0) = x = i(x)$.
        \item
          $\forall x \in V, G(x, 1) = F(x, 1) = p = e_p(x)$.
      \end{itemize}
      Thus $i$ is indeed nullhomotopic.
  \end{itemize}
\end{proof}

\begin{lem}
  The neighborhood $V$ that we find in Problem 5 is connected.
\end{lem}

\begin{proof}
  Suppose otherwise.
  Let $A, B$ denote a separation of $V$.
  Without loss of generality, we assume $p \in A$.
  Let $q \in B$.
  ($B$ must be nonempty since $A, B$ are a separation.)

  Let $F$ be the homotopy we defined in the solution for Problem 5 from the inclusion map to the constant map at $p$.
  Let $f: I \rightarrow V$ be defined such that $f(t) = F(q, t)$.
  Then $f$ is a path from $f(0) = F(q, 0) = q$ to $f(1) = F(q, 1) = p$ in $V$.
  Since $f$ is continuous, $f^{-1}(A)$ and $f^{-1}(B)$ are open in $I$.
  Moreover, $I = f^{-1}(V) = f^{-1}(A) \cup f^{-1}(B)$ and $\emptyset = f^{-1}(\emptyset) = f^{-1}(A \cap B) = f^{-1}(A) \cap f^{-1}(B)$.
  Since $1 \in f^{-1}(p) \subset f^{-1}(A)$ and $0 \in f^{-1}(q) \subset f^{-1}(B)$, $f^{-1}(A)$ and $f^{-1}(B)$ are nonempty.
  Therefore, $f^{-1}(A)$ and $f^{-1}(B)$ form a separation of $I$.
  However, this is impossible because $I$ is connected.
\end{proof}

\begin{exer}{(Exercise 6(a), Chapter 0)}
  Let $X$ be the subspace of $\mathbb{R}^2$ consisting of the horizontal segment $[0, 1] \times \{ 0 \}$ together with all the vertical segments $\{ r \} \times [0, 1 - r]$ for $r$ a rational number in $[0, 1]$.
  Show that $X$ deformation retracts to any point in the segment $[0, 1] \times \{ 0 \}$, but not to any other point.
\end{exer}

\begin{proof}
  Let $(a, 0) \in [0, 1] \times \{ 0 \}$ be given.
  Let $F: X \times I \rightarrow X$ be defined such that

  \begin{align*}
    F((x, y), t) &= \begin{cases}
      (x, (1 - 2t)y) & (0 \leq t \leq 1 / 2) \\
      (x + (a - x)(2t - 1), 0) & (1 / 2 \leq t \leq 1).
    \end{cases}
  \end{align*}

  $F$ is well defined because when $t = 1 / 2$:
  \begin{itemize}
    \item
      $(x, (1 - 2t)y) = (x, 0)$.
    \item
      $(x + (a - x)(2t - 1), 0) = (x, 0)$.
  \end{itemize}
  Moreover, by the pasting lemma, $F$ is continuous.

  Then $F$ is a deformation retract of $X$ onto $(a, 0)$ because

  \begin{itemize}
    \item
      \begin{align*}
        F((a, 0), t) = \begin{cases}
          (a, 0(1 - 2t)) = (a, 0) & (t \in [0, 1 / 2]) \\
          (a + (a - a)(2t - 1), 0) = (a, 0) & (t \in [1 / 2, 1]).
        \end{cases}
      \end{align*}
      Therefore, $F((a, 0), t) = (a, 0)$ for any $t \in I$.
    \item
      $F((x, y), 0) = (x, y)$ for any $(x, y) \in x$.
    \item
      $F((x, y), 1) = (a, 0)$ for any $(x, y) \in x$.
  \end{itemize}

  Therefore, $F$ is indeed a deformation retract of $X$ onto $(a, 0)$.

  Suppose that there exists a point $(a, b) \in X$ to which $X$ deformation retracts onto such that $b \ne 0$.
  Let $G: X \times I \rightarrow X$ denote such a deformation retract.
  Consider the open subset $U = B((a, b), b) \cap X$.
  Note that $U$ is disjoint from the segment $[0, 1] \times \{ 0 \}$.
  Then $U$ is a neighborhood of $(a, b)$, a point to which $X$ deformation retracts onto.
  By Problem 5 (Chapter 0), there must exist a neighborhood $V \subset U$ of $x$ such that the inclusion map $V \rightarrow U$ is nullhomotopic.
  By the Lemma we showed above, $V$ must be connected.
  Since $V$ is an open subset of $X$, there must exist an $r > 0$ such that $B((a, b), r) \cap X \subset V$.
  Let $c$ be an irrational number in $(a, a + r)$.
  Then $V \cap ((-\infty, c) \times \mathbb{R})$ and $V \cap ((c, \infty) \times \mathbb{R})$ form a separation of $V$.
  This is a contradiction, so our initial assumption that $X$ deformation retracts onto $(a, b)$ was wrong.
  Therefore, $X$ deformation retracts to any point in the segment $[0, 1] \times \{ 0 \}$, but not to any other point.
\end{proof}

\begin{exer}{(Exercise 9, Chapter 0)}
  Show that a retract of a contractible space is contractible.
\end{exer}

\begin{proof}
  Let $X$ be a contractible space.
  Then $\Id_X$ is homotopic to a constant map.
  This implies the existence of a fixed point $p \in X$ and a continuous function $F: X \times I \rightarrow X$ such that
  \begin{itemize}
    \item
      $\forall x \in X, F(x, 0) = x$,
    \item
      $\forall x \in X, F(x, 1) = p$.
  \end{itemize}
  Let $A \subset X$ be a retract of $X$, and let $r: X \rightarrow A$ denote a retraction.
  In other words, $r(X) = A$ and $r \mid_A = \Id_A$.

  Let $G: A \times I \rightarrow A$ be the restriction of $r \circ F$ to $A \times I$.
  This makes sense because $F$ maps $A \times I$ into $X$, and $r$ maps $X$ into $A$.
  We claim that $G$ is a homotopy between $\Id_A$ and the constant map $e_{r(p)}$ such that $e_{r(p)}(a) = r(p)$ for all $a \in A$.

  \begin{itemize}
    \item
      $r \circ F$ is continuous since it is a composition of continuous functions.
      $G$ is a restriction of a continuous function, so $G$ is continuous.
    \item
      $G(a, 0) = r(F(a, 0)) = r(a) = a = \Id_A(a)$.
    \item
      $G(a, 1) = r(F(a, 1)) = r(p) = e_{r(p)}(a)$.
  \end{itemize}

  Therefore, $G$ is indeed a homotopy between $\Id_A$ and the constant map at $r(p)$.
  Since the identity map is homotopic to a constant map, $A$ is contractible.
\end{proof}

\begin{exer}{(Exercise 13, Chapter 0)}
  Show that any two deformation retractions $r_t^0$ and $r_t^1$ of a space $X$ onto a subspace $A$ can be joined by a continuous family of deformation retractions $r^s_t$, $0 \leq s \leq 1$, of $X$ onto $A$, where continuity means that the map $X \times I \times I \rightarrow X$ sending $(x, s, t)$ to $r^s_t(x)$ is continuous.
\end{exer}

\begin{proof}
  Let $F: X \times I \times I \rightarrow X$ be defined such that

  \begin{align*}
    F(x, t, s) &= \begin{cases}
      r_{t(1 - 2s)}^0(x) & (s \in [0, 1/2]) \\
      r_{t(2s - 1)}^1(x) & (s \in [1/2, 1]).
    \end{cases}
  \end{align*}

  We claim that $F$ is well-defined and satisfies the desired properties.

  \begin{itemize}
    \item
      Let $s = 1 / 2$.
      $r_{t(1 - 2s)}^0(x) = r_0^0(x) = x$ because $r^0_t$ is a deformation retraction.
      Similarly, $r_{t(2s - 1)}^1(x) = r_0^1(x) = x$ because $r^0_t$ is a deformation retraction.
      Therefore, $F$ is well defined when $s = 1/2$.
      Moreover, by the pasting lemma, $F$ is continuous.
      This is because the intersection $X \times I \times [0, 1/2] \cap X \times I \times [1/2, 1] = X \times I \times \{ 1/2 \}$ is closed.
    \item
      $F(x, t, 0) = r_t^0(x)$ for any $x \times t \in X \times I$.
    \item
      $F(x, t, 1) = r_t^1(x)$ for any $x \times t \in X \times I$.
  \end{itemize}

  Therefore, $F$ maps $X \times I \times I \rightarrow X$ continuously sending $(x, s, t)$ to $r^s_t(x)$.
\end{proof}

\begin{exer}{(Exercise 7, Chapter 1.1)}
  Define $f: S^1 \times I \rightarrow S^1 \times I$ by $f(\theta, s) = (\theta + 2\pi s, s)$, so $f$ restricts to the identity on the two boundary circles of $S^1 \times I$.
  Show that $f$ is homotopic to the identity by a homotopy $f_t$ that is stationary on one of the boundary circles, but not by any homotopy $f_t$ that is stationary on both boundary circles.
\end{exer}

\begin{proof}
  Define $F: (S^1 \times I) \times I \rightarrow S^1 \times I$ such that $F((\theta, s), t) &= t(\theta, s) + (1 - t)f(\theta, s)$.
  Then $F$ is a homotopy between $f$ and the identity map that is stationary on $S^1 \times \{ 0 \}$.
  This is because $F((\theta, 0), t) = t(\theta, 0) + (1 - t)f(\theta, 0) = (t\theta, 0) + ((1 - t)\theta, 0) = (\theta, 0)$ for any $(\theta, t) \in S^1 \times I$.

  Suppose that there exists a homotopy $G: (S^1 \times I) \times I \rightarrow S^1 \times I$ between $f$ and the identity map that is stationary on both boundary circles.
  Let $H: I \times I \rightarrow S^1$ be defined such that $H(s, t) = \pi_1(F((0, t), s))$ where $\pi_1$ denotes the projection of the first coordinate.

  \begin{itemize}
    \item
      $H(s, 0) = \pi_1(G((0, 0), s)) = \pi_1(0, 0) = 0$ because $G$ is stationary on the circle $S^1 \times \{ 0 \}$.
    \item
      $H(s, 1) = \pi_1(G((0, 1), s)) = \pi_1(0, 1) = 0$ because $G$ is stationary on the circle $S^1 \times \{ 1 \}$.
    \item
      $H(0, t) = \pi_1(G((0, t), 0)) = \pi_1(f(0, t)) = \pi_1(2\pi t, t) = 2\pi t$.
    \item
      $H(1, t) = \pi_1(G((0, t), 1)) = \pi_1(0, t) = 0$.
  \end{itemize}

  Then $t \mapsto H(0, t)$ corresponds to the $\omega$ in Theorem 1.7, and $t \mapsto H(1, t)$ corresponds to a constant map.
  In other words, $H$ is a homotopy between $\omega$ and a constant map in $S^1$.
  However, this is a contradiction because Theorem 1.7 states that $\pi_1(S^1)$ is the infinite cyclic group generated by $\omega$.
  Therefore, such a homotopy $G$ does not exist.
\end{proof}

\begin{exer}{(Exercise 16, Chapter 1.1)}
  Show that there are no retractions $r: X \rightarrow A$ in the following cases:
  \begin{itemize}
    \item
      $X = \mathbb{R}^3$ with $A$ any subspace homeomorphic to $S^1$.
    \item
      $X = S^1 \times D^2$ with $A$ its boundary torus $S^1 \times S^1$.
  \end{itemize}
\end{exer}

\begin{proof}
  $ $
  \begin{itemize}
    \item
      Suppose that there exists a retract $r: X \rightarrow A$.
      In other words, $r$ is a continuous map such that $r \mid_A = \Id_A$ and $r(X) \subset A$.
      Let $\phi: S^1 \rightarrow A$ be a homeomorphism.
      Let $\omega$ be the loop defined in Theorem 1.7.
      Then $\pi_1(S^1, (1, 0))$ is the infinite cyclic group generated by $[\omega]$.
      Consider the following two paths:
      \begin{itemize}
        \item
          $\phi \circ \omega: I \rightarrow A$.
        \item
          $e_0: I \rightarrow A$ such that $e_0(t) = (\phi \circ \omega)(0) = \phi(1, 0)$.
      \end{itemize}
      They are paths in $A$, and $A$ is a subset of $\mathbb{R}^3$, so they are paths in $\mathbb{R}^3$.
      Since $\mathbb{R}^3$ is convex, we can define a linear homotopy between them.
      Let $F: I \times I \rightarrow \mathbb{R}^3$ such that $F(s, t) = t(\phi \circ \omega)(s) + (1 - t)e_0(s)$.
      Therefore, the two paths are homotopic in $\mathbb{R}^3$.

      We will consider $\phi^{-1} \circ r \circ F$ that maps $I \times I \rightarrow S$.
      Since it is a composition of continuous functions, it is continuous.

      \begin{align*}
        (\phi^{-1} \circ r \circ F)(s, 0)
          &= \phi^{-1}(r(F(s, 0))) \\
          &= \phi^{-1}(r(e_0(s))) \\
          &= \phi^{-1}(e_0(s)) & \text{(since $e_0(s) \in A$)} \\
          &= (1, 0). \\
        (\phi^{-1} \circ r \circ F)(s, 1)
          &= \phi^{-1}(r(F(s, 1))) \\
          &= \phi^{-1}(r((\phi \circ \omega)(s))) \\
          &= \phi^{-1}(r(\phi(\omega(s)))) \\
          &= \phi^{-1}(\phi(\omega(s))) & \text{(since $\phi(\omega(s)) \in A$)}\\
          &= \omega(s). \\
        (\phi^{-1} \circ r \circ F)(0, t)
          &= \phi^{-1}(r(\phi(1, 0))) \\
          &= \phi^{-1}(\phi(1, 0)) \\
          &= (1, 0). \\
        (\phi^{-1} \circ r \circ F)(1, t)
          &= \phi^{-1}(r(\phi(1, 0))) \\
          &= \phi^{-1}(\phi(1, 0)) \\
          &= (1, 0).
      \end{align*}

      Therefore, $\phi^{-1} \circ r \circ F$ is a path homotopy between $\omega$ and the constant loop at $(1, 0)$.
      This implies that $[w]$ is the identity element in $\pi_1(S^1)$.
      However, this is a contradiction because Theorem 1.7 states that $\pi_1(S^1)$ is the infinite cyclic group generated by $[w]$.

      Hence, there exists no such retraction $r$.
    \item
      Suppose $X$ retracts onto $A$.
      By Proposition 1.17, the homomorphism $i_*: \pi_1(A, x_0) \rightarrow \pi_1(X, x_0)$ induced by the inclusion $i: A \rightarrow X$ is injective.
      By Theorem 1.7, $\pi_1(S^1)$ is isomorphic to $\mathbb{Z}$.
      $\pi_1(D^2) = 0$ because $D^2$ is a convex subset and thus a linear homotopy connects any paths.
      By Proposition 1.12, $\pi_1(X) = \pi_1(S^1) \times \pi_1(D^2) = \mathbb{Z} \times 0 = \mathbb{Z}$ and $\pi_1(A) = \pi_1(S^1) \times \pi_1(S^1) = \mathbb{Z} \times \mathbb{Z}$.
      Let $f: \mathbb{Z} \times \mathbb{Z} \rightarrow \mathbb{Z}$ be any homomorphism.
      Let $a = f(1, 0), b = f(0, 1)$.
      If $a = 0$ or $b = 0$, $f$ is not injective because $f(0, 0) = 0$.
      Suppose otherwise.
      Then $f(b, 0) = ab = f(0, a)$, so $f$ is not injective.

      Therefore, there exists no injection from $\mathbb{Z} \times \mathbb{Z} \rightarrow \mathbb{Z}$.
      Hence, $X$ does not retract onto $A$.
  \end{itemize}
\end{proof}

\begin{exer}{(Exercise 20, Chapter 1.1)}
  Suppose $f_t: X \rightarrow X$ is a homotopy such that $f_0$ and $f_1$ are each the identity map.
  Use Lemma 1.19 to show that for any $x_0 \in X$, the loop $f_t(x_0)$ represents an element of the center of $\pi_1(X, x_0)$.
\end{exer}

\begin{proof}
  Let $x_0 \in X$ be given.
  Let $f: I \rightarrow X$ be the loop defined such that $f(t) = f_t(x_0)$.
  \begin{itemize}
    \item
      $f_t: X \rightarrow X$ is a homotopy.
    \item
      $f$ is a path formed by the images of the base point $x_0$.
  \end{itemize}
  By Lemma 1.19, the following diagram commutes.

  \begin{center}
  \begin{tikzpicture}
  \matrix (m) [matrix of math nodes,row sep=3em,column sep=4em,minimum width=2em]
  {
                  & \pi_1(X, f_1(x_0)) \\
    \pi_1(X, x_0) &                    \\
                  & \pi_1(X, f_0(x_0)) \\};
  \path[-stealth]
    (m-2-1) edge node [below] {$(f_1)_{*}$} (m-1-2)
            edge node [below] {$(f_0)_{*}$} (m-3-2)
    (m-1-2) edge node [right] {$\beta_f$} (m-3-2);
  \end{tikzpicture}
  \end{center}

  $(f_0)_* = (f_1)_* = (\Id_X)_* = \Id_{\pi_1(X, x_0)}$ by a basic property of induced homomorphisms (P.34 of Hatcher).
  Since $f_0 = f_1 = \Id_X$, $f_0(x_0) = f_1(x_0) = x_0$.
  Therefore, the diagram above can be simplified as following:

  \begin{center}
  \begin{tikzpicture}
  \matrix (m) [matrix of math nodes,row sep=3em,column sep=4em,minimum width=2em]
  {
                  & \pi_1(X, x_0) \\
    \pi_1(X, x_0) &                    \\
                  & \pi_1(X, x_0) \\};
  \path[-stealth]
    (m-2-1) edge node [below] {$\Id$} (m-1-2)
            edge node [below] {$\Id$} (m-3-2)
    (m-1-2) edge node [right] {$\beta_f$} (m-3-2);
  \end{tikzpicture}
  \end{center}

  Let $[g] \in \pi_1(X, x_0)$.
  Then by the diagram above, we have $\Id([g]) = \Id(\beta_f([g]))$.
  This implies $[g] = [f \cdot g \cdot \overline{f}]$.
  Therefore, $[g] \cdot [f] = [f] \cdot [g]$, so $[f]$ commutes with every element in $\pi_1(X, x_0)$.
  Hence, $[f] \in Z(\pi_1(X, x_0))$.

\end{proof}

\end{document}
