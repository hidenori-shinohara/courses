\documentclass[12pt, psamsfonts]{amsart}

%-------Packages---------
\usepackage{amssymb,amsfonts}
\usepackage{fullpage}
\usepackage{todonotes}
\usepackage{physics}
\usepackage[all,arc]{xy}
\usepackage{enumerate}
\usepackage{enumitem}
\usepackage{mathrsfs}
\usepackage{theoremref}
\usepackage{graphicx}
\usepackage[bookmarks]{hyperref}

%--------Theorem Environments--------
%theoremstyle{plain} --- default
\newtheorem{thm}{Theorem}[section]
\newtheorem{cor}[thm]{Corollary}
\newtheorem{prop}[thm]{Proposition}
\newtheorem{lem}[thm]{Lemma}
\newtheorem{conj}[thm]{Conjecture}
\newtheorem{quest}[thm]{Question}

\theoremstyle{definition}
\newtheorem{defn}[thm]{Definition}
\newtheorem{defns}[thm]{Definitions}
\newtheorem{con}[thm]{Construction}
\newtheorem{exmp}[thm]{Example}
\newtheorem{exmps}[thm]{Examples}
\newtheorem{notn}[thm]{Notation}
\newtheorem{notns}[thm]{Notations}
\newtheorem{addm}[thm]{Addendum}
\newtheorem*{exer}{Exercise}

\theoremstyle{remark}
\newtheorem{rem}[thm]{Remark}
\newtheorem{rems}[thm]{Remarks}
\newtheorem{warn}[thm]{Warning}
\newtheorem{sch}[thm]{Scholium}

\DeclareMathOperator{\Hom}{Hom}
\DeclareMathOperator{\Id}{Id}

\makeatletter
\let\c@equation\c@thm
\makeatother
\numberwithin{equation}{section}

\bibliographystyle{plain}

\begin{document}

\title{math 601 (Due 10/9)}
\author{Hidenori Shinohara}
\maketitle
\tableofcontents

\section{Modules}

\begin{exer}{(Problem 1)}
  For each of the $\mathbb{Z}$-modules listed in the handout, answer the questions in the handout.
\end{exer}

\begin{proof}
$ $
  \begin{enumerate}[label=(\alph*)]
    \item 
      $M = \mathbb{Z}^3 \times \mathbb{Z} / 86\mathbb{Z}$.
      \todo[inline]{
        Solve this problem!
      }
    \item 
      $M = \prod_{n \geq 1} \mathbb{Z} / n\mathbb{Z}$.
      \todo[inline]{
        Solve this problem!
      }
    \item 
      $M = \mathbb{Z}[1/p] \subset \mathbb{Q}$.
      \todo[inline]{
        Solve this problem!
      }
    \item 
      $M = \mathbb{Q} / \mathbb{Z}_{(p)}$.
      \todo[inline]{
        Solve this problem!
      }
  \end{enumerate}
\end{proof}

\section{Rings of Fractions}

\begin{exer}{(Problem 3)}
  Let $T \subset R$ be the subset consisting of all nonzero divisors.
  \begin{itemize}
    \item
      Show that $T$ is a multiplicative set.
    \item
      Let $s \in T$ and let $S = \{ 1, s, s^2, s^3, \cdots \} \subset T$.
      Show that the following rings are isomorphic: $S^{-1}R$, the subring $R[1/s] \subset T^{-1}R$, and the quotient ring $R[x]/(sx - 1)$.
  \end{itemize}
\end{exer}

\begin{proof}
$ $
  \begin{itemize}
    \item
     \todo[inline]{
       Prove this!
     }
    \item
     \todo[inline]{
       Prove this!
     }
  \end{itemize}
\end{proof}

\section{The Quadratic Equation}

\begin{exer}{(Problem 20)}
  \todo[inline]{
  }
\end{exer}

\begin{exer}{(Problem 21)}
  \todo[inline]{
  }
\end{exer}

\begin{exer}{(Problem 22)}
  \todo[inline]{
  }
\end{exer}

\section{Factorization in Integral Domains}

\begin{exer}{(Problem 5)}
  $ $
  \begin{itemize}
    \item
      Let $k$ be a field and let $a \in k$.
      Construct a $k$-algebra isomorphism, $k[x, y] / (x - a) \rightarrow k[y]$.
      Justify your answer.
    \item
      Let $f(x, y) \in k[x, y]$.
      What is the image of $f(x, y)$ under the above isomorphism?
  \end{itemize}
\end{exer}

\begin{proof}
  $ $
  \todo[inline]{
    Ask Professor Schoen after class on Friday.
  }
  \begin{itemize}
    \item
      Let $\phi$ be defined such that $\phi(f(x, y) + (x - a)) = f(a, y)$.
      \begin{itemize}
        \item
          Well-defined?
          Let $f(x, y) + (x - a) = g(x, y) + (x - a)$.
          Then $g(x, y) = f(x, y) + h(x, y)(x - a)$.
          \begin{align*}
            \phi(g(x, y) + (x - a))
              &= \phi((f(x, y) + h(x, y)(x - a)) + (x - a)) \\
              &= f(a, y) + h(a, y)(a - a) \\
              &= f(a, y) \\
              &= \phi(f(x, y)).
          \end{align*}
        \item
          $k$-algebra homomorphism?
          Let $c \in k, f, g \in k[x, y]$ be given.
          \begin{align*}
            \phi(c(f + (x - a))
              &= \phi(cf + (x - a)) \\
              &= cf(a, y) \\
              &= c\phi(f + (x - a)). \\
            \phi((f + g) + (x - a))
              &= (f + g)(a, y) \\
              &= f(a, y) + g(a, y) \\
              &= \phi(f + (x - a)) + \phi(g + (x - a)). \\
            \phi((fg) + (x - a))
              &= (fg)(a, y) \\
              &= f(a, y)g(a, y) \\
              &= \phi(f + (x - a))\phi(g + (x - a)).
          \end{align*}
      \end{itemize}
    \item
      $\phi(f(x, y) + (x - a)) = f(a, y)$.
  \end{itemize}
\end{proof}

\begin{exer}{(Problem 6)}
  $ $
  \begin{itemize}
    \item
      Give an example of a field $k$, an element $a \in k$ and a reducible polynomial $f(x, y) \in k[x, y]$ of degree $n$ in $y$ such that $f(a, y) \in k[y]$ is irreducible and has degree $n$.
    \item
      \todo[inline]{
      }
    \item
      \todo[inline]{
      }
  \end{itemize}
\end{exer}

\begin{proof}
  $ $
  \begin{itemize}
    \item
      Let $k = \mathbb{Q}, a = 1, f(x, y) = xy$.
      Then the degree of $f(x, y)$ in $y$ is 1.
      $f(x, y) = xy \in k[x, y]$ is reducible since $x$ and $y$ are not units in $k[x, y]$.
      However, $f(a, y) = 1y = y$ is irreducible in $k[y]$.
  \end{itemize}
\end{proof}

\end{document}

