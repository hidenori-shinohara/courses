\documentclass[12pt, psamsfonts]{amsart}

%-------Packages---------
\usepackage{amssymb,amsfonts}
\usepackage{tikz-cd}
\usepackage{fullpage}
\usepackage{todonotes}
\usepackage{physics}
\usepackage[all,arc]{xy}
\usepackage{mathtools}
\usepackage{enumerate}
\usepackage{enumitem}
\usepackage{mathrsfs}
\usepackage{theoremref}
\usepackage{graphicx}
\usepackage[bookmarks]{hyperref}

%--------Theorem Environments--------
%theoremstyle{plain} --- default
\newtheorem{thm}{Theorem}[section]
\newtheorem{cor}[thm]{Corollary}
\newtheorem{prop}[thm]{Proposition}
\newtheorem{lem}[thm]{Lemma}
\newtheorem{conj}[thm]{Conjecture}
\newtheorem{quest}[thm]{Question}

\theoremstyle{definition}
\newtheorem{defn}[thm]{Definition}
\newtheorem{defns}[thm]{Definitions}
\newtheorem{con}[thm]{Construction}
\newtheorem{exmp}[thm]{Example}
\newtheorem{exmps}[thm]{Examples}
\newtheorem{notn}[thm]{Notation}
\newtheorem{notns}[thm]{Notations}
\newtheorem{addm}[thm]{Addendum}
\newtheorem*{exer}{Exercise}

\theoremstyle{remark}
\newtheorem{rem}[thm]{Remark}
\newtheorem{rems}[thm]{Remarks}
\newtheorem{warn}[thm]{Warning}
\newtheorem{sch}[thm]{Scholium}

\DeclareMathOperator{\Hom}{Hom}
\DeclareMathOperator{\Id}{Id}
\DeclareMathOperator{\ord}{ord}
\DeclareMathOperator{\cont}{cont}

\makeatletter
\let\c@equation\c@thm
\makeatother
\numberwithin{equation}{section}

\bibliographystyle{plain}

\begin{document}

\title{math 601 (Due 10/9)}
\author{Hidenori Shinohara}
\maketitle
\tableofcontents

\section{Rings of Fractions}

\begin{exer}{(Problem 3)}
  Let $T \subset R$ be the subset consisting of all non zero divisors.
  \begin{itemize}
    \item
      Show that $T$ is a multiplicative set.
    \item
      Let $s \in T$ and let $S = \{ 1, s, s^2, s^3, \cdots \} \subset T$.
      Show that the following rings are isomorphic: $S^{-1}R$, the subring $R[1/s] \subset T^{-1}R$, and the quotient ring $R[x]/(sx - 1)$.
  \end{itemize}
\end{exer}

\begin{proof}
  $ $
  \begin{itemize}
    \item
      \begin{itemize}
        \item
          Let $a, b \in T$.
          Let $c \in R$ be given.
          If $(ab)c = 0$, then $a(bc) = 0$.
          Since $a$ is a non zero divisor, $bc = 0$.
          Since $b$ is a non zero divisor, $c = 0$.
          Since $R$ is a commutative ring throughout this handout, there is no need to check the case that $c(ab) = 0$.
          Thus $ab$ is a non zero divisor, so $T$ is closed under multiplication.
        \item
          $1 \in T$ since $\forall c \in R, c \cdot 1 = 0 \implies c = 0$.
      \end{itemize}

      Therefore, $T$ is indeed a multiplicative set.
    \item
      \todo[inline]{
        I have some idea, but I don't know how to solve this.
        There are a few mapping properties that we've covered:
      }
      \begin{itemize}
        \item
          The universal mapping property of the quotient.
          (Proposition 6 on Commutative Rings)
          Given $\pi: R \rightarrow R / I$ and $f: R \rightarrow S$ with some nice properties, there exists $\overline{f}: R / I \rightarrow S$ such that $\overline{f} \circ \pi = f$.
        \item
          The mapping property of polynomials.
          (Proposition 1 on Commutative Rings)
          Given $\phi_0:R \rightarrow S$ and $s \in S$, there exists $\phi:R[x] \rightarrow S$.
        \item
          The universal property of rings of fractions.
          (Proposition (iv) of the Ring of Fractions.)
          Given $i: R \rightarrow S^{-1}R$ and $h: R \rightarrow T$ with some nice properties, there exists $\lambda: S^{-1}R \rightarrow T$ such that $h = \lambda \circ i$.
      \end{itemize}
      Let $\pi$ be the canonical map from $R[x]$ into $R[x] / (sx - 1)$.
      Let $f: R[x] \rightarrow S^{-1}R$ be the homomorphism associated to the inclusion map $R \rightarrow S^{-1}R$ and the element $1 / s \in S^{-1}R$.
      By the mapping property of polynomials, the existence of $f$ is guaranteed.

      By the universal property of the quotient, universal mapping property of the ring of fractions, there exist homomorphisms $\overline{f}, \overline{\pi}$, respectively, such that the following diagram commutes:
      \begin{center}
        \begin{tikzcd}
          R[x] \arrow[r, "\pi"] \arrow[dr, "f"] & R[x]/(sx - 1) \arrow[d, "\overline{f}"] \\
                                                & S^{-1}R \arrow[u, shift left, "\overline{\pi}"]
        \end{tikzcd}
      \end{center}
      Since $\pi$ and $f$ are both surjective, $\overline{f}$ and $\overline{\pi}$ must be surjective in order for the diagram to commute.
      Then $\overline{f} \circ \overline{\pi} \circ f = \overline{f} \circ \pi = f$.
      Since $f$ is surjective, this implies that $\overline{f} \circ \overline{\pi} = \Id_{S^{-1}R}$.
      Similarly, $\overline{\pi} \circ \overline{f} = \Id_{R[x]/(sx - 1)}$.
      Therefore, $\overline{\pi}$ and $\overline{f}$ are the inverse homomorphism of each other, so they are isomorphisms.
      \todo[inline]{
        What about $R[1/s]$?
      }
  \end{itemize}
\end{proof}

\section{Modules}

\begin{exer}{(Problem 1)}
  For each of the $\mathbb{Z}$-modules listed in the handout, answer the questions in the handout.
\end{exer}

\begin{proof}
$ $
  \begin{enumerate}[label=(\alph*)]
    \item 
      $M = \mathbb{Z}^3 \times \mathbb{Z} / 86\mathbb{Z}$.
      \todo[inline]{
        Solve this problem!
      }
    \item 
      $M = \prod_{n \geq 1} \mathbb{Z} / n\mathbb{Z}$.
      \todo[inline]{
        Solve this problem!
      }
    \item 
      $M = \mathbb{Z}[1/p] \subset \mathbb{Q}$.
      \todo[inline]{
        Solve this problem!
      }
    \item 
      $M = \mathbb{Q} / \mathbb{Z}_{(p)}$.
      \todo[inline]{
        Solve this problem!
      }
  \end{enumerate}
\end{proof}

\section{The Quadratic Equation}

\begin{exer}{(Problem 20)}
  Construct ring isomorphisms $\mathbb{Z}[x]/(x^2 - 2) \rightarrow \mathbb{Z}[\sqrt{2}]$ and $\mathbb{Z}/(p)[x]/(x^2 - 2) \rightarrow \mathbb{Z}[\sqrt{2}]/(p)$.
\end{exer}

\begin{proof}
  Let $i: \mathbb{Z} \rightarrow \mathbb{Z}[\sqrt{2}]$ be the inclusion and $s = \sqrt{2} \in \mathbb{Z}[\sqrt{2}]$.
  By the mapping property of polynomials, there exists a ring homomorphism $\phi: \mathbb{Z}[x] \rightarrow \mathbb{Z}[\sqrt{2}]$ such that $\phi(\sum_{i=0}^{n} r_ix^i) = \sum_{i=0}^{n} i(r_i)s^i$.
  In other words, $\phi$ maps $f(x)$ into $f(\sqrt{2})$.
  For each $a + b\sqrt{2} \in \mathbb{Z}[\sqrt{2}]$, $\phi(a + bx) = a + b\sqrt{2}$, so $\phi$ is surjective.
  We claim that $\ker(\phi) = (x^2 - 2)$.
  \begin{itemize}
    \item
      Since $\sqrt{2}^2 - 2 = 2 - 2 = 0$, $x^2 - 2 \in \ker(\phi)$.
      Moreover, $(x^2 - 2) \subset \ker(\phi)$.
    \item
      Let $f(x) \in \ker(\phi)$.
      Since $\mathbb{Z}[x]$ is a Euclidean domain, $f(x) = q(x)(x^2 - 2) + ax + b$ for some $q(x) \in \mathbb{Z}[x], a, b \in \mathbb{Z}$.
      Since $ax + b = f(x) - q(x)(x^2 - 2)$, $a\sqrt{2} + b = 0$.
      Since $a, b$ are integers, $a = b = 0$.
      This implies $f(x) \in (x^2 - 2)$.
  \end{itemize}
  Therefore, $\ker(\phi) = (x^2 - 2)$.
  By the first isomorphism theorem(Theorem 16 on P.97, Dummit and Foote), $\tilde{\phi} : \mathbb{Z}[x]/(x^2 - 2) \rightarrow \mathbb{Z}[\sqrt{2}]$ induced by $\phi$ is an isomorphism.

  We will solve the second part using the same approach.
  We will assume that $p$ is a prime.
  Consider the inclusion $\mathbb{Z}/(p) \xhookrightarrow{} \mathbb{Z}[\sqrt{2}]/(p)$ and the element $\sqrt{2} + (p) \in \mathbb{Z}[\sqrt{2}]/(p)$.
  Let $\Phi: \mathbb{Z}/(p)[x] \rightarrow \mathbb{Z}[\sqrt{2}]/(p)$ be a ring homomorphism associated to the inclusion and element.
  We will examine how $\Phi$ behaves.
  \begin{align*}
    \Phi(\sum_{i=0}^{n}(a_i + (p))x^i)
      &= \sum_{i=0}^{n}(a_i + (p))(\sqrt{2} + (p))^i \\
      &= \sum_{i=0}^{n}(a_i + (p))(\sqrt{2}^i + (p)) \\
      &= \sum_{i=0}^{n}(a_i 2^{i/2} + (p)).
  \end{align*}
  For any $a + b\sqrt{2} + (p) \in \mathbb{Z}[\sqrt{2}]/(p)$, $\Phi(a + bx) = a + b\sqrt{2} + (p)$, so $\Phi$ is surjective.
  We claim that $\ker(\Phi) = (x^2 - 2)$.
  Here, by $x^2 - 2$, we mean $(1 + (p))x^2 - (2 + (p))$.
  \begin{itemize}
    \item
      Since $\sqrt{2}^2 - 2 = 0$, $(x^2 - 2) \in \ker(\Phi)$.
    \item
      Let $f(x) \in \ker(\Phi) \subset \mathbb{Z}/(p)[x]$.
      Since $p$ is a prime, $\mathbb{Z}/(p)$ is a field.
      Thus $\mathbb{Z}/(p)[x]$ is a Euclidean domain.
      Choose $q(x) \in \mathbb{Z}/(p)[x]$ and $a + (p), b + (p) \in \mathbb{Z}/(p)$ such that $f(x) = (x^2 - 2)q(x) + ((a + (p))x + (b + (p))$.
      Then $0 = \Phi(f(x)) = \Phi(x^2 - 2)\Phi(q(x)) + \Phi((a + (p))x + (b + (p))) = 0 + \Phi((a + (p))x + (b + (p)))$.
      Thus $\Phi((a + (p))x + (b + (p))) = (a + (p))(\sqrt{2} + (p)) + (b + (p)) = (a\sqrt{2} + b) + (p)$.
      Therefore, $a\sqrt{2} + b \in (p)$.
      Since $a, b \in \mathbb{Z}$, this is possible only if $a = 0$ and $b \in (p)$.
      In other words, this is possible only if $a + (p) = b + (p) = 0$.
      Therefore, $f(x) = (x^2 - 2)q(x) \in (x^2 - 2)$.
  \end{itemize}
  Therefore, $\ker(\Phi) = (x^2 - 2)$, so the homomorphism $\tilde{\Phi}$ induced by $\Phi$ is an isomorphism from $\mathbb{Z}/(p)[x]/(x^2 - 2) \rightarrow \mathbb{Z}[\sqrt{2}]/(p)$ by the first isomorphism theorem.
\end{proof}

\begin{exer}{(Problem 21)}
  Let $p \in \mathbb{Z}$ be an odd prime.
  Show that $\mathbb{Z}[\sqrt{2}]/(p)$ is an integral domain if and only if $(x^2 - 2)$ is an irreducible element of $\mathbb{Z}/(p)[x]$.
  Show that this occurs if and only if 2 is not a square in $\mathbb{Z}/(p)$.
\end{exer}

\begin{exer}{(Problem 22)}
  By Problem 20, $\mathbb{Z}[\sqrt{2}]/(p)$ is isomorphic to $\mathbb{Z}/(p)[x]/(x^2 - 2)$.
  Thus it suffices to show that $\mathbb{Z}/(p)[x]/(x^2 - 2)$ is an integral domain if and only if $x^2 - 2$ is an irreducible element of $\mathbb{Z}/(p)[x]$.
  By Corollary 4 on P.300 (Dummit and Foote), since $\mathbb{Z}/(p)$ is a field, $\mathbb{Z}/(p)[x]$ is a UFD.
  By Proposition 12 on P.286, a nonzero element generates a prime ideal if and only if it is irreducible.
  By Proposition 13 on P.255, $(x^2 - 2)$ is a prime ideal if and only if $\mathbb{Z}/(p)[x]/(x^2 - 2)$ is an integral domain.
  Therefore, $\mathbb{Z}/(p)[x]/(x^2 - 2)$ is an integral domain if and only if $x^2 - 2$ is an irreducible element.

  \todo[inline]{
    Show the second part.
  }
\end{exer}

\section{Factorization in Integral Domains}

\begin{exer}{(Problem 5)}
  $ $
  \begin{itemize}
    \item
      Let $k$ be a field and let $a \in k$.
      Construct a $k$-algebra isomorphism, $k[x, y] / (x - a) \rightarrow k[y]$.
      Justify your answer.
    \item
      Let $f(x, y) \in k[x, y]$.
      What is the image of $f(x, y)$ under the above isomorphism?
  \end{itemize}
\end{exer}

\begin{proof}
  $ $
  \begin{itemize}
    \item
      Let $\phi$ be defined such that $\phi(f(x, y) + (x - a)) = f(a, y)$.
      \begin{itemize}
        \item
          Well-defined?
          Let $f(x, y) + (x - a) = g(x, y) + (x - a)$.
          Then $g(x, y) = f(x, y) + h(x, y)(x - a)$.
          \begin{align*}
            \phi(g(x, y) + (x - a))
              &= \phi((f(x, y) + h(x, y)(x - a)) + (x - a)) \\
              &= f(a, y) + h(a, y)(a - a) \\
              &= f(a, y) \\
              &= \phi(f(x, y)).
          \end{align*}
        \item
          $k$-algebra homomorphism?
          Let $c \in k, f, g \in k[x, y]$ be given.
          \begin{align*}
            \phi(c(f + (x - a))
              &= \phi(cf + (x - a)) \\
              &= cf(a, y) \\
              &= c\phi(f + (x - a)). \\
            \phi((f + g) + (x - a))
              &= (f + g)(a, y) \\
              &= f(a, y) + g(a, y) \\
              &= \phi(f + (x - a)) + \phi(g + (x - a)). \\
            \phi((fg) + (x - a))
              &= (fg)(a, y) \\
              &= f(a, y)g(a, y) \\
              &= \phi(f + (x - a))\phi(g + (x - a)).
          \end{align*}
      \end{itemize}
    \item
      $\phi(f(x, y) + (x - a)) = f(a, y)$.
  \end{itemize}
\end{proof}

\begin{exer}{(Problem 6)}
  $ $
  \begin{itemize}
    \item
      Give an example of a field $k$, an element $a \in k$ and a reducible polynomial $f(x, y) \in k[x, y]$ of degree $n$ in $y$ such that $f(a, y) \in k[y]$ is irreducible and has degree $n$.
    \item
      Suppose given a polynomial $f \in k[x, y]$ which when viewed as an element of $k(x)[y]$ has degree $n$ (in $y$) and content 1.
      Suppose there is some $a \in k$ such that $f(a, y) \in k[y]$ is irreducible and has degree $n$.
      Show that $f(x, y) \in k[x, y]$ is irreducible.
    \item
      Give an example of a field $k$, an element, $a \in k$, and a reducible polynomial $f(x, y) \in k[x, y]$, which when viewed as an element of $k(x)[y]$ has degree $n$ and content 1 such that $f(a, y) \in k[y]$ is irreducible.
  \end{itemize}
\end{exer}

\begin{proof}
  $ $
  \begin{itemize}
    \item
      Let $k = \mathbb{Q}, a = 1, f(x, y) = xy$.
      Then the degree of $f(x, y)$ in $y$ is 1.
      $f(x, y) = xy \in k[x, y]$ is reducible since $x$ and $y$ are not units in $k[x, y]$.
      However, $f(a, y) = 1y = y$ is irreducible in $k[y]$.
    \item
      Choose $f_1, \cdots, f_n \in k[x]$ such that $f(x, y) = f_n(x)y^n + \cdots + f_1(x)y^1 + f_0(x)$.
      Then $f(a, y) = f_n(a)y^n + \cdots + f_1(a)y^1 + f_0(a)$.
      Let $h_1(x, y), h_2(x, y) \in k[x]$ be given such that $f(x, y) = h_1(x, y)h_2(x, y)$.
      Then $f(a, y) = h_1(a, y)h_2(a, y)$.
      Then $h_1(a, y)$ or $h_2(a, y)$ is a unit in $k[y]$ since $f(a, y)$ is irreducible in $k[y]$.
      Without loss of generality, we will assume $h_1(a, y)$ is a unit in $k[y]$.

      It is given that $\deg_y(f(a, y))$, the degree of $f(a, y)$ in $y$, is $n$.
      Thus $\deg_y(h_1(a, y)) + \deg_y(h_2(a, y)) = n$.
      Since $\deg_y(h_1(a, y)) = 0$, $\deg_y(h_2(a, y)) = n$.
      Therefore, $\deg_y(h_2(x, y)) \geq n$.

      On the other hand, $\deg_y(f(x, y)) = \deg_y(h_1(x, y)) + \deg_y(h_2(x, y))$, so $\deg_y(h_2(x, y)) \leq n$.
      Thus $\deg_y(h_2(x, y)) = n$.
      Let $g_1(x), \cdots, g_n(x) \in k[x]$ such that $h_2(x, y) = g_n(x)y^n + \cdots + g_1(x)y^1 + g_0(x)$.
      Then $f(x, y) = h_1(x, y)h_2(x, y) = (h_1(x, y)g_n(x))y^n + \cdots + (h_1(x, y)g_1(x))y^1 + h_1(x, y)g_0(x)$.

      Since $\deg_y(h_2(x, y)) = n$, $\deg_y(h_1(x, y)) = 0$.
      Thus, $h_1(x, y) \in k[x]$, so $h_1(x, y)g_i(x) \in k[x]$ for each $i$.
      Therefore, $h_1(x, y)g_i(x) = f_i(x)$ for each $i$.

      Let $p \in k[x]$ be an irreducible.
      If $p \mid h_1(x, y)$, then $p \mid f_i(x) = h_1(x, y)g_i(x)$ for each $i$, so $\ord_p(f_i) \geq 1$ for each $i$.
      Therefore, $o_p(f(x, y)) \geq 1$, and thus $p \mid \cont(f(x, y))$.
      However, since $\cont(f(x, y)) = 1$, $p \nmid h_1(x, y)$.
      Thus $h_1(x, y)$ is a unit in $k[x]$ since it cannot be divided by any irreducibles.
      Since $h_1(x, y)$ is a unit in $k[x]$ and $k[y]$, it must consist only of a constant term, which is a unit in $k$.
      Hence, $h_1(x, y)$ is a unit in $k[x, y]$.
      
      We have shown that for any $h_1(x, y), h_2(x, y) \in k[x, y]$, $h_1h_2 = f$ implies one of $h_1$ or $h_2$ is a unit.
      Therefore, $f(x, y)$ is an irreducible in $k[x, y]$.

    \item
      Let $k = \mathbb{Q}, a = 1, f(x, y) = (x - 1)y^2 + y$.
      Then $f(x, y)$, which when viewed as an element of $k(x)[y]$ has degree 1.
      \begin{itemize}
        \item
          The coefficient of $y$ is 1, and $\ord_p(1) = 0$ for any $p$ because $1 \in k[x]^*$.
        \item
          The coefficient of $y^2$, when $f(x, y)$ is viewed as an element of $k(x)[y]$ is $x - 1$.
          Thus for any irreducible element $p \in k[x]$, $\ord_p(x - 1) \geq 0$.
      \end{itemize}
      Therefore, $o_p(f(x)) = 0$ for any irreducible element $p \in k[x]$.
      Thus $\cont(f(x, y)) = 1$.
      
      $f(a, y) = y \in k[y]$.
      This is irreducible because if $f_1f_2 = y$ for some $f_1, f_2 \in k[y]$, then $\deg(f_1) + \deg(f_2) = 1$ implies that one of $f_1$ or $f_2$ is a unit in $k$.
  \end{itemize}
\end{proof}

\end{document}

