\documentclass[12pt, psamsfonts]{amsart}

%-------Packages---------
\usepackage{amssymb,amsfonts}
\usepackage[all,arc]{xy}
\usepackage{enumerate}
\usepackage{mathrsfs}
\usepackage{theoremref}
\usepackage{graphicx}
\usepackage[bookmarks]{hyperref}

%--------Theorem Environments--------
%theoremstyle{plain} --- default
\newtheorem{thm}{Theorem}[section]
\newtheorem{cor}[thm]{Corollary}
\newtheorem{prop}[thm]{Proposition}
\newtheorem{lem}[thm]{Lemma}
\newtheorem{conj}[thm]{Conjecture}
\newtheorem{quest}[thm]{Question}

\theoremstyle{definition}
\newtheorem{defn}[thm]{Definition}
\newtheorem{defns}[thm]{Definitions}
\newtheorem{con}[thm]{Construction}
\newtheorem{exmp}[thm]{Example}
\newtheorem{exmps}[thm]{Examples}
\newtheorem{notn}[thm]{Notation}
\newtheorem{notns}[thm]{Notations}
\newtheorem{addm}[thm]{Addendum}
\newtheorem*{exer}{Exercise}

\theoremstyle{remark}
\newtheorem{rem}[thm]{Remark}
\newtheorem{rems}[thm]{Remarks}
\newtheorem{warn}[thm]{Warning}
\newtheorem{sch}[thm]{Scholium}

\DeclareMathOperator{\Hom}{Hom}
\DeclareMathOperator{\Id}{Id}

\makeatletter
\let\c@equation\c@thm
\makeatother
\numberwithin{equation}{section}

\bibliographystyle{plain}

\begin{document}

\title{Math 601 Homework (Due 9/4)}
\author{Hidenori Shinohara}
\maketitle

\begin{exer}{(2.1)}
  Show that the function $g: \mathbb{R} \rightarrow S^1$, $g(r) = \exp(2\pi ir)$, where $i^2 = -1$, satisfies the property that $g(r) = g(r')$ if and only if $r \sim r'$.
  Use this to explicitly construct a bijective map from the orbit space of the action to $S^1$, $\mathrm{g}: \mathbb{R} / \sim= \mathbb{Z} \backslash \mathbb{R} \rightarrow S^1$.
\end{exer}

\begin{proof}
$ $
  \begin{itemize}
    \item
      Let $r, r' \in \mathbb{R}$ such that $r \sim r'$.
      Let $k \in \mathbb{Z}$ such that $k * r' = r$.
      Therefore, $k + r' = r$.

      \begin{align*}
        g(r)
          &= \exp(2\pi i r) \\
          &= \exp(2\pi i (k + r')) \\
          &= \exp(2\pi i k + 2\pi i r') \\
          &= \exp(2\pi i k)\exp(2\pi i r') \\
          &= \exp(2\pi i r') \\
          &= g(r').
      \end{align*}
    \item
      Let $r, r' \in \mathbb{R}$ such that $g(r) = g(r')$.
      \begin{align*}
        \exp(2\pi ir) = \exp(2 \pi ir')
          &\implies \exp(2\pi i(r - r')) = 1 \\
          &\implies \cos(2\pi (r - r')) + i\sin (2\pi (r - r')) = 1 \\
          &\implies \sin (2\pi (r - r')) = 0 \\
          &\implies r - r' \in \mathbb{Z} \\
          &\implies \exists k \in \mathbb{Z},  r = k * r' \\
          &\implies r \sim r'.
      \end{align*}
  \end{itemize}
  Let $\mathrm{g}: \mathbb{Z} \backslash \mathbb{R} \rightarrow S^1$ be defined such that $\mathrm{g}([r]) = g(r)$ for each $[r] \in \mathbb{Z} \backslash \mathbb{R}$.
  \begin{itemize}
    \item
      Well-defined?
      Let $[r] = [r'] \in \mathbb{Z} \backslash \mathbb{R}$.
      Then $r \sim r'$.
      We showed that $g(r) = g(r')$ if $r \sim r'$ earlier.
      Therefore, $\mathrm{g}$ is indeed well-defined.
    \item
      Injective?
      Let $[r], [r'] \in \mathrm{Z} \backslash \mathbb{R}$.
      Suppose $\mathrm{g}([r]) = \mathrm{g}([r'])$.
      Then $g(r) = g(r')$.
      We showed earlier that this implies $r \sim r'$.
      In other words, $[r] = [r']$.
      Therefore, $\mathrm{g}$ is injective.
    \item
      Surjective?
      Let $z \in S^1$.
      Express $z$ as $re^{i\theta}$ where $r, \theta \in \mathbb{R}$.
      Since $\lvert z \rvert = 1$, we can assume that $r = 1$ without loss of generality.  (If $r = -1$, then $e^{i\pi} = -1$, so $\theta$ can be redefined as $\theta + \pi$.)

      Then $[\theta / 2\pi]$ is an element in $\mathbb{Z} \backslash \mathbb{R}$, and $\mathrm{g}([\theta / 2\pi]) = g(\theta / 2\pi) = \exp(2\pi i \cdot \theta / 2\pi) = \exp(i\theta) = z$.
      Therefore, $\mathrm{g}$ is indeed surjective.
  \end{itemize}
\end{proof}

\begin{exer}{(2.2)}
  Let $*: G \times S \rightarrow S$ be a left action of $G$.
  Show that $s \star g = g^{-1} * s$ defines a right action of $G$ on $S$.
\end{exer}

\begin{proof}
  Let $s \in S, g, h \in G$ be given.
  \begin{align*}
    (s \star g) \star h
      &= h^{-1} * (s \star g) \\
      &= h^{-1} * (g^{-1} * s) \\
      &= (h^{-1}g^{-1}) * s \\
      &= (gh)^{-1} * s \\
      &= s \star (gh).
  \end{align*}

  Let $e \in G$ denote the identity element and let $s \in S$ be given.
  \begin{align*}
    s \star e
      &= e^{-1} * s \\
      &= e * s \\
      &= s.
  \end{align*}

  Therefore, $\star$ is indeed a right action of $G$ on $S$.
\end{proof}

\begin{exer}{(2.3)}
$ $
 \begin{enumerate}
   \item
     Let $h, h' \in G$ lie in the same conjugacy class.
     Show that $h$ and $h'$ have the same order.
   \item
    Give an example of a group and two elements of the same order which do not line in the same conjugacy class.
 \end{enumerate}
\end{exer}

\begin{proof}
  \begin{enumerate}
    \item
      Since $h$ and $h'$ lie in the same conjugacy class, there must exist an element $g \in G$ such that $h = g * h'$.
      In other words, $h = g \cdot h' \cdot g^{-1}$.
      We will show that $h^n = g \cdot (h')^n \cdot g^{-1}$ for all $n \in \mathbb{N}$ using mathematical induction.
      \begin{itemize}
        \item
          When $n = 1$, the statement is true.
        \item
          Suppose $h^n = g \cdot (h')^n \cdot g^{-1}$ for some $n \in \mathbb{N}$.
          \begin{align*}
            h^{n + 1}
              &= h^n \cdot h \\
              &= (g \cdot (h')^n \cdot g^{-1}) \cdot (g \cdot h' \cdot g^{-1}) \\
              &= g \cdot (h')^n \cdot (g^{-1} \cdot g) \cdot h' \cdot g^{-1} \\
              &= g \cdot (h')^n \cdot h' \cdot g^{-1} \\
              &= g \cdot (h')^{n + 1} \cdot g^{-1}.
          \end{align*}
      \end{itemize}
      Therefore, $h^n = g \cdot (h')^n \cdot g^{-1}$ for all $n \in \mathbb{N}$.

      For any $n \in \mathbb{N}$, if $h^n = e$, then $g \cdot (h')^n \cdot g^{-1} = e$, so $(h')^n = g^{-1}g = e$.
      For any $n \in \mathbb{N}$, If $(h')^n = e$, then $h^n = geg^{-1} = e$.
      Therefore, $\forall n \in \mathbb{N}, h^n = e \iff (h')^n = e$.

      This implies that if the order of one of $h$ or $h'$ is infinite, the other has to be infinite as well.
      On the other hand, if the order of one of $h$ or $h'$ is finite, the other has to be finite as well.
      Suppose that the orders of $h$ and $h'$ are finite and let $n$ denote the order of $h$.
      Then $h^n = e$ and $h^m \ne e$ for each natural number $m < n$.
      Then $(h')^n = e$ and $(h')^m \ne e$ for each natural number $m < n$.
      Therefore, the order of $h'$ is $n$ as well.

      We showed that, regardless of whether the order is finite, $h$ and $h'$ have the same order.
    \item
      We will consider the Klein 4-group $K = \\{ e, a, b, c \\}$.
      Since $a^2 = b^2 = e$, $a$ and $b$ have the order $2$.
      Suppose that $a$ and $b$ lie in the same conjugacy class.
      Then there must exist a $g \in K$ such that $a = gbg^{-1}$.
      Since $K$ is abelian, $a = gbg^{-1} = gg^{-1}b = eb = b$.
      This is a contradiction, so there $a$ and $b$ do not lie in the same conjugacy class.
      Thus we found two elements of the same order which do not lie in the same conjugacy class.
  \end{enumerate}
\end{proof}

\end{document}


