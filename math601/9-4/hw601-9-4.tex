\documentclass[12pt, psamsfonts]{amsart}

%-------Packages---------
\usepackage{amssymb,amsfonts}
\usepackage[all,arc]{xy}
\usepackage{enumerate}
\usepackage{mathrsfs}
\usepackage{theoremref}
\usepackage{graphicx}
\usepackage[bookmarks]{hyperref}

%--------Theorem Environments--------
%theoremstyle{plain} --- default
\newtheorem{thm}{Theorem}[section]
\newtheorem{cor}[thm]{Corollary}
\newtheorem{prop}[thm]{Proposition}
\newtheorem{lem}[thm]{Lemma}
\newtheorem{conj}[thm]{Conjecture}
\newtheorem{quest}[thm]{Question}

\theoremstyle{definition}
\newtheorem{defn}[thm]{Definition}
\newtheorem{defns}[thm]{Definitions}
\newtheorem{con}[thm]{Construction}
\newtheorem{exmp}[thm]{Example}
\newtheorem{exmps}[thm]{Examples}
\newtheorem{notn}[thm]{Notation}
\newtheorem{notns}[thm]{Notations}
\newtheorem{addm}[thm]{Addendum}
\newtheorem*{exer}{Exercise}

\theoremstyle{remark}
\newtheorem{rem}[thm]{Remark}
\newtheorem{rems}[thm]{Remarks}
\newtheorem{warn}[thm]{Warning}
\newtheorem{sch}[thm]{Scholium}

\DeclareMathOperator{\Hom}{Hom}
\DeclareMathOperator{\Id}{Id}

\makeatletter
\let\c@equation\c@thm
\makeatother
\numberwithin{equation}{section}

\bibliographystyle{plain}

\begin{document}

\title{Math 601 Homework (Due 9/4)}
\author{Hidenori Shinohara}
\maketitle

\begin{exer}{(2.1)}
  Show that the function $g: \mathbb{R} \rightarrow S^1$, $g(r) = \exp(2\pi ir)$, where $i^2 = -1$, satisfies the property that $g(r) = g(r')$ if and only if $r \sim r'$.
  Use this to explicitly construct a bijective map from the orbit space of the action to $S^1$, $g: \mathbb{R} / \sim= \mathbb{Z} \ \mathbb{R} \rightarrow S^1$.
\end{exer}

\begin{proof}
$ $
  \begin{itemize}
    \item
      Let $r, r' \in \mathbb{R}$ such that $r \sim r'$.
      Let $k \in \mathbb{Z}$ such that $k * r' = r$.
      Therefore, $k + r' = r$.

      \begin{align*}
        g(r)
          &= \exp(2\pi i r) \\
          &= \exp(2\pi i (k + r')) \\
          &= \exp(2\pi i k + 2\pi i r') \\
          &= \exp(2\pi i k)\exp(2\pi i r') \\
          &= \exp(2\pi i r') \\
          &= g(r').
      \end{align*}
    \item
      Let $r, r' \in \mathbb{R}$ such that $g(r) = g(r')$.
      \begin{align*}
        \exp(2\pi ir) = \exp(2 \pi ir')
          &\implies \exp(2\pi i(r - r')) = 1 \\
          &\implies \cos(2\pi (r - r')) + i\sin (2\pi (r - r')) = 1 \\
          &\implies \sin (2\pi (r - r')) = 0 \\
          &\implies r - r' \in \mathbb{Z} \\
          &\implies \exists k \in \mathbb{Z},  r = k * r' \\
          &\implies r \sim r'.
      \end{align*}
  \end{itemize}
  TODO
\end{proof}

\begin{exer}{(2.2)}
  Let $*: G \times S \rightarrow S$ be a left action of $G$.
  Show that $s \star g = g^{-1} * s$ defines a right action of $G$ on $S$.
\end{exer}

\begin{proof}
  Let $s \in S, g, h \in G$ be given.
  \begin{align*}
    (s \star g) \star h
      &= h^{-1} * (s \star g) \\
      &= h^{-1} * (g^{-1} * s) \\
      &= (h^{-1}g^{-1}) * s \\
      &= (gh)^{-1} * s \\
      &= s \star (gh).
  \end{align*}

  Let $e \in G$ denote the identity element and let $s \in S$ be given.
  \begin{align*}
    s \star e
      &= e^{-1} * s \\
      &= e * s \\
      &= s.
  \end{align*}

  Therefore, $\star$ is indeed a right action of $G$ on $S$.
\end{proof}

\end{document}


