\documentclass[12pt, psamsfonts]{amsart}

%-------Packages---------
\usepackage{amssymb,amsfonts}
\usepackage[all,arc]{xy}
\usepackage{enumerate}
\usepackage{mathrsfs}
\usepackage{theoremref}
\usepackage{graphicx}
\usepackage[bookmarks]{hyperref}

%--------Theorem Environments--------
%theoremstyle{plain} --- default
\newtheorem{thm}{Theorem}[section]
\newtheorem{cor}[thm]{Corollary}
\newtheorem{prop}[thm]{Proposition}
\newtheorem{lem}[thm]{Lemma}
\newtheorem{conj}[thm]{Conjecture}
\newtheorem{quest}[thm]{Question}

\theoremstyle{definition}
\newtheorem{defn}[thm]{Definition}
\newtheorem{defns}[thm]{Definitions}
\newtheorem{con}[thm]{Construction}
\newtheorem{exmp}[thm]{Example}
\newtheorem{exmps}[thm]{Examples}
\newtheorem{notn}[thm]{Notation}
\newtheorem{notns}[thm]{Notations}
\newtheorem{addm}[thm]{Addendum}
\newtheorem*{exer}{Exercise}

\theoremstyle{remark}
\newtheorem{rem}[thm]{Remark}
\newtheorem{rems}[thm]{Remarks}
\newtheorem{warn}[thm]{Warning}
\newtheorem{sch}[thm]{Scholium}

\DeclareMathOperator{\Hom}{Hom}
\DeclareMathOperator{\Id}{Id}

\makeatletter
\let\c@equation\c@thm
\makeatother
\numberwithin{equation}{section}

\bibliographystyle{plain}

\begin{document}

\title{Math 601 Homework (Due 9/4)}
\author{Hidenori Shinohara}
\maketitle

\begin{exer}{(2.1)}
  Show that the function $g: \mathbb{R} \rightarrow S^1$, $g(r) = \exp(2\pi ir)$, where $i^2 = -1$, satisfies the property that $g(r) = g(r')$ if and only if $r \sim r'$.
  Use this to explicitly construct a bijective map from the orbit space of the action to $S^1$, $\mathrm{g}: \mathbb{R} / \sim= \mathbb{Z} \backslash \mathbb{R} \rightarrow S^1$.
\end{exer}

\begin{proof}
$ $
  \begin{itemize}
    \item
      Let $r, r' \in \mathbb{R}$ such that $r \sim r'$.
      Let $k \in \mathbb{Z}$ such that $k * r' = r$.
      Therefore, $k + r' = r$.

      \begin{align*}
        g(r)
          &= \exp(2\pi i r) \\
          &= \exp(2\pi i (k + r')) \\
          &= \exp(2\pi i k + 2\pi i r') \\
          &= \exp(2\pi i k)\exp(2\pi i r') \\
          &= \exp(2\pi i r') \\
          &= g(r').
      \end{align*}
    \item
      Let $r, r' \in \mathbb{R}$ such that $g(r) = g(r')$.
      \begin{align*}
        \exp(2\pi ir) = \exp(2 \pi ir')
          &\implies \exp(2\pi i(r - r')) = 1 \\
          &\implies \cos(2\pi (r - r')) + i\sin (2\pi (r - r')) = 1 \\
          &\implies \sin (2\pi (r - r')) = 0 \\
          &\implies r - r' \in \mathbb{Z} \\
          &\implies \exists k \in \mathbb{Z},  r = k * r' \\
          &\implies r \sim r'.
      \end{align*}
  \end{itemize}
  Let $\mathrm{g}: \mathbb{Z} \backslash \mathbb{R} \rightarrow S^1$ be defined such that $\mathrm{g}([r]) = g(r)$ for each $[r] \in \mathbb{Z} \backslash \mathbb{R}$.
  \begin{itemize}
    \item
      Well-defined?
      Let $[r] = [r'] \in \mathbb{Z} \backslash \mathbb{R}$.
      Then $r \sim r'$.
      We showed that $g(r) = g(r')$ if $r \sim r'$ earlier.
      Therefore, $\mathrm{g}$ is indeed well-defined.
    \item
      Injective?
      Let $[r], [r'] \in \mathrm{Z} \backslash \mathbb{R}$.
      Suppose $\mathrm{g}([r]) = \mathrm{g}([r'])$.
      Then $g(r) = g(r')$.
      We showed earlier that this implies $r \sim r'$.
      In other words, $[r] = [r']$.
      Therefore, $\mathrm{g}$ is injective.
    \item
      Surjective?
      Let $z \in S^1$.
      Express $z$ as $re^{i\theta}$ where $r, \theta \in \mathbb{R}$.
      Since $\lvert z \rvert = 1$, we can assume that $r = 1$ without loss of generality.  (If $r = -1$, then $e^{i\pi} = -1$, so $\theta$ can be redefined as $\theta + \pi$.)

      Then $[\theta / 2\pi]$ is an element in $\mathbb{Z} \backslash \mathbb{R}$, and $\mathrm{g}([\theta / 2\pi]) = g(\theta / 2\pi) = \exp(2\pi i \cdot \theta / 2\pi) = \exp(i\theta) = z$.
      Therefore, $\mathrm{g}$ is indeed surjective.
  \end{itemize}
\end{proof}

\begin{exer}{(2.2)}
  Let $*: G \times S \rightarrow S$ be a left action of $G$.
  Show that $s \star g = g^{-1} * s$ defines a right action of $G$ on $S$.
\end{exer}

\begin{proof}
  Let $s \in S, g, h \in G$ be given.
  \begin{align*}
    (s \star g) \star h
      &= h^{-1} * (s \star g) \\
      &= h^{-1} * (g^{-1} * s) \\
      &= (h^{-1}g^{-1}) * s \\
      &= (gh)^{-1} * s \\
      &= s \star (gh).
  \end{align*}

  Let $e \in G$ denote the identity element and let $s \in S$ be given.
  \begin{align*}
    s \star e
      &= e^{-1} * s \\
      &= e * s \\
      &= s.
  \end{align*}

  Therefore, $\star$ is indeed a right action of $G$ on $S$.
\end{proof}

\begin{exer}{(2.3)}
$ $
 \begin{enumerate}
   \item
     Let $h, h' \in G$ lie in the same conjugacy class.
     Show that $h$ and $h'$ have the same order.
   \item
    Give an example of a group and two elements of the same order which do not line in the same conjugacy class.
 \end{enumerate}
\end{exer}

\begin{proof}
  \begin{enumerate}
    \item
      Since $h$ and $h'$ lie in the same conjugacy class, there must exist an element $g \in G$ such that $h = g * h'$.
      In other words, $h = g \cdot h' \cdot g^{-1}$.
      We will show that $h^n = g \cdot (h')^n \cdot g^{-1}$ for all $n \in \mathbb{N}$ using mathematical induction.
      \begin{itemize}
        \item
          When $n = 1$, the statement is true.
        \item
          Suppose $h^n = g \cdot (h')^n \cdot g^{-1}$ for some $n \in \mathbb{N}$.
          \begin{align*}
            h^{n + 1}
              &= h^n \cdot h \\
              &= (g \cdot (h')^n \cdot g^{-1}) \cdot (g \cdot h' \cdot g^{-1}) \\
              &= g \cdot (h')^n \cdot (g^{-1} \cdot g) \cdot h' \cdot g^{-1} \\
              &= g \cdot (h')^n \cdot h' \cdot g^{-1} \\
              &= g \cdot (h')^{n + 1} \cdot g^{-1}.
          \end{align*}
      \end{itemize}
      Therefore, $h^n = g \cdot (h')^n \cdot g^{-1}$ for all $n \in \mathbb{N}$.

      For any $n \in \mathbb{N}$, if $h^n = e$, then $g \cdot (h')^n \cdot g^{-1} = e$, so $(h')^n = g^{-1}g = e$.
      For any $n \in \mathbb{N}$, If $(h')^n = e$, then $h^n = geg^{-1} = e$.
      Therefore, $\forall n \in \mathbb{N}, h^n = e \iff (h')^n = e$.

      This implies that if the order of one of $h$ or $h'$ is infinite, the other has to be infinite as well.
      On the other hand, if the order of one of $h$ or $h'$ is finite, the other has to be finite as well.
      Suppose that the orders of $h$ and $h'$ are finite and let $n$ denote the order of $h$.
      Then $h^n = e$ and $h^m \ne e$ for each natural number $m < n$.
      Then $(h')^n = e$ and $(h')^m \ne e$ for each natural number $m < n$.
      Therefore, the order of $h'$ is $n$ as well.

      We showed that, regardless of whether the order is finite, $h$ and $h'$ have the same order.
    \item
      We will consider the Klein 4-group $K = \{ e, a, b, c \}$.
      Since $a^2 = b^2 = e$, $a$ and $b$ have the order $2$.
      Suppose that $a$ and $b$ lie in the same conjugacy class.
      Then there must exist a $g \in K$ such that $a = gbg^{-1}$.
      Since $K$ is abelian, $a = gbg^{-1} = gg^{-1}b = eb = b$.
      This is a contradiction, so there $a$ and $b$ do not lie in the same conjugacy class.
      Thus we found two elements of the same order which do not lie in the same conjugacy class.
  \end{enumerate}
\end{proof}

\begin{exer}{(2.4)}
  Construct a bijection between $\mathbb{P}^n_k$ and the set of all one-dimensional subspaces of the vector space, $k^{n + 1}$.
\end{exer}

\begin{proof}
  Let $F$ be the mapping from $\mathbb{P}^n_k$ to the set of all one-dimensional subspaces of $k^{n + 1}$ defined by $F(x_0 : \cdots : x_n) = \{ (tx_0, \cdots, tx_n) \mid t \in k \}$.
  We claim that this is a bijection.
  \begin{itemize}
    \item
      Well-defined?
      Let $(x_0 : \cdots : x_n) = (y_0 : \cdots : y_n) \in \mathbb{P}^n_k$ be given.
      Then there must exist a $t \in k^{\times}$ such that $(x_0, \cdots, x_n) = (ty_0, \cdots, ty_n)$.
      \begin{itemize}
        \item
          For any $(sx_0, \cdots, sx_n) \in F(x_0 : \cdots : x_n)$, $(sx_0, \cdots, sx_n) = (sty_0, \cdots, sty_n) \in F(y_0 : \cdots : y_n)$.
        \item
          For any $(sy_0, \cdots, sy_n) \in F(y_0 : \cdots : y_n)$, $(sy_0, \cdots, sy_n) = ((s / t)ty_0, \cdots, (s / t)ty_n) = ((s / t)x_0, \cdots, (s / t)x_n) \in F(x_0 : \cdots : x_n)$.
      \end{itemize}
      Therefore, $F(x_0 : \cdots : x_n) = F(y_0 : \cdots : y_n)$.
    \item
      Injective?
      Let $(x_0 : \cdots : x_n), (y_0 : \cdots : y_n) \in \mathbb{P}^n_k$ be given.
      Then $(x_0, \cdots, x_n) \ne (0, \cdots, 0)$ and $(y_0, \cdots, y_n) \ne (0, \cdots, 0)$.
      Suppose that $F(x_0 : \cdots : x_n) = F(y_0 : \cdots : y_n)$.
      Since $(x_0, \cdots, x_n) = (1x_0, \cdots, 1x_n) \in F(x_0 : \cdots : x_n) = F(y_0 : \cdots : y_n)$, there must exist a $t \in k$ such that $(x_0, \cdots, x_n) = (ty_0, \cdots, ty_n)$.
      Since $(x_0, \cdots, x_n) \ne (0, \cdots, 0)$, $t \ne 0$.
      Then $t \in k^{\times}$.
      Therefore, $(x_0, \cdots, x_n) = t * (y_0, \cdots, y_n)$, so $(x_0 : \cdots : x_n) = (y_0 : \cdots : y_n)$.
    \item
      Surjective?
      Let $V$ be a one-dimensional subspace of $k^{n + 1}$.
      Let $\{ (a_0, \cdots, a_n) \}$ be a basis of $V$.
      Then $V = \{ (ta_0, \cdots, ta_n) \mid t \in k \}$.
      Since $(a_0, \cdots, a_n)$ is a basis element, it is nonzero.
      Therefore, $(a_0 : \cdots : a_n) \in \mathbb{P}^n_k$.
      Then $F(a_0 : \cdots : a_n) = V$.
  \end{itemize}
  $F$ is indeed a bijection between $\mathbb{P}^n_k$ and the set of all one-dimensional subspaces of $k^{n + 1}$.
\end{proof}

\begin{exer}{(2.5)}
  The set $\pm 1$ is a group with group law given by multiplication.
  This group acts on the unit sphere, $S^n := \{ (x_0, \cdots, x_n) \in \mathbb{R}^{n + 1} : x_0^2 + \cdots + x_n^2 = 1 \}$ by scalar multiplication $\pm 1 * (x_0, \cdots, x_n) = \pm 1 \cdot (x_0, \cdots, x_n)$.
  Let $\sim$ denote the corresponding equivalence relation on $S^n$.
  Construct a natural bijective map, $S^n \slash \sim \rightarrow \mathbb{P}^n_{\mathbb{R}}$.
\end{exer}

\begin{proof}
  Let $f: S^n \slash \sim \rightarrow \mathbb{P}^n_{\mathbb{R}}$ be defined such that $f([(x_0, \cdots, x_n)]) = (x_0 : \cdots : x_n)$.
  We claim that $f$ is a bijection.
  \begin{itemize}
    \item
      Well-defined?
      Let $(x_0, \cdots, x_n) \sim (y_0, \cdots, y_n) \in S^n$ be given.
      Since $(x_0, \cdots, x_n) \sim (y_0, \cdots, y_n)$, there exists a $t \in \{ -1, 1 \}$ such that $(x_0, \cdots, x_n) = t \cdot (y_0, \cdots, y_n) = (ty_0, \cdots, ty_n)$.
      Since $t \in \mathbb{R}^*$, $(x_0 : \cdots : x_n) = (y_0 : \cdots : y_n)$.
      This implies that $f([(x_0, \cdots, x_n)]) = f([(y_0, \cdots, y_n)])$.
    \item
      Injective?
      Let $(x_0, \cdots, x_n), (y_0, \cdots, y_n) \in S^n$ be given.
      Suppose $f([(x_0, \cdots, x_n)]) = f([(y_0, \cdots, y_n)])$.
      Then $(x_0 : \cdots : x_n) = (y_0 : \cdots : y_n)$.
      Therefore, there must exist a $t \in \mathbb{R}^*$ such that $(x_0, \cdots, x_n) = (ty_0, \cdots, ty_n)$.
      \begin{align*}
        (ty_0)^2 + \cdots + (ty_n)^2
          &= t^2(y_0^2 + \cdots y_n^2) \\
          &= t^2 \cdot 1 \\
          &= t^2.
      \end{align*}
      On the other hand, $x_0^2 + \cdots + x_n^2 = 1$, so $t^2 = 1$.
      This implies that $t \in \{ -1, 1 \}$.
      Since $(x_0, \cdots, x_n) = t(y_0, \cdots, y_n)$ for some $t \in \{ \pm 1 \}$, $[(x_0, \cdots, x_n)] = [(y_0, \cdots, y_n)]$ in $S^n \slash \sim$.
    \item
      Surjective?
      Let $(x_0 : \cdots : x_n) \in \mathbb{P}^n_{\mathbb{R}}$.
      Then $(x_0, \cdots, x_n) \ne 0$, so $x_0^2 + \cdots + x_n^2 \ne 0$.
      Let $c = x_0^2 + \cdots + x_n^2$.
      Let $y_i = x_i / \sqrt{c}$ for each $i$.
      This makes sense because $c$ is a positive real number, so $1 / \sqrt{c}$ exists.
      Then $y_0^2 + \cdots + y_n^2 = 1$, so $(y_0, \cdots, y_n) \in S^1$.
      Since $\sqrt{c}(y_0, \cdots, y_n) = (x_0, \cdots, x_n)$, $(y_0 : \cdots : y_n) = (x_0 : \cdots : x_n)$.
      Therefore, $f(y_0, \cdots, y_n) = (x_0 : \cdots : x_n)$, and thus $f$ is indeed surjective.
  \end{itemize}

\end{proof}

\begin{exer}{(2.6)}
  $ $
  \begin{itemize}
    \item
      Determine the number of elements in the group $GL_2(R)$ when $R$ is the ring $\mathbb{Z} / p\mathbb{Z}$, with $p$ a prime number.
    \item
      $GL_2(\mathbb{Z} / p)$ acts on $(\mathbb{Z} / p)^2$ by multiplying column vectors on the left by matrices.
      Determine the distinct orbits of this action.
    \item
      Describe the stabilizer subgroup in $GL_2(\mathbb{Z} / p)$ of the element $\begin{bmatrix} 1 \\ 0 \end{bmatrix} \in (\mathbb{Z} / p)^2$.
  \end{itemize}
\end{exer}

\begin{proof}
  $ $
  \begin{itemize}
    \item
      From linear algebra, we know that a matrix has a nonzero determinant if and only if the set of the column vectors is linearly independent.
      Since we are working with $2 \times 2$ matrices here, it is equivalent to checking whether each vector is a scalar multiple of the other.
      \begin{itemize}
        \item
          There are $p^2 - 1$ columns that can be in an invertible matrix.
          This is because the only column vector that no invertible matrix has is the zero vector.
        \item
          Suppose we pick one of the $p^2 - 1$ columns as the left column vector.
          Then there are exactly $(p^2 - 1) - (p - 1)$ column vectors that we can pick as the right column vector to create an invertible matrix.
          This is because any nonzero scalar multiple of $p^2 - 1$ cannot be the right column vector, and if the right column vector is not a scalar multiple of the left vector, the matrix will be invertible.
      \end{itemize}
      Therefore, there are exactly $(p^2 - 1)((p^2 - 1) - (p - 1)) = p^4 - p^3 - p^2 + p$ matrices in $GL_2(R)$.
    \item
      We claim that there are two distinct orbits.
      Let $v = \begin{bmatrix} a \\ b \end{bmatrix} \in (\mathbb{Z}/p)^2$.
      \begin{itemize}
        \item
          If $v = 0$, then the orbit is $\{ v \}$.
        \item
          If $v \ne 0$, then we claim that the orbit is the collection of all the nonzero vectors.
          First, we claim that the column vector $\begin{bmatrix} 1 \\ 1 \end{bmatrix}$ is in the orbit of $v$.

          \begin{itemize}
            \item
              If $a \ne 0$ and $b \ne 0$, then $\begin{bmatrix} a^{-1} & 0 \\ 0 & b^{-1} \end{bmatrix}\begin{bmatrix} a \\ b \end{bmatrix} = \begin{bmatrix} 1 \\ 1 \end{bmatrix}$.
            \item
              If $a \ne 0$ and $b = 0$, then $\begin{bmatrix} a^{-1} & 0 \\ a^{-1} & 1 \end{bmatrix}\begin{bmatrix} a \\ b \end{bmatrix} = \begin{bmatrix} 1 \\ 1 \end{bmatrix}$.
            \item
              If $a = 0$ and $b \ne 0$, then $\begin{bmatrix} 1 & b^{-1} \\ 0 & b^{-1} \end{bmatrix}\begin{bmatrix} a \\ b \end{bmatrix} = \begin{bmatrix} 1 \\ 1 \end{bmatrix}$.
          \end{itemize}
          Since this argument can be applied to any nonzero column vector, it implies that the orbit of any nonzero column vector contains $\begin{bmatrix} 1 \\ 1 \end{bmatrix}$.
          Since the orbits are equivalence classes, they are either disjoint or identical.
          In other words, this implies that every nonzero column vector has the same orbit.

          Moreover, for any $w \in (\mathbb{Z}/p)^2$, $Iw = w$ where $I$ denote the identity matrix.
          Thus the orbit of a nonzero vector contains any nonzero vector.
          
          Therefore, the orbit of any nonzero vector is the set of all the nonzero vectors.
      \end{itemize}
    \item
      \begin{align*}
        \begin{bmatrix} a & b \\ c & d \end{bmatrix} \begin{bmatrix} 1 \\ 0 \end{bmatrix} = \begin{bmatrix} 1 \\ 0 \end{bmatrix} 
          &\implies \begin{bmatrix} a \\ c \end{bmatrix} = \begin{bmatrix} 1 \\ 0 \end{bmatrix} 
      \end{align*}
      Therefore, it suffices to determine what the values of $b$ and $d$ are.
      $\begin{bmatrix} 1 & b \\ 0 & d \end{bmatrix}$ is an upper triangular matrix, so it is invertible if and only if $d \ne 0$.
      Therefore,
      \begin{align*}
        G_{\begin{bmatrix} 1 \\ 0 \end{bmatrix}} = \{ \begin{bmatrix} 1 & b \\ 0 & d \end{bmatrix} \mid b, d \in \mathbb{Z}_p, d \ne 0 \}.
      \end{align*}
  \end{itemize}
\end{proof}

\end{document}


