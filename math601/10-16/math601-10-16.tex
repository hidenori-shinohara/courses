\documentclass[12pt, psamsfonts]{amsart}

%-------Packages---------
\usepackage{amssymb,amsfonts}
\usepackage{fullpage}
\usepackage{tikz-cd}
\usepackage{todonotes}
\usepackage{physics}
\usepackage[all,arc]{xy}
\usepackage{enumerate}
\usepackage{enumitem}
\usepackage{mathrsfs}
\usepackage{theoremref}
\usepackage{graphicx}
\usepackage[bookmarks]{hyperref}

%--------Theorem Environments--------
%theoremstyle{plain} --- default
\newtheorem{thm}{Theorem}[section]
\newtheorem{cor}[thm]{Corollary}
\newtheorem{prop}[thm]{Proposition}
\newtheorem{lem}[thm]{Lemma}
\newtheorem{conj}[thm]{Conjecture}
\newtheorem{quest}[thm]{Question}

\theoremstyle{definition}
\newtheorem{defn}[thm]{Definition}
\newtheorem{defns}[thm]{Definitions}
\newtheorem{con}[thm]{Construction}
\newtheorem{exmp}[thm]{Example}
\newtheorem{exmps}[thm]{Examples}
\newtheorem{notn}[thm]{Notation}
\newtheorem{notns}[thm]{Notations}
\newtheorem{addm}[thm]{Addendum}
\newtheorem*{exer}{Exercise}

\theoremstyle{remark}
\newtheorem{rem}[thm]{Remark}
\newtheorem{rems}[thm]{Remarks}
\newtheorem{warn}[thm]{Warning}
\newtheorem{sch}[thm]{Scholium}

\DeclareMathOperator{\Hom}{Hom}
\DeclareMathOperator{\End}{End}
\DeclareMathOperator{\Id}{Id}

\makeatletter
\let\c@equation\c@thm
\makeatother
\numberwithin{equation}{section}

\bibliographystyle{plain}

\begin{document}

\title{Math 601 Homework (Due 10/16)}
\author{Hidenori Shinohara}
\maketitle

\tableofcontents

\section{Jordan Canonical Form}

Let $k$ be a field, $V$ a finite dimensional $k$-vector space, and $T \in \End_k(V)$ a linear transformation.

\begin{exer}{(Problem 1)}
  Show that the set $\{ p(x) \in k[x] \mid p(T) = 0 \in \End_k(V) \}$ is an ideal, $I \subset k[x]$.
  Also, show that $I \ne 0$.
\end{exer}

\begin{proof}
  $ $
  \begin{itemize}
    \item
     Claim 1: $I$ is nonempty.
     \todo[inline]{
       Use Cayley-Hamilton to find a non-trivial element.
       I'm still trying to understand C-H.
       One thing I learned today is that the determinant function is independent of the choice of the basis.
     }
    \item
     Claim 2: $I$ is closed under subtraction.
     Let $p(x), q(x) \in I$.
     Then $p(x) - q(x) \in I$ because $p(T) - q(T) = 0 - 0 = 0$.
    \item
     Claim 3: $I$ is closed under multiplication by elements in $k[x]$.
     Let $p(x) \in I, r(x) \in k[x]$.
     Then $p(T)r(T) = 0r(T) = 0$, so $r(x)p(x) \in I$.
  \end{itemize}
  By Claim 1 and 2, $I$ is a subgroup of $k[x]$ under addition.
  Then Claim 3 implies that $I$ is an ideal.
  By Claim 1, $I \ne 0$.
\end{proof}

\begin{exer}{(Problem 2)}
  Let $p(x) \in k[x]$ be a nonzero polynomial such that $p(T) = 0 \in \End_k(V)$.
  Show that if $p(x) \in k[x]$ is a product of linear polynomials, then there is a $k$-basis for $V$ with respect to which the matrix for $T$ is in Jordan normal form.
\end{exer}

\begin{proof}
  \todo[inline,caption={}]{
    I'm not sure what to do here.
    \begin{itemize}
      \item
        If I use the theorem, this problem will be too easy and the first part of the problem will be unnecessary, so I don't think I can just use the theorem.
      \item
        Initially, I thought that I could just do Step 3 and 4 in the proof of the theorem on PP.3-4 of the Jordan Canonical Form handout.
        However, I realized that Step 3 and 4 require the characteristic polynomial, but $p(x)$ is not necessarily the characteristic polynomial.
        I don't think there is anything that I can do with $p(x)$ but not with the characteristic polynomial, though.
        If there is some cool stuff I can do with $p(x)$ then I should be able to do that with the characteristic polynomial because $p(x)$ may be the characteristic polynomial.
      \item
        Maybe... I can't assume that $k$ is algebraically closed.
        But then that means I can't use the C-H theorem for the first part.
    \end{itemize}
  }
\end{proof}

\begin{exer}{(Problem 3)}
  Suppose that the field $k$ contains $m$ distinct $m$-th roots of 1.
  Suppose that $T^m = \Id_V \in \End_k(V)$.
  Show that there is a basis of $V$ with respect to which, the matrix for $T$ is diagonal.
  What can you say about the diagonal entries?
\end{exer}

\begin{proof}
  $ $
  \todo[inline,caption={}]{
    Some ideas...
    \begin{itemize}
      \item
        Assume $k = \mathbb{C}$.
      \item
        Let $r_l = \exp(\frac{2\pi il}{m})$ for each $l = 1, \cdots, m$.
      \item
        $x^m - 1 = (x - r_1) \cdots (x - r_m)$.
        Thus $T^m - \Id_V = (T - r_1\Id_V) \cdots (T - r_m\Id_V)$.
      \item
        Let $M$ denote the diagonal matrix for $T$.
        Then $M^m$ must be the identity matrix.
        Moreover, each entry of $M^m$ is simply the $m$-th power of the corresponding entry of $M$.
        Thus each of the diagonal entries in $M$ must be an $m$-th root of 1.
        On the other hand, any diagonal matrix where each entry is an $m$-th root of 1 has this property that when raised to the $m$-th power, it becomes the identity.
    \end{itemize}
  }
\end{proof}

\end{document}


