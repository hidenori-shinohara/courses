\documentclass[12pt, psamsfonts]{amsart}

%-------Packages---------
\usepackage{amssymb,amsfonts}
\usepackage{fullpage}
\usepackage{tikz-cd}
\usepackage{todonotes}
\usepackage{physics}
\usepackage[all,arc]{xy}
\usepackage{enumerate}
\usepackage{enumitem}
\usepackage{mathrsfs}
\usepackage{theoremref}
\usepackage{graphicx}
\usepackage[bookmarks]{hyperref}

%--------Theorem Environments--------
%theoremstyle{plain} --- default
\newtheorem{thm}{Theorem}[section]
\newtheorem{cor}[thm]{Corollary}
\newtheorem{prop}[thm]{Proposition}
\newtheorem{lem}[thm]{Lemma}
\newtheorem{conj}[thm]{Conjecture}
\newtheorem{quest}[thm]{Question}

\theoremstyle{definition}
\newtheorem{defn}[thm]{Definition}
\newtheorem{defns}[thm]{Definitions}
\newtheorem{con}[thm]{Construction}
\newtheorem{exmp}[thm]{Example}
\newtheorem{exmps}[thm]{Examples}
\newtheorem{notn}[thm]{Notation}
\newtheorem{notns}[thm]{Notations}
\newtheorem{addm}[thm]{Addendum}
\newtheorem*{exer}{Exercise}

\theoremstyle{remark}
\newtheorem{rem}[thm]{Remark}
\newtheorem{rems}[thm]{Remarks}
\newtheorem{warn}[thm]{Warning}
\newtheorem{sch}[thm]{Scholium}

\DeclareMathOperator{\Hom}{Hom}
\DeclareMathOperator{\End}{End}
\DeclareMathOperator{\Id}{Id}
\DeclareMathOperator{\Coker}{Coker}

\makeatletter
\let\c@equation\c@thm
\makeatother
\numberwithin{equation}{section}

\bibliographystyle{plain}

\begin{document}

\title{Math 601 Homework (Due 10/16)}
\author{Hidenori Shinohara}
\maketitle

\tableofcontents

\section{Modules}

\begin{exer}{(Problem 2)}
  Consider the $m \times n$ matrices given below as presentation matrices for $\mathbb{Z}$-modules.
  That is think of the given matrix, $H$, as giving a linear transformation, $\mathbb{Z}^n \rightarrow \mathbb{Z}^m$, $x \mapsto Hx$ and thus giving a presentation of $\Coker(H) = \mathbb{Z}^m/\Im(H)$.
  Give in each case a familiar finitely generated $\mathbb{Z}$-module which is isomorphic to the $\mathbb{Z}$-module which $H$ presents.
  \begin{itemize}
    \item
      $H = 6$.
    \item
      $H = \begin{bmatrix} 2 & 1 \end{bmatrix}$.
    \item
      $H = \begin{bmatrix} 2 & 3 \\ 4 & 5 \end{bmatrix}$.
    \item
      $H = \begin{bmatrix} 4 & 12 \\ 6 & 2 \end{bmatrix}$.
    \item
      $H = \begin{bmatrix} 3 & 6 \\ 8 & 4 \\ 10 & 5 \end{bmatrix}$.
    \item
      $H = \begin{bmatrix} 36 & 12 & 24 \\ 30 & 18 & 24 \\ 15 & -6 & 12 \end{bmatrix}$.
  \end{itemize}
\end{exer}

\begin{proof}
  In each case, we will compute a Smith normal form because a smith normal form allows us to find invariant factors easily.
  Moreover, elementary row and column operations over integers of $H$ correspond to a change of basis of $\mathbb{Z}^m$ and $\mathbb{Z}^n$.
  Therefore, it does not change the module represented by the matrix.
  \begin{itemize}
    \item
      This $H$ generates the exact sequence

      \begin{center}
        \begin{tikzcd} \mathbb{Z}^1 \arrow[r, "H"] &\mathbb{Z}^1 \arrow[r, "p"] & \mathbb{Z}^1/6\mathbb{Z} \arrow[r, "0"] & 0 \end{tikzcd}
      \end{center}

      where $p$ is the map $k \mapsto k + 6\mathbb{Z}$.
      Thus $\mathbb{Z}/6\mathbb{Z}$ is what $H$ represents.
    \item
      This $H$ generates the exact sequence

      \begin{center}
        \begin{tikzcd} \mathbb{Z}^2 \arrow[r, "H"] &\mathbb{Z}^1 \arrow[r, "p"] & \mathbb{Z}^1/\Im(H) \arrow[r, "0"] & 0 \end{tikzcd}
      \end{center}

      where $p$ is the map $k \mapsto k + \Im(H)$.
      The Smith normal form of $H$ is $\begin{bmatrix} 1 & 0 \end{bmatrix}$ since
      \begin{align*}
        \begin{bmatrix} 2 & 1 \end{bmatrix}
          &\sim \begin{bmatrix} 1 & 2 \end{bmatrix} \\
          &\sim \begin{bmatrix} 1 & 0 \end{bmatrix} \\
      \end{align*}
      Thus $H$ represents $\mathbb{Z}/\mathbb{Z} \cong 0$.
    \item
      \todo[inline,caption={}]{
      }
  \end{itemize}
\end{proof}

\begin{exer}{(Problem 3)}
  To what familiar abelian group is the following abelian group isomorphic to?
  The group generated by $a, b, c$ for which the module of relations is generated by the following relations, $6a - 10b + 4c = 0$ and $8a - 20c = 0$.
\end{exer}

\begin{proof}
  \todo[inline,caption={}]{
    Solve this!
  }
  \begin{itemize}
    \item
      Abelian group = $\mathbb{Z}$ module.
    \item
      \begin{center}
        \begin{tikzcd} \mathbb{Z}^2 \arrow[r, "h"] &\mathbb{Z}^3 \arrow[r, "q"] & M \arrow[r, "0"] & 0 \end{tikzcd}
      \end{center}
    \item
      $M$ is an abelian group generated by $a, b, c$ with some relations.
    \item
      $\ker(q)$ is the module of relations, and $\ker(q) = \langle 6a - 10b + 4c, 8a - 20c \rangle$.
    \item
      $h = \begin{bmatrix} 6 & 8 \\ -10 & 0 \\ 2 & -20 \end{bmatrix}$
    \item
      $q(1, 0, 0) = a, q(0, 1, 0) = b, q(0, 0, 1) = c$.
    \item
      $M = \langle a, b, c \mid 6a - 10b + 4c, 8a - 20c \rangle$.
    \item
      Is the answer just $\mathbb{Z}^3 / \langle (6, -10, 4), (8, 0, -20) \rangle$?
      I'm certainly not familiar with that abelian group.
      If I mod $\mathbb{Z}^3$ by two independent vectors, does that leave $\mathbb{Z}$?
  \end{itemize}
\end{proof}

\begin{exer}{(Problem 4)}
  How many isomorphism classes of abelian groups with $27783 = 3^47^3$ elements are there?
\end{exer}

\begin{proof}
  Let $M$ be an abelian group with 27783 elements.
  Then $M$ is a $\mathbb{Z}$-module with 27783 elements.
  By the theorem on PP.8-9 of the Module handout, $M \simeq \mathbb{Z}/(d_1) \times \cdots \times \mathbb{Z}/(d_n) \times \mathbb{Z}^{m - s}$.
  Since $M$ only contains finitely many elements and $\mathbb{Z}$ contains infinitely many elements, $M \simeq \mathbb{Z}/(d_1) \times \cdots \times \mathbb{Z}/(d_n)$.
  $\gcd(a, b) = 1$ if and only if $\mathbb{Z}/(a)$ is isomorphic to $\mathbb{Z}/(b)$.
  \begin{itemize}
    \item
      $\mathbb{Z}_3 \times \mathbb{Z}_3 \times \mathbb{Z}_3 \times \mathbb{Z}_3$, $\mathbb{Z}_9 \times \mathbb{Z}_3 \times \mathbb{Z}_3$,
      $\mathbb{Z}_9 \times \mathbb{Z}_9$, $\mathbb{Z}_{27} \times \mathbb{Z}_3$, $\mathbb{Z}_{81}$.
    \item
      $\mathbb{Z}_7 \times \mathbb{Z}_7 \times \mathbb{Z}_7$, $\mathbb{Z}_{49} \times \mathbb{Z}_7$, $\mathbb{Z}_{343}$.
  \end{itemize}
  Thus the combinations of the above are exactly all the distinct classes of abelian groups with 27783 elements, so there are exactly $3 \times 5 = 15$ classes.
\end{proof}

\section{The Quadratic Equation}

\begin{exer}{(Problem 23)}
  Show that if $x^2 - 2y^2 = n$, $n \ne 0$ has one solution, then it has infinitely many.
  If $n$ is prime in $\mathbb{Z}$, describe all the solutions.
\end{exer}

\begin{proof}
  Let $n \in \mathbb{Z}$ be given.
  Suppose $x^2 - 2y^2 = n$ for some $x, y \in \mathbb{Z}$.
  For each $k \in \mathbb{N}$, pick $a_k, b_k \in \mathbb{Z}$ such that $a_k + b_k\sqrt{2} = u_0^{2k}$ where $u_0 = 1 + \sqrt{2}$.
  We showed that $u_0^{2k}$ is a unit element for each $k \in \mathbb{N}$.
  Since $(a_k + b_k\sqrt{2})(a_k - b_k\sqrt{2}) = N(a_k + b_k\sqrt{2}) = N(u_0)^{2k} = 1$ by Problem 2 and 3.
  Moreover, $u_0^k \ne u_0^{k'}$ whenever $k \ne k'$ since $u_0 \ne 0$ and $\abs{u_0} \ne 1$.

  $n = x^2 - 2y^2 = (x + \sqrt{2}y)(x - \sqrt{2}y)$.
  Then $(x + \sqrt{2}y)(a_k - b_k\sqrt{2}) = (a_kx - 2b_ky) + (b_kx - a_ky)\sqrt{2}$, and $(x - \sqrt{2}y)(a_k + b_k\sqrt{2}) = (a_kx - 2b_ky) - (b_kx - a_ky)\sqrt{2}$.
  \begin{align*}
    (a_kx - 2b_ky)^2 - 2(xb_k - a_ky)^2
      &= N((a_kx - 2b_ky) + (xb_k - a_ky)\sqrt{2}) \\
      &= N(x + \sqrt{2}y)N(a_k - b_k\sqrt{2}) \\
      &= N(x + \sqrt{2}y)(a_k - b_k\sqrt{2})\gamma(a_k + b_k\sqrt{2}) \\
      &= N(x + \sqrt{2}y)(a_k + b_k\sqrt{2})\gamma(a_k - b_k\sqrt{2}) \\
      &= N(x + \sqrt{2}y)N(a_k + b_k\sqrt{2}) \\
      &= N(x + \sqrt{2}y) \cdot 1 \\
      &= N(x + \sqrt{2}y) \\
      &= x^2 - 2y^2 = n.
  \end{align*}

  If $k \ne k'$, then $a_k - b_k\sqrt{2} \ne a_{k'} - b_{k'}\sqrt{2}$.
  Thus $(x + \sqrt{2}y)(a_k - b_k\sqrt{2}) \ne (x + \sqrt{2}y)(a_k - b_k\sqrt{2})$, so $(a_kx - 2b_ky, xb_k - a_ky) \ne (a_{k'}x - 2b_{k'}y, xb_{k'} - a_{k'}y)$.
  Thus we get different solutions for different values of $k$.

  \todo[inline,caption={}]{
    Prime?
  }
\end{proof}

\begin{exer}{(Problem 24)}
  For which $\overline{n} \in \mathbb{Z}/(8)$ does $\overline{x}^2 - \overline{2}\overline{y}^2 = \overline{n}$ have solutions?
\end{exer}

\begin{proof}
  $ $
  \begin{itemize}
    \item $0^2 - 2 \cdot 0^2 = 0$
    \item $1^2 - 2 \cdot 0^2 = 1$
    \item $2^2 - 2 \cdot 1^2 = 2$
    \item $2^2 - 2 \cdot 0^2 = 4$
    \item $0^2 - 2 \cdot 1^2 = 6$
    \item $1^2 - 2 \cdot 1^2 = 7$
  \end{itemize}
  By Problem 25 below, there exist no solutions to $\overline{x}^2 - \overline{2}\overline{y}^2 = \overline{n}$ when $\overline{n} = 3, 5$.

\end{proof}

\begin{exer}{(Problem 25)}
  Show that if $n \equiv \pm 3 \pmod 8$, then $x^2 - 2y^2 = n$ has no solutions.
\end{exer}

\begin{proof}
  We consider $x \mapsto x^2 \pmod 8$ for each $x$.
  $0 \mapsto 0, 1 \mapsto 1, 2 \mapsto 4, 3 \mapsto 1, 4 \mapsto 0, 5 \mapsto 1, 6 \mapsto 4, 7 \mapsto 1$.
  It suffices to check $x = 0, \cdots, 7$ because every integer is equivalent to one of these 8 numbers $\pmod 8$.
  Thus $x^2 - 2y^2 \equiv a - 2b \pmod 8$ where $a, b \in \{ 0, 1, 4 \}$ for any $x, y \in \mathbb{Z}$.
  By checking those $3 \times 3 = 9$ possibilities, we can conclude that there exists no $x, y$ such that $x^2 - 2y^2 \equiv \pm 3 \pmod 8$.

  \begin{itemize}
    \item
      $0 - 2 \cdot 0 \equiv 0$
    \item
      $0 - 2 \cdot 1 \equiv 6$
    \item
      $0 - 2 \cdot 4 \equiv 0$
    \item
      $1 - 2 \cdot 0 \equiv 1$
    \item
      $1 - 2 \cdot 1 \equiv 7$
    \item
      $1 - 2 \cdot 4 \equiv 1$
    \item
      $4 - 2 \cdot 0 \equiv 4$
    \item
      $4 - 2 \cdot 1 \equiv 2$
    \item
      $4 - 2 \cdot 4 \equiv 4$
  \end{itemize}
\end{proof}

\begin{exer}{(Problem 26)}
  Let $p \in \mathbb{Z}$ be an odd prime.
  Quadratic reciprocity says that 2 is a square mod $p$ if and only if $p \equiv \pm 1 \pmod 8$.
  Conclude that $x^2 - 2y^2 = p$ has a solution if and only if $p \equiv \pm 1 \pmod 8$.
\end{exer}


\begin{proof}
  \todo[inline,caption={}]{
    By Problem 19, $x^2 - 2y^2 = p$ has a solution if and only if $p$ is not irreducible in $\mathbb{Z}[\sqrt{2}]$.

    By Problem 21, 2 is not a square in $\mathbb{Z}/(p)$ if and only if $\mathbb{Z}[\sqrt{2}]/(p)$ is an integral domain.

    Therefore, $x^2 - 2y^2 = p$ has a solution if and only if 2 is a square in $\mathbb{Z}/(p)$.
    By Quadratic reciprocity, 2 is a square in $\mathbb{Z}/(p)$ if and only if $p \equiv \pm 1 \pmod 8$.

    Thus $x^2 - 2y^2 = p$ has a solution if and only if $p \equiv \pm 1 \pmod 8$.
  }
\end{proof}

\section{Jordan Canonical Form}

Let $k$ be a field, $V$ a finite dimensional $k$-vector space, and $T \in \End_k(V)$ a linear transformation.

\begin{exer}{(Problem 1)}
  Show that the set $\{ p(x) \in k[x] \mid p(T) = 0 \in \End_k(V) \}$ is an ideal, $I \subset k[x]$.
  Also, show that $I \ne 0$.
\end{exer}

\begin{proof}
  $ $
  \begin{itemize}
    \item
      Claim 1: $I$ is nonempty.
      Let $v_1, \cdots, v_n$ be a basis of $V$.
      Such a basis must exist since the dimension of $V$ is finite.
      Let $M$ be the $n \times n$ matrix associated to $V$ with respect to the basis $\{ v_1, \cdots, v_n \}$.
      In other words, for any $v \in V$, $Mv = T(v)$ where $Mv$ is the product.
      Since $M$ is an $n \times n$ matrix, the set $\{ M^0, \cdots, M^{n^2} \}$ is linearly dependent.
      Thus there exist $a_{n^2}, \cdots, a_0 \in k$ such that
      \begin{itemize}
        \item
          $a_{n^2}M^{n^2} + \cdots + a_0M^0 = 0$.
        \item
          $a_{n^2}, \cdots, a_0$ are not all zero.
      \end{itemize}
      Then for any $v \in V$,
      \begin{align*}
        0 &= (a_{n^2}M^{n^2} + \cdots + a_0M^0)v \\
          &= a_{n^2}M^{n^2}v + \cdots + a_0M^0v \\
          &= a_{n^2}T^{n^2}(v) + \cdots + a_0T^0(v) \\
          &= (a_{n^2}T^{n^2} + \cdots + a_0T^0)(v).
      \end{align*}

      Therefore, $p(x) = a_{n^2}x^{n^2} + \cdots + a_0x^0 \ne 0$ and $p(T) = 0$.
      Thus $p(x) \in I$, so $I$ is nonempty.
    \item
      Claim 2: $I$ is closed under subtraction.
      Let $p(x), q(x) \in I$.
      Then $p(x) - q(x) \in I$ because $p(T) - q(T) = 0 - 0 = 0$.
    \item
      Claim 3: $I$ is closed under multiplication by elements in $k[x]$.
      Let $p(x) \in I, r(x) \in k[x]$.
      Then $p(T)r(T) = 0r(T) = 0$, so $r(x)p(x) \in I$.
  \end{itemize}
  By Claim 1 and 2, $I$ is a subgroup of $k[x]$ under addition.
  Then Claim 3 implies that $I$ is an ideal.
  By Claim 1, $I \ne 0$.
\end{proof}

\begin{exer}{(Problem 2)}
  Let $p(x) \in k[x]$ be a nonzero polynomial such that $p(T) = 0 \in \End_k(V)$.
  Show that if $p(x) \in k[x]$ is a product of linear polynomials, then there is a $k$-basis for $V$ with respect to which the matrix for $T$ is in Jordan normal form.
\end{exer}

\begin{proof}
  \todo[inline,caption={}]{
    Since $k$ is just a field, I can't assume that $k$ is algebraically closed.
    \begin{itemize}
      \item
        $p(x) = (x - a_1)^{m_1} \cdots (x - a_n)^{m_n}$.
      \item
        Let $N = \dim(V)$.
      \item
        Let $q(\lambda) = \det(T - \lambda \Id)$ be the characteristic polynomial of $T$.
      \item
        Let $v_1, \cdots, v_N$ be a basis of $V$.
    \end{itemize}
    For each $i$, $(p(T))(v_i) = 0$.
    In other words, there exists a $j$ such that $(T - a_j\Id)(v) = 0$ for some nonzero $v$.
    This can be found by applying each linear factor to $v_i$ and figure out the point where it turns into 0.
    In other words, $\det(T - a_j\Id) = 0$.
    This implies that $a_j$ is a root of the characteristic polynomial $q(\lambda)$ of $T$.
    Thus $\lambda - a_j$ divides $q(\lambda)$.

    But I'm not sure what to do next.
    We want to find the largest number $r_j$ such that $(\lambda - a_j)^{r_j}$ divides $q(\lambda)$.
    What happens next?
  }
\end{proof}

\begin{exer}{(Problem 3)}
  Suppose that the field $k$ contains $m$ distinct $m$-th roots of 1.
  Suppose that $T^m = \Id_V \in \End_k(V)$.
  Show that there is a basis of $V$ with respect to which, the matrix for $T$ is diagonal.
  What can you say about the diagonal entries?
\end{exer}

\begin{proof}
  $ $
  \todo[inline,caption={}]{
    \begin{itemize}
      \item
        Let $r_i, \cdots, r_m$ denote the $m$ distinct $m$th roots of 1.
      \item
        Then each $x - r_i$ divides $x^m - 1$.
        Thus $x^m - 1 = (x - r_1) \cdots (x - r_m)$.
        This means that $p(x) = x^m - 1$ is a polynomial such that $p(T) = 0$ and it is a product of linear polynomials.
        Then I think that we can use an approach similar to the previous problem.
      \item
        Let $M$ denote the diagonal matrix for $T$.
        Then $M^m$ must be the identity matrix.
        Moreover, the $i$th diagonal entry of $M^m$ is simply the $m$-th power of the $i$th diagonal entry of $M$.
        Thus each of the diagonal entries in $M$ must be an $m$-th root of 1.
        On the other hand, any diagonal matrix where each entry is an $m$-th root of 1 becomes the identity when raised to the $m$th power.
    \end{itemize}
  }
\end{proof}

\begin{exer}{(Problem 4)}
  Let $V$ be a 9 dimensional $k$-vector space.
  Let $T \in \End_k(V)$ have minimal polynomial, $x^2(x - 1)^3$.
  What are the possible Jordan canonical forms for $T$?
\end{exer}

\begin{proof}
  $ $
  \todo[inline,caption={}]{
    For any $a, b \in \{ 0, 1 \}$,

    \begin{align*}
      \begin{bmatrix}
        1 & 0 & \cdots \\
        a & 1 & 0 & \cdots \\
        0 & b & 1 & 0 & \cdots \\
        0 & 0 & 0 & 0 & 0 & \cdots \\
        \vdots & \vdots & & & & \ddots \\
      \end{bmatrix}
    \end{align*}

    satisfies $x^2(x - 1)^3$.
  }
\end{proof}

\end{document}


