\documentclass[12pt, psamsfonts]{amsart}

%-------Packages---------
\usepackage{amssymb,amsfonts}
\usepackage{fullpage}
\usepackage{tikz-cd}
\usepackage{todonotes}
\usepackage{physics}
\usepackage[all,arc]{xy}
\usepackage{enumerate}
\usepackage{enumitem}
\usepackage{mathrsfs}
\usepackage{theoremref}
\usepackage{graphicx}
\usepackage[bookmarks]{hyperref}

%--------Theorem Environments--------
%theoremstyle{plain} --- default
\newtheorem{thm}{Theorem}[section]
\newtheorem{cor}[thm]{Corollary}
\newtheorem{prop}[thm]{Proposition}
\newtheorem{lem}[thm]{Lemma}
\newtheorem{conj}[thm]{Conjecture}
\newtheorem{quest}[thm]{Question}

\theoremstyle{definition}
\newtheorem{defn}[thm]{Definition}
\newtheorem{defns}[thm]{Definitions}
\newtheorem{con}[thm]{Construction}
\newtheorem{exmp}[thm]{Example}
\newtheorem{exmps}[thm]{Examples}
\newtheorem{notn}[thm]{Notation}
\newtheorem{notns}[thm]{Notations}
\newtheorem{addm}[thm]{Addendum}
\newtheorem*{exer}{Exercise}

\theoremstyle{remark}
\newtheorem{rem}[thm]{Remark}
\newtheorem{rems}[thm]{Remarks}
\newtheorem{warn}[thm]{Warning}
\newtheorem{sch}[thm]{Scholium}

\DeclareMathOperator{\Hom}{Hom}
\DeclareMathOperator{\End}{End}
\DeclareMathOperator{\Id}{Id}

\makeatletter
\let\c@equation\c@thm
\makeatother
\numberwithin{equation}{section}

\bibliographystyle{plain}

\begin{document}

\title{Math 601 Homework (Due 10/16)}
\author{Hidenori Shinohara}
\maketitle

\tableofcontents

\section{Jordan Canonical Form}

Let $k$ be a field, $V$ a finite dimensional $k$-vector space, and $T \in \End_k(V)$ a linear transformation.

\begin{exer}{(Problem 1)}
  Show that the set $\{ p(x) \in k[x] \mid p(T) = 0 \in \End_k(V) \}$ is an ideal, $I \subset k[x]$.
  Also, show that $I \ne 0$.
\end{exer}

\begin{proof}
$ $
 \begin{itemize}
   \item
    Claim 1: $I$ is nonempty.
    \todo[inline]{
      Use Cayley-Hamilton to find a non-trivial element.
      I'm still trying to understand C-H.
      One thing I learned today is that the determinant function is independent of the choice of the basis.
    }
   \item
    Claim 2: $I$ is closed under subtraction.
    Let $p(x), q(x) \in I$.
    Then $p(x) - q(x) \in I$ because $p(T) - q(T) = 0 - 0 = 0$.
   \item
    Claim 3: $I$ is closed under multiplication by elements in $k[x]$.
    Let $p(x) \in I, r(x) \in k[x]$.
    Then $p(T)r(T) = 0r(T) = 0$, so $r(x)p(x) \in I$.
 \end{itemize}
 By Claim 1 and 2, $I$ is a subgroup of $k[x]$ under addition.
 Then Claim 3 implies that $I$ is an ideal.
 By Claim 1, $I \ne 0$.
\end{proof}

\end{document}


