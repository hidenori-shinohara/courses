\documentclass[12pt, psamsfonts]{amsart}

%-------Packages---------
\usepackage{amssymb,amsfonts}
\usepackage{todonotes}
\usepackage{fullpage}
\usepackage{physics}
\usepackage[all,arc]{xy}
\usepackage{enumerate}
\usepackage{mathrsfs}
\usepackage{theoremref}
\usepackage{graphicx}
\usepackage[bookmarks]{hyperref}

%--------Theorem Environments--------
%theoremstyle{plain} --- default
\newtheorem{thm}{Theorem}[section]
\newtheorem{cor}[thm]{Corollary}
\newtheorem{prop}[thm]{Proposition}
\newtheorem{lem}[thm]{Lemma}
\newtheorem{conj}[thm]{Conjecture}
\newtheorem{quest}[thm]{Question}

\theoremstyle{definition}
\newtheorem{defn}[thm]{Definition}
\newtheorem{defns}[thm]{Definitions}
\newtheorem{con}[thm]{Construction}
\newtheorem{exmp}[thm]{Example}
\newtheorem{exmps}[thm]{Examples}
\newtheorem{notn}[thm]{Notation}
\newtheorem{notns}[thm]{Notations}
\newtheorem{addm}[thm]{Addendum}
\newtheorem*{exer}{Exercise}

\theoremstyle{remark}
\newtheorem{rem}[thm]{Remark}
\newtheorem{rems}[thm]{Remarks}
\newtheorem{warn}[thm]{Warning}
\newtheorem{sch}[thm]{Scholium}

\DeclareMathOperator{\Hom}{Hom}
\DeclareMathOperator{\Id}{Id}

\makeatletter
\let\c@equation\c@thm
\makeatother
\numberwithin{equation}{section}

\bibliographystyle{plain}

\begin{document}

\title{Math 601 Homework (Due 9/18)}
\author{Hidenori Shinohara}
\maketitle

\begin{exer}
  Let $R$ be a commutative ring with one.
  Explain why there is a unique ring homomorphism, $\mathbb{Z} \rightarrow R$.
\end{exer}

\begin{proof}
  The existence of a ring homomorphism is clear since $\phi(n) = 1_R + \cdots + 1_R$ and $\phi(-n) = -\phi(n)$ define a homomorphism.

  We will show the uniqueness of a ring homomorphism.
  Let $\phi_1, \phi_2: \mathbb{Z} \rightarrow R$ be ring homomorphisms.

  We claim that $\phi_1(n) = \phi_2(n)$ for each $n \in \mathbb{N}$.
  \begin{itemize}
    \item
      By definition, $\phi_1(1) = \phi_2(1) = 1_R$.
    \item
      Suppose $\phi_1(n) = \phi_2(n)$ for some $n \in \mathbb{N}$.
      Then $\phi_1(n + 1) = \phi_1(n) + \phi_1(1) = \phi_2(n) + \phi_2(1) = \phi_2(n + 1)$.
  \end{itemize}
  By mathematical induction, $\phi_1(n) = \phi_2(n)$ for each $n \in \mathbb{N}$.

  For every $n \in \mathbb{N}$, $\phi_1(-n) = -\phi_1(n) = -\phi_2(n) = \phi_2(-n)$.
  Finally, $\phi_1(0) = \phi_1(0 + 0) = \phi_1(0) + \phi_1(0)$, so $\phi_1(0) = 0_R$.
  Similarly, $\phi_2(0) = 0_R$.
  Thus $\phi_1(0) = \phi_2(0)$.

  Hence, we have shown that $\phi_1 = \phi_2$.
\end{proof}

\begin{exer}{(Problem 2)}
  Let $I \subset R$ be an ideal in a commutative ring.
  Describe a bijective correspondence between ideals in $R / I$ and certain ideals in $R$.
\end{exer}

\begin{proof}
  \todo[inline]{
    Tried this for about 10 minutes.
    I think this must be related to some special ideals, so I checked the annihilator, but that doesn't really work.
    I suspect that this problem is fairly simple once I notice what it is, but it'll take time until I notice it.
  }
\end{proof}

\begin{exer}{(Problem 3)}
  Let $I, J \subset R$ be ideals in a commutative ring.
  Let $I + J \subset R$ denote the smallest ideal containing $I$ and $J$.
  Observe that $I + J = \{ i + j \in R : i \in I, j \in J \}$.
  Let $\overline{J} \subset R / I$ denote the image of $J$ under the canonical quotient map, $R \rightarrow R / I$.
  Observe that $\overline{J}$ is an ideal in $S := R / I$.
  Use the universal mapping property of the quotient to show that $R / (I + J) \simeq S / \overline{J}$.
\end{exer}

\begin{proof}
  \todo[inline]{
    Tried this for 20 minutes.
      The problem seems complicated, but it seems that we just need some sort of category theoretical approach to solve this problem.
      I think I can finish it in the next 20 minutes.
      The universal mapping property of the quotient is proposition 6 in the handouts.
    }
  \begin{figure}
    \includegraphics[width=.5\linewidth]{problem3_partial_delete.jpeg}
      \caption{deletethis}
    \label{fig:mydelete}
  \end{figure}
\end{proof}

\begin{exer}{(Problem 4)}
  Let $R$ be a commutative ring and $f(x) = \sum_{i=0}^{n} a_ix^i \in R[x]$ a non-zero polynomial of degree $n$.
  Suppose that $a_n \in R^{\times}$.
  Let $J = (f(x))$.
  Prove that every element of $R[x]/J$ may be written in exactly one way in the form $\sum_{i=0}^{n - 1}r_ix^i + J$ with $r_0, r_1, \cdots, r_{n - 1} \in R$.
\end{exer}

\begin{proof}
  \todo[inline]{
    Tried this for 10 minutes.
    This problem seems wrong.
    See Figure \ref{fig:counterexampledelete}.
  }
  \begin{figure}
    \includegraphics[width=.5\linewidth]{problem4_wrong_delete.jpeg}
    \caption{Problem 4}
    \label{fig:counterexampledelete}
  \end{figure}
\end{proof}

\end{document}


