\documentclass[12pt, psamsfonts]{amsart}

%-------Packages---------
\usepackage{amssymb,amsfonts}
\usepackage{fullpage}
\usepackage{physics}
\usepackage[all,arc]{xy}
\usepackage{enumerate}
\usepackage{mathrsfs}
\usepackage{theoremref}
\usepackage{graphicx}
\usepackage[bookmarks]{hyperref}

%--------Theorem Environments--------
%theoremstyle{plain} --- default
\newtheorem{thm}{Theorem}[section]
\newtheorem{cor}[thm]{Corollary}
\newtheorem{prop}[thm]{Proposition}
\newtheorem{lem}[thm]{Lemma}
\newtheorem{conj}[thm]{Conjecture}
\newtheorem{quest}[thm]{Question}

\theoremstyle{definition}
\newtheorem{defn}[thm]{Definition}
\newtheorem{defns}[thm]{Definitions}
\newtheorem{con}[thm]{Construction}
\newtheorem{exmp}[thm]{Example}
\newtheorem{exmps}[thm]{Examples}
\newtheorem{notn}[thm]{Notation}
\newtheorem{notns}[thm]{Notations}
\newtheorem{addm}[thm]{Addendum}
\newtheorem*{exer}{Exercise}

\theoremstyle{remark}
\newtheorem{rem}[thm]{Remark}
\newtheorem{rems}[thm]{Remarks}
\newtheorem{warn}[thm]{Warning}
\newtheorem{sch}[thm]{Scholium}

\DeclareMathOperator{\Hom}{Hom}
\DeclareMathOperator{\Id}{Id}

\makeatletter
\let\c@equation\c@thm
\makeatother
\numberwithin{equation}{section}

\bibliographystyle{plain}

\begin{document}

\title{Math 601 Homework (Due 9/18)}
\author{Hidenori Shinohara}
\maketitle

\begin{exer}
  Let $R$ be a commutative ring with one.
  Explain why there is a unique ring homomorphism, $\mathbb{Z} \rightarrow R$.
\end{exer}

\begin{proof}
  The existence of a ring homomorphism is clear since $\phi(n) = 1_R + \cdots + 1_R$ and $\phi(-n) = -\phi(n)$ define a homomorphism.

  We will show the uniqueness of a ring homomorphism.
  Let $\phi_1, \phi_2: \mathbb{Z} \rightarrow R$ be ring homomorphisms.

  We claim that $\phi_1(n) = \phi_2(n)$ for each $n \in \mathbb{N}$.
  \begin{itemize}
    \item
      By definition, $\phi_1(1) = \phi_2(1) = 1_R$.
    \item
      Suppose $\phi_1(n) = \phi_2(n)$ for some $n \in \mathbb{N}$.
      Then $\phi_1(n + 1) = \phi_1(n) + \phi_1(1) = \phi_2(n) + \phi_2(1) = \phi_2(n + 1)$.
  \end{itemize}
  By mathematical induction, $\phi_1(n) = \phi_2(n)$ for each $n \in \mathbb{N}$.

  For every $n \in \mathbb{N}$, $\phi_1(-n) = -\phi_1(n) = -\phi_2(n) = \phi_2(-n)$.
  Finally, $\phi_1(0) = \phi_1(0 + 0) = \phi_1(0) + \phi_1(0)$, so $\phi_1(0) = 0_R$.
  Similarly, $\phi_2(0) = 0_R$.
  Thus $\phi_1(0) = \phi_2(0)$.

  Hence, we have shown that $\phi_1 = \phi_2$.
\end{proof}

\end{document}


