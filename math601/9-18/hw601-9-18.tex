\documentclass[12pt, psamsfonts]{amsart}

%-------Packages---------
\usepackage{amssymb,amsfonts}
\usepackage{todonotes}
\usepackage{fullpage}
\usepackage{physics}
\usepackage[all,arc]{xy}
\usepackage{enumerate}
\usepackage{mathrsfs}
\usepackage{theoremref}
\usepackage{graphicx}
\usepackage[bookmarks]{hyperref}

%--------Theorem Environments--------
%theoremstyle{plain} --- default
\newtheorem{thm}{Theorem}[section]
\newtheorem{cor}[thm]{Corollary}
\newtheorem{prop}[thm]{Proposition}
\newtheorem{lem}[thm]{Lemma}
\newtheorem{conj}[thm]{Conjecture}
\newtheorem{quest}[thm]{Question}

\theoremstyle{definition}
\newtheorem{defn}[thm]{Definition}
\newtheorem{defns}[thm]{Definitions}
\newtheorem{con}[thm]{Construction}
\newtheorem{exmp}[thm]{Example}
\newtheorem{exmps}[thm]{Examples}
\newtheorem{notn}[thm]{Notation}
\newtheorem{notns}[thm]{Notations}
\newtheorem{addm}[thm]{Addendum}
\newtheorem*{exer}{Exercise}

\theoremstyle{remark}
\newtheorem{rem}[thm]{Remark}
\newtheorem{rems}[thm]{Remarks}
\newtheorem{warn}[thm]{Warning}
\newtheorem{sch}[thm]{Scholium}

\DeclareMathOperator{\Hom}{Hom}
\DeclareMathOperator{\Id}{Id}

\makeatletter
\let\c@equation\c@thm
\makeatother
\numberwithin{equation}{section}

\bibliographystyle{plain}

\begin{document}

\title{Math 601 Homework (Due 9/18)}
\author{Hidenori Shinohara}
\maketitle

\begin{exer}{(Problem 1)}
  Let $R$ be a commutative ring with one.
  Explain why there is a unique ring homomorphism, $\mathbb{Z} \rightarrow R$.
\end{exer}

\begin{proof}
  The existence of a ring homomorphism is clear since $\phi(n) = 1_R + \cdots + 1_R$ and $\phi(-n) = -\phi(n)$ define a homomorphism.

  We will show the uniqueness of a ring homomorphism.
  Let $\phi_1, \phi_2: \mathbb{Z} \rightarrow R$ be ring homomorphisms.

  We claim that $\phi_1(n) = \phi_2(n)$ for each $n \in \mathbb{N}$.
  \begin{itemize}
    \item
      By definition, $\phi_1(1) = \phi_2(1) = 1_R$.
    \item
      Suppose $\phi_1(n) = \phi_2(n)$ for some $n \in \mathbb{N}$.
      Then $\phi_1(n + 1) = \phi_1(n) + \phi_1(1) = \phi_2(n) + \phi_2(1) = \phi_2(n + 1)$.
  \end{itemize}
  By mathematical induction, $\phi_1(n) = \phi_2(n)$ for each $n \in \mathbb{N}$.

  For every $n \in \mathbb{N}$, $\phi_1(-n) = -\phi_1(n) = -\phi_2(n) = \phi_2(-n)$.
  Finally, $\phi_1(0) = \phi_1(0 + 0) = \phi_1(0) + \phi_1(0)$, so $\phi_1(0) = 0_R$.
  Similarly, $\phi_2(0) = 0_R$.
  Thus $\phi_1(0) = \phi_2(0)$.

  Hence, we have shown that $\phi_1 = \phi_2$.
\end{proof}

\begin{exer}{(Problem 2)}
  Let $I \subset R$ be an ideal in a commutative ring.
  Describe a bijective correspondence between ideals in $R / I$ and certain ideals in $R$.
\end{exer}

\begin{proof}
  \todo[inline]{
    Tried this for about 10 minutes.
    I think this must be related to some special ideals, so I checked the annihilator, but that doesn't really work.
    I suspect that this problem is fairly simple once I notice what it is, but it'll take time until I notice it.
  }
\end{proof}

\begin{exer}{(Problem 3)}
  Let $I, J \subset R$ be ideals in a commutative ring.
  Let $I + J \subset R$ denote the smallest ideal containing $I$ and $J$.
  Observe that $I + J = \{ i + j \in R : i \in I, j \in J \}$.
  Let $\overline{J} \subset R / I$ denote the image of $J$ under the canonical quotient map, $R \rightarrow R / I$.
  Observe that $\overline{J}$ is an ideal in $S := R / I$.
  Use the universal mapping property of the quotient to show that $R / (I + J) \simeq S / \overline{J}$.
\end{exer}

\begin{proof}
  \todo[inline]{
    Tried this for 20 minutes.
      The problem seems complicated, but it seems that we just need some sort of category theoretical approach to solve this problem.
      I think I can finish it in the next 20 minutes.
      The universal mapping property of the quotient is proposition 6 in the handouts.
    }
  \begin{figure}
    \includegraphics[width=.5\linewidth]{problem3_partial_delete.jpeg}
      \caption{deletethis}
    \label{fig:mydelete}
  \end{figure}
\end{proof}

\begin{exer}{(Problem 4)}
  Let $R$ be a commutative ring and $f(x) = \sum_{i=0}^{n} a_ix^i \in R[x]$ a non-zero polynomial of degree $n$.
  Suppose that $a_n \in R^{\times}$.
  Let $J = (f(x))$.
  Prove that every element of $R[x]/J$ may be written in exactly one way in the form $\sum_{i=0}^{n - 1}r_ix^i + J$ with $r_0, r_1, \cdots, r_{n - 1} \in R$.
\end{exer}

\begin{proof}
  \todo[inline]{
    Tried this for 10 minutes.
    This problem seems wrong.
    See Figure \ref{fig:counterexampledelete}.
  }
  \begin{figure}
    \includegraphics[width=.5\linewidth]{problem4_wrong_delete.jpeg}
    \caption{Problem 4}
    \label{fig:counterexampledelete}
  \end{figure}
\end{proof}

\begin{exer}{(Problem 5)}
  $ $
  \begin{enumerate}
    \item
      Consider the subring $S := \mathbb{Z}[(1 + \sqrt{5})/2] \subset \mathbb{R}$.
      Find a generating set for the abelian group $(S, +)$ with the minimal possible cardinality and justify your answer.
    \item
      Find an explicit principal ideal, $I \subset \mathbb{Z}[x]$, and an explicit ring isomorphism, $\mathbb{Z}[x]/I \simeq S$.
      In the course of justifying your answer make explicit use of the mapping property of polynomials, the universal mapping property of the quotient, and division with remainder.
    \item
      To what familiar ring is $\mathbb{Z}[(1 + \sqrt{5}) / 2] / ((3 - \sqrt{5}) / 2))$ isomorphic?
    \item
      To what familiar ring is $\mathbb{Z}[(1 + \sqrt{5}) / 2] / (2 + \sqrt{5})$ isomorphic?
  \end{enumerate}
\end{exer}

\begin{proof}
  $ $
  \begin{enumerate}
    \item 
      Suppose a generating set is a singleton.
      Let $x \in S$ be such an element.
      Then $kx = 1$ for some $k \in \mathbb{Z}$ because we must be able to obtain $1$ by adding or subracting $x$ finitely many times.
      $k \ne 0$, so this implies that $x = 1 / k$.
      Then $x \in \mathbb{Q}$.
      However, $(1 + \sqrt{5}) / 2 \notin \mathbb{Q}$.
      $(\mathbb{Q}, +)$ is an abelian group, so it is closed under addition and subtraction.
      Therefore, a generating set cannot be a singleton.

      We claim that $\{ 1, (1 + \sqrt{5}) / 2 \}$ is a generating set.
      Let $s \in S$ be given.
      Then $s$ is a real number such that $s = \sum_{i=0}^{\infty} r_i((1 + \sqrt{5}) / 2)^i$.
      Since this is $\mathbb{R}$, the $\sum$ means limits.
      Since $\abs{((1 + \sqrt{5}) / 2)^i} > 1$ for each $i > 0$, there must exist an $N \in \mathbb{N}$ such that $\forall i \geq N, r_i = 0$.
      Then $s = \sum_{i=0}^{N} r_i((1 + \sqrt{5}) / 2)^i$.

      Since $(1 + \sqrt{5}) / 2$ is a root to the equation $x^2 - x - 1 = 0$, we know that it satisfies $x^2 = x + 1$.
      By applying this repeatedly, $((1 + \sqrt{5}) / 2)^n$ can be expressed as a linear combination of $(1 + \sqrt{5}) / 2$ and $1$ over $\mathbb{Z}$.
      Therefore, $s$ can be expressed as a linear combination of $(1 + \sqrt{5}) / 2$ and $1$ over $\mathbb{Z}$.
      A linear combination of two numbers over $\mathbb{Z}$ can be expressed as a finite sequence of addition and subtraction of the two numbers, so $\{ 1, (1 + \sqrt{5}) / 2 \}$ is indeed a generator of $(S, +)$.
  \end{enumerate}
\end{proof}

\end{document}


