\documentclass[12pt, psamsfonts]{amsart}

%-------Packages---------
\usepackage{amssymb,amsfonts}
\usepackage{fullpage}
\usepackage{todonotes}
\usepackage{physics}
\usepackage[all,arc]{xy}
\usepackage{enumerate}
\usepackage{mathrsfs}
\usepackage{theoremref}
\usepackage{graphicx}
\usepackage[bookmarks]{hyperref}

%--------Theorem Environments--------
%theoremstyle{plain} --- default
\newtheorem{thm}{Theorem}[section]
\newtheorem{cor}[thm]{Corollary}
\newtheorem{prop}[thm]{Proposition}
\newtheorem{lem}[thm]{Lemma}
\newtheorem{conj}[thm]{Conjecture}
\newtheorem{quest}[thm]{Question}

\theoremstyle{definition}
\newtheorem{defn}[thm]{Definition}
\newtheorem{defns}[thm]{Definitions}
\newtheorem{con}[thm]{Construction}
\newtheorem{exmp}[thm]{Example}
\newtheorem{exmps}[thm]{Examples}
\newtheorem{notn}[thm]{Notation}
\newtheorem{notns}[thm]{Notations}
\newtheorem{addm}[thm]{Addendum}
\newtheorem*{exer}{Exercise}

\theoremstyle{remark}
\newtheorem{rem}[thm]{Remark}
\newtheorem{rems}[thm]{Remarks}
\newtheorem{warn}[thm]{Warning}
\newtheorem{sch}[thm]{Scholium}

\DeclareMathOperator{\Hom}{Hom}
\DeclareMathOperator{\cont}{cont}
\DeclareMathOperator{\ord}{ord}
\DeclareMathOperator{\Id}{Id}

\makeatletter
\let\c@equation\c@thm
\makeatother
\numberwithin{equation}{section}

\bibliographystyle{plain}

\begin{document}

\title{Math 601 (Due 10/2)}
\author{Hidenori Shinohara}
\maketitle

\tableofcontents

\section{Factorization in Integral Domains}

\begin{exer}{(Problem 1)}
  Let $R = \mathbb{Z}$.
  Compute the content of the following polynomials in $\mathbb{Q}[x]$.
  The content is an element of the quotient group, $\mathbb{Q}^* / \mathbb{Z}^* \simeq \mathbb{Q}^* / \{ \pm 1 \}$.
  \begin{itemize}
    \item
      $f(x) = 2x^2 - 6x + 28$.
    \item
      $g(x) = \frac{2}{3}x^2 - \frac{3}{5}x + \frac{7}{11}$.
  \end{itemize}
\end{exer}

\begin{proof}
  $ $
  \begin{itemize}
    \item
      By property (i) of the content, $\cont(f(x)) = \gcd(2, -6, 28) = 2$.
    \item
      $\cont(g(x)) = 2^{o_2(g(x))}3^{o_3(g(x))}5^{o_5(g(x))}\cdots$.
      \begin{align*}
        o_2(f(x))
          &= \min \{ \ord_2(2/3), \ord_2(-3/5), \ord_2(7/11) \} \\
          &= \min \{ \ord_2(2) - \ord_2(3), \ord_2(-3) - \ord_2(5), \ord_2(7) - \ord_2(11) \} \\
          &= \min \{ 1 - 0, 0 - 0, 0 - 0 \} \\
          &= 0. \\
        o_3(f(x))
          &= \min \{ \ord_3(2/3), \ord_3(-3/5), \ord_3(7/11) \} \\
          &= \min \{ \ord_3(2) - \ord_3(3), \ord_3(-3) - \ord_3(5), \ord_3(7) - \ord_3(11) \} \\
          &= \min \{ 0 - 1, 1 - 0, 0 - 0 \} \\
          &= -1. \\
        o_5(f(x))
          &= \min \{ \ord_5(2/3), \ord_5(-3/5), \ord_5(7/11) \} \\
          &= \min \{ \ord_5(2) - \ord_5(3), \ord_5(-3) - \ord_5(5), \ord_5(7) - \ord_5(11) \} \\
          &= \min \{ 0 - 0, 0 - 1, 0 - 0 \} \\
          &= -1. \\
        o_7(f(x))
          &= \min \{ \ord_7(2/3), \ord_7(-3/5), \ord_7(7/11) \} \\
          &= \min \{ \ord_7(2) - \ord_7(3), \ord_7(-3) - \ord_7(5), \ord_7(7) - \ord_7(11) \} \\
          &= \min \{ 0 - 0, 0 - 0, 1 - 0 \} \\
          &= 0. \\
        o_{11}(f(x))
          &= \min \{ \ord_{11}(2/3), \ord_{11}(-3/5), \ord_{11}(7/11) \} \\
          &= \min \{ \ord_{11}(2) - \ord_{11}(3), \ord_{11}(-3) - \ord_{11}(5), \ord_{11}(7) - \ord_{11}(11) \} \\
          &= \min \{ 0 - 0, 0 - 0, 0 - 1 \} \\
          &= -1.
      \end{align*}
      Therefore, $\cont(g(x)) = 2^03^{-1}5^{-1}7^{0}11^{-1} = \frac{1}{165}$.
  \end{itemize}
\end{proof}

\begin{exer}{(Problem 2)}
  $ $
  \begin{itemize}
    \item
      Prove that if $f(x) = \sum_{i=0}^{n} a_ix^i \in R[x]$, then $\cont(f(x)) = \gcd(a_0, \cdots, a_n)$.
    \item
      For $b \in F^*$, $\cont(b \cdot f(x)) = b \cdot \cont(f(x))$.
  \end{itemize}
\end{exer}

\begin{proof}
  $ $
  \begin{itemize}
    \item
      By Proposition 13 of P.287 (Dummit and Foote), $\gcd(up_1^{a_1} \cdots p_n^{a_n}, vp_1^{b_1} \cdots p_n^{b_n}) = p_1^{\min \{ a_1, b_1 \}} \cdots p_n^{\min \{ a_n, b_n \}}$.
      By mathematical induction, this property holds for greatest common divisors of $n + 1$ elements of $R$.
      Let $f(x) = \sum_{i=0}^{n} a_ix^i \in R[x]$ be given.
      Choose distinct (not equivalent) irreducible elements $p_1, \cdots, p_m \in R$, non-negative integers $a_{i, j}$, and units $u_i$ such that $a_i = u_ip_1^{a_{i, 1}} \cdots p_m^{a_{i, m}}$.
      Since $R$ is a UFD, it is possible to pick such $p_i, a_{i, j}, u_i$.
      Then $o_{p_j} = \min \{ a_{0, j}, \cdots, a_{n, j} \}$ for each $j$.
      Thus $\cont(f(x)) = \prod p_j^{o_{p_j}} = \prod p_j^{\min \{ a_{0, j}, \cdots, a_{n, j} \}} = \gcd(a_0, \cdots, a_n)$
    \item
      As discussed below the definition of $\ord_p$ in the handout ``Factorization in Integral Domains," $\ord_p$ is a group homomorphism from a multiplicative group $F^*$ to an additive group $\mathbb{Z}$.

      Let $f(x) = \sum_{i=0}^{n} a_ix^i \in R[x], b \in R$ be given.
      Choose distinct (not equivalent) irreducible elements $p_1, \cdots, p_m \in R$, non-negative integers $a_{i, j}, b_j$, and units $u_i, w$ such that $a_i = u_ip_1^{a_{i, 1}} \cdots p_m^{a_{i, m}}$ and $b = wp_1^{b_1} \cdots p_m^{b_m}$.
      Then $b \cdot f(x) = \sum_{i=0}^{n} (ba_i)x_i$, and $ba_i = (u_iw)p_1^{a_{i, 1} + b_1} \cdots p_m^{a_{i, m} + b_m}$.
      Since $\ord_p$ is a group homomorphism for each $p$, we have that for each $i, j$, $\ord_{p_j}(ba_i) = \ord_{p_j}(b) + \ord_{p_j}(a_i)$.
      \begin{align*}
        o_{p_j}(b \cdot f(x))
          &= \min\{ \ord_{p_j}(ba_i) \mid 0 \leq i \leq n \} \\
          &= \min\{ \ord_{p_j}(b) + \ord_{p_j}(a_i) \mid 0 \leq i \leq n \} \\
          &= \ord_{p_j}(b) + \min\{ \ord_{p_j}(a_i) \mid 0 \leq i \leq n \} \\
          &= \ord_{p_j}(b) + o_{p_j}(f(x)).
      \end{align*}

      Therefore, 
      \begin{align*}
        \cont(f(x))
          &= \prod_{j} p_j^{o_{p_j}(b \cdot f(x))} \\
          &= \prod_{j} p_j^{\ord_{p_j}(b) + o_{p_j}(f(x))} \\
          &= \prod_{j} p_j^{\ord_{p_j}(b)} \prod_j p_j^{o_{p_j}(f(x))} \\
          &= b\prod_j p_j^{o_{p_j}(f(x))} \\
          &= b\cont(f(x)).
      \end{align*}
  \end{itemize}
\end{proof}

\begin{exer}{(Problem 3)}
  Determine if the given polynomial is an irreducible element of the given integral domain.
  \begin{itemize}
    \item
      $3x^3 - 15x^2 - 21 \in \mathbb{Z}[x]$.
    \item
      $3x^3 - 15x^2 - 21 \in \mathbb{Q}[x]$.
  \end{itemize}
\end{exer}

\begin{proof}
  $ $
  \begin{itemize}
    \item
      $3x^3 - 15x^2 - 21 = 3(x^3 - 5x^2 - 7)$.
      Since neither 3 nor $x^3 - 5x^2 - 7$ is a unit, $3x^3 - 15x^2 - 21$ is not irreducible.
    \item
      We claim that $f(x) = 3x^3 - 15x^2 - 21 \in \mathbb{Q}[x]$ is irreducible.
      The content is $\gcd(3, -15, -21) = 3$, so let $f_0(x) = f(x) / 3 = x^3 - 5x^2 - 7$.
      Then $f_0(x)$ is primitive in $\mathbb{Z}[x]$.
      As discussed in class on 9/27, $f_0(x)$ is irreducible if and only if it has no linear factors.
      If $mx + n$ is a factor of $f_0(x)$, then $m \mid 1$ and $n \mid -7$.
      Thus $m \in \{ -1, 1 \}$ and $n \in \{ -1, 1, -7, 7 \}$.
      This implies that it is sufficient to check $x + 1, x + 7, x - 1, x - 7$ because the other possibilities can be obtained by multiplying -1.

      \begin{itemize}
        \item
          $f(x) = (3 x^{2} - 18 x + 18)(x + 1) - 39$.
        \item
          $f(x) = (3 x^{2} - 12 x - 12)(x - 1) - 33$.
        \item
          $f(x) = (3 x^{2} - 36 x + 252)(x + 7) - 1785$.
        \item
          $f(x) = (3 x^{2} + 6 x + 42)(x - 7) + 273$.
      \end{itemize}

      Since none of them is a factor of $f(x)$, $f_0(x)$ is irreducible.
      Since $3$ is a unit in $\mathbb{Q}[x]$, $f(x) = 3f_0(x)$ is irreducible in $\mathbb{Q}[x]$.
  \end{itemize}
\end{proof}

\begin{exer}{(Problem 4(i))}
  The polynomial $f(x) = x^5 + 8x^4 + 7x^3 - 30x^2 + 42x - 36 \in \mathbb{Z}[x]$, reduced modulo 7 and 11 factors as a product of irreducible polynomials as follows:
  \begin{itemize}
    \item
      $x^5 + 8x^4 + 7x^3 - 30x^2 + 42x - 36 = (x^2 - 3x + 1)(x^3 + 4x^2 + 4x - 1) \in \mathbb{Z} / 7\mathbb{Z}[x]$.
    \item
      $x^5 + 8x^4 + 7x^3 - 30x^2 + 42x - 36 = (x^2 + 4x + 5)(x^3 + 4x^2 + 8x + 6) \in \mathbb{Z} / 11\mathbb{Z}[x]$.
  \end{itemize}
\end{exer}

\begin{proof}
  Suppose $f(x) = (x^2 + ax + b)(x^3 + cx^2 + dx + e)$ for some $a, b, c, d, e \in \mathbb{Z}$.
  Since the leading coefficient of $f(x)$ is 1, if $f(x)$ is a product of two polynomials, we can assume that their leading coefficients are 1 without loss of generality.
  If $f(x)$ factors as above, then we can obtain a factorization of $f(x)$ in $\mathbb{Z} / 7\mathbb{Z}[x]$ and $\mathbb{Z} / 11\mathbb{Z}[x]$ by taking modulo 7 and 11 of each coefficient.
  We will try to reverse engineer that.
  \begin{itemize}
    \item
      $a \equiv -3 \pmod 7$ and $a \equiv 4 \pmod {11}$ are satisfied by $a = 4$.
    \item
      $b \equiv 1 \pmod 7$ and $b \equiv 5 \pmod {11}$ are satisfied by $b = -6$.
    \item
      $c \equiv 4 \pmod 7$ and $c \equiv 4 \pmod {11}$ are satisfied by $c = 4$.
    \item
      $d \equiv 4 \pmod 7$ and $d \equiv 8 \pmod {11}$ are satisfied by $d = -3$.
    \item
      $c \equiv -1 \pmod 7$ and $c \equiv 6 \pmod {11}$ are satisfied by $c = 6$.
  \end{itemize}
  There are other values that satisfy such equations (e.g., $a = 4 + 77 = 81$), but it seems reasonable to start with numbers with small absolute values since each coefficient of $f(x)$ has a relatively small absolute value.
  It turns out that this set of coefficients indeed gives a factorization of $f(x)$.
  In other words, $f(x) = (x^2 + 4x - 6)(x^3 + 4x^2 - 3x + 6)$.

  We will check if $x^2 + 4x - 6$ and $x^3 + 4x^2 - 3x + 6$ are irreducible.
  \begin{itemize}
    \item
      $x^2 + 4x - 6$ is irreducible by the Eisenstein irreducibility criterion.
      Let $P = (2)$.
      $4 \in P = (2)$ and $6 \in P = (2)$, but $6 \notin P^2$.
      Thus $x^2 + 4x - 6$ is irreducible.
    \item
      Is $x^3 + 4x^2 - 3x + 6$ irreducible?
      Since the content is $1$, this polynomial cannot be factored as a product of a non-unit integer and a polynomial of degree 3.
      Thus if this polynomial is not irreducible, it must factor as a product of a polynomial of degree 1 and one of degree 2.
      In other words, $x^3 + 4x^2 - 3x + 6 = (x + a)(x^2 + bx + c)$ for some $a, b, c \in \mathbb{Z}$.
      We can assume that the leading coefficient of each factor is 1 because the leading coefficient of $x^3 + 4x^2 - 3x + 6$ is 1.
      Then $ac = 6$, so $a \mid 6$.
      Thus $a \in \{ -1, 1, -2, 2, 3, -3, 6, -6 \}$.
      \begin{align*}
        x^3 + 4x^2 - 3x + 6 &= (x^{2} + 10 x + 57)(x - 6) + 348, \\
        x^3 + 4x^2 - 3x + 6 &= (x^{2} + 7 x + 18)(x - 3) + 60, \\
        x^3 + 4x^2 - 3x + 6 &= (x^{2} + 6 x + 9)(x - 2) + 24, \\
        x^3 + 4x^2 - 3x + 6 &= (x^{2} + 5 x + 2)(x - 1) + 8, \\
        x^3 + 4x^2 - 3x + 6 &= (x^{2} + 3 x - 6)(x + 1) + 12, \\
        x^3 + 4x^2 - 3x + 6 &= (x^{2} + 2 x - 7)(x + 2) + 20, \\
        x^3 + 4x^2 - 3x + 6 &= (x^{2} + x - 6)(x + 3) + 24, \\
        x^3 + 4x^2 - 3x + 6 &= (x^{2} - 2 x + 9)(x + 6) + -48.
      \end{align*}
      Therefore, $x^3 + 4x^2 - 3x + 6$ is irreducible.
  \end{itemize}
  Hence, $f(x)$ is a product of an irreducible polynomial of degree 2 and one of degree 3.
\end{proof}

\section{Rings of Fractions}
\begin{exer}{(Problem 1 (iii))}
  Prove that the natural map $i: R \rightarrow S^{-1}R$, which maps $r$ to $\frac{r}{1}$ is an injective ring homomorphism.
\end{exer}

\begin{proof}
  $ $
  \begin{itemize}
    \item
      Ring homomorphism?
      \begin{itemize}
        \item
          For all $r, s \in R$, $i(rs) = \frac{rs}{1} = \frac{r}{1} \cdot \frac{s}{1} = i(r)i(s)$.
        \item
          For all $r, s \in R$, $i(r + s) = \frac{r + s}{1} = \frac{r}{1} + \frac{s}{1} = i(r) + i(s)$.
      \end{itemize}
      Therefore, $i$ is indeed a ring homomorphism.
    \item
      Injective?
      It suffices to check that $\ker(i) = \{ 1 \}$.
      Let $r \in R$ such that $\ker(r)$ is the multiplicative identity in $S^{-1}R$.
      By definition, $\ker(r) = \frac{1}{1}$.
      Thus $\frac{r}{1} = \frac{1}{1}$, so $r \cdot 1 - 1 \cdot 1 = 0$.
      This means $r = 1$, so $\ker(i) = \{ 1 \}$.
  \end{itemize}
  Therefore, $i$ is indeed an injective ring homomorphism.
\end{proof}

\begin{exer}{(Problem 1(iv))}
  Prove that given a ring homomorphism $h: R \rightarrow T$, such that $h(s) \in T^{\star}$ for every $s \in S$, there exists a unique ring homomorphism $\lambda: S^{-1}R \rightarrow T$, such that $h = \lambda \circ i$.
\end{exer}

\begin{proof}
  Suppose such a $\lambda$ exists.
  Then for all $r \in R$, $h(r) = (\lambda \circ i)(r) = \lambda(r / 1)$.
  Therefore, $\lambda(r / 1) = h(r)$.
  Let $s \in S$.
  Then $1_T = \lambda(1 / 1) = \lambda((s / 1) \cdot (1 / s)) = \lambda(s / 1)\lambda(1 / s)$.
  Therefore, $\lambda(1 / s) = \lambda(s / 1)^{-1} = h(s)^{-1}$.
  This implies that $\lambda(r / s) = \lambda(r / 1)\lambda(1 / s) = h(r)h(s)^{-1}$.

  In other words, if such a $\lambda$ exists, it must map $r / s$ to $h(r)h(s)^{-1}$.
  This proves the uniqueness.
  We will show that such a function is indeed well defined and it is a ring homomorphism.
  \begin{itemize}
    \item
      Well-defined?
      Since $h(s) \in T^{\star}$ for each $s \in S$, $h(s)^{-1}$ is well defined.
      Let $r / s = r' / s' \in S^{-1}R$ be given.
      Then $rs' = r's$.
      Since $h$ is a ring homomorphism, $h(r)h(s') = h(r')h(s)$.
      Therefore, $\lambda(r / s) = h(r)h(s)^{-1} = h(r')h(s')^{-1} = \lambda(r' / s')$.
    \item
      Ring homomorphism?
      Let $r / s, r' / s' \in S^{-1}R$.
      \begin{align*}
        \lambda(\frac{r}{s} \cdot \frac{r'}{s'})
          &= \lambda(\frac{rr'}{ss'}) \\ 
          &= h(rr')h(ss')^{-1} \\
          &= h(r)h(r')h(s)^{-1}h(s')^{-1} \\
          &= h(r)h(s)^{-1}h(r')h(s')^{-1} \\
          &= \lambda(\frac{r}{s})\lambda(\frac{r'}{s'}). \\
        \lambda(\frac{r}{s} + \frac{r'}{s'})
          &= \lambda(\frac{rs' + r's}{ss'}) \\
          &= h(rs' + r's)h(ss')^{-1} \\
          &= (h(r)h(s') + h(r')h(s))h(s)^{-1}h(s')^{-1} \\
          &= h(r)h(s)^{-1} + h(r')h(s')^{-1} \\
          &= \lambda(\frac{r}{s}) + \lambda(\frac{r'}{s'}).
      \end{align*}
    \item
      Commutes?
      For any $r \in R$, $\lambda(i(r)) = \lambda(r / 1) = h(r)h(1)^{-1} = h(r)$.
      Therefore, $\lambda \circ i$ is indeed $h$.
  \end{itemize}
\end{proof}

\section{The Quadratic Equation $x^2 - 2y^2 = n$}
\begin{exer}{(Problem 15)}
  Find a solution to $x^2 - 2y^2 = 7$.
\end{exer}

\begin{proof}
  $3^2 - 2 \cdot 1^2 = 9 - 2 = 7$.
  Thus $(x, y) = (3, 1)$ is a solution to $x^2 - 2y^2 = 7$.
\end{proof}

\begin{exer}{(Problem 16)}
  Is 7 irreducible in $\mathbb{Z}[\sqrt{2}]$?
  If not, find a factorization into irreducible elements.
\end{exer}

\begin{proof}
  By Problem 3 from the previous assignment, we know that $\alpha \in \mathbb{Z}[\sqrt{2}]$ is a unit if and only if $N(\alpha) = \pm 1$.
  We will use this result in this solution.

  By Problem 15, we know that $7 = (3 + \sqrt{2})(3 - \sqrt{2})$.
  Since $N(3 + \sqrt{2}) = N(3 - \sqrt{2}) = 7 \ne \pm 1$, 7 can be expressed as a product of two non-unit elements, so 7 is not irreducible.

  Suppose $3 + \sqrt{2} = (a + b\sqrt{2})(c + d\sqrt{2})$ for some $a, b, c, d \in \mathbb{Z}$.
  By Problem 2 from the previous assignment, we know that $N(3 + \sqrt{2}) = N(a + b\sqrt{2})N(c + d\sqrt{2})$.
  Since $N$ maps $\mathbb{Z}[\sqrt{2}]$ into integers, exactly one of $N(a + b\sqrt{2})$ and $N(c + d\sqrt{2}$ must be 1 or -1, and the other one is 7 or -7.
  Therefore, one of $a + b\sqrt{2}$ or $c + d\sqrt{2}$ is a unit, so $3 + \sqrt{2}$ is irreducible.

  Similarly, if $3 - \sqrt{2} = (a' + b'\sqrt{2})(c' + d'\sqrt{2})$, then $7 = N(3 - \sqrt{2}) = N(a' + b'\sqrt{2})N(c' + d'\sqrt{2})$.
  Therefore, one of $a' + b'\sqrt{2}$ or $c' + d'\sqrt{2}$ is a unit, so $3 - \sqrt{2}$ is irreducible.
\end{proof}

\begin{exer}{(Problem 17)}
  Let $p \in \mathbb{Z} \setminus \{ 0 \}$ and suppose $\alpha\beta = p$ in $\mathbb{Z}[\sqrt{2}]$.
  Show that $\beta = c\gamma(\alpha)$ with $c \in \mathbb{Q}$.
\end{exer}

\begin{proof}
  Choose $a, b, c, d \in \mathbb{Z}$ such that $a + b\sqrt{2} = \beta, c + d\sqrt{2} = \alpha$.
  Since $\alpha\beta = p \ne 0$, $\alpha \ne 0$.
  This implies at least one of $c$ or $d$ is nonzero.
  Therefore, $\gamma(\alpha) = c - d\sqrt{2} \ne 0$.

  We have $\alpha\beta = (ac + 2bd) + \sqrt{2}(ad + bc)$.
  Since $\alpha\beta \in \mathbb{Z}$, $ad + bc = 0$.

  \begin{align*}
    \frac{\beta}{\gamma(\alpha)}
      &= \frac{a + b\sqrt{2}}{c - d\sqrt{2}} \\
      &= \frac{(a + b\sqrt{2})(c + d\sqrt{2})}{c^2 - 2d^2} \\
      &= \frac{(ac + 2bd) + (ad + bc)\sqrt{2}}{c^2 - 2d^2} \\
      &= \frac{ac + 2bd}{c^2 - 2d^2}.
  \end{align*}

  Therefore, $\frac{\beta}{\gamma(\alpha)}  = \frac{ac + 2bd}{c^2 - 2d^2} \in \mathbb{Q}$.
  In other words, $\beta = \frac{ac + 2bd}{c^2 - 2d^2}\gamma(\alpha)$.
\end{proof}

\begin{exer}{(Problem 18)}
  Let $p \in \mathbb{Z}$ be an odd prime.
  Show that $p = N(\alpha)$ for some $\alpha \in \mathbb{Z}[\sqrt{2}]$ if and only if $p$ is not irreducible as an element of $\mathbb{Z}[\sqrt{2}]$.
\end{exer}

\begin{proof}
  By Problem 3 from the previous assignment, we know that $\alpha \in \mathbb{Z}[\sqrt{2}]$ is a unit if and only if $N(\alpha) = \pm 1$.
  We will use this result in this solution.

  Suppose $p = N(\alpha)$ for some $\alpha \in \mathbb{Z}[\sqrt{2}]$.
  Since $N(\alpha) = \alpha\gamma(\alpha)$, $p$ can be written as a product of $\alpha$ and $\gamma(\alpha)$.
  \begin{itemize}
    \item
      $N(\alpha) = p \ne \pm 1$, so $\alpha$ is not a unit.
    \item
      Since $N(\gamma(\alpha)) = \gamma(\alpha)\gamma(\gamma(\alpha)) = \gamma(\alpha)\alpha = N(\alpha) = p \ne \pm 1$, $\gamma(\alpha)$ is not a unit.
  \end{itemize}
  Therefore, $p$ is a product of two non-unit elements $\alpha, \gamma(\alpha)$, so $p$ is not irreducible.

  On the other hand, suppose that $p$ is not irreducible as an element of $\mathbb{Z}[\sqrt{2}]$.
  Then $p = \alpha\beta$ where $\alpha, \beta \in \mathbb{Z}[\sqrt{2}]$ are non-unit elements.
  Then $N(p) = N(\alpha)N(\beta)$.

  \begin{itemize}
    \item
      $N(p) = p^2$ because $p$ is an integer.
    \item
      $N(\alpha) \ne \pm 1$ because $\alpha$ is not a unit.
    \item
      $N(\beta) \ne \pm 1$ because $\beta$ is not a unit.
  \end{itemize}
  Since $N(\alpha), N(\beta)$ are both integers, $N(\alpha) = N(\beta) = p$ or $N(\alpha) = N(\beta) = -p$.
  If $N(\alpha) = p$, then we are done.
  If $N(\alpha) = -p$, then $N(\alpha(1 + \sqrt{2})) = N(\alpha)N(1 + \sqrt{2}) = (-p)(-1) = p$.
\end{proof}

\begin{exer}{(Problem 19)}
  Let $p \in \mathbb{Z}$ be an odd prime.
  Show that $x^2 - 2y^2 = p$ has a solution if and only if $p$ is not irreducible in $\mathbb{Z}[\sqrt{2}]$.
\end{exer}

\begin{proof}
  Let an odd prime $p$ be given.
  There exists an $\alpha \in \mathbb{Z}[\sqrt{2}]$ such that $p = N(\alpha)$ if and only if there exist $x, y \in \mathbb{Z}$ such that $p = x^2 - 2y^2$ because $N(x + \sqrt{2}y) = x^2 - 2y^2$.
  By combining this with the results of Problem 18, we have $x^2 - 2y^2 = p$ has a solution if and only if $p$ is not irreducible in $\mathbb{Z}[\sqrt{2}]$.
\end{proof}

\end{document}
