\documentclass[12pt, psamsfonts]{amsart}

%-------Packages---------
\usepackage{amssymb,amsfonts}
\usepackage{fullpage}
\usepackage{todonotes}
\usepackage{physics}
\usepackage[all,arc]{xy}
\usepackage{enumerate}
\usepackage{mathrsfs}
\usepackage{theoremref}
\usepackage{graphicx}
\usepackage[bookmarks]{hyperref}

%--------Theorem Environments--------
%theoremstyle{plain} --- default
\newtheorem{thm}{Theorem}[section]
\newtheorem{cor}[thm]{Corollary}
\newtheorem{prop}[thm]{Proposition}
\newtheorem{lem}[thm]{Lemma}
\newtheorem{conj}[thm]{Conjecture}
\newtheorem{quest}[thm]{Question}

\theoremstyle{definition}
\newtheorem{defn}[thm]{Definition}
\newtheorem{defns}[thm]{Definitions}
\newtheorem{con}[thm]{Construction}
\newtheorem{exmp}[thm]{Example}
\newtheorem{exmps}[thm]{Examples}
\newtheorem{notn}[thm]{Notation}
\newtheorem{notns}[thm]{Notations}
\newtheorem{addm}[thm]{Addendum}
\newtheorem*{exer}{Exercise}

\theoremstyle{remark}
\newtheorem{rem}[thm]{Remark}
\newtheorem{rems}[thm]{Remarks}
\newtheorem{warn}[thm]{Warning}
\newtheorem{sch}[thm]{Scholium}

\DeclareMathOperator{\Hom}{Hom}
\DeclareMathOperator{\Id}{Id}

\makeatletter
\let\c@equation\c@thm
\makeatother
\numberwithin{equation}{section}

\bibliographystyle{plain}

\begin{document}

\title{Math 601 (Due 10/2)}
\author{Hidenori Shinohara}
\maketitle

\tableofcontents

\section{The Quadratic Equation $x^2 - 2y^2 = n$}
\begin{exer}{(Problem 15)}
  Find a solution to $x^2 - 2y^2 = 7$.
\end{exer}

\begin{proof}
  $3^2 - 2 \cdot 1^2 = 9 - 2 = 7$.
  Thus $(x, y) = (3, 1)$ is a solution to $x^2 - 2y^2 = 7$.
\end{proof}

\begin{exer}{(Problem 16)}
  Is 7 irreducible in $\mathbb{Z}[\sqrt{2}]$?
  If not, find a factorization into irreducible elements.
\end{exer}

\begin{proof}
  By Problem 3 from the previous assignment, we know that $\alpha \in \mathbb{Z}[\sqrt{2}]$ is a unit if and only if $N(\alpha) = \pm 1$.
  We will use this result in this solution.

  By Problem 15, we know that $7 = (3 + \sqrt{2})(3 - \sqrt{2})$.
  Since $N(3 + \sqrt{2}) = N(3 - \sqrt{2}) = 7 \ne \pm 1$, 7 can be expressed as a product of two non-unit elements, so 7 is not irreducible.

  Suppose $3 + \sqrt{2} = (a + b\sqrt{2})(c + d\sqrt{2})$ for some $a, b, c, d \in \mathbb{Z}$.
  By Problem 2 from the previous assignment, we know that $N(3 + \sqrt{2}) = N(a + b\sqrt{2})N(c + d\sqrt{2})$.
  Since $N$ maps $\mathbb{Z}[\sqrt{2}]$ into integers, exactly one of $N(a + b\sqrt{2})$ and $N(c + d\sqrt{2}$ must be 1 or -1, and the other one is 7 or -7.
  Therefore, one of $a + b\sqrt{2}$ or $c + d\sqrt{2}$ is a unit, so $3 + \sqrt{2}$ is irreducible.

  Similarly, if $3 - \sqrt{2} = (a' + b'\sqrt{2})(c' + d'\sqrt{2})$, then $7 = N(3 - \sqrt{2}) = N(a' + b'\sqrt{2})N(c' + d'\sqrt{2})$.
  Therefore, one of $a' + b'\sqrt{2}$ or $c' + d'\sqrt{2}$ is a unit, so $3 - \sqrt{2}$ is irreducible.
\end{proof}

\end{document}
