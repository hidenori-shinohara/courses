\documentclass[12pt, psamsfonts]{amsart}

%-------Packages---------
\usepackage{amssymb,amsfonts}
\usepackage{fullpage}
\usepackage{tikz-cd}
\usepackage{todonotes}
\usepackage{physics}
\usepackage{listings}
\usepackage[all,arc]{xy}
\usepackage{enumerate}
\usepackage{enumitem}
\usepackage{mathrsfs}
\usepackage{theoremref}
\usepackage{graphicx}
\usepackage[bookmarks]{hyperref}

%--------Theorem Environments--------
%theoremstyle{plain} --- default
\newtheorem{thm}{Theorem}[section]
\newtheorem{cor}[thm]{Corollary}
\newtheorem{prop}[thm]{Proposition}
\newtheorem{lem}[thm]{Lemma}
\newtheorem{conj}[thm]{Conjecture}
\newtheorem{quest}[thm]{Question}

\theoremstyle{definition}
\newtheorem{defn}[thm]{Definition}
\newtheorem{defns}[thm]{Definitions}
\newtheorem{con}[thm]{Construction}
\newtheorem{exmp}[thm]{Example}
\newtheorem{exmps}[thm]{Examples}
\newtheorem{notn}[thm]{Notation}
\newtheorem{notns}[thm]{Notations}
\newtheorem{addm}[thm]{Addendum}
\newtheorem*{exer}{Exercise}

\theoremstyle{remark}
\newtheorem{rem}[thm]{Remark}
\newtheorem{rems}[thm]{Remarks}
\newtheorem{warn}[thm]{Warning}
\newtheorem{sch}[thm]{Scholium}

\DeclareMathOperator{\Hom}{Hom}
\DeclareMathOperator{\Id}{Id}
\DeclareMathOperator{\End}{End}
\DeclareMathOperator{\Aut}{Aut}
\DeclareMathOperator{\ord}{ord}

\makeatletter
\let\c@equation\c@thm
\makeatother
\numberwithin{equation}{section}

\bibliographystyle{plain}

\begin{document}

\title{MyTitle}
\author{Hidenori Shinohara}
\maketitle

\tableofcontents

\section{Modules}

\begin{exer}{(Problem 6)}
  Take four $4 \times 4$ matrices with integer entries and check if the abelian group presented by the matrix is cyclic.
\end{exer}

\begin{proof}
  \begin{align*}
    \begin{bmatrix} -166 & -74 & 254 & 347\\ 140 & -93 & 246 & 425\\ -196 & 57 & -363 & 202\\ 325 & 257 & 314 & -389 \end{bmatrix}
      &\rightarrow \begin{bmatrix} 18444530375 & 1 & 1 & 1 \end{bmatrix} \\
    \begin{bmatrix} 237 & -81 & 332 & -132\\ 95 & 268 & 229 & 498\\ 387 & 213 & 46 & 55\\ 88 & -126 & -380 & -447 \end{bmatrix}
      &\rightarrow \begin{bmatrix} 2610768268 & 1 & 1 & 1 \end{bmatrix} \\
    \begin{bmatrix} -275 & -22 & -207 & -276\\ -469 & -342 & 240 & -101\\ -41 & 455 & 51 & -151\\ 267 & -450 & 98 & -40 \end{bmatrix}
      &\rightarrow \begin{bmatrix} 33644517767 & 1 & 1 & 1 \end{bmatrix} \\
    \begin{bmatrix} 48 & 29 & 22 & -481\\ 388 & -468 & -137 & -491\\ 84 & -352 & 85 & -384\\ -226 & -486 & 102 & -156 \end{bmatrix}
       &= \begin{bmatrix} 13267264454 & 1 & 1 & 1 \end{bmatrix} \\
  \end{align*}
  Each of the groups contains 4 generators, so none of them are cyclic.
\end{proof}

\section{Galois Theory}

\begin{exer}{(Problem 1)}
  Let $F = \mathbb{Q}$.
  Let $L = \mathbb{Q}(\sqrt{7}, \sqrt{-11})$.
  To what familiar group is $\Aut(L/F)$ is isomorphic?
\end{exer}

\begin{proof}
  $[K:\mathbb{Q}(\sqrt{7})] = [K:\mathbb{Q}(\sqrt{-11})] = 2$.
  Since the characteristic of $K$ is not 2, by the argument presented on P.3 of the Galous Theory handout, $\Aut(K/\mathbb{Q}(\sqrt{7}))$ and $\Aut(K/\mathbb{Q}(\sqrt{-11}))$ have 2 elements.
  For instance, $\alpha = \sqrt{7}$ and the minimal monic polynomial is $x^2 - 7$.
  This gives $D = 28$ and two automorphisms in $\Aut(K/\mathbb{Q}(\sqrt{7}))$, the identity map, and $\sigma: \sqrt{D} \mapsto -\sqrt{D}$ as discussed in the handout.
  Similarly, $\Aut(K/\mathbb{Q}(\sqrt{-11}))$ contains the identity map and $\sigma: \sqrt{D} \mapsto -\sqrt{D}$ where $D = -44$.
  \todo[inline,caption={}]{
    Finish this proof.
  }
\end{proof}

\end{document}


