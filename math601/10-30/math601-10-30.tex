\documentclass[12pt, psamsfonts]{amsart}

%-------Packages---------
\usepackage{amssymb,amsfonts}
\usepackage{fullpage}
\usepackage{tikz-cd}
\usepackage{todonotes}
\usepackage{physics}
\usepackage{listings}
\usepackage[all,arc]{xy}
\usepackage{enumerate}
\usepackage{enumitem}
\usepackage{mathrsfs}
\usepackage{theoremref}
\usepackage{graphicx}
\usepackage[bookmarks]{hyperref}

%--------Theorem Environments--------
%theoremstyle{plain} --- default
\newtheorem{thm}{Theorem}[section]
\newtheorem{cor}[thm]{Corollary}
\newtheorem{prop}[thm]{Proposition}
\newtheorem{lem}[thm]{Lemma}
\newtheorem{conj}[thm]{Conjecture}
\newtheorem{quest}[thm]{Question}

\theoremstyle{definition}
\newtheorem{defn}[thm]{Definition}
\newtheorem{defns}[thm]{Definitions}
\newtheorem{con}[thm]{Construction}
\newtheorem{exmp}[thm]{Example}
\newtheorem{exmps}[thm]{Examples}
\newtheorem{notn}[thm]{Notation}
\newtheorem{notns}[thm]{Notations}
\newtheorem{addm}[thm]{Addendum}
\newtheorem*{exer}{Exercise}

\theoremstyle{remark}
\newtheorem{rem}[thm]{Remark}
\newtheorem{rems}[thm]{Remarks}
\newtheorem{warn}[thm]{Warning}
\newtheorem{sch}[thm]{Scholium}

\DeclareMathOperator{\Hom}{Hom}
\DeclareMathOperator{\Id}{Id}
\DeclareMathOperator{\End}{End}
\DeclareMathOperator{\Aut}{Aut}
\DeclareMathOperator{\ord}{ord}

\makeatletter
\let\c@equation\c@thm
\makeatother
\numberwithin{equation}{section}

\bibliographystyle{plain}

\begin{document}

\title{Math 601 (Due 10/30)}
\author{Hidenori Shinohara}
\maketitle

\tableofcontents

\section{Factoring Polynomials with coefficients in Finite Fields}

\begin{exer}{(Problem 1)}
  Consider the Frobenius homomorphism, $F_p: \mathbb{F}_q \rightarrow \mathbb{F}_q$.
  Show that this homomorphism is bijective.
  If $q = p$, identify it with a familiar homomorphism.
\end{exer}

\begin{proof}
  Since $\mathbb{F}_q$ is finite, it suffices to show that $F_p$ is injective.
  $F_p(a) = F_p(b) \implies a^p + (-b)^p = 0 \implies a - b = 0$ if $p \geq 3$.
  The case when $p = 2$ is similar.
  If $q = p$, $\mathbb{F}_q \cong \mathbb{Z}/p\mathbb{Z}$, which is a cyclic additive group generated by 1.
  Since $F_p(1) = 1$, $F_p$ must be the identity homomorphism.
\end{proof}

\begin{exer}{(Problem 2)}
  Let $K$ be a field of characteristic $p$.
  Which polynomials $f(x) \in K[x]$ satisfies $f'(x) = 0$?
\end{exer}

\begin{proof}
  $f'(x) = \sum_{i=1}^{n} ia_ix^i = 0 \iff (\forall i, i \notin (p) \implies a_i = 0)$ since if $i \in (p)$, $ia_i = 0$ regardless of what $a_i$ is.
\end{proof}

\section{Modules}

\begin{exer}{(Problem 6)}
  Take four $4 \times 4$ matrices with integer entries and check if the abelian group presented by the matrix is cyclic.
\end{exer}

\begin{proof}
  \begin{align*}
    \begin{bmatrix} -166 & -74 & 254 & 347\\ 140 & -93 & 246 & 425\\ -196 & 57 & -363 & 202\\ 325 & 257 & 314 & -389 \end{bmatrix}
      &\rightarrow \begin{bmatrix} 18444530375 & 1 & 1 & 1 \end{bmatrix} \\
    \begin{bmatrix} 237 & -81 & 332 & -132\\ 95 & 268 & 229 & 498\\ 387 & 213 & 46 & 55\\ 88 & -126 & -380 & -447 \end{bmatrix}
      &\rightarrow \begin{bmatrix} 2610768268 & 1 & 1 & 1 \end{bmatrix} \\
    \begin{bmatrix} -275 & -22 & -207 & -276\\ -469 & -342 & 240 & -101\\ -41 & 455 & 51 & -151\\ 267 & -450 & 98 & -40 \end{bmatrix}
      &\rightarrow \begin{bmatrix} 33644517767 & 1 & 1 & 1 \end{bmatrix} \\
    \begin{bmatrix} 48 & 29 & 22 & -481\\ 388 & -468 & -137 & -491\\ 84 & -352 & 85 & -384\\ -226 & -486 & 102 & -156 \end{bmatrix}
       &= \begin{bmatrix} 13267264454 & 1 & 1 & 1 \end{bmatrix} \\
  \end{align*}
  Each of the groups contains 4 generators, so none of them are cyclic.
\end{proof}

\section{Galois Theory}

\begin{exer}{(Problem 1)}
  Let $F = \mathbb{Q}$.
  Let $L = \mathbb{Q}(\sqrt{7}, \sqrt{-11})$.
  To what familiar group is $\Aut(L/F)$ is isomorphic?
\end{exer}

\begin{proof}
  $[K:\mathbb{Q}(\sqrt{7})] = [K:\mathbb{Q}(\sqrt{-11})] = 2$.
  Since the characteristic of $K$ is not 2, by the argument presented on P.3 of the Galous Theory handout, $\Aut(K/\mathbb{Q}(\sqrt{7}))$ and $\Aut(K/\mathbb{Q}(\sqrt{-11}))$ have 2 elements.
  For instance, $\alpha = \sqrt{7}$ and the minimal monic polynomial is $x^2 - 7$.
  This gives $D = 28$ and two automorphisms in $\Aut(K/\mathbb{Q}(\sqrt{7}))$, the identity map, and $\sigma: \sqrt{D} \mapsto -\sqrt{D}$ as discussed in the handout.
  Similarly, $\Aut(K/\mathbb{Q}(\sqrt{-11}))$ contains the identity map and $\sigma: \sqrt{D} \mapsto -\sqrt{D}$ where $D = -44$.
  \todo[inline,caption={}]{
    Finish this proof.
  }
\end{proof}

\begin{exer}{(Problem 2)}
  Let $F \subset K$ be a field extension.
  \begin{enumerate}
    \item 
      Prove in at most two sentences that each $\sigma \in \Aut(K / F)$ is an $F$-linear transformation of the $F$-vector space, $K$.
    \item
      Does the same condition hold in general for $\sigma \in \Aut(K)$?
      Prove or give a counterexample.
  \end{enumerate}
\end{exer}

\begin{proof}
  $ $
  \begin{enumerate}
    \item 
      For any $a \in F$ and $v, w \in K$, $\sigma(av + w) = \sigma(a)\sigma(v) + \sigma(w) = a\sigma(v) + \sigma(w)$, so $\sigma$ is indeed an $F$-linear transformation.
    \item
      Let $F = \mathbb{Q}(\sqrt{7})$ and $K = \mathbb{Q}(\sqrt{7}, \sqrt{-11})$.
      Let $\sigma \in \Aut(K / \mathbb{Q})$ such that $\sigma(\sqrt{7}) = -\sqrt{7}, \sigma(\sqrt{-11}) = -\sqrt{-11}$.
      The existence of such an automorphism is shown in the solution to Problem 1.
      $K$ is an $F$-vector space.
      However, $\sigma(\sqrt{7} \cdot 1) = -\sqrt{7} \ne \sqrt{7} = \sqrt{7}(\sigma(1))$, so $\sigma$ is not an $F$-linear transformation.
  \end{enumerate}
\end{proof}

\begin{exer}{(Problem 3)}
  Let $\zeta = \exp(2\pi i / 3) \in \mathbb{C}$.
  Consider the following subfields of $\mathbb{C}$.
  Let $F = \mathbb{Q}(\zeta)$.
  For $i \in \{ 0, 1, 2 \}$, let $K_i = \mathbb{Q}(\zeta^i7^{1/3})$.
  Let $L = \mathbb{Q}(7^{1/3}, \zeta7^{1/3}, \zeta^27^{1/3})$.
\end{exer}

\begin{proof}
  $ $
  \begin{enumerate}
    \item 
      $[F:\mathbb{Q}] = 3$.
    \item
      $\Aut(F/\mathbb{Q})$ consists of two maps, the identity map and another map that swaps $\zeta$ and $\zeta^2$.
    \item
      $[K_i:\mathbb{Q}] = 3$ for each $i$ because $\{ 1, \zeta^i7^{1/3}, (\zeta^{i}7^{i/3})^2 \}$ is a $\mathbb{Q}$-basis.
    \item
      \todo[inline,caption={}]{
        Finish the rest!
      }
  \end{enumerate}
\end{proof}

\end{document}


