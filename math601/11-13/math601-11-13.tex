\documentclass[12pt, psamsfonts]{amsart}

%-------Packages---------
\usepackage{amssymb,amsfonts}
\usepackage{fullpage}
\usepackage{tikz-cd}
\usepackage{todonotes}
\usepackage{physics}
\usepackage[all,arc]{xy}
\usepackage{enumerate}
\usepackage{enumitem}
\usepackage{mathrsfs}
\usepackage{theoremref}
\usepackage{graphicx}
\usepackage[bookmarks]{hyperref}

%--------Theorem Environments--------
%theoremstyle{plain} --- default
\newtheorem{thm}{Theorem}[section]
\newtheorem{cor}[thm]{Corollary}
\newtheorem{prop}[thm]{Proposition}
\newtheorem{lem}[thm]{Lemma}
\newtheorem{conj}[thm]{Conjecture}
\newtheorem{quest}[thm]{Question}

\theoremstyle{definition}
\newtheorem{defn}[thm]{Definition}
\newtheorem{defns}[thm]{Definitions}
\newtheorem{con}[thm]{Construction}
\newtheorem{exmp}[thm]{Example}
\newtheorem{exmps}[thm]{Examples}
\newtheorem{notn}[thm]{Notation}
\newtheorem{notns}[thm]{Notations}
\newtheorem{addm}[thm]{Addendum}
\newtheorem*{exer}{Exercise}

\theoremstyle{remark}
\newtheorem{rem}[thm]{Remark}
\newtheorem{rems}[thm]{Remarks}
\newtheorem{warn}[thm]{Warning}
\newtheorem{sch}[thm]{Scholium}

\DeclareMathOperator{\Hom}{Hom}
\DeclareMathOperator{\Id}{Id}
\DeclareMathOperator{\End}{End}
\DeclareMathOperator{\ord}{ord}
\DeclareMathOperator{\Aut}{Aut}

\makeatletter
\let\c@equation\c@thm
\makeatother
\numberwithin{equation}{section}

\bibliographystyle{plain}

\begin{document}

\title{Math 601 (Due 11/13)}
\author{Hidenori Shinohara}
\maketitle

\tableofcontents

\section{Factoring Polynomials with Coefficients in Finite Fields}

\begin{exer}{(Problem 14)}
  For $a \in \mathbb{F}_q$, what are the possible values for $a^{(q - 1)/2}$?
  How many different $a$ take each value?
\end{exer}

\begin{proof}
  Let $\ev{\alpha} = (\mathbb{F}_q)^*$.
  Let $k \in \mathbb{Z}$.
  If $k$ is even, then $(\alpha^{k})^{(q - 1)/2} = (\alpha^{k/2})^{q - 1} = 1$.
  If $k = 2l + 1$ for some $l$, then $(\alpha^{k})^{(q - 1)/2} = \alpha^{l(q - 1)}\cdot\alpha^{(q - 1)/2} = \alpha^{(q - 1) / 2} = -1$ because -1 has degree 2 and $\alpha^{(q - 1)/2}$ is the only element in $\ev{\alpha}$ of degree 2.
  Therefore, 
  \begin{align*}
    a^{(q - 1) / 2} &= \begin{cases}
      0 & (a = 0) \\
      1 & (\exists l \in \mathbb{Z}, a = \alpha^{2l}) \\
      -1 & (\exists l \in \mathbb{Z}, a = \alpha^{2l + 1}).
    \end{cases}
  \end{align*}
  This is well defined because every nonzero element in $\mathbb{Z}_q$ is in $\ev{\alpha}$ and $2 \mid \abs{\ev{\alpha}} = q - 1$, so the parity of the exponent does not depend on the choice of $k$.
  Hence, 1 value gives 0, $(q - 1)/2$ values give 1, and $(q - 1) / 2$ values give $-1$.
\end{proof}

\begin{exer}{(Problem 15)}
  Let $f(x)$ be as in problem 13 and let $h \in \mathbb{F}_q[x]$ be a randomly chosen polynomial.
  What is the probability that $h^{(q^r - 1)/2} = \pm 1$ in the ring $\mathbb{F}_q[x]/(f(x))$.
\end{exer}

\begin{proof}
  As shown in Problem 13 last week, there exists an isomorphism $\Phi: \mathbb{F}_q[x]/(f(x)) \rightarrow \mathbb{F}_q[x]/(f_1(x)) \times \cdots \times \mathbb{F}_q[x]/(f_m(x))$ by the Chinese Remainder Theorem.
  For any $h \in \mathbb{F}_q[x]$, $\Phi(h + (f)) = (h + (f_1), \cdots, h + (f_m))$.
  Moreover, $\Phi(h^{(q - 1)/2} + (f)) = (h^{(q - 1)/2} + (f_1), \cdots, h^{(q - 1)/2} + (f_m))$.
  Therefore, $h^{(q - 1)/2} + (f) = 1$ if and only if $h^{(q - 1)/2} + (f_1), \cdots, h^{(q - 1)/2} + (f_m)$ all equal 1.

  Let $\alpha_1, \cdots, \alpha_m$ be generators of $(\mathbb{F}_q[x]/(f_1(x)))^*, \cdots, (\mathbb{F}_q[x]/(f_m(x)))^*$.
  For each $i$, $h^{(q - 1)/2} + (f_i) = 1$ if and only if $h \in \ev{ \alpha_i^2 }$ by Problem 14.
  Therefore, $h^{(q - 1)/2} + (f) = 1$ if and only if $(h + (f_1), \cdots, h + (f_m)) \in \ev{ \alpha_1^2 } \times \cdots \times \ev{ \alpha_m^2 }$.
  There are exactly $((q^r - 1)/2)^m$ elements that satisfy that.
  Therefore,
  \begin{align*}
    \frac{(\frac{q^r - 1}{2})^m}{(q^r)^m} = \Big(\frac{q^r - 1}{2q^r}\Big)^m = \Big(\frac{1}{2} - \frac{1}{2q^r}\Big)^m.
  \end{align*}
  is the probability that $h^{(q^r - 1)/2} = 1$ in $\mathbb{F}_q[x]/(f(x))$.

  Using the exact same argument, we can derive that the probability that $h^{(q^r - 1)/2} = -1$ is exactly the same value.
\end{proof}

\begin{exer}{(Problem 16)}
  With $f(x)$ as in problem 13, write $f(x) = g_1(x) \cdots g_m(x)$ for the factorization into irreducible factors.
  Express $\gcd(f(x), h^{(q^r - 1)/2} - 1)$ in terms of the $g_i(x)$'s.
\end{exer}

\begin{proof}
  $\gcd(f(x), h^{(q^r - 1)/2} - 1)$ is the product of $g_i(x)$'s that divide $h^{(q^r - 1)/2} - 1$.
  It is divisible by $g_i(x)$ if and only if $h \in \ev{ \alpha_i^2 }$ from Problem 15.
\end{proof}

\begin{exer}{(Problem 17)}
  Describe a probabilistic factoring algorithm which has a very high probability of finding the irreducible factors of a polynomial $f(x) \in \mathbb{F}_q[x]$, provided one knows ahead of time that $f(x)$ is a product of m distinct irreducible polynomials of degree $r$.
\end{exer}

\begin{proof}
  Let $i_0$ be fixed.
  Given a random $h(x) \in \mathbb{F}_q[x]$, the probability that $h^{(q - 1)/2} - 1 \in (f_{i_0})$ is $1/2 - 1/(2q^r)$, which is slightly smaller than 50\%.
  Therefore, it is likely that given a random $h(x) \in \mathbb{F}_q[x]$, the probability that $h^{(q - 1)/2} - 1 \in (f_i)$ for \textit{some} $i$'s is high.
  However, the probability that $h^{(q - 1)/2} - 1 \in (f_i)$ in \textit{all} $i$'s is low.

  In other words, the probability that $h^{(q - 1)/2} - 1$ is a proper divisor of $f$ is high.
  Therefore, we can expect to factor $f(x)$ by 

  \begin{itemize}
    \item
      Step 1: Generate a random polynomial $h(x) \in \mathbb{F}_q[x]/(f(x))$.
    \item
      Step 2: Calculate $h^{(q^r - 1)/2} - 1$.
      This step can be done efficiently by exponentiation by squaring.
    \item
      Step 3: Calculate $d(x) = \gcd(f(x), h^{(q^r - 1)/2} - 1)$.
      This step can be done efficiently by the Euclid algorithm.
    \item
      Step 4: If $1 \leq \deg(d(x)) \leq \deg(f(x))$, then factorize $f(x)/d(x)$ and $d(x)$ further by going back to Step 1 unless it is degree $r$.
      Otherwise, we were unlucky, so we go back to Step 1.
  \end{itemize}
\end{proof}

\end{document}
