\documentclass[12pt, psamsfonts]{amsart}

%-------Packages---------
\usepackage{amssymb,amsfonts}
\usepackage{fullpage}
\usepackage{tikz-cd}
\usepackage{todonotes}
\usepackage{physics}
\usepackage[all,arc]{xy}
\usepackage{enumerate}
\usepackage{enumitem}
\usepackage{mathrsfs}
\usepackage{theoremref}
\usepackage{graphicx}
\usepackage[bookmarks]{hyperref}

%--------Theorem Environments--------
%theoremstyle{plain} --- default
\newtheorem{thm}{Theorem}[section]
\newtheorem{cor}[thm]{Corollary}
\newtheorem{prop}[thm]{Proposition}
\newtheorem{lem}[thm]{Lemma}
\newtheorem{conj}[thm]{Conjecture}
\newtheorem{quest}[thm]{Question}

\theoremstyle{definition}
\newtheorem{defn}[thm]{Definition}
\newtheorem{defns}[thm]{Definitions}
\newtheorem{con}[thm]{Construction}
\newtheorem{exmp}[thm]{Example}
\newtheorem{exmps}[thm]{Examples}
\newtheorem{notn}[thm]{Notation}
\newtheorem{notns}[thm]{Notations}
\newtheorem{addm}[thm]{Addendum}
\newtheorem*{exer}{Exercise}

\theoremstyle{remark}
\newtheorem{rem}[thm]{Remark}
\newtheorem{rems}[thm]{Remarks}
\newtheorem{warn}[thm]{Warning}
\newtheorem{sch}[thm]{Scholium}

\DeclareMathOperator{\Hom}{Hom}
\DeclareMathOperator{\Id}{Id}
\DeclareMathOperator{\End}{End}
\DeclareMathOperator{\ord}{ord}
\DeclareMathOperator{\Aut}{Aut}

\makeatletter
\let\c@equation\c@thm
\makeatother
\numberwithin{equation}{section}

\bibliographystyle{plain}

\begin{document}

\title{Math 601 (Due 11/13)}
\author{Hidenori Shinohara}
\maketitle

\tableofcontents

\section{Factoring Polynomials with Coefficients in Finite Fields}

\begin{exer}{(Problem 14)}
  For $a \in \mathbb{F}_q$, what are the possible values for $a^{(q - 1)/2}$?
  How many different $a$ take each value?
\end{exer}

\begin{proof}
  Let $\ev{\alpha} = (\mathbb{F}_q)^*$.
  Let $k \in \mathbb{Z}$.
  If $k$ is even, then $(\alpha^{k})^{(q - 1)/2} = (\alpha^{k/2})^{q - 1} = 1$.
  If $k = 2l + 1$ for some $l$, then $(\alpha^{k})^{(q - 1)/2} = \alpha^{l(q - 1)}\cdot\alpha^{(q - 1)/2} = \alpha^{(q - 1) / 2} = -1$ because -1 has degree 2 and $\alpha^{(q - 1)/2}$ is the only element in $\ev{\alpha}$ of degree 2.
  Therefore, 
  \begin{align*}
    a^{(q - 1) / 2} &= \begin{cases}
      0 & (a = 0) \\
      1 & (\exists l \in \mathbb{Z}, a = \alpha^{2l}) \\
      -1 & (\exists l \in \mathbb{Z}, a = \alpha^{2l + 1}).
    \end{cases}
  \end{align*}
  This is well defined because every nonzero element in $\mathbb{Z}_q$ is in $\ev{\alpha}$ and $2 \mid \abs{\ev{\alpha}} = q - 1$, so the parity of the exponent does not depend on the choice of $k$.
  Hence, 1 value gives 0, $(q - 1)/2$ values give 1, and $(q - 1) / 2$ values give $-1$.
\end{proof}

\end{document}
