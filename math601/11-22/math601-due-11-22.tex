\documentclass[12pt, psamsfonts]{amsart}

%-------Packages---------
\usepackage{amssymb,amsfonts}
\usepackage{fullpage}
\usepackage{tikz-cd}
\usepackage{todonotes}
\usepackage{physics}
\usepackage[all,arc]{xy}
\usepackage{enumerate}
\usepackage{enumitem}
\usepackage{mathrsfs}
\usepackage{theoremref}
\usepackage{graphicx}
\usepackage[bookmarks]{hyperref}

%--------Theorem Environments--------
%theoremstyle{plain} --- default
\newtheorem{thm}{Theorem}[section]
\newtheorem{cor}[thm]{Corollary}
\newtheorem{prop}[thm]{Proposition}
\newtheorem{lem}[thm]{Lemma}
\newtheorem{conj}[thm]{Conjecture}
\newtheorem{quest}[thm]{Question}

\theoremstyle{definition}
\newtheorem{defn}[thm]{Definition}
\newtheorem{defns}[thm]{Definitions}
\newtheorem{con}[thm]{Construction}
\newtheorem{exmp}[thm]{Example}
\newtheorem{exmps}[thm]{Examples}
\newtheorem{notn}[thm]{Notation}
\newtheorem{notns}[thm]{Notations}
\newtheorem{addm}[thm]{Addendum}
\newtheorem*{exer}{Exercise}

\theoremstyle{remark}
\newtheorem{rem}[thm]{Remark}
\newtheorem{rems}[thm]{Remarks}
\newtheorem{warn}[thm]{Warning}
\newtheorem{sch}[thm]{Scholium}

\DeclareMathOperator{\Hom}{Hom}
\DeclareMathOperator{\Id}{Id}
\DeclareMathOperator{\End}{End}
\DeclareMathOperator{\ord}{ord}
\DeclareMathOperator{\Aut}{Aut}
\DeclareMathOperator{\Gal}{Gal}

\makeatletter
\let\c@equation\c@thm
\makeatother
\numberwithin{equation}{section}

\bibliographystyle{plain}

\begin{document}

\title{Math 601 (Due 11/22)}
\author{Hidenori Shinohara}
\maketitle

\tableofcontents
\section{THE THEOREM ON SYMMETRIC POLYNOMIALS}

\begin{exer}{(Problem 1)}
  By substituting $u_4 = 0$, we get $u_1^2u_2u_3 + u_1u_2^2u_3 + u_1u_2u_3^2 = s_3s_1$.
  $s_3s_1$ with 4 variables expands to $u_{1}^{2} u_{2} u_{3} + u_{1}^{2} u_{2} u_{4} + u_{1}^{2} u_{3} u_{4} + u_{1} u_{2}^{2} u_{3} + u_{1} u_{2}^{2} u_{4} + u_{1} u_{2} u_{3}^{2} + 4 u_{1} u_{2} u_{3} u_{4} + u_{1} u_{2} u_{4}^{2} + u_{1} u_{3}^{2} u_{4} + u_{1} u_{3} u_{4}^{2} + u_{2}^{2} u_{3} u_{4} + u_{2} u_{3}^{2} u_{4} + u_{2} u_{3} u_{4}^{2}$.
  Then $s_3s_1 - f$ where $f$ is the original polynomial gives us $4u_1u_2u_3u_4 = 4s_4$.
  Therefore, $f = s_3s_1 - 4s_4$.
\end{exer}

\begin{exer}{(Problem 2)}
  We are given that $\abs{M - xI} = x^3 - ax^2 + bx - c$.
  This implies that $\abs{M - (-x)I} = -x^3 - ax^2 - bx - c$.
  Since the determinant function preserves multiplication, $\abs{M - xI}\abs{M - (-x)I} = \abs{M^2 - x^2I}$.
  This implies $\abs{M^2 - x^2I} = -x^6 + (a^2 - 2b)x^4 + (b^2 + 2ac)x^2 + c^2$.
  Therefore, the characteristic polynomial of $M$ is $-x^3 + (a^2 - 2b)x^2 + (b^2 + 2ac)x + c^2$.
\end{exer}

\section{Galois Theory VI}

\begin{exer}{(Problem 3)}
  \begin{enumerate}[label=(\alph*)]
    \item
      $\{ (123), (132), e \}$ is clearly a subgroup of the stabilizer group $S_v$ of $v$.
      Since $(12) \notin S_v$, $3 \leq \abs{S_v} \leq 5$.
      By Lagrange's Theorem, $S_v = \ev{(123)}$. 
    \item
      By (i), $S_3v$ contains only $[S_3:S_v] = 2$ elements.
      Thus $v' = (12) \cdot v = u_2u_1^2 + u_1u_3^2 + u_3u_2^2$.
    \item
      By substituting $u_3 = 0$ for $v + v'$, we get $u_1u_2^2 + u_2u_1^2 = s_1s_2$.
      Then $v + v' - s_1s_2 = -3u_1u_2u_3 = -3s_3$.
      Therefore, $v + v' = s_1s_2 + 3s_3$.
    \item
      We will use the fundamental theorem of Galois Theory.
      $F(v) = K^{\ev{(123)}}$, so $\abs{\ev{(123)}} = 3 = [K:F(v)]$.
      Moreover, $\abs{\ev{\Gal(K/F)}} = [K:F]$.
      Therefore, $[F(v):F] = [K:F]/[K:F(v)] = \abs{\ev{\Gal(K/F)}}/3$.
    \item
      Calculation shows that $vv' = 9s_3^2 + s_3s_1^3 - 6s_3s_1s_2 + s_2^3$.
      By substituting $s_1 = 0, s_2 = p, s_3 = q$, we get $9q^2 + p^3$.
  \end{enumerate}
\end{exer}

\end{document}


