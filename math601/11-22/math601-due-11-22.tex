\documentclass[12pt, psamsfonts]{amsart}

%-------Packages---------
\usepackage{amssymb,amsfonts}
\usepackage{fullpage}
\usepackage{tikz-cd}
\usepackage{todonotes}
\usepackage{physics}
\usepackage[all,arc]{xy}
\usepackage{enumerate}
\usepackage{enumitem}
\usepackage{mathrsfs}
\usepackage{theoremref}
\usepackage{graphicx}
\usepackage[bookmarks]{hyperref}

%--------Theorem Environments--------
%theoremstyle{plain} --- default
\newtheorem{thm}{Theorem}[section]
\newtheorem{cor}[thm]{Corollary}
\newtheorem{prop}[thm]{Proposition}
\newtheorem{lem}[thm]{Lemma}
\newtheorem{conj}[thm]{Conjecture}
\newtheorem{quest}[thm]{Question}

\theoremstyle{definition}
\newtheorem{defn}[thm]{Definition}
\newtheorem{defns}[thm]{Definitions}
\newtheorem{con}[thm]{Construction}
\newtheorem{exmp}[thm]{Example}
\newtheorem{exmps}[thm]{Examples}
\newtheorem{notn}[thm]{Notation}
\newtheorem{notns}[thm]{Notations}
\newtheorem{addm}[thm]{Addendum}
\newtheorem*{exer}{Exercise}

\theoremstyle{remark}
\newtheorem{rem}[thm]{Remark}
\newtheorem{rems}[thm]{Remarks}
\newtheorem{warn}[thm]{Warning}
\newtheorem{sch}[thm]{Scholium}

\DeclareMathOperator{\Hom}{Hom}
\DeclareMathOperator{\Id}{Id}
\DeclareMathOperator{\End}{End}
\DeclareMathOperator{\ord}{ord}
\DeclareMathOperator{\Aut}{Aut}

\makeatletter
\let\c@equation\c@thm
\makeatother
\numberwithin{equation}{section}

\bibliographystyle{plain}

\begin{document}

\title{Math 601 (Due 11/22)}
\author{Hidenori Shinohara}
\maketitle

\tableofcontents
\section{THE THEOREM ON SYMMETRIC POLYNOMIALS}

\begin{exer}{(Problem 1)}
  By substituting $u_4 = 0$, we get $u_1^2u_2u_3 + u_1u_2^2u_3 + u_1u_2u_3^2 = s_3s_1$.
  $s_3s_1$ with 4 variables expands to $u_{1}^{2} u_{2} u_{3} + u_{1}^{2} u_{2} u_{4} + u_{1}^{2} u_{3} u_{4} + u_{1} u_{2}^{2} u_{3} + u_{1} u_{2}^{2} u_{4} + u_{1} u_{2} u_{3}^{2} + 4 u_{1} u_{2} u_{3} u_{4} + u_{1} u_{2} u_{4}^{2} + u_{1} u_{3}^{2} u_{4} + u_{1} u_{3} u_{4}^{2} + u_{2}^{2} u_{3} u_{4} + u_{2} u_{3}^{2} u_{4} + u_{2} u_{3} u_{4}^{2}$.
  Then $s_3s_1 - f$ where $f$ is the original polynomial gives us $4u_1u_2u_3u_4 = 4s_4$.
  Therefore, $f = s_3s_1 - 4s_4$.
\end{exer}

\end{document}


