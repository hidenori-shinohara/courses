\documentclass[12pt, psamsfonts]{amsart}

%-------Packages---------
\usepackage{amssymb,amsfonts}
\usepackage[all,arc]{xy}
\usepackage{enumerate}
\usepackage{physics}
\usepackage{mathrsfs}
\usepackage{theoremref}
\usepackage{graphicx}
\usepackage[bookmarks]{hyperref}

%--------Theorem Environments--------
%theoremstyle{plain} --- default
\newtheorem{thm}{Theorem}[section]
\newtheorem{cor}[thm]{Corollary}
\newtheorem{prop}[thm]{Proposition}
\newtheorem{lem}[thm]{Lemma}
\newtheorem{conj}[thm]{Conjecture}
\newtheorem{quest}[thm]{Question}

\theoremstyle{definition}
\newtheorem{defn}[thm]{Definition}
\newtheorem{defns}[thm]{Definitions}
\newtheorem{con}[thm]{Construction}
\newtheorem{exmp}[thm]{Example}
\newtheorem{exmps}[thm]{Examples}
\newtheorem{notn}[thm]{Notation}
\newtheorem{notns}[thm]{Notations}
\newtheorem{addm}[thm]{Addendum}
\newtheorem*{exer}{Exercise}

\theoremstyle{remark}
\newtheorem{rem}[thm]{Remark}
\newtheorem{rems}[thm]{Remarks}
\newtheorem{warn}[thm]{Warning}
\newtheorem{sch}[thm]{Scholium}

\DeclareMathOperator{\Hom}{Hom}
\DeclareMathOperator{\Id}{Id}

\makeatletter
\let\c@equation\c@thm
\makeatother
\numberwithin{equation}{section}

\bibliographystyle{plain}

\begin{document}

\title{Math 601 Homework (Due 9/11)}
\author{Hidenori Shinohara}
\maketitle

\begin{exer}{(1)}
  Show that $2 \times 2$ matrices give a functor, $M_2$, from the category of rings to itself, $R \mapsto M_2(R)$.
\end{exer}

\begin{proof}
  Let $R, R'$ be rings and $\phi \in \Hom(R, R')$.
  Let $M_2(\phi): M_2(R) \rightarrow M_2(R')$ be defined such that

  \begin{align*}
    (M_2(\phi))\begin{pmatrix}\begin{bmatrix} a & b \\ c & d \end{bmatrix} \end{pmatrix} = 
    \begin{bmatrix} \phi(a) & \phi(b) \\ \phi(c) & \phi(d) \end{bmatrix}.
  \end{align*}

  We claim that $M_2$ is indeed a functor.

 \begin{itemize}
   \item
     Claim 1: For any $\phi \in \Hom(R, R')$, $M_2(\phi) \in \Hom(M_2(R), M_2(R'))$.
     In other words, we want to show that $M_2(\phi)$ is a ring homomorphism for any $\phi$.
     \begin{align*}
       (M_2(\phi))\begin{pmatrix}\begin{bmatrix} a & b \\ c & d \end{bmatrix} + \begin{bmatrix} e & f \\ g & h \end{bmatrix} \end{pmatrix} 
         &= (M_2(\phi))\begin{pmatrix}\begin{bmatrix} a + e & b + f \\ c + g & d + h \end{bmatrix} \end{pmatrix} \\
         &= \begin{bmatrix} \phi(a + e) & \phi(b + f) \\ \phi(c + g) & \phi(d + h) \end{bmatrix} \\
         &= \begin{bmatrix} \phi(a) + \phi(e) & \phi(b) + \phi(f) \\ \phi(c) + \phi(g) & \phi(d) + \phi(h) \end{bmatrix} \\
         &= \begin{bmatrix} \phi(a) & \phi(b) \\ \phi(c) & \phi(d) \end{bmatrix} + \begin{bmatrix} \phi(e) & \phi(f) \\ \phi(g) & \phi(h) \end{bmatrix} \\
         &= (M_2(\phi))\begin{bmatrix} a & b \\ c & d \end{bmatrix} + (M_2(\phi))\begin{bmatrix} e & f \\ g & h \end{bmatrix} \\
     \end{align*}
     \begin{align*}
       &(M_2(\phi))\begin{pmatrix}\begin{bmatrix} a & b \\ c & d \end{bmatrix} \begin{bmatrix} e & f \\ g & h \end{bmatrix} \end{pmatrix} \\
         &= (M_2(\phi))\begin{pmatrix}\begin{bmatrix} ae + bg & af + bh \\ ce + dg & cf + dh \end{bmatrix} \end{pmatrix} \\
         &= \begin{bmatrix} \phi(ae + bg) & \phi(af + bh) \\ \phi(ce + dg) & \phi(cf + dh) \end{bmatrix} \\
         &= \begin{bmatrix} \phi(a)\phi(e) + \phi(b)\phi(g) & \phi(a)\phi(f) + \phi(b)\phi(h) \\ \phi(c)\phi(e) + \phi(d)\phi(g) & \phi(c)\phi(f) + \phi(d)\phi(h)) \end{bmatrix} \\
         &= \begin{bmatrix} \phi(a) & \phi(b) \\ \phi(c) & \phi(d) \end{bmatrix} \begin{bmatrix} \phi(e) & \phi(f) \\ \phi(g) & \phi(h) \end{bmatrix} \\
         &= (M_2(\phi))\begin{pmatrix}\begin{bmatrix} a & b \\ c & d \end{bmatrix} \end{pmatrix} (M_2(\phi))\begin{pmatrix}\begin{bmatrix} e & f \\ g & h \end{bmatrix} \end{pmatrix}
     \end{align*}
     Therefore, $M_2(\phi)$ is indeed a ring homomorphism.
   \item
     For any ring $R$ and the identity function $\Id_R$, $M_2(\Id_R)$ is the identity map on $M_2(R)$ because it maps each element in a given matrix to itself.
   \item
     Let $f \in \Hom(A, B), g \in \Hom(B, C)$.
     \begin{align*}
       (M_2(f \circ g))\begin{pmatrix}\begin{bmatrix} a & b \\ c & d \end{bmatrix}\end{pmatrix}
        &= \begin{bmatrix} (f \circ g)(a) & (f \circ g)(b) \\ (f \circ g)(c) & (f \circ g)(d) \end{bmatrix} \\
        &= \begin{bmatrix} f(g(a)) & f(g(b)) \\ f(g(c)) & f(g(d)) \end{bmatrix} \\
        &= M_2(f)\begin{pmatrix}\begin{bmatrix} g(a) & g(b) \\ g(c) & g(d) \end{bmatrix}\end{pmatrix}\\
        &= M_2(f)\begin{pmatrix}M_2(g)\begin{pmatrix}\begin{bmatrix} a & b \\ c & d \end{bmatrix}\end{pmatrix}\end{pmatrix} \\
        &= (M_2(f) \circ M_2(g))\begin{pmatrix}\begin{bmatrix} a & b \\ c & d \end{bmatrix}\end{pmatrix}.
     \end{align*}
 \end{itemize}
 Therefore, $M_2$ is indeed a functor.
\end{proof}

\begin{exer}{(Problem 8 from More exercises)}
  Consider the subgroup, $D_5 = \langle (12345), (14)(23) \rangle \subset S_5$.
  \begin{enumerate}
    \item
      Set $a = (12345)$ and compute $a^{-1}$.
    \item
      Set $b = (14)(23)$ and compute $aba^{-1}$.
    \item
      Show that every element in $D_5$ may be written in the form $a^ib^j$ for some $i, j \in \mathbb{Z}$.
    \item
      Compute $\abs{D_5}$.
  \end{enumerate}
\end{exer}

\begin{proof}
  $ $
  \begin{enumerate}
    \item
      $a$ sends 1 to 2, 2 to 3, $\cdots$.
      We want $a^{-1}$ to do the opposite.
      Thus $a^{-1} = (15432)$.
      Since $(12345)(15432) = (15432)(12345) = (1)$, $(15432)$ is indeed $a^{-1}$.
    \item
      $aba^{-1} = (a(1)a(4))(a(2)a(3)) = (25)(34)$.
    \item
      $ba = (14)(23)(12345) = (13)(45)$, and $a^{-1}b = (15432)(14)(23) = (13)(45)$.
      Therefore, $ba = a^{-1}b$.
      We claim that $ba^n = a^{-n}b$ for every $n \in \mathbb{N}$.
      Suppose $ba^n = a^{-n}b$ for some $n \in \mathbb{N}$.
      Then $ba^{n + 1} = (ba^n)a = (a^{-n}b)a = a^{-n}(ba) = a^{-n}a^{-1}b = a^{-n-1}b$.
      By mathematical induction, $ba^n = a^{-n}b$ for every $n \in \mathbb{N}$.

      For any $n \in \mathbb{N}$, $ba^n = a^{-n}b$, so $a^nba^n = b$, and thus $a^nb = ba^{-n}$.
      Therefore, we have $ba^k = a^{-k}b$ for every $k \in \mathbb{Z}$.

      We claim that for any $i, j \in \mathbb{Z}$, $b^ja^i$ can be written in the desired form.
      Since $b^2 = e$, we consider two cases based on the parity of $j$.
      If $j$ is even, then $b^j = e$, so $b^ja^i = a^i$.
      If $j$ is odd, then $b^j = b$, so $b^ja^i = ba^i = a^{-i}b$ as shown above.

      We will prove the general case.
      By the argument above, it suffices to show that every element in $D_5$ can be represented as a word of length $\leq 2$.
      Let $x_1^{i_1} \cdots x_k^{i_k} \in D_5$ be given where $i_1, \cdots, i_k \in \mathbb{Z}$ and each $x_i$ is either $a$ or $b$.
      Since $D_5$ is generated by $a, b$, every element can be represented in this form.
      We will show that every element in $D_5$ can be represented as a word of length $\leq 2$ by using strong induction.
      If $k \leq 2$, then we are done.
      Suppose that we can represent every element in $D_5$ of length $\leq k$ as a word of length $\leq 2$ for some $k \geq 2$.
      Let $x = x_1^{i_1} \cdots x_{k + 1}^{i_{k + 1}} \in D_5$.
      If $x_1 = x_2$, then $x = x_2^{i_1 + i_2}x_3^{i_3} \cdots x_{k + 1}^{i_{k + 1}}$, so by the inductive hypothesis, this can be represented as a word of length $\leq 2$.
      If $x_2 = x_3$, then $x = x_1^{i_1}x_2^{i_2 + i_3}x_4^{i_4} \cdots x_{k + 1}^{i_{k + 1}}$, so by the inductive hypothesis, this can be represented as a word of length $\leq 2$.
      Suppose $x_1 \ne x_2$ and $x_2 \ne x_3$.
      Then there are two cases:
      \begin{itemize}
        \item
          Case 1: $(x_1, x_2, x_3) = (a, b, a)$.
          By the argument above, $b^{i_2}a^{i_3}$ can be represented as $a^ib^j$ for some $i, j \in \mathbb{Z}$.
          Therefore, $a^{i_1}(b^{i_2}a^{i_3}) = a^{i_1}(a^ib^j) = a^{i_1 + i}b^j$, so $x$ can be represented as a word of length $k$.
          By the inductive hypothesis, $x$ can be represented as a word of length $\leq 2$.
        \item
          Case 2: $(x_1, x_2, x_3) = (b, a, b)$.
          By the argument above, $b^{i_1}a^{i_2}$ can be represented as $a^ib^j$ for some $i, j \in \mathbb{Z}$.
          Therefore, $(b^{i_1}a^{i_2})b^{i_3} = (a^ib^j)b^{i_3} = a^ib^{j + i_3}$.
          By the inductive hypothesis, $x$ can be represented as a word of length $\leq 2$.
      \end{itemize}
    \item
      TODO
  \end{enumerate}
\end{proof}


\end{document}


