\documentclass[12pt, psamsfonts]{amsart}

%-------Packages---------
\usepackage{amssymb,amsfonts}
\usepackage{fullpage}
\usepackage{todonotes}
\usepackage{physics}
\usepackage[all,arc]{xy}
\usepackage{enumerate}
\usepackage{mathrsfs}
\usepackage{theoremref}
\usepackage{graphicx}
\usepackage[bookmarks]{hyperref}

%--------Theorem Environments--------
%theoremstyle{plain} --- default
\newtheorem{thm}{Theorem}[section]
\newtheorem{cor}[thm]{Corollary}
\newtheorem{prop}[thm]{Proposition}
\newtheorem{lem}[thm]{Lemma}
\newtheorem{conj}[thm]{Conjecture}
\newtheorem{quest}[thm]{Question}

\theoremstyle{definition}
\newtheorem{defn}[thm]{Definition}
\newtheorem{defns}[thm]{Definitions}
\newtheorem{con}[thm]{Construction}
\newtheorem{exmp}[thm]{Example}
\newtheorem{exmps}[thm]{Examples}
\newtheorem{notn}[thm]{Notation}
\newtheorem{notns}[thm]{Notations}
\newtheorem{addm}[thm]{Addendum}
\newtheorem*{exer}{Exercise}

\theoremstyle{remark}
\newtheorem{rem}[thm]{Remark}
\newtheorem{rems}[thm]{Remarks}
\newtheorem{warn}[thm]{Warning}
\newtheorem{sch}[thm]{Scholium}

\DeclareMathOperator{\Hom}{Hom}
\DeclareMathOperator{\Id}{Id}

\makeatletter
\let\c@equation\c@thm
\makeatother
\numberwithin{equation}{section}

\bibliographystyle{plain}

\begin{document}

\title{Math 601 (Due 9/25)}
\author{Hidenori Shinohara}
\maketitle

\begin{exer}{(Problem 1)}
  Define $\gamma: \mathbb{Z}[\sqrt{2}] \rightarrow \mathbb{Z}[\sqrt{2}]$ by $\gamma(a + b\sqrt{2}) = a - b\sqrt{2}$.
  Show that $\gamma$ is a ring isomorphism and compute its inverse.
\end{exer}

\begin{proof}
  Let $a + b\sqrt{2}, c + d\sqrt{2} \in \mathbb{Z}[\sqrt{2}]$ be given.
  \begin{align*}
    \gamma((a + b\sqrt{2}) + (c + d\sqrt{2}))
      &= \gamma((a + c) + (b + d)\sqrt{2}) \\
      &= (a + c) - (b + d)\sqrt{2} \\
      &= (a - b\sqrt{2}) + (c - d\sqrt{2}) \\
      &= \gamma(a + b\sqrt{2}) + \gamma(c + d\sqrt{2}). \\
    \gamma((a + b\sqrt{2})(c + d\sqrt{2}))
      &= \gamma((ac + 2bd) + (ad + bc)\sqrt{2}) \\
      &= (ac + 2bd) - (ad + bc)\sqrt{2} \\
      &= (ac + 2(-b)(-d)) + (a(-d) + (-b)c)\sqrt{2} \\
      &= (a - b\sqrt{2})(c - d\sqrt{2}) \\
      &= \gamma(a + b\sqrt{2})\gamma(c + d\sqrt{2}). \\
  \end{align*}
  Moreover, $\gamma(1) = 1 - 0\sqrt{2} = 1$.
  Therefore, $\gamma$ is a ring homomorphism.
  For any $a + b\sqrt{2}$, $\gamma(\gamma(a + b\sqrt{2})) = \gamma(a - b\sqrt{2}) = a + b\sqrt{2}$.
  Therefore, $\gamma$ has an inverse, and the inverse of $\gamma$ is $\gamma$.
  This implies that $\gamma$ is bijective.

  In conclusion, $\gamma$ is an isomorphism and its inverse is itself.
\end{proof}

\begin{exer}{(Problem 2)}
  Define $N: \mathbb{Z}[\sqrt{2}] \rightarrow \mathbb{Z}$ by $N(a + b\sqrt{2}) = (a + b\sqrt{2})\gamma(a + b\sqrt{2})$.
  Show that $N(\alpha\beta) = N(\alpha)N(\beta)$.
\end{exer}

\begin{proof}
  Let $a + b\sqrt{2}, c + d\sqrt{2}$ be given.
  \begin{align*}
    N((a + b\sqrt{2})(c + d\sqrt{2}))
      &= N((ac + 2bd) + (ad + bc)\sqrt{2}) \\
      &= ((ac + 2bd) + (ad + bc)\sqrt{2})\gamma((ac + 2bd) + (ad + bc)\sqrt{2}) \\
      &= (a + b\sqrt{2})(c + d\sqrt{2})\gamma((a + b\sqrt{2})(c + d\sqrt{2})) \\
      &= (a + b\sqrt{2})(c + d\sqrt{2})\gamma(a + b\sqrt{2})\gamma(c + d\sqrt{2}) \\
      &= (a + b\sqrt{2})\gamma(a + b\sqrt{2})(c + d\sqrt{2})\gamma(c + d\sqrt{2}) \\
      &= N(a + b\sqrt{2})N(c + d\sqrt{2}).
  \end{align*}
\end{proof}

\begin{exer}{(Problem 3)}
  Write $\mathbb{Z}[\sqrt{2}]^*$ for the group of units in $\mathbb{Z}[\sqrt{2}]$.
  Show that $\alpha \in \mathbb{Z}[\sqrt{2}]^*$ if and only if $N(\alpha) = \pm 1$.
\end{exer}

\begin{proof}
  We have $N(1) = 1\gamma(1) = 1$.

  Let $\alpha$ be a unit and $\beta$ be the inverse.
  Then $N(\alpha\beta) = N(1) = 1$.
  Thus $1 = N(\alpha)N(\beta)$.
  Since $N(\alpha), N(\beta) \in \mathbb{Z}$, $N(\alpha) = \pm 1$.

  On the other hand, suppose that $N(\alpha) = \pm 1$ for some $\alpha$.
  \begin{itemize}
    \item
      Case 1: $N(\alpha) = 1$. 
      Then $\alpha\gamma(\alpha) = 1$, so $\gamma(\alpha)$ is an inverse of $\alpha$.
      Therefore, $\alpha$ is a unit.
    \item
      Case 2: $N(\alpha) = -1$. 
      Then $\alpha\gamma(\alpha) = -1$, so $-\gamma(\alpha)$ is an inverse of $\alpha$.
      Therefore, $\alpha$ is a unit.
  \end{itemize}
  In each case, $\alpha$ is a unit.

  Therefore, $N(\alpha) = \pm 1$ if and only if $\alpha$ is a unit.
\end{proof}

\begin{exer}{(Problem 4)}
  What does finding the units in $\mathbb{Z}[\sqrt{2}]$ have to do with solving the equation $x^2 - 2y^2 = \pm 1$?
\end{exer}

\begin{proof}
  Let $(a, b)$ be a solution to the equation.
  Then $a^2 - 2b^2 = \pm 1$, so $(a + b\sqrt{2})(a - b\sqrt{2}) = \pm 1$.
  This implies that $a \pm b\sqrt{2}$ is a unit in $\mathbb{Z}[\sqrt{2}]$.

  On the other hand, let $a + b\sqrt{2}$ be a unit in $\mathbb{Z}[\sqrt{2}]$.
  By Problem 3, $N(a + b\sqrt{2}) = \pm 1$.
  Thus $\pm 1 = N(a + b\sqrt{2}) = (a + b\sqrt{2})(a - b\sqrt{2}) = a^2 - b^2$.
  Hence, $(a, b)$ is a solution to $x^2 - 2y^2 = \pm 1$.

  In conclusion, there exists a bijective correspondence between the units in $\mathbb{Z}[\sqrt{2}]$ and the solutions to $x^2 - 2y^2 = \pm 1$.
\end{proof}

\begin{exer}{(Problem 5)}
  Show that $\mathbb{Z}[\sqrt{2}]$ has no smallest positive element.
\end{exer}

\begin{proof}
  We have $0 < \sqrt{2} - 1 < 1$.
  Since $\forall n \in \mathbb{N}, (\sqrt{2} - 1)^n \in \mathbb{Z}[\sqrt{2}]$ and $\lim_{n \rightarrow \infty} (\sqrt{2} - 1)^n = 0$, there exists no smallest positive element in $\mathbb{Z}[\sqrt{2}]$.
\end{proof}

\begin{exer}{(Problem 6)}
  Find an element $u \in \mathbb{Z}[\sqrt{2}]^*$ with $u > 1$.
\end{exer}

\begin{proof}
  $(\sqrt{2} + 1)(\sqrt{2} - 1) = 2 - 1 = 1$.
  Thus $u = \sqrt{2} + 1$ is a unit such that $u > 1$.
\end{proof}
\end{document}


