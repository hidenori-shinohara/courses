\documentclass[12pt, psamsfonts]{amsart}

%-------Packages---------
\usepackage{amssymb,amsfonts}
\usepackage{fullpage}
\usepackage{todonotes}
\usepackage{physics}
\usepackage[all,arc]{xy}
\usepackage{enumerate}
\usepackage{mathrsfs}
\usepackage{theoremref}
\usepackage{graphicx}
\usepackage[bookmarks]{hyperref}

%--------Theorem Environments-------- %theoremstyle{plain} --- default
\newtheorem{thm}{Theorem}[section]
\newtheorem{cor}[thm]{Corollary}
\newtheorem{prop}[thm]{Proposition}
\newtheorem{lem}[thm]{Lemma}
\newtheorem{conj}[thm]{Conjecture}
\newtheorem{quest}[thm]{Question}

\theoremstyle{definition}
\newtheorem{defn}[thm]{Definition}
\newtheorem{defns}[thm]{Definitions}
\newtheorem{con}[thm]{Construction}
\newtheorem{exmp}[thm]{Example}
\newtheorem{exmps}[thm]{Examples}
\newtheorem{notn}[thm]{Notation}
\newtheorem{notns}[thm]{Notations}
\newtheorem{addm}[thm]{Addendum}
\newtheorem*{exer}{Exercise}

\theoremstyle{remark}
\newtheorem{rem}[thm]{Remark}
\newtheorem{rems}[thm]{Remarks}
\newtheorem{warn}[thm]{Warning}
\newtheorem{sch}[thm]{Scholium}

\DeclareMathOperator{\Hom}{Hom}
\DeclareMathOperator{\Id}{Id}

\makeatletter
\let\c@equation\c@thm
\makeatother
\numberwithin{equation}{section}

\bibliographystyle{plain}

\begin{document}

\title{Math 601 (Due 9/25)}
\author{Hidenori Shinohara}
\maketitle

\begin{exer}{(Problem 1)}
  Define $\gamma: \mathbb{Z}[\sqrt{2}] \rightarrow \mathbb{Z}[\sqrt{2}]$ by $\gamma(a + b\sqrt{2}) = a - b\sqrt{2}$.
  Show that $\gamma$ is a ring isomorphism and compute its inverse.
\end{exer}

\begin{proof}
  Let $a + b\sqrt{2}, c + d\sqrt{2} \in \mathbb{Z}[\sqrt{2}]$ be given.
  \begin{align*}
    \gamma((a + b\sqrt{2}) + (c + d\sqrt{2}))
      &= \gamma((a + c) + (b + d)\sqrt{2}) \\
      &= (a + c) - (b + d)\sqrt{2} \\
      &= (a - b\sqrt{2}) + (c - d\sqrt{2}) \\
      &= \gamma(a + b\sqrt{2}) + \gamma(c + d\sqrt{2}). \\
    \gamma((a + b\sqrt{2})(c + d\sqrt{2}))
      &= \gamma((ac + 2bd) + (ad + bc)\sqrt{2}) \\
      &= (ac + 2bd) - (ad + bc)\sqrt{2} \\
      &= (ac + 2(-b)(-d)) + (a(-d) + (-b)c)\sqrt{2} \\
      &= (a - b\sqrt{2})(c - d\sqrt{2}) \\
      &= \gamma(a + b\sqrt{2})\gamma(c + d\sqrt{2}). \\
  \end{align*}
  Moreover, $\gamma(1) = 1 - 0\sqrt{2} = 1$.
  Therefore, $\gamma$ is a ring homomorphism.
  For any $a + b\sqrt{2}$, $\gamma(\gamma(a + b\sqrt{2})) = \gamma(a - b\sqrt{2}) = a + b\sqrt{2}$.
  Therefore, $\gamma$ has an inverse, and the inverse of $\gamma$ is $\gamma$.
  This implies that $\gamma$ is bijective.

  In conclusion, $\gamma$ is an isomorphism and its inverse is itself.
\end{proof}

\begin{exer}{(Problem 2)}
  Define $N: \mathbb{Z}[\sqrt{2}] \rightarrow \mathbb{Z}$ by $N(a + b\sqrt{2}) = (a + b\sqrt{2})\gamma(a + b\sqrt{2})$.
  Show that $N(\alpha\beta) = N(\alpha)N(\beta)$.
\end{exer}

\begin{proof}
  Let $a + b\sqrt{2}, c + d\sqrt{2}$ be given.
  \begin{align*}
    N((a + b\sqrt{2})(c + d\sqrt{2}))
      &= N((ac + 2bd) + (ad + bc)\sqrt{2}) \\
      &= ((ac + 2bd) + (ad + bc)\sqrt{2})\gamma((ac + 2bd) + (ad + bc)\sqrt{2}) \\
      &= (a + b\sqrt{2})(c + d\sqrt{2})\gamma((a + b\sqrt{2})(c + d\sqrt{2})) \\
      &= (a + b\sqrt{2})(c + d\sqrt{2})\gamma(a + b\sqrt{2})\gamma(c + d\sqrt{2}) \\
      &= (a + b\sqrt{2})\gamma(a + b\sqrt{2})(c + d\sqrt{2})\gamma(c + d\sqrt{2}) \\
      &= N(a + b\sqrt{2})N(c + d\sqrt{2}).
  \end{align*}
\end{proof}

\begin{exer}{(Problem 3)}
  Write $\mathbb{Z}[\sqrt{2}]^*$ for the group of units in $\mathbb{Z}[\sqrt{2}]$.
  Show that $\alpha \in \mathbb{Z}[\sqrt{2}]^*$ if and only if $N(\alpha) = \pm 1$.
\end{exer}

\begin{proof}
  We have $N(1) = 1\gamma(1) = 1$.

  Let $\alpha$ be a unit and $\beta$ be the inverse.
  Then $N(\alpha\beta) = N(1) = 1$.
  Thus $1 = N(\alpha)N(\beta)$.
  Since $N(\alpha), N(\beta) \in \mathbb{Z}$, $N(\alpha) = \pm 1$.

  On the other hand, suppose that $N(\alpha) = \pm 1$ for some $\alpha$.
  \begin{itemize}
    \item
      Case 1: $N(\alpha) = 1$. 
      Then $\alpha\gamma(\alpha) = 1$, so $\gamma(\alpha)$ is an inverse of $\alpha$.
      Therefore, $\alpha$ is a unit.
    \item
      Case 2: $N(\alpha) = -1$. 
      Then $\alpha\gamma(\alpha) = -1$, so $-\gamma(\alpha)$ is an inverse of $\alpha$.
      Therefore, $\alpha$ is a unit.
  \end{itemize}
  In each case, $\alpha$ is a unit.

  Therefore, $N(\alpha) = \pm 1$ if and only if $\alpha$ is a unit.
\end{proof}

\begin{exer}{(Problem 4)}
  What does finding the units in $\mathbb{Z}[\sqrt{2}]$ have to do with solving the equation $x^2 - 2y^2 = \pm 1$?
\end{exer}

\begin{proof}
  Let $(a, b)$ be a solution to the equation.
  Then $a^2 - 2b^2 = \pm 1$, so $(a + b\sqrt{2})(a - b\sqrt{2}) = \pm 1$.
  This implies that $a \pm b\sqrt{2}$ is a unit in $\mathbb{Z}[\sqrt{2}]$.

  On the other hand, let $a + b\sqrt{2}$ be a unit in $\mathbb{Z}[\sqrt{2}]$.
  By Problem 3, $N(a + b\sqrt{2}) = \pm 1$.
  Thus $\pm 1 = N(a + b\sqrt{2}) = (a + b\sqrt{2})(a - b\sqrt{2}) = a^2 - b^2$.
  Hence, $(a, b)$ is a solution to $x^2 - 2y^2 = \pm 1$.

  In conclusion, there exists a bijective correspondence between the units in $\mathbb{Z}[\sqrt{2}]$ and the solutions to $x^2 - 2y^2 = \pm 1$.
\end{proof}

\begin{exer}{(Problem 5)}
  Show that $\mathbb{Z}[\sqrt{2}]$ has no smallest positive element.
\end{exer}

\begin{proof}
  We have $0 < \sqrt{2} - 1 < 1$.
  Since $\forall n \in \mathbb{N}, (\sqrt{2} - 1)^n \in \mathbb{Z}[\sqrt{2}]$ and $\lim_{n \rightarrow \infty} (\sqrt{2} - 1)^n = 0$, there exists no smallest positive element in $\mathbb{Z}[\sqrt{2}]$.
\end{proof}

\begin{exer}{(Problem 6)}
  Find an element $u \in \mathbb{Z}[\sqrt{2}]^*$ with $u > 1$.
\end{exer}

\begin{proof}
  $(\sqrt{2} + 1)(\sqrt{2} - 1) = 2 - 1 = 1$.
  Thus $u = \sqrt{2} + 1$ is a unit such that $u > 1$.
\end{proof}

\begin{exer}{(Problem 7)}
  Let $u \in \mathbb{Z}[\sqrt{2}]^*$ with $u > 1$.
  Write $u = a + b\sqrt{2}$ with $a, b \in \mathbb{Z}$.
  Show $a > 0$ and $b > 0$.
\end{exer}

\begin{proof}
  Since $u$ is a unit, $N(u) = \pm 1$ from Problem 3.
  In other words, $(a + b\sqrt{2})(a - b\sqrt{2}) = a^2 - 2b^2 = \pm 1$.
  Then $a^2 = \pm 1 + 2b^2 \equiv 1 \pmod 2$, so $a$ is odd.
  Specifically, $a \ne 0$.
  \begin{itemize}
    \item
      Case 1: $a < 0$.
      Since $a$ is an integer, $a \leq -1$.
      Since $u = a + b\sqrt{2} > 1$, $b > 0$.
      Since $b$ is an integer, $b \geq 1$.
      This implies that $a - b\sqrt{2} \leq -1 - \sqrt{2} < -1$.

      This means $(a + b\sqrt{2})(a - b\sqrt{2}) < -1$ because $a + b\sqrt{2} > 1$.
      However, this is impossible because $(a + b\sqrt{2})(a - b\sqrt{2}) = \pm 1$.
      This is a contradiction, so $a$ is not negative.
    \item
      Case 2: $a > 0$ and $b < 0$.
      Since $a, b$ are integers, this implies $a \geq 1$ and $b \leq -1$.
      Then $a - b\sqrt{2} \geq 1 + \sqrt{2} > 2$.
      Since $a + b\sqrt{2} > 1$, this implies $(a + b\sqrt{2})(a - b\sqrt{2}) > 1 \cdot 2 = 2$.
      This is a contradiction because we have $(a + b\sqrt{2})(a - b\sqrt{2}) = \pm 1$.
  \end{itemize}
  Therefore, both $a$ and $b$ must be positive.
\end{proof}

\begin{exer}{(Problem 8)}
  Show that among all $u$ satisfying the conditions of 7, there is a least element $u_0$.
  What is $u_0$?
\end{exer}

\begin{proof}
  Since we know that $a \geq 1$ and $b \geq 1$, $1 + \sqrt{2}$ is less than or equal to all such $u$.
  Since $(1 + \sqrt{2})(\sqrt{2} - 1) = 1$, $1 + \sqrt{2}$ is indeed a unit.
  Therefore, $1 + \sqrt{2}$ is the least element in $\mathbb{Z}[\sqrt{2}]^*$.
\end{proof}

\begin{exer}{(Problem 9)}
  Show that every element of $\mathbb{Z}[\sqrt{2}]^*$ is of the form $\pm u_0^n$, $n \in \mathbb{Z}$.
\end{exer}

\begin{proof}
  Let $u \in \mathbb{Z}[\sqrt{2}]^*$.
  \begin{itemize}
    \item
      Case 1: $1 < u$.
      Since $1 + \sqrt{2}$ is the least element among all units greater than 1, there must exist an $n \in \mathbb{N}$ such that $(1 + \sqrt{2})^n \leq u < (1 + \sqrt{2})^{n + 1}$.
      This implies that $1 \leq \frac{u}{(1 + \sqrt{2})^n} < 1 + \sqrt{2}$.
      Since $u$ and $1 + \sqrt{2}$ are both units, $\frac{u}{(1 + \sqrt{2})^n}$ is a unit in $\mathbb{Z}[\sqrt{2}]$ as well.
      Since $1 + \sqrt{2}$ is the least element among all units greater than 1, $u / (1 + \sqrt{2})^n = 1$.
      Therefore, $u = (1 + \sqrt{2})^n$.
    \item
      Case 2: $u = 1$.
      Then $u = (1 + \sqrt{2})^0$.
    \item
      Case 3: $0 < u < 1$.
      Then $1 / u \in \mathbb{Z}[\sqrt{2}]^*$, and $1 < 1 / u$.
      By Case 1, $1 / u = (1 + \sqrt{2})^n$ for some $n \in \mathbb{Z}$.
      Therefore, $u = (1 + \sqrt{2})^{-n}$.
    \item
      Case 4: $-1 < u < 0$.
      Then $-u \in \mathbb{Z}[\sqrt{2}]^*$ and $0 < -u < 1$.
      By Case 3, $-u = (1 + \sqrt{2})^{n}$ for some $n \in \mathbb{Z}$.
      Thus $u = -(1 + \sqrt{2})^n$.
    \item
      Case 5: $u = -1$.
      Then $u = -(1 + \sqrt{2})^0$.
    \item
      Case 6: $u < -1$.
      Then $-u \in \mathbb{Z}[\sqrt{2}]^*$ and $1 < -u$.
      By Case 1, $-u = (1 + \sqrt{2})^n$ for some $n \in \mathbb{Z}$.
      Therefore, $u = -(1 + \sqrt{2})^n$.
  \end{itemize}
  Therefore, $u$ is indeed of the form $\pm(1 + \sqrt{2})^n$ with $n \in \mathbb{Z}$.
\end{proof}

\begin{exer}{(Problem 10)}
  Describe all solutions to $x^2 - 2y^2 = 1$.
\end{exer}

\begin{proof}
  We claim that $(x, y) \in \mathbb{Z}^2$ is a solution to $x^2 - 2y^2 = 1$ if and only if $x + y\sqrt{2} = (1 + \sqrt{2})^{2n}$ for some $n \in \mathbb{Z}$.

  Let $x, y \in \mathbb{Z}$.
  \begin{itemize}
    \item
      $x^2 - 2y^2 = 1$ if and only if $N(x + \sqrt{2}y) = 1$.
    \item
      We showed in Problem 3 that $x + \sqrt{2}y \in \mathbb{Z}[\sqrt{2}]^*$ if and only if $N(x + \sqrt{2}y) = \pm 1$.
    \item
      We showed in Problem 9 that every element in $\mathbb{Z}[\sqrt{2}]^*$ is of the form $\pm u_0^n$ for some $n \in \mathbb{Z}$.
  \end{itemize}

  Therefore, we will first check which $\pm u_0^n$ satisfies $N(\pm u_0^n) = 1$.
  We claim that $N(u_0^{2n}) = N(-u_0^{2n}) = 1$ for all $n \in \mathbb{Z}$.
  \begin{itemize}
    \item
      When $n = 0$, this is clearly true.
    \item
      Suppose that $N(u_0^{2n}) = 1$ for some $n \in \mathbb{N}$.
      Let $x + \sqrt{2}y = u_0^{2n}$ where $x, y \in \mathbb{Z}$.
      Then $u_0^{2n + 2} = (x + \sqrt{2}y)(1 + \sqrt{2})^2 = (x + \sqrt{2}y)(3 + 2\sqrt{2}) = (3x + 4y) + (2x + 3y)\sqrt{2}$.
      \begin{align*}
        N(u_0^{2n + 2})
          &= ((3x + 4y) + (2x + 3y)\sqrt{2})((3x + 4y) - (2x + 3y)\sqrt{2}) \\
          &= (9x^2 + 24xy + 16y^2) - 2(4x^2 + 12xy + 9y^2) \\
          &= x^2 - 2y^2 \\
          &= N(u_0^{2n}) = 1.
      \end{align*}

      By mathematical induction, $N(u_0^{2n}) = 1$ for all $n \in \mathbb{N}$.
    \item
      Let $n \in \mathbb{N}$.
      Let $x + y\sqrt{2} = u_0^{2n}$ where $x, y \in \mathbb{Z}$.
      \begin{align*}
        \frac{1}{u_0^{2n}}
          &= \frac{1}{x + y\sqrt{2}} \\
          &= \frac{x - y\sqrt{2}}{x^2 - 2y^2} \\
          &= \frac{x - y\sqrt{2}}{N(x + y\sqrt{2})} \\
          &= \frac{x - y\sqrt{2}}{N(u_0^{2n})} \\
          &= x - y\sqrt{2}.
      \end{align*}
      Since $N(x - y\sqrt{2}) = N(x + y\sqrt{2}) = 1$, $N(u_0^{-2n}) = 1$ for all $n \in \mathbb{N}$.
    \item
      Let $n \in \mathbb{Z}$.
      Let $x + y\sqrt{2} = u_0^{2n}$.
      \begin{align*}
        N(-u_0^{2n})
          &= N(-x - y\sqrt{2}) \\
          &= (-x - y\sqrt{2})(-x + y\sqrt{2}) \\
          &= (x + y\sqrt{2})(x - y\sqrt{2}) \\
          &= N(x + y\sqrt{2}) \\
          &= N(u_0^{2n}) = 1.
      \end{align*}
      Thus $N(-u_0^{2n}) = 1$ for all $n \in \mathbb{Z}$.
  \end{itemize}
  Therefore, $N(\pm u_0^{2n}) = 1$ for any sign and $n \in \mathbb{Z}$.
  We now claim that $N(\pm u_0^{2n + 1}) = -1$ for any sign and $n \in \mathbb{Z}$.
  Let $x + y\sqrt{2} = \pm u_0^{2n}$ for some sign and $n \in \mathbb{Z}$.
  Then $(x + y\sqrt{2})(1 + \sqrt{2}) = (x + 2y) + (x + y)\sqrt{2}$.
  \begin{align*}
    N((x + y\sqrt{2})(1 + \sqrt{2}))
      &= N((x + 2y) + (x + y)\sqrt{2}) \\
      &= ((x + 2y) + (x + y)\sqrt{2})((x + 2y) - (x + y)\sqrt{2}) \\
      &= (x + 2y)^2 - 2(x + y)^2 \\
      &= (x^2 + 4xy + 4y^2) - (2x^2 + 4xy + 2y^2) \\
      &= -x^2 + 2y^2 \\
      &= -(x^2 - 2y^2) \\
      &= -N(x + y\sqrt{2}) \\
      &= -1.
  \end{align*}
  Therefore, $N(\pm u_0^{2n + 1}) = -1$ for any sign and any $n \in \mathbb{Z}$.
  Hence, $\{ (x, y) \in \mathbb{Z}^2 \mid x + \sqrt{2}y \in \{ -u_0^{2n}, u_0^{2n} \mid n \in \mathbb{Z} \} \}$ is the set of all solutions to $x^2 - 2y^2 = 1$.
\end{proof}

\begin{exer}{(Problem 11)}
  Construct a group isomorphism $\mathbb{Z}[\sqrt{2}]^* \rightarrow \mathbb{Z} \times \mathbb{Z} / 2\mathbb{Z}$.
\end{exer}

\begin{proof}
  By Problem 9, every element in $\mathbb{Z}[\sqrt{2}]^*$ can be represented as $(-1)^a(1 + \sqrt{2})^{2k}$ for some $(k, a) \in \mathbb{Z} \times \mathbb{Z} / 2 \mathbb{Z}$.
  Let $\phi: \mathbb{Z}[\sqrt{2}]^* \rightarrow \mathbb{Z} \times \mathbb{Z} / 2\mathbb{Z}$ be defined such that $\phi((-1)^a(1 + \sqrt{2})^{2k}) = (k, a)$.
  \begin{itemize}
    \item
      Well-defined?
      Every element in $\mathbb{Z}[\sqrt{2}]^*$ can be expressed unique as $(-1)^a(1 + \sqrt{2})^{2k}$ for some $(k, a) \in \mathbb{Z} \times \mathbb{Z} / 2 \mathbb{Z}$.
      Thus $\phi$ is well defined.
    \item
      Group homomorphism?
      \begin{align*}
        \phi((-1)^a(1 + \sqrt{2})^{2k}(-1)^b(1 + \sqrt{2})^{2l})
          &= \phi((-1)^{a + b}(1 + \sqrt{2})^{2(k + l)}) \\
          &= (k + l, a + b) \\
          &= (k, a) + (l, b) \\
          &= \phi((-1)^a(1 + \sqrt{2})^{2k})\phi((-1)^b(1 + \sqrt{2})^{2l}).
      \end{align*}
    \item
      Injective?
      $\phi((-1)^a(1 + \sqrt{2})^k) = (0, 0)$ implies that $k = a = 0$.
      Therefore, 1 is the only number in the kernel of $\phi$.
      Since the kernel of $\phi$ only contains the identity element, $\phi$ is injective.
    \item
      Surjective?
      For any $(k, a) \in \mathbb{Z} \times \mathbb{Z} / 2\mathbb{Z}$, $(-1)^a(1 + \sqrt{2})^{2k} \in \mathbb{Z}[\sqrt{2}]^*$.
  \end{itemize}
  Therefore, $\phi$ is a group isomorphism.
\end{proof}



\begin{exer}{(Problem 12)}
  Show that $\mathbb{Z}[\sqrt{2}]$ is an integral domain.
\end{exer}

\begin{proof}
  $\mathbb{Z}[\sqrt{2}]$ is a commutative ring because multiplication of real numbers is commutative.
  Moreover, $\mathbb{Z}[\sqrt{2}] \subset \mathbb{R}$ where $\mathbb{R}$ is a field.
  Thus $\mathbb{Z}[\sqrt{2}]$ has no zero divisors.
  Therefore, $\mathbb{Z}[\sqrt{2}]$ is an integral domain.
\end{proof}

\begin{exer}{(Problem 13)}
  Define $\sigma: \mathbb{Z}[\sqrt{2}] \setminus \{ 0 \} \rightarrow \{ 0, 1, 2, \cdots, \}$ by $\sigma(\alpha) = \abs{N(\alpha)}$.
  Show that $(\mathbb{Z}[\sqrt{2}], \sigma)$ is a Euclidean domain.
\end{exer}

\begin{proof}
  Let $a + b\sqrt{2}, c + d\sqrt{2} \in \mathbb{Z}[\sqrt{2}]$ be given such that $c + d\sqrt{2} \ne 0$.
  Consider

  \begin{align*}
    \frac{a + b\sqrt{2}}{c + d\sqrt{2}} = \frac{ac - 2bd}{c^2 - 2d^2} + \frac{bc - ad}{c^2 - 2d^2}\sqrt{2}.
  \end{align*}

  Let $p, q \in \mathbb{Z}$ be chosen such that

  \begin{align*}
    \abs{\frac{ac - 2bd}{c^2 - 2d^2} - p} \leq \frac{1}{2}, \abs{\frac{bc - ad}{c^2 - 2d^2} - q} \leq \frac{1}{2}.
  \end{align*}

  Such $p, q$ are guaranteed to exist.
  Let $\alpha + \beta\sqrt{2}$ denote $\frac{a + b\sqrt{2}}{c + d\sqrt{2}} - (p + q\sqrt{2})$.
  Then $\abs{\alpha} \leq 1/2, \abs{\beta} \leq 1/2$.

  Let $\epsilon = (a + b\sqrt{2}) - (c + d\sqrt{2})(p + q\sqrt{2})$.
  If $\epsilon = 0$, we are done.
  Suppose otherwise.
  Then we have $a + b\sqrt{2} = (c + d\sqrt{2})(p + q\sqrt{2}) + \epsilon$.

  \begin{align*}
    \epsilon
      &= (a + b\sqrt{2}) - (c + d\sqrt{2})(p + q\sqrt{2}) \\
      &= (c + d\sqrt{2})(\frac{a + b\sqrt{2}}{c + d\sqrt{2}} - (p + q\sqrt{2})) \\
      &= (c + d\sqrt{2})(\alpha + \beta\sqrt{2}) \\
      &= (\alpha c + 2\beta d) + (c\beta + \alpha d)\sqrt{2}.
  \end{align*}

  This implies that

  \begin{align*}
    N(\epsilon)
      &= (\alpha c + 2\beta d)^2 - 2(c\beta + \alpha d)^2 \\
      &= (\alpha^2 c^2 + 2\alpha\beta cd + 4\beta^2d^2) - 2(c^2\beta^2 + 2\alpha\beta cd + \alpha^2 d^2) \\
      &= \alpha^2(c^2 - 2d^2) - 2\beta^2(c^2 - 2d^2) \\
      &= (c^2 - 2d^2)(\alpha^2 - 2\beta^2) \\
      &= (\alpha^2 - 2\beta^2)N(c + d\sqrt{2}).
  \end{align*}

  Therefore, $\sigma(\epsilon) = \abs{\alpha^2 - 2\beta^2}\sigma(c + d\sqrt{2})$.
  Since $\abs{\alpha^2 - 2\beta^2} \leq \abs{\alpha}^2 + 2\abs{\beta}^2 \leq 1/4 + 2 \cdot 1/4 = 3/4$, $\sigma(\epsilon) < \sigma(c + d\sqrt{2})$.
\end{proof}

\begin{exer}{(Problem 14)}
  Conclude that $\mathbb{Z}[\sqrt{2}]$ is a principal ideal domain and a unique factorization domain.
\end{exer}

\begin{proof}
  In class, we proved that every principal ideal domain is a unique factorization domain.
  Therefore, it suffices to show that $\mathbb{Z}[\sqrt{2}]$ is a principal ideal domain.
  Let $I$ be an ideal of $\mathbb{Z}[\sqrt{2}]$.
  If $I = (0)$, we are done.
  Suppose otherwise.
  Let $S = \{ \abs{N(\alpha)} \mid \alpha \in I, \alpha \ne 0 \}$.
  Since $S$ is a nonempty set of positive integers, there exists a minimum value $m$.
  Let $\beta \in I$ be an element such that $\abs{N(\beta)} = m$.
  We claim that $I = (\beta)$.

  Suppose otherwise.
  Let $\alpha \in I \setminus (\beta)$.
  By Problem 13, there exist $\delta, \epsilon \in \mathbb{Z}[\sqrt{2}]$ such that $\alpha = \beta\delta + \epsilon$ with $\abs{N(\epsilon)} < \abs{N(\beta)}$.
  $\epsilon$ cannot be $0$ because $\alpha \notin (\beta)$.
  Since $I$ is an ideal, $\beta\delta \in I$.
  This implies that $\epsilon = \alpha - \beta\delta \in I$.
  However, this is a contradiction because $\beta$ was chosen because $\abs{N(\beta)} \leq \abs{N(\beta')}$ for all $\beta' \in I$.
  Therefore, $I = (\beta)$, and thus $\mathbb{Z}[\sqrt{2}]$ is a principal ideal domain and a unique factorization domain.
\end{proof}

\end{document}
