\documentclass[12pt, psamsfonts]{amsart}

%-------Packages---------
\usepackage{amssymb,amsfonts}
\usepackage{fullpage}
\usepackage{tikz-cd}
\usepackage{todonotes}
\usepackage{physics}
\usepackage[all,arc]{xy}
\usepackage{enumerate}
\usepackage{enumitem}
\usepackage{mathrsfs}
\usepackage{theoremref}
\usepackage{graphicx}
\usepackage[bookmarks]{hyperref}

%--------Theorem Environments--------
%theoremstyle{plain} --- default
\newtheorem{thm}{Theorem}[section]
\newtheorem{cor}[thm]{Corollary}
\newtheorem{prop}[thm]{Proposition}
\newtheorem{lem}[thm]{Lemma}
\newtheorem{conj}[thm]{Conjecture}
\newtheorem{quest}[thm]{Question}

\theoremstyle{definition}
\newtheorem{defn}[thm]{Definition}
\newtheorem{defns}[thm]{Definitions}
\newtheorem{con}[thm]{Construction}
\newtheorem{exmp}[thm]{Example}
\newtheorem{exmps}[thm]{Examples}
\newtheorem{notn}[thm]{Notation}
\newtheorem{notns}[thm]{Notations}
\newtheorem{addm}[thm]{Addendum}
\newtheorem*{exer}{Exercise}

\theoremstyle{remark}
\newtheorem{rem}[thm]{Remark}
\newtheorem{rems}[thm]{Remarks}
\newtheorem{warn}[thm]{Warning}
\newtheorem{sch}[thm]{Scholium}

\DeclareMathOperator{\Hom}{Hom}
\DeclareMathOperator{\Id}{Id}
\DeclareMathOperator{\End}{End}
\DeclareMathOperator{\ord}{ord}
\DeclareMathOperator{\Aut}{Aut}

\makeatletter
\let\c@equation\c@thm
\makeatother
\numberwithin{equation}{section}

\bibliographystyle{plain}

\begin{document}

\title{Math 601 (Due 11/6)}
\author{Hidenori Shinohara}
\maketitle

\tableofcontents

\section{Galois Theory II (P.2)}

\begin{exer}{(Problem 1)}
  Let $f(x) \in F[x]$ be an irreducible polynomial of degree $d$.
  Let $F \subset K$ be a field extension such that $f(x)$ factors as a product of linear polynomials in $K[x]$.
  Show that $f(x)$ is separable if and only if there exist $d$ distinct $F$-algebra homomorphisms, $F[x]/(f(x)) \rightarrow K$.
\end{exer}

\begin{proof}
  Without loss of generality, assume $f(x)$ is monic and $f(x) = \prod_{i=1}^{d} (x - a_i)$ for some $a_i \in K$.

  Suppose $f(x)$ is separable.
  Then $a_i \ne a_j$ for all $i \ne j$.
  For each $i$, let $\phi_i: F[x]/(f(x)) \rightarrow K$ be an $F$-algebra homomorphism such that $x \mapsto a_i$ and $a \mapsto a$ for all $a \in F$.
  Then each $\phi_i$ is distinct because $\phi_i(x) \ne \phi_j(x)$ whenever $i \ne j$.
  Thus we showed the existence of $d$ distinct $F$-algebra homomorphisms.

  Suppose there exist $d$ distinct homomorphisms $\phi_i$ for $i = 1, \cdots, d$.
  For any $j$, $\prod_{i=1}^{d}(\phi_j(x) - a_i) = \phi_j(\prod_{i=1}^{d}(x - a_i)) = \phi_j(f(x)) = 0$, so $\phi_j(x) \in K$ is a root of $f(x)$.
  Thus $x - \phi_i(x)$ divides $f(x)$ for each $i$.
  Since $\phi_i$ is uniquely determined by the value $\phi_i(x)$, $\phi_i(x) \ne \phi_j(x)$ whenever $i \ne j$.
  Thus $f(x) = \prod_{i=1}^{d}(x - \phi_i(x))$, and $f(x)$ is separable.
\end{proof}

\begin{exer}{(Problem 2)}
  Let $F \subset F[v_1, \cdots, v_r] = K$ be an algebraic field extension such that the irreducible monic polynomial, $f_i(x) \in F[x]$, for $v_i$ is separable for each $i$.
  Let $F \subset L$ be a splitting field of $f(x) := \prod_{i=1}^r f_i(x) \in F[x]$.
  Let $w \in K$ and let $g(x) \in F[x]$ be the minimal manic polynomial of $w$.
  Set $d = \deg(g(x))$.
  Show that there are exactly $d$ distinct $F$-algebra homomorphisms, $F[w] \rightarrow L$.
\end{exer}

\begin{proof}
  $ $
  \todo[inline,caption={}]{
     Because of Problem 3, I don't think I'm supposed to show that $g$ is separable.
  }
\end{proof}

\begin{exer}{(Problem 3)}
  Let $F \subset F[v_1, \cdots, v_r] = K$ be as in the previous problem.
  Let $w \in K$.
  Show that the monic irreducible polynomial of $w$ is separable.
\end{exer}

\begin{proof}
  By Problem 1 and 2, this is trivial because $F[w]$ is isomorphic to $F[x]/(f(x))$ by Lemma 2.1 (Field Extension handout).
\end{proof}

\section{Galois Theory II (P.8)}

\begin{exer}{(Problem 1)}
  Recall that $p$ is prime and $q$ is a power of $p$.
  Define $F_q: \mathbb{F}_{q^r} \rightarrow \mathbb{F}_{q^r}$ by $F_q(a) = a^q$.
  Show that $F_q \in \Aut(\mathbb{F}_{q^r}/\mathbb{F}_q)$.
\end{exer}

\begin{proof}
  $F_q(a + b) = (a + b)^q = a^q + b^q$ since $p \mid \binom{q}{i}$ for $1 \leq i \leq q - 1$.
  Thus $F_q$ preserves addition, and it is clear that $F_q$ preserves multiplication, so $F_q$ is a homomorphism.
  Moreover, any element in $\mathbb{F}_q$ satisfies $x^q - x = 0$, so $F_q(a) = a^q = a$ for any $a \in \mathbb{F}_q$.

  Finally, in order to show that $F_q$ is bijective, it suffices to check if it is injective since $\mathbb{F}_{q^r}$ is finite.
  $F_q(a) = 0 \implies a^q = 0 \implies a = 0$, so $F_q$ is indeed injective.
\end{proof}

\begin{exer}{(Problem 2)}
  Show that $F_p: \mathbb{F}_{q^r} \rightarrow \mathbb{F}_{q^r}$, $F_p(a) = a^p$ is not an element of $\Aut(\mathbb{F}_{q^r} / \mathbb{F}_q)$ unless $q = p$.
\end{exer}

\begin{proof}
  If $q = p$, we are done.
  Suppose $q > p$.
  Let $\ev{\alpha} = (\mathbb{F}_{q})^{*}$.
  Then the order of $\alpha$ is $q - 1$, so $F_p(\alpha) = \alpha^p \ne \alpha$.
\end{proof}

\begin{exer}{(Problem 3)}
  Let $f(x) \in \mathbb{F}_q[x]$ be a monic irreducible polynomial of degree $r$.
  Explain why $f(x)$ has a root $\alpha \in \mathbb{F}_{q^r}$.
\end{exer}

\begin{proof}
  Let $f(x) = \sum_{i=0}^{r} a_ix^i$.
  Since $\ev{f(x)}$ is a maximal ideal, $\mathbb{F}_q[x]/\ev{f(x)}$ is a field with an $\mathbb{F}_q$-basis $\{ 1, x, \cdots, x^{d - 1} \}$.
  Thus the field contains $q^r$ elements.
  By the uniqueness of a finite field, there exists an isomorphism $\phi: \mathbb{F}_{q^r} \rightarrow \mathbb{F}_q[x]/\ev{f(x)}$.
  Let $\alpha = \phi^{-1}(x)$.
  Then $\phi(\sum_{i=0}^{r} a_i\alpha^i) = \sum_{i=0}^{r}a_ix^i = 0$.
  Thus $\mathbb{F}_{q^r}$ contains a root of $f(x)$.
\end{proof}

\begin{exer}{(Problem 4)}
  With $f(x)$ as in the previous problem, show that $f(x) = \prod_{i=0}^{r-1} (x - \alpha^{q^i}) \in \mathbb{F}_{q^r}[x]$.
  Conclude that $\mathbb{F}_{q^r}$ is a splitting field for $f(x)$ over $\mathbb{F}_q$.
  In other words, $\alpha^{q^i}$ is a root of $f(x)$ for any $i \in \mathbb{N}$.
  \todo[inline,caption={}]{
    How do I show that $\alpha^{q^i} \ne \alpha^{q^j}$ if $0 \leq i < j \leq r - 1$?
  }
\end{exer}

\begin{proof}
  Let $f(x) = \sum_{i=0}^{r} a_ix^i$.
  Then $(f(x))^q = (\sum_{i=0}^{r} a_ix^i)^q = \sum_{i=0}^{r}a_i^q(x^q)^i = \sum_{i=0}^r a_i(x^q)^i$.
  Thus the $q$th power of any root $\beta$ of $f(x)$ is a root of $f(x)$.
\end{proof}

\section{Factoring Polynomials with Coefficients in Finite Fields}

\begin{exer}{(Problem 9)}
  Let $\mathbb{F}_q$ be a field with $q = p^m$ elements.
  Let $f(x) \in \mathbb{F}_q[x]$ be square free.
  Describe $\gcd(x^q - x, f(x))$ in terms of the linear factors of $f(x)$.
\end{exer}

\begin{proof}
  Since $(x^q - x)' = -1$, $\gcd(x^q - x, (x^q - x)') = 1$.
  Thus $x^q - x$ is square free by Problem 7 from last week.
  Thus $x^q - x = \prod_{i=1}^{q} (x - a_i)$ where $\mathbb{F}_q = \{ a_1, \cdots, a_q \}$.
  Each linear factor (if any) of $f(x)$ is associate to $x - a_i$ for some $i$.
  Since $f(x)$ is square free, $\gcd(x^q - x, f(x))$ is the product of all the linear factors of $f(x)$.
\end{proof}

\begin{exer}{(Problem 10)}
  Let $f(x) \in \mathbb{F}_q[x]$ be square free.
  Describe, $h(x) = \gcd(x^{q^2} - x, f(x))$, in terms of the irreducible quadratic polynomials which divide $f(x)$ and whatever other information is necessary.
\end{exer}

\begin{proof}
  Since every element in $\mathbb{F}_q$ is a root of $x^{q^2} - x$, $h(x)$ is divisible by all the linear polynomials that divide $f(x)$.

  Let $g(x) \in \mathbb{F}_q[x]$ be an irreducible monic quadratic polynomial.
  Then $\mathbb{F}_q[x]/(g(x)) \cong \mathbb{F}_{q^2}$ with an isomorphism $\phi$.
  Then $\phi(x)$ is a root of $g(x)$.
  Thus $g = (x - \alpha)(x - \beta)$ in $\mathbb{F}_{q^2}[x]$.

  Moreover, every element in $\mathbb{F}_{q^2}$ is a root of $x^{q^2} - x$.
  Thus $g = (x - \alpha)(x - \beta) \mid x^{q^2} - x$.
  Therefore, $h(x)$ is divisible by all the irreducible monic quadratic polynomials that divide $f(x)$.

  Finally, the set of roots of $x^{q^2} - x$ is exactly $\mathbb{F}_{q^2}$.
  Since $[\mathbb{F}_{q^2}:\mathbb{F}_q] = 2$, the degree of the minimal polynomial of each element must be either 1 or 2.
  In other words, $x^{q^2} - x$ is a product of some linear and quadratic polynomials in $\mathbb{F}_q[x]$.

  Therefore, $h(x)$ is exactly the product of all the irreducible monic polynomials of degree 1 or 2 that divide $f(x)$.
  ($x^{q^2} - x$ may or may not be square free, but $f(x)$ is square free, so $h(x)$ must be square free.)
\end{proof}

\begin{lem}\label{mylem}
  Suppose $f \in \mathbb{F}_q[x]$ is irreducible.
  Let $d \in \mathbb{N}$.
  Then $f \mid (x^{q^d} - x)$ if and only if $\deg(f) \mid d$.
\end{lem}

\begin{proof}
  Let $d \in \mathbb{N}$ be given.
  Let $n = \deg(f)$.
  Then $\mathbb{F}_q[x]/(f(x)) = \mathbb{F}_{q^n}$ contains a root $\alpha$ of $f(x)$.

  Suppose $n \mid d$.
  $\alpha^{q^n} - \alpha = 0$ implies $0 = (\alpha^{q^n} - \alpha)^{q^n} = \alpha^{q^{2n}} - \alpha^{q^n} = \alpha^{q^{2n}} - \alpha$.
  By repeating this process, we get $\alpha^{q^{d}} - \alpha = 0$ since $n \mid d$.
  Thus $\alpha$ satisfies $f(x)$ and $x^{q^d} - x$, and $f(x)$ is irreducible.
  Thus $f \mid x^{q^d} - x$.

  Suppose $f(x) \mid (x^{q^d} - x)$.
  Since $f(x)$ is an irreducible polynomial with a root $\alpha$, it must be the minimal polynomial of $\alpha$.
  Thus $[\mathbb{F}_q(\alpha):\mathbb{F}_q] = n$.
  $f(x) \mid (x^{q^d} - x)$ implies that $\alpha$ satisfies $x^{q^d} - x$.
  Thus $\alpha \in \mathbb{F}_{q^d}$.
  Then $d = [\mathbb{F}_{q^d}:\mathbb{F}_q(\alpha)][\mathbb{F}_q(\alpha):\mathbb{F}_q]$, so $n \mid d$.
\end{proof}

\begin{exer}{(Problem 11)}
  Given a square free polynomial $f(x) \in \mathbb{F}_q[x]$, describe how to use repeated $\gcd$ calculations to factor $f(x)$ as $f = f_1f_2 \cdots f_r$, where each $f_i$ is a product of distinct irreducible factors of degree $i$.
\end{exer}

\begin{proof}
  We will use Lemma \ref{mylem} above.
  We will start with $n = 1$.
  \begin{itemize}
    \item
      If $f(x)$ is a unit, terminate.
    \item
      Calculate $h(x) = \gcd(x^{q^n} - x, f(x))$.
      This is the product of all irreducible polynomials of $f(x)$ of degree $n$ by Lemma \ref{mylem}.
    \item
      Record $h(x)$.
      Set $f(x) = f(x) / h(x)$ and $n = n + 1$.
      Repeat.
  \end{itemize}
  Then the $h$'s that we record are the products of distinct irreducible of factors of degree $i$ for each $i$.
\end{proof}

\begin{exer}{(Problem 12)}
  Prove the following criterion for a degree $n$ polynomial $f(x) \in \mathbb{F}_q[x]$ to be irreducible:
  $f(x)$ is irreducible if and only if
  \begin{itemize}
    \item
      $\gcd(f(x), x^{q^n} - x) = f(x)$, and
    \item
      For each proper divisor $d$ of $n$, $\gcd(f(x), x^{q^d} - x) = 1$.
  \end{itemize}
\end{exer}

\begin{proof}
  Suppose $f(x)$ is irreducible.
  By Lemma \ref{mylem}, $\gcd(f(x), x^{q^n} - x) = f(x)$.
  Since the same lemma implies that $x^{q^d} - x$ cannot be divided by any irreducible polynomial of degree $> d$, $\gcd(f(x), x^{q^d} - x) = 1$.

  Suppose the two conditions are met.
  We will show that $f(x)$ is irreducible.
  Let $g(x)$ be an irreducible polynomial that divides $f(x)$.
  Since $\gcd(f(x), x^{q^d} - x) = 1$ for each proper divisor $d$ of $n$, $\gcd(g(x), x^{q^d} - x) = 1$ as well.
  By the lemma, $\deg(g(x)) \nmid d$.
  Since $\gcd(f, x^{q^n} - x) = f$, $\gcd(g, x^{q^n} - x) = g$.
  By the lemma, $\deg(g) \mid n$.
  Therefore, $\deg(g)$ is a divisor of $n$ that is not a proper divisor of $n$.
  In other words, $\deg(g) = n$, so $f$ is irreducible.
\end{proof}

\begin{exer}{(Problem 13)}
  Suppose $f(x) \in \mathbb{F}_q[x]$ is a product of $m$ distinct monic irreducible polynomials of degree $r$.
  To what ring is $\mathbb{F}_q[x]/(f(x))$ isomorphic?
\end{exer}

\begin{proof}
  By the Chinese remainder theorem, $\mathbb{F}_q[x]/(f(x)) = \mathbb{F}_q[x]/(f_1) \times \cdots \times \mathbb{F}_q[x]/(f_m)$.
  Thus $\mathbb{F}_q[x]/(f(x)) = \mathbb{F}_{q^r} \times \cdots \times \mathbb{F}_{q^r}$ ($m$ times)
\end{proof}



\end{document}


