\documentclass[12pt, psamsfonts]{amsart}

%-------Packages---------
\usepackage{amssymb,amsfonts}
\usepackage{fullpage}
\usepackage{tikz-cd}
\usepackage{todonotes}
\usepackage{physics}
\usepackage[all,arc]{xy}
\usepackage{enumerate}
\usepackage{enumitem}
\usepackage{mathrsfs}
\usepackage{theoremref}
\usepackage{graphicx}
\usepackage[bookmarks]{hyperref}

%--------Theorem Environments--------
%theoremstyle{plain} --- default
\newtheorem{thm}{Theorem}[section]
\newtheorem{cor}[thm]{Corollary}
\newtheorem{prop}[thm]{Proposition}
\newtheorem{lem}[thm]{Lemma}
\newtheorem{conj}[thm]{Conjecture}
\newtheorem{quest}[thm]{Question}

\theoremstyle{definition}
\newtheorem{defn}[thm]{Definition}
\newtheorem{defns}[thm]{Definitions}
\newtheorem{con}[thm]{Construction}
\newtheorem{exmp}[thm]{Example}
\newtheorem{exmps}[thm]{Examples}
\newtheorem{notn}[thm]{Notation}
\newtheorem{notns}[thm]{Notations}
\newtheorem{addm}[thm]{Addendum}
\newtheorem*{exer}{Exercise}

\theoremstyle{remark}
\newtheorem{rem}[thm]{Remark}
\newtheorem{rems}[thm]{Remarks}
\newtheorem{warn}[thm]{Warning}
\newtheorem{sch}[thm]{Scholium}

\DeclareMathOperator{\Hom}{Hom}
\DeclareMathOperator{\Id}{Id}
\DeclareMathOperator{\End}{End}
\DeclareMathOperator{\ord}{ord}
\DeclareMathOperator{\Aut}{Aut}

\makeatletter
\let\c@equation\c@thm
\makeatother
\numberwithin{equation}{section}

\bibliographystyle{plain}

\begin{document}

\title{Math 601 (Due 11/6)}
\author{Hidenori Shinohara}
\maketitle

\tableofcontents

\section{Galois Theory II (P.2)}

\begin{exer}{(Problem 1)}
  Let $f(x) \in F[x]$ be an irreducible polynomial of degree $d$.
  Let $F \subset K$ be a field extension such that $f(x)$ factors as a product of linear polynomials in $K[x]$.
  Show that $f(x)$ is separable if and only if there exist $d$ distinct $F$-algebra homomorphisms, $F[x]/(f(x)) \rightarrow K$.
\end{exer}

\begin{proof}
  Without loss of generality, assume $f(x)$ is monic and $f(x) = \prod_{i=1}^{d} (x - a_i)$ for some $a_i \in K$.

  Suppose $f(x)$ is separable.
  Then $a_i \ne a_j$ for all $i \ne j$.
  For each $i$, let $\phi_i: F[x]/(f(x)) \rightarrow K$ be an $F$-algebra homomorphism such that $x \mapsto a_i$ and $a \mapsto a$ for all $a \in F$.
  Then each $\phi_i$ is distinct because $\phi_i(x) \ne \phi_j(x)$ whenever $i \ne j$.
  Thus we showed the existence of $d$ distinct $F$-algebra homomorphisms.

  Suppose there exist $d$ distinct homomorphisms $\phi_i$ for $i = 1, \cdots, d$.
  For any $j$, $\prod_{i=1}^{d}(\phi_j(x) - a_i) = \phi_j(\prod_{i=1}^{d}(x - a_i)) = \phi_j(f(x)) = 0$, so $\phi_j(x) \in K$ is a root of $f(x)$.
  Thus $x - \phi_i(x)$ divides $f(x)$ for each $i$.
  Since $\phi_i$ is uniquely determined by the value $\phi_i(x)$, $\phi_i(x) \ne \phi_j(x)$ whenever $i \ne j$.
  Thus $f(x) = \prod_{i=1}^{d}(x - \phi_i(x))$, and $f(x)$ is separable.
\end{proof}

\begin{exer}{(Problem 2)}
  Let $F \subset F[v_1, \cdots, v_r] = K$ be an algebraic field extension such that the irreducible manic polynomial, $f_i(x) \in F[x]$, for $v_i$ is separable for each $i$.
  Let $F \subset L$ be a splitting field of $f(x) := \prod_{i=1}^r f_i(x) \in F[x]$.
  Let $w \in K$ and let $g(x) \in F[x]$ be the minimal manic polynomial of $w$.
  Set $d = \deg(g(x))$.
  Show that there are exactly $d$ distinct $F$-algebra homomorphisms, $F[w] \rightarrow L$.
\end{exer}

\begin{proof}
  $ $
  \todo[inline,caption={}]{
     Because of Problem 3, I don't think I'm supposed to show that $g$ is separable.
  }
\end{proof}

\begin{exer}{(Problem 3)}
  Let $F \subset F[v_1, \cdots, v_r] = K$ be as in the previous problem.
  Let $w \in K$.
  Show that the monic irreducible polynomial of $w$ is separable.
\end{exer}

\begin{proof}
$ $
 \todo[inline,caption={}]{
   Can I just use the results of Problem 1 and 2?
 }
\end{proof}

\section{Galois Theory II (P.8)}

\begin{exer}{(Problem 1)}
  Recall that $p$ is prime and $q$ is a power of $p$.
  Define $F_q: \mathbb{F}_{q^r} \rightarrow \mathbb{F}_{q^r}$ by $F_q(a) = a^q$.
  Show that $F_q \in \Aut(\mathbb{F}_{q^r}/\mathbb{F}_q)$.
\end{exer}

\begin{proof}
  $F_q(a + b) = (a + b)^q = a^q + b^q$ since $p \mid \binom{q}{i}$ for $1 \leq i \leq q - 1$.
  Thus $F_q$ preserves addition, and it is clear that $F_q$ preserves multiplication, so $F_q$ is a homomorphism.
  Moreover, any element in $\mathbb{F}_q$ satisfies $x^q - x = 0$, so $F_q(a) = a^q = a$ for any $a \in \mathbb{F}_q$.
\end{proof}

\section{Factoring Polynomials with Coefficients in Finite Fields}

\begin{exer}{(Problem 9)}
  Let $\mathbb{F}_q$ be a field with $q = p^m$ elements.
  Let $f(x) \in \mathbb{F}_q[x]$ be square free.
  Describe $\gcd(x^q - x, f(x))$ in terms of the linear factors of $f(x)$.
\end{exer}

\begin{proof}
  Since $(x^q - x)' = -1$, $\gcd(x^q - x, (x^q - x)') = 1$.
  Thus $x^q - x$ is square free by Problem 7 from last week.
  Thus $x^q - x = \prod_{i=1}^{q} (x - a_i)$ where $\mathbb{F}_q = \{ a_1, \cdots, a_q \}$.
  Each linear factor (if any) of $f(x)$ is associate to $x - a_i$ for some $i$.
  Since $f(x)$ is square free, $\gcd(x^q - x, f(x))$ is the product of all the linear factors of $f(x)$.
\end{proof}

\end{document}


