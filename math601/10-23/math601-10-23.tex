\documentclass[12pt, psamsfonts]{amsart}

%-------Packages---------
\usepackage{amssymb,amsfonts}
\usepackage{fullpage}
\usepackage{tikz-cd}
\usepackage{todonotes}
\usepackage{physics}
\usepackage[all,arc]{xy}
\usepackage{enumerate}
\usepackage{enumitem}
\usepackage{mathrsfs}
\usepackage{theoremref}
\usepackage{graphicx}
\usepackage[bookmarks]{hyperref}

%--------Theorem Environments--------
%theoremstyle{plain} --- default
\newtheorem{thm}{Theorem}[section]
\newtheorem{cor}[thm]{Corollary}
\newtheorem{prop}[thm]{Proposition}
\newtheorem{lem}[thm]{Lemma}
\newtheorem{conj}[thm]{Conjecture}
\newtheorem{quest}[thm]{Question}

\theoremstyle{definition}
\newtheorem{defn}[thm]{Definition}
\newtheorem{defns}[thm]{Definitions}
\newtheorem{con}[thm]{Construction}
\newtheorem{exmp}[thm]{Example}
\newtheorem{exmps}[thm]{Examples}
\newtheorem{notn}[thm]{Notation}
\newtheorem{notns}[thm]{Notations}
\newtheorem{addm}[thm]{Addendum}
\newtheorem*{exer}{Exercise}

\theoremstyle{remark}
\newtheorem{rem}[thm]{Remark}
\newtheorem{rems}[thm]{Remarks}
\newtheorem{warn}[thm]{Warning}
\newtheorem{sch}[thm]{Scholium}

\DeclareMathOperator{\Hom}{Hom}
\DeclareMathOperator{\Id}{Id}
\DeclareMathOperator{\End}{End}
\DeclareMathOperator{\ord}{ord}
\DeclareMathOperator{\GCD}{GCD}

\makeatletter
\let\c@equation\c@thm
\makeatother
\numberwithin{equation}{section}

\bibliographystyle{plain}

\begin{document}

\title{Math 601 (Due 10/23)}
\author{Hidenori Shinohara}
\maketitle

\tableofcontents

\section{Field Extension}

\begin{exer}{(Problem 1)}
  Let $p$ be a prime number.
  Let $K = \mathbb{Z} / p\mathbb{Z}(t)$ be the fraction field of $\mathbb{Z} / p\mathbb{Z}[t]$.
  \begin{enumerate}[label=(\roman*)]
    \item
      What is the characteristic of $K$?
    \item
      What is the characteristic of any extension field of $K$?
    \item
      Show that the Frobenius endormophism, $F: K \rightarrow K$ is not a ring isomorphism.
    \item
      Let $f(x) = x^p - t \in K[x]$.
      Prove that $f(x)$ is irreducible.
    \item
      Prove that $f(x)$ is not a separable polynomial.
    \item
      Construct an explicit field extension $K \subset L$ such that $f(x) \in L[x]$ has a factor of positive degree $< p$.
    \item
      With $f$ and $L$ above find all the roots of $f(x)$ in $L$ and determine their multiplicities.
  \end{enumerate}
\end{exer}

\begin{proof}
  $ $
  \begin{enumerate}[label=(\roman*)]
    \item
      We will prove in general that if $R \subset S$ are both commutative rings with 1, they have the same characteristic.
      Let $i: R \rightarrow S$ be the inclusion map.
      Let $\phi: \mathbb{Z} \rightarrow R$ be the unique ring homomorphism.

      Then $i \circ \phi: \mathbb{Z} \rightarrow S$ is a ring homomorphism, and this is the only homomorphism from $\mathbb{Z}$ to $S$ by the uniqueness.
      \begin{align*}
        a \in \ker(\phi)
          &\iff \phi(a) = 0 \\
          &\iff i(\phi(a)) = 0 & \text{($i$ is injective)} \\
          &\iff a \in \ker(i \circ \phi).
      \end{align*}
      Thus $\ker(\phi) = \ker(i \circ \phi)$, so $R$ and $S$ have the same characteristic.

      Therefore, $\mathbb{Z}/p\mathbb{Z}$ has the same characteristic as $K$.
      The kernel of $\psi: \mathbb{Z} \rightarrow \mathbb{Z}/p\mathbb{Z}$ is $(p)$, so the characteristic of $K$ is $p$.
    \item
      Using the result that we proved in (i), we conclude that the characteristic of any extension field of $K$ is $p$.
    \item
      Suppose that it is a ring isomorphism.
      Let $a / b \in K$ be chosen such that $F(a / b) = t$.
      \begin{align*}
        \big(\frac{a}{b}\big)^p = t
          &\implies a^p = tb^p \\
          &\implies p\deg(a) = \deg(t) + p\deg(b) \\
          &\implies p(\deg(a) - \deg(b)) = 1.
      \end{align*}
      However, $p \geq 2$, so this is impossible.
      Therefore, $F$ is not a ring isomorphism.
    \item
      $t$ is an irreducible element in $\mathbb{Z}/p\mathbb{Z}[t]$ because $t = ab$ implies that the degree of $a$ or $b$ must be 0, which implies that one of them is a unit.
      By Corollary 4 on P.300 (Dummit and Foote), $\mathbb{Z}/p\mathbb{Z}[t]$ is a principal ideal domain and unique factorization domain.
      By Proposition 2 on P.284 (Dummit and Foote), $t$ is a prime element in $\mathbb{Z}/p\mathbb{Z}[t]$.
      By the Eisenstein irreducibility criterion from the Factorization in Integral Domain handout, $x^p - t$ is irreducible in $K[x]$ because $-t \in (t)$ but $-t \notin (t^2)$.
    \item
      $f'(x) = px^{p - 1} = 0$.
      Thus $f(x) \in \GCD(f(x), f'(x))$ and $f(x) = x^p - t$ is not a unit.
      By Lemma 3.2 of the Field Extension handout, $f(x)$ is not separable.
    \item
      Let $L = K[y] / (y^p - t)$.
      Since $y^p - t$ is irreducible in $K[y]$, $(y^p - t)$ is a maximal ideal in $K[y]$.
      Thus $L$ is a field.
      Then $x^p - t$ has a root in $L$ because $y^p - t = 0$.
      This implies the existence of a linear factor of $x^p - t$.
    \item
      In $L[x]$, $(x - y)^p = \sum_{i=0}^{p} \binom{p}{i} x^i(-y)^{p - i} = x^p - y^p$ because $p \mid \binom{p}{i}$ for $1 \leq i \leq p - 1$.
      Since $y^p = t$, $x^p - y^p = x^p - t$.
      Therefore, the only root is $y$ and the multiplicity is $p$.
  \end{enumerate}
\end{proof}

\begin{exer}{(Problem 2)}
  Let $F$ be a field of characteristic 0.
  Let $f(x) \in F[x]$ be an irreducible polynomial.
  Then $f(x)$ is separable.
\end{exer}

\begin{proof}
  Let $f(x) = \sum_{i=0}^{n} a_ix^i \in F[x]$ be an irreducible polynomial with $a_n \ne 0$.
  Since $f(x)$ is irreducible, $f(x)$ is neither a unit nor 0.
  Since $F$ is a field, all polynomials of degree 0 are units.
  Thus $\deg(f(x)) = n \geq 1$.
  It suffices to show that $\GCD(f(x), f'(x)) = F^*$ by Lemma 3.2.
  Let $g(x) \in F[x]$ be given such that $g(x) \mid f(x), g(x) \mid f'(x)$.
  Since $f(x)$ is irreducible, either $g(x)$ is a unit or there exists a unit $u \in F^*$ such that $g(x) = uf(x)$.
  Suppose $g(x)$ is not a unit.
  Since $g(x) \mid f'(x)$, $f'(x) = h(x)g(x) = uh(x)f(x)$ for some $h(x) \in F[x]$.
  Thus $\deg(f'(x)) = \deg(uh(x)) + \deg(f(x))$.
  \begin{itemize}
    \item
      $f'(x) = \sum_{i = 1}^{n} ia_ix^{i - 1}, n \geq 1$ and $a_n \ne 0$.
      Since $F$ is a field of characteristic 0, $na_n \ne 0$.
      Therefore, $\deg(f'(x)) = n - 1$.
    \item
      $\deg(uh(x)) \geq 0$.
    \item
      $\deg(f(x)) = n$.
  \end{itemize}
  However, this implies that $n - 1 \geq 0 + n = n$.
  This is a contradiction, so $g(x)$ must be a unit.
  Therefore, $\GCD(f(x), f'(x)) = F^*$.
\end{proof}

\begin{exer}{(Problem 3)}
  Let $F$ be a field.
  Let $f(x) \in F[x]$ be an irreducible polynomial which is not separable.
  Show that $f'(x) = 0 \in F[x]$.
\end{exer}

\begin{proof}
  Suppose $f(x)$ is irreducible.
  Then $f(x) \ne 0$ and $f(x)$ is not a unit by definition.
  Thus $\deg(f(x)) \geq 1$.

  Since $f(x)$ is not separable, there exists a non-unit $g(x) \in F[x]$ such that $g(x) \mid f(x)$ and $g(x) \mid f'(x)$ by Lemma 3.2 from the Field Extension handout.
  Since $f(x)$ is irreducible and $g(x)$ is not a unit, $f(x)$ is the product of $g(x)$ and a unit.
  This implies that $\deg(f(x)) = \deg(g(x))$.

  Since $g(x) \mid f'(x)$, $f'(x) = h(x)g(x)$.
  If $f'(x) = 0$, we are done.
  Suppose otherwise.
  Then $\deg(f'(x)) = \deg(h(x)) + \deg(g(x)) = \deg(h(x)) + \deg(f(x)) \geq \deg(f(x))$.
  However, by the definition of the $'$ operator, $\deg(f'(x)) < \deg(f(x))$.
  This is a contradiction, so $f'(x) = 0$.
\end{proof}

\begin{exer}{(Problem 4)}
  Let $F$ be a field of prime characteristic $p$.
  Let $f(x) = \sum_{i=0}^n a_ix^i \in F[x]$ be an irreducible polynomial.
  Give a necessary and sufficient criterion for $f(x)$ to be inseparable in terms of the coefficients $a_i$.
\end{exer}

\begin{proof}
  We claim that $\forall i, (i \notin p\mathbb{Z} \implies a_i = 0)$ is a necessary and sufficient criterion.
  \begin{itemize}
    \item
      Suppose $f(x)$ is inseparable.
      By Lemma 5.5 from the Field Extension handout, $f'(x) = 0$.
      If $f'(x) = 0$, then $ia_i = 0$ for each $i$.
      Since $p$ is a prime, $a_i$ must be $0$ if $i \notin p\mathbb{Z}$.
    \item
      Suppose $\forall i, (i \notin p\mathbb{Z} \implies a_i = 0)$.
      Then $f'(x) = 0$, so $f(x) \mid f(x), f(x) \mid f'(x)$ and $f(x)$ is not a unit since $f(x)$ is irreducible.
      Therefore, $GCD(f(x), f'(x)) \ne F^{\times}$, so $f$ is inseparable by Lemma 3.2.
  \end{itemize}
  Hence, $\forall i, (i \notin p\mathbb{Z} \implies a_i = 0)$ is a necessary and sufficient criterion.
\end{proof}

\begin{exer}{(Problem 5)}
  What is the characteristic of the ring $\mathbb{Z} \times \mathbb{Z} / 2\mathbb{Z} \times \mathbb{Z} / 10\mathbb{Z}$?
\end{exer}

\begin{proof}
  Let $\phi$ be the only ring homomorphism from $\mathbb{Z} \rightarrow \mathbb{Z} \times \mathbb{Z} / 2\mathbb{Z} \times \mathbb{Z} / 10\mathbb{Z}$.
  Then $\phi(a) = (a, a + (2), a + (10))$ for any $a \in \mathbb{Z}$.
  If $\phi(a) = (0, 0, 0)$, then $a = 0$.
  Since $\ker(\phi) = (0)$, the characteristic is 0.
\end{proof}

\begin{exer}{(Problem 6)}
  Let $K$ be a finite field of characteristic $p$.
  Let $a, b \in K^*$ be two elements which have the same order in this finite group.
  Show that $\mathbb{Z}/p[a] = \mathbb{Z}/p[b]$ as subfields of $K$.
\end{exer}

\begin{proof}
  We will consider the subgroups $\ev{ a }, \ev{ b } \leq K^*$.
  $\ev{ a }, \ev{ b }$ contain all powers of $a, b$, respectively.
  Since $a, b$ have the same order, $\ev{ a }, \ev{ b }$ must have the same number of elements.
  Since $K^*$ is a cyclic group, $\ev{ a } = \ev{ b }$.
  (Theorem 7(3), P.58, Dummit and Foote)
  $\mathbb{Z}/p[a] = \{ \sum_{i=0}^{n} a_ia^i \mid n \geq 0, a_i \in \mathbb{Z}/p \}$, so it is the collection of all linear combinations of $\ev{ a }$ over $\mathbb{Z}/p$.
  Similarly, $\mathbb{Z}/p[b]$ is the collection of all linear combinations of $\ev{ b }$ over $\mathbb{Z}/p$.
  Since $\ev{ a } = \ev{ b }$, $\mathbb{Z}/p[a] = \mathbb{Z}/p[b]$.
\end{proof}

\begin{exer}{(Problem 7)}
  Let $K$ be a field with 81 elements.
  List all positive integers $n$ which are orders of elements in the group, $K^*$.
  Now compute the function $d(n) = [\mathbb{Z}/3[a]:\mathbb{Z}/3]$, where $a \in K^*$ is any element of order $n$.
  Present your results in the form of a table with entries $n$ and $d(n)$.
\end{exer}

\begin{proof}
  $ $
  \todo[inline,caption={}]{
    \begin{itemize}
      \item
        Problem 6 shows that $d(n)$ is well defined.
      \item
        $K$ is considered to be an extension field of $\mathbb{Z} / 3$.
      \item
        Let $K^* = \ev{ \alpha \mid \alpha^{80} = 1 }$.
      \item
        $\alpha^{40}$ is the only element of order 2, and $2 \in \mathbb{Z} / 3$ is an element of order 2.
        Thus $\alpha^{40} = 2$.
      \item
        Let $a$ be given.
        Let $n$ be the order of $a$.
        By Problem 6, we can assume $a = \alpha^k$ where $k = 80 / n$.
        This is because $(\mathbb{Z}/3)[a] = (\mathbb{Z}/3)[\alpha^k]$.
      \item
        Let $a$ be given.
        Then $(\mathbb{Z}/3\mathbb{Z}[a])^* \leq K^*$.
        Let $\alpha^k$ be a generator of $(\mathbb{Z}/3\mathbb{Z}[a])^*$.
        This $k$ can't be anything because $\alpha^{40}$ is always in $\mathbb{Z} / 3[a]$.
        For instance, $\mathbb{Z}/3\mathbb{Z}[\alpha^{20}]  = \{ 0, 1, 2, \alpha^{20}, 2\alpha^{20} \}$, so the degree of extension is 2 because $\{ 1, \alpha^{20} \}$ is a $\mathbb{Z}/3\mathbb{Z}$-basis.
    \end{itemize}
  }
\end{proof}

\section{Factorization in Integral Domain}

\begin{exer}{(Problem 7)}
  Define $\mathbb{Z}_{(p)} = \{ a / b \in \mathbb{Q} : p \nmid p \}$.
  Now $\mathbb{Z}_{(p)}$ is a subring of $\mathbb{Q}$ and $p\mathbb{Z}_{(p)}$ is a maximal ideal.
  \begin{enumerate}[label=(\roman*)]
    \item
      Prove that there is a ring isomorphism, $\mathbb{Z}/p\mathbb{Z} \simeq \mathbb{Z}_{(p)} / p\mathbb{Z}_{(p)}$.
    \item
      Suppose given $f \in \mathbb{Z}_{(p)}[x, y]$ such that when viewed as an element of $\mathbb{Q}(x)[y]$, $f$ has content 1 and degree $n$ in $y$.
      Prove that if the reduction mod $p$ of $f$, $f_0 \in \mathbb{Z}/p[x, y]$ is irreducible and of degree $n$ in $y$, then $f$ is irreducible in $\mathbb{Q}[x, y]$.
  \end{enumerate}
\end{exer}

\begin{proof}
  $ $
  \begin{enumerate}[label=(\roman*)]
    \item
      Define $\phi: \mathbb{Z}_{(p)} \rightarrow \mathbb{Z}/p\mathbb{Z}$ such that $a / b \mapsto ab^{-1}$.
      \begin{itemize}
        \item
          Claim: $\phi$ is well-defined.
          If $p \nmid b$, then $b \notin \mathbb{Z}/p\mathbb{Z}$, so $b^{-1}$ exists.
          Moreover, if $a / b = c / d \in \mathbb{Z}(p)$, then $ad = bc$, so $ab^{-1} = cd^{-1}$.
        \item
          Claim: $\phi$ is surjective.
          For all $a \in \mathbb{Z}/p\mathbb{Z}$, $\phi(a / 1) = a$.
        \item
          Claim: $\ker(\phi) = p\mathbb{Z}_{(p)}$.
          \begin{align*}
            \frac{a}{b} \in \ker(\phi)
              &\iff ab^{-1} = 0 \\
              &\iff p \mid a \\
              &\iff \frac{a}{b} \in p\mathbb{Z}_{(p)}.
          \end{align*}
      \end{itemize}
      By the first isomorphism theorem for rings (Theorem 7, P.243, Dummit and Foote), $\mathbb{Z}_{(p)} / p\mathbb{Z}_{(p)} \simeq \mathbb{Z}/p\mathbb{Z} $.
    \item
      \todo[inline,caption={}]{
        \begin{itemize}
          \item
            Suppose $f(x, y) = f_1(x, y)f_2(x, y)$ in $\mathbb{Q}[x, y]$ where $f_1, f_2$ are not units.
            Since the content of $f$ is 1, $\deg_y(f_1(x, y)) \geq 1$ and $\deg_y(f_2(x, y)) \geq 1$.
            Let $f_{1, 0}, f_{2, 0}$ be the reduction of $f_1, f_2$.
            Since $f_0 = f_{1, 0}f_{2, 0}$, $f_{1, 0} \in \mathbb{Z}/p[x, y]$ is a unit without loss of generality.
            Therefore, the leading coefficient of $f_1 \in \mathbb{Q}(x)[y]$ is in $p\mathbb{Z}[x]$.
            This means the leading coefficient of $f$ is in $p\mathbb{Z}[x]$.
            However, this means the degree of $f_0$ is less than $n$.
            This is a contradiction.
        \end{itemize}
      }
  \end{enumerate}
\end{proof}

\begin{exer}{(Problem 10)}
  Prove that $x^4 + x^3 + x^2 + x + 3 \in \mathbb{Q}[x]$ is irreducible.
\end{exer}

\begin{proof}
  By the third properties of the content from the factorization in integral domains handout, $f(x) = x^4 + x^3 + x^2 + x + 3$ is primitive.
  By Corollary 1(ii) of the factorization in integral domains handout, it suffices to show that $f(x)$ is irreducible in $\mathbb{Z}[x]$.
  Since $\deg(f(x)) = 4$, if $f(x)$ is not irreducible it must have a factor of degree 1 or 2.

  If there exists a factor of degree 1, then $f(x)$ must have a root in $\mathbb{Z}[x]$.
  If $x(x^3 + x^2 + x^1 + 1) = -3$, $x$ must divide 3.
  In other words, the only values that may be a root of $f(x)$ are $\pm 1, \pm 3$.
  However, none of them are actually roots because $f(3) = 123, f(-3) = 63, f(1) = 7, f(-1) = 3$.

  If there exists a factor of degree 2, then $f(x) = (x^2 + ax + b)(x^2 + cx + d) = x^4 + (a + c)x^3 + (b + ac + d)x^2 + (bc + ad)x + bd$ for some $a, b, c, d \in \mathbb{Z}$.
  Then $bd = 3$.
  This implies that $(b, d) = (1, 3), (-1, -3), (3, 1), (-3, -1)$.
  By symmetry, it suffices to only check $(1, 3), (-1, -3)$.
  \begin{itemize}
    \item
      If $(b, d) = (1, 3)$, then we have a system of equations
      \begin{align*}
        \begin{cases}
          a + c &= 1 \\
          c + 3a &= 1.
        \end{cases}
      \end{align*}
      Thus $a = 0, c = 1$.
      However, $b + ac + d = 1 + 0 + 3 = 4 \ne 1$.
    \item
      If $(b, d) = (-1, -3)$, then we have a system of equations
      \begin{align*}
        \begin{cases}
          a + c &= 1 \\
          -c - 3a &= 1.
        \end{cases}
      \end{align*}
      Thus $a = -1, c = 2$.
      However, $b + ac + d = -1 + -2 + -3 = -6 \ne 1$.
  \end{itemize}
  Therefore, there exist no such $a, b, c, d$, so $f(x)$ must be irreducible.
\end{proof}

\begin{exer}{(Problem 11)}
  Let $R$ be a commutative ring and $f(x) = \sum_{i=0}^{n} a_ix^i \in R[x]$ a nonzero polynomial of degree $d$.
  Suppose that $a_n \in R^*$.
  Show that $R[x]/(f(x))$ is a free $R$-module with basis, $1, x, x^2, \cdots, x^{n - 1}$.
  In other words, using the notation, $I := (f(x))$, show that every element of $R[x]/I$ may be written as an $R$-linear combination of $1 + I, \cdots, x^{n - 1} + I$ in exactly one way.
\end{exer}

\begin{proof}
  $ $
  \todo[inline,caption={}]{
    Finish this!
  }
\end{proof}

\end{document}
