\documentclass[12pt, psamsfonts]{amsart}

%-------Packages---------
\usepackage{amssymb,amsfonts}
\usepackage{fullpage}
\usepackage{tikz-cd}
\usepackage{todonotes}
\usepackage{physics}
\usepackage[all,arc]{xy}
\usepackage{enumerate}
\usepackage{enumitem}
\usepackage{mathrsfs}
\usepackage{theoremref}
\usepackage{graphicx}
\usepackage[bookmarks]{hyperref}

%--------Theorem Environments--------
%theoremstyle{plain} --- default
\newtheorem{thm}{Theorem}[section]
\newtheorem{cor}[thm]{Corollary}
\newtheorem{prop}[thm]{Proposition}
\newtheorem{lem}[thm]{Lemma}
\newtheorem{conj}[thm]{Conjecture}
\newtheorem{quest}[thm]{Question}

\theoremstyle{definition}
\newtheorem{defn}[thm]{Definition}
\newtheorem{defns}[thm]{Definitions}
\newtheorem{con}[thm]{Construction}
\newtheorem{exmp}[thm]{Example}
\newtheorem{exmps}[thm]{Examples}
\newtheorem{notn}[thm]{Notation}
\newtheorem{notns}[thm]{Notations}
\newtheorem{addm}[thm]{Addendum}
\newtheorem*{exer}{Exercise}

\theoremstyle{remark}
\newtheorem{rem}[thm]{Remark}
\newtheorem{rems}[thm]{Remarks}
\newtheorem{warn}[thm]{Warning}
\newtheorem{sch}[thm]{Scholium}

\DeclareMathOperator{\Hom}{Hom}
\DeclareMathOperator{\Id}{Id}
\DeclareMathOperator{\End}{End}
\DeclareMathOperator{\ord}{ord}

\makeatletter
\let\c@equation\c@thm
\makeatother
\numberwithin{equation}{section}

\bibliographystyle{plain}

\begin{document}

\title{Math 601 (Due 10/23)}
\author{Hidenori Shinohara}
\maketitle

\tableofcontents

\section{Field Extension}

\begin{exer}{(Problem 1)}
  Let $p$ be a prime number.
  Let $K = \mathbb{Z} / p\mathbb{Z}(t)$ be the fraction field of $\mathbb{Z} / p\mathbb{Z}[t]$.
  \begin{enumerate}[label=(\roman*)]
    \item
      What is the characteristic of $K$?
    \item
      What is the characteristic of any extension field of $K$?
    \item
      Show that the Frobenius endormophism, $F: K \rightarrow K$ is not a ring isomorphism.
    \item
      Let $f(x) = x^p - t \in K[x]$.
      Prove that $f(x)$ is irreducible.
    \item
      Prove that $f(x)$ is not a separable polynomial.
    \item
      Construct an explicit field extension $K \subset L$ such that $f(x) \in L[x]$ has a factor of positive degree $< p$.
    \item
      With $f$ and $L$ above find all the roots of $f(x)$ in $L$ and determine their multiplicities.
  \end{enumerate}
\end{exer}

\begin{proof}
  $ $
  \begin{enumerate}[label=(\roman*)]
    \item
      We will write $k \cdot 1$ to denote $1 + 1 + \cdots + 1$ ($k$ times).
      Since $p \cdot 1 = 0$ in $K$, the characteristic of $K$ is at most $p$.
      Let $k$ denote the characteristic of $K$.
      Let $i: \mathbb{Z}/p\mathbb{Z} \rightarrow (\mathbb{Z}/p\mathbb{Z})[t], i': \mathbb{Z}/p\mathbb{Z}[t] \rightarrow K$ be inclusions.
      Then $i' \circ i: \mathbb{Z} / p\mathbb{Z} \rightarrow K$ is an injective ring homomorphism.
      $k \cdot 1 \ne 0$ in $\mathbb{Z}/p\mathbb{Z}$.
      Thus $(i' \circ i)(k \cdot 1) = k \cdot (i' \circ i)(1) = k' \cdot 1 = 0$.
      Since $i' \circ i$ is injective, this implies $k \cdot 1 = 0$.
      Therefore, $k \geq p$, so $k$ must be equal to $p$.
  \end{enumerate}
\end{proof}

\begin{exer}{(Problem 2)}
  Let $F$ be a field of characteristic 0.
  Let $f(x) \in F[x]$ be an irreducible polynomial.
  Then $f(x)$ is separable.
\end{exer}

\begin{proof}
  Since $f(x)$ is irreducible, $f(x)$ is not a unit.
  Since $F$ is a field, all polynomials of degree 0 are units.
  Thus $\deg(f(x)) \geq 1$.
  \todo[inline,caption={}]{
    Finish the proof.
    Check Notability for a sketch.
  }
\end{proof}

\begin{exer}{(Problem 3)}
  Let $F$ be a field.
  Let $f(x) \in F[x]$ be an irreducible polynomial which is not separable.
  Show that $f'(x) = 0 \in F[x]$.
\end{exer}

\begin{proof}
  Suppose $f(x)$ is irreducible.
  Then $f(x) \ne 0$ and $f(x)$ is not a unit by definition.
  Thus $\deg(f(x)) \geq 1$.

  Since $f(x)$ is not separable, there exists a non-unit $g(x) \in F[x]$ such that $g(x) \mid f(x)$ and $g(x) \mid f'(x)$ by Lemma 3.2 from the Field Extension handout.
  Since $f(x)$ is irreducible and $g(x)$ is not a unit, $f(x)$ is the product of $g(x)$ and a unit.
  This implies that $\deg(f(x)) = \deg(g(x))$.

  Since $g(x) \mid f'(x)$, $f'(x) = h(x)g(x)$.
  If $f'(x) = 0$, we are done.
  Suppose otherwise.
  Then $\deg(f'(x)) = \deg(h(x)) + \deg(g(x)) = \deg(h(x)) + \deg(f(x)) \geq \deg(f(x))$.
  However, by the definition of the $'$ operator, $\deg(f'(x)) < \deg(f(x))$.
  This is a contradiction, so $f'(x) = 0$.
\end{proof}

\begin{exer}{(Problem 4)}
  Let $F$ be a field of prime characteristic $p$.
  Let $f(x) = \sum_{i=0}^n a_ix^i \in F[x]$ be an irreducible polynomial.
  Give a necessary and sufficient criterion for $f(x)$ to be inseparable in terms of the coefficients $a_i$.
\end{exer}

\begin{proof}
  We claim that $\forall i, (i \notin p\mathbb{Z} \implies a_i = 0)$ is a necessary and sufficient criterion.
  \begin{itemize}
    \item
      Suppose $f(x)$ is inseparable.
      By Lemma 5.5 from the Field Extension handout, $f'(x) = 0$.
      If $f'(x) = 0$, then $ia_i = 0$ for each $i$.
      Since $p$ is a prime, $a_i$ must be $0$ if $i \notin p\mathbb{Z}$.
    \item
      Suppose $\forall i, (i \notin p\mathbb{Z} \implies a_i = 0)$.
      Then $f'(x) = 0$, so $f(x) \mid f(x), f(x) \mid f'(x)$ and $f(x)$ is not a unit since $f(x)$ is irreducible.
      Therefore, $GCD(f(x), f'(x)) \ne F^{\times}$, so $f$ is inseparable by Lemma 3.2.
  \end{itemize}
  Hence, $\forall i, (i \notin p\mathbb{Z} \implies a_i = 0)$ is a necessary and sufficient criterion.
\end{proof}

\begin{exer}{(Problem 5)}
  What is the characteristic of the ring $\mathbb{Z} \times \mathbb{Z} / 2\mathbb{Z} \times \mathbb{Z} / 10\mathbb{Z}$?
\end{exer}

\begin{proof}
  Define $\phi: \mathbb{Z} \rightarrow \mathbb{Z} \times \mathbb{Z} / 2\mathbb{Z} \times \mathbb{Z} / 10\mathbb{Z}$ such that $\phi(k) = (k, 0, 0)$.
  Then $\phi$ is injective, so the characteristic is 0.
\end{proof}

\begin{exer}{(Problem 6)}
  Let $K$ be a finite field of characteristic $p$.
  Let $a, b \in K^*$ be two elements which have the same order in this finite group.
  Show that $\mathbb{Z}/p[a] = \mathbb{Z}/p[b]$ as subfields of $K$.
\end{exer}

\begin{proof}
  \todo[inline,caption={}]{
    Can I just say they both have the same number of elements and use the lemma from class?
  }
\end{proof}

\end{document}
