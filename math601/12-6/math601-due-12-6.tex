\documentclass[12pt, psamsfonts]{amsart}

%-------Packages---------
\usepackage{amssymb,amsfonts}
\usepackage{semantic}
\usepackage{fullpage}
\usepackage{tikz-cd}
\usepackage{todonotes}
\usepackage{physics}
\usepackage[all,arc]{xy}
\usepackage{enumerate}
\usepackage{enumitem}
\usepackage{mathrsfs}
\usepackage{theoremref}
\usepackage{graphicx}
\usepackage[bookmarks]{hyperref}

%--------Theorem Environments--------
%theoremstyle{plain} --- default
\newtheorem{thm}{Theorem}[section]
\newtheorem{cor}[thm]{Corollary}
\newtheorem{prop}[thm]{Proposition}
\newtheorem{lem}[thm]{Lemma}
\newtheorem{conj}[thm]{Conjecture}
\newtheorem{quest}[thm]{Question}

\theoremstyle{definition}
\newtheorem{defn}[thm]{Definition}
\newtheorem{defns}[thm]{Definitions}
\newtheorem{con}[thm]{Construction}
\newtheorem{exmp}[thm]{Example}
\newtheorem{exmps}[thm]{Examples}
\newtheorem{notn}[thm]{Notation}
\newtheorem{notns}[thm]{Notations}
\newtheorem{addm}[thm]{Addendum}
\newtheorem*{exer}{Exercise}

\theoremstyle{remark}
\newtheorem{rem}[thm]{Remark}
\newtheorem{rems}[thm]{Remarks}
\newtheorem{warn}[thm]{Warning}
\newtheorem{sch}[thm]{Scholium}

\DeclareMathOperator{\Hom}{Hom}
\DeclareMathOperator{\Id}{Id}
\DeclareMathOperator{\End}{End}
\DeclareMathOperator{\ord}{ord}
\DeclareMathOperator{\Aut}{Aut}
\DeclareMathOperator{\Gal}{Gal}

\makeatletter
\let\c@equation\c@thm
\makeatother
\numberwithin{equation}{section}

\bibliographystyle{plain}

\begin{document}

\title{Math 601 (Due 12/6)}
\author{Hidenori Shinohara}
\maketitle

\tableofcontents

\section{Galois Theory VI}

\begin{exer}{(Problem 1)}
  Let $u_1, u_2, u_3, u_4$ be the variables of the elementary symmetric polynomials $s_1, s_2, s_3, s_4$.
  Then $f(x) = (x - u_1)(x - u_2)(x - u_3)(x - u_4)$.
  Every element in $F = \mathbb{C}(s_1, \cdots, s_4)$ is a symmetric polynomial in the $u_i$ divided by a symmetric polynomial in the $u_i$.
  $f(x)$ does not have a linear factor in $F[x]$ because the $-u_i$ in $x - u_i$ is not symmetric.
  Moreover, it does not have a quadratic factor in $F[x]$ because the $u_i + u_j$ in $(x - u_i)(x - u_j) = x^2 - (u_i + u_j)x + u_iu_j$ is not symmetric, so $(x - u_i)(x - u_j) \notin F[x]$.
  Therefore, we will use the method developed in Galois Theory IV.
  Let $h(y) = y^2 - \delta$ where $\delta$ is the discriminant of $f(x)$.
  $h$ factors if and only if the discriminant is a perfect square.
  $\sigma(\prod_{i < j} (u_i - u_j)) = -\prod_{i < j} (u_i - u_j)$ under $\sigma$ that corresponds to the permutation $(12)$, so $\delta$ is not a perfect square.

  \todo[inline,caption={}]{
    Finish the $g(y)$ part.
    Or come up with something new.
  }

  The roots of $f(x)$ are expressible by radicals relative to $F$ because, as shown in Problem 3 below, every transitive subgroup of $S_4$ is solvable.
\end{exer}

\begin{exer}{(Problem 2)}
  $f(x) = x^6 - 2$ is irreducible over $\mathbb{Q}$ by Eisenstein ($p = 2$).
  The roots are $\{ \zeta^{i} \sqrt[6]{2} \mid i = 0, \cdots, 5 \}$ where $\zeta = e^{2\pi i / 6} = (1 + \sqrt{-3}) / 2$.
  Then the splitting field $L$ is $\mathbb{Q}(\zeta^0\sqrt[6]{2}, \cdots, \zeta^5\sqrt[6]{2}) = \mathbb{Q}(\zeta, \sqrt[6]{2})$.
  Let $\sigma \in \Aut(L/\mathbb{Q})$.
  The minimal polynomial of $\sqrt[6]{2}$ is $x^6 - 2$, so $\sigma(\sqrt[6]{2}) = \zeta^i\sqrt[6]{2}$ for some $i$.
  The minimal polynomial of $\zeta$ is $x^2 - x + 1$, so $\sigma(\zeta) = \zeta, \overline{\zeta}$.
  Thus there are $6 \cdot 2 = 12$ automorphisms.
  This is isomorphic to $D_6$ because $\sqrt[6]{2} \mapsto \zeta\sqrt[6]{2}$ corresponds to rotation and $\zeta \mapsto \overline{\zeta}$ corresponds to reflection.
\end{exer}

\begin{exer}{(Problem 3)}
  As discussed in the Galois Theory IV handout, the only transitive subgroups of $S_4$ are $S_4, A_4, V_4, C_4$, and groups with 8 elements.
  Clearly, $V_4, C_4$ are solvable.
  We showed below (Problem 2 from the Cauchy handout) that every $p$-group is solvable.
  Thus any group with 8 elements is solvable.
  The handout mentions $V_4 \transitive S_4$, so clearly $V_4 \trianglelefteq A_4$.

  Moreover, $A_4 / V_4$ has only 3 elements, so it is abelian.
  Thus $\{ e \} \subset V_4 \subset A_4 \subset S_4$ is a filtration because $A_4$ is an index-2 subgroup of $S_4$.
  Therefore, all the transitive subgroups of $S_4$ are solvable, so all the roots of any quartic polynomial are expressible by radicals.
\end{exer}

\section{Cauchy's Theorem, Finite $p$-groups, The Sylow theorems}

\begin{exer}{(Problem 2)}
  Let a prime number $p$ be given.
  We will show that any group $G$ of order $p^n$ for some $n$ is solvable by induction on $n$.
  When $n = 1$, $G \cong \mathbb{Z}_p$, which is abelian, so it is solvable.
  Suppose we have shown the proposition for some $n \in \mathbb{N}$, and let $G$ be a group of order $p^{n + 1}$.
  By Corollary 1 right above this problem statement in the handout, the center $H$ of $G$ is a nontrivial subgroup.
  Moreover, $H$ is clearly a normal subgroup of $G$.
  Thus it makes sense to consider $G / H$.
  The order of $G / H$ must be $p^m$ for some $1 \leq m \leq n - 1$.
  By the inductive hypothesis, $G / H$ is solvable.
  Since every subgroup of $G / H$ can be realized as the quotient of a subgroup of $G$ by $H$[Theorem 20(1), P.99, Dummit and Foote], there must exist a sequence of subgroups $H = G_0 \leq G_1 \leq \cdots \leq G_l = G$ such that $G_0 / H \trianglelefteq G_1 / H \trianglelefteq \cdots \trianglelefteq G_l / H$ and $(G_{i + 1} / H) / (G_{i} / H)$ is abelian for each $i$.
  By Theorem 19 [P.98, Dummit and Foote], $(G_{i + 1} / H) / (G_{i} / H) \cong G_{i + 1} / G_{i}$, so $G_{i + 1} / G_i$ is abelian for each $i$.
  $G_i / H \trianglelefteq G_{i + 1} / H$ implies $G_i \trianglelefteq G_{i + 1}$ for each $i$ by Theorem 20(5) [P.99, Dummit and Foote].

  We showed the existence of a sequence $H = G_0 \trianglelefteq G_1 \trianglelefteq \cdots \trianglelefteq G_l = G$ such that $G_{i + 1} / G_i$ is abelian for each $i$.
  By the inductive hypothesis, there exists a similar sequence of subgroups from $\{ e \}$ to $H$.
  Therefore, $G$ is solvable.
\end{exer}

\begin{exer}{(Problem 3)}
  Let $m = 3, p = 7$.
  Then $\abs{G} = 21 = pm$ with $p \nmid m$.
  Let $t$ be the number of Sylow $p$-subgroups.
  By the third Sylow theorem, $t \mid m$ and $t \equiv 1 \pmod p$.
  The only number that satisfies this is 1, so every group of order 21 has a unique Sylow 7-subgroup.
\end{exer}

\begin{exer}{(Problem 4)}
  Using the same idea as Problem 2 above, we will construct a filtration.
  Let $G$ be an extension of $H$ by $Q$.
  Suppose $H$ and $Q$ are both solvable.
  Since $Q$ is solvable, there exists a filtration $\{ e \} = Q_0 \trianglelefteq \cdots \trianglelefteq Q_n = Q$.
  Let $\phi$ be an isomorphism from $Q$ to $G / H$.
  Then the $\phi(Q_i)$'s form a filtration of $G / H$ and $\phi(Q_i) = G_i / H$ for some subgroup $G_i$ by the same theorems that we used in Problem 2.
  Moreover, $G_i$'s form a filtration from $H$ to $G$.
  Since $H$ is solvable, there exists a filtration from $\{ e \}$ to $H$.
  By concatenating them, we obtain a filtration from $\{ e \}$ to $G$, so $G$ is solvable.
\end{exer}

\begin{exer}{(Problem 5)}
  By Problem 3, $G$ has a unique group $H$ of order 7.
  Since conjugation preserves the order of a group, the group must be normal.
  Then $H \trianglelefteq G$ and $G / H \cong \mathbb{Z}_3$.
  Any group of prime order is abelian and thus solvable.
  Therefore, $G$ is an extension of a solvable group $\mathbb{Z}_7$ by a solvable group $\mathbb{Z}_3$, so it must be solvable.
\end{exer}

\begin{exer}{(Problem 7)}
  Since we are given that $\mathbb{Q}(\alpha)$ is the splitting field, every root of $f(x)$ can be expressed by multiplying, adding, dividing and subtracting rational numbers and $\alpha$.
  This implies that $\sigma \in G = \Aut(\mathbb{Q}(\alpha)/\mathbb{Q})$ is uniquely determined by $\sigma(\alpha)$.
  Therefore, $\abs{G} \leq 80$.

  \todo[inline,caption={}]{
    The Galois group should be a subgroup of $C_{80}$, but I don't know why.
    Alternatively, I can show that every transitive subgroup of order $\leq 80$ of $S_{80}$ is solvable, but that sounds much harder.
  }
\end{exer}

\begin{exer}{(Problem 10)}
  By the Corollary 1 indicated in the hint, we obtain a nontrivial center $C$ of $G$.
  By Lagrange, $\abs{C} = p, p^2$.
  If $\abs{C} = p^2$, then $G$ is abelian, so $G$ must be isomorphic to $\mathbb{Z} / (p^2)$ or $(\mathbb{Z} / p)^2$.
  Suppose $\abs{C} = p$.
  Since $C$ is normal, we will consider $G / C$, which is isomorphic to $\mathbb{Z} / p$.
  Let $x + C$ be a generator of $G / C$ and $y$ be a generator of $C$.
  Then every element in $G$ can be expressed as $x^iy^j$ for some $i, j \in \mathbb{Z}/p$.
  However, this implies that $C = G$ because for any $i, j, k, l$, $(x^iy^j)(x^ky^l) = x^ix^ky^jy^l = x^kx^iy^ly^j = (x^ky^l)(x^iy^j)$ because a power of $y$ commutes with any element.
  This is a contradiction, so $\abs{C} \ne p$.
\end{exer}

\end{document}
