\documentclass[12pt, psamsfonts]{amsart}

%-------Packages---------
\usepackage{amssymb,amsfonts}
\usepackage{semantic}
\usepackage{fullpage}
\usepackage{tikz-cd}
\usepackage{todonotes}
\usepackage{physics}
\usepackage[all,arc]{xy}
\usepackage{enumerate}
\usepackage{enumitem}
\usepackage{mathrsfs}
\usepackage{theoremref}
\usepackage{graphicx}
\usepackage[bookmarks]{hyperref}

%--------Theorem Environments--------
%theoremstyle{plain} --- default
\newtheorem{thm}{Theorem}[section]
\newtheorem{cor}[thm]{Corollary}
\newtheorem{prop}[thm]{Proposition}
\newtheorem{lem}[thm]{Lemma}
\newtheorem{conj}[thm]{Conjecture}
\newtheorem{quest}[thm]{Question}

\theoremstyle{definition}
\newtheorem{defn}[thm]{Definition}
\newtheorem{defns}[thm]{Definitions}
\newtheorem{con}[thm]{Construction}
\newtheorem{exmp}[thm]{Example}
\newtheorem{exmps}[thm]{Examples}
\newtheorem{notn}[thm]{Notation}
\newtheorem{notns}[thm]{Notations}
\newtheorem{addm}[thm]{Addendum}
\newtheorem*{exer}{Exercise}

\theoremstyle{remark}
\newtheorem{rem}[thm]{Remark}
\newtheorem{rems}[thm]{Remarks}
\newtheorem{warn}[thm]{Warning}
\newtheorem{sch}[thm]{Scholium}

\DeclareMathOperator{\Hom}{Hom}
\DeclareMathOperator{\Id}{Id}
\DeclareMathOperator{\End}{End}
\DeclareMathOperator{\ord}{ord}
\DeclareMathOperator{\Aut}{Aut}
\DeclareMathOperator{\Gal}{Gal}

\makeatletter
\let\c@equation\c@thm
\makeatother
\numberwithin{equation}{section}

\bibliographystyle{plain}

\begin{document}

\title{Math 601 (Due 12/6)}
\author{Hidenori Shinohara}
\maketitle

\tableofcontents

\section{Jordan canonical form}

\begin{exer}{(Problem 3)}
  By the theorem in the Jordan canonical form handout, there exists a basis for which the matrix $M$ for $T$ consists of blocks in the specified form.
  Let $B$ a block of size $\geq 2$ where the diagonal elements are all $\lambda$.
  Then the diagonal elements in $B^m$ are all $\lambda^m$ and the sub-diagonal elements in $B^m$ are all $m\lamdbda^{m - 1}$.
  Since $M^m = I$, $m\lambda^{m - 1} = 0$.
  Then $\lambda = 0$.
  However, if $\lambda = 0$, then $\lambda^m \ne 1$.
  This is a contradiction, so all the blocks must be of size 1, so $M$ is diagonal.
  Let $a_1, \cdots, a_m$ be the diagonal elements of $M$.
  Then $M^m$ is a diagonal matrix with $a_1^m, \cdots, a_m^m$.
  Therefore, each $a_i$ is an $m$-th root of unity.
\end{exer}

\section{Galois Theory VI}


\begin{exer}{(Problem 1)}
  Let $u_1, u_2, u_3, u_4$ be the variables of the elementary symmetric polynomials $s_1, s_2, s_3, s_4$.
  Then $f(x) = (x - u_1)(x - u_2)(x - u_3)(x - u_4)$.
  For any permutation $\sigma \in S_4$, $\phi \in \Aut(F(u_1, \cdots, u_n))$ determined by $\phi(u_i) = u_{\sigma_i}$ is an automorphism that fixes $F$ because every elementary symmetric polynomial $s_i$ is symmetric.
  Therefore, the Galois group is isomorphic to $S_4$.

  The roots of $f(x)$ are expressible by radicals relative to $F$ because, as shown in Problem 3 below, $S_4$ is solvable.
\end{exer}

\begin{exer}{(Problem 2)}
  $f(x) = x^6 - 2$ is irreducible over $\mathbb{Q}$ by Eisenstein ($p = 2$).
  The roots are $\{ \zeta^{i} \sqrt[6]{2} \mid i = 0, \cdots, 5 \}$ where $\zeta = e^{2\pi i / 6} = (1 + \sqrt{-3}) / 2$.
  Then the splitting field $L$ is $\mathbb{Q}(\zeta^0\sqrt[6]{2}, \cdots, \zeta^5\sqrt[6]{2}) = \mathbb{Q}(\zeta, \sqrt[6]{2})$.
  Let $\sigma \in \Aut(L/\mathbb{Q})$.
  The minimal polynomial of $\sqrt[6]{2}$ is $x^6 - 2$, so $\sigma(\sqrt[6]{2}) = \zeta^i\sqrt[6]{2}$ for some $i$.
  The minimal polynomial of $\zeta$ is $x^2 - x + 1$, so $\sigma(\zeta) = \zeta, \overline{\zeta}$.
  Thus there are $6 \cdot 2 = 12$ automorphisms.
  This is isomorphic to $D_6$ because $\sqrt[6]{2} \mapsto \zeta\sqrt[6]{2}$ corresponds to rotation and $\zeta \mapsto \overline{\zeta}$ corresponds to reflection.
\end{exer}

\begin{exer}{(Problem 3)}
  As discussed in the Galois Theory IV handout, the only transitive subgroups of $S_4$ are $S_4, A_4, V_4, C_4$, and groups with 8 elements.
  Clearly, $V_4, C_4$ are solvable.
  We showed below (Problem 2 from the Cauchy handout) that every $p$-group is solvable.
  Thus any group with 8 elements is solvable.
  The handout mentions $V_4 \transitive S_4$, so clearly $V_4 \trianglelefteq A_4$.

  Moreover, $A_4 / V_4$ has only 3 elements, so it is abelian.
  Thus $\{ e \} \subset V_4 \subset A_4 \subset S_4$ is a filtration because $A_4$ is an index-2 subgroup of $S_4$.
  Therefore, all the transitive subgroups of $S_4$ are solvable, so all the roots of any quartic polynomial are expressible by radicals.
\end{exer}

\section{Cauchy's Theorem, Finite $p$-groups, The Sylow theorems}

\begin{exer}{(Problem 2)}
  Let a prime number $p$ be given.
  We will show that any group $G$ of order $p^n$ for some $n$ is solvable by induction on $n$.
  When $n = 1$, $G \cong \mathbb{Z}_p$, which is abelian, so it is solvable.
  Suppose we have shown the proposition for some $n \in \mathbb{N}$, and let $G$ be a group of order $p^{n + 1}$.
  By Corollary 1 right above this problem statement in the handout, the center $H$ of $G$ is a nontrivial subgroup.
  Moreover, $H$ is clearly a normal subgroup of $G$.
  Thus it makes sense to consider $G / H$.
  The order of $G / H$ must be $p^m$ for some $1 \leq m \leq n - 1$.
  By the inductive hypothesis, $G / H$ is solvable.
  Since every subgroup of $G / H$ can be realized as the quotient of a subgroup of $G$ by $H$[Theorem 20(1), P.99, Dummit and Foote], there must exist a sequence of subgroups $H = G_0 \leq G_1 \leq \cdots \leq G_l = G$ such that $G_0 / H \trianglelefteq G_1 / H \trianglelefteq \cdots \trianglelefteq G_l / H$ and $(G_{i + 1} / H) / (G_{i} / H)$ is abelian for each $i$.
  By Theorem 19 [P.98, Dummit and Foote], $(G_{i + 1} / H) / (G_{i} / H) \cong G_{i + 1} / G_{i}$, so $G_{i + 1} / G_i$ is abelian for each $i$.
  $G_i / H \trianglelefteq G_{i + 1} / H$ implies $G_i \trianglelefteq G_{i + 1}$ for each $i$ by Theorem 20(5) [P.99, Dummit and Foote].

  We showed the existence of a sequence $H = G_0 \trianglelefteq G_1 \trianglelefteq \cdots \trianglelefteq G_l = G$ such that $G_{i + 1} / G_i$ is abelian for each $i$.
  By the inductive hypothesis, there exists a similar sequence of subgroups from $\{ e \}$ to $H$.
  Therefore, $G$ is solvable.
\end{exer}

\begin{exer}{(Problem 3)}
  Let $m = 3, p = 7$.
  Then $\abs{G} = 21 = pm$ with $p \nmid m$.
  Let $t$ be the number of Sylow $p$-subgroups.
  By the third Sylow theorem, $t \mid m$ and $t \equiv 1 \pmod p$.
  The only number that satisfies this is 1, so every group of order 21 has a unique Sylow 7-subgroup.
\end{exer}

\begin{exer}{(Problem 4)}
  Using the same idea as Problem 2 above, we will construct a filtration.
  Let $G$ be an extension of $H$ by $Q$.
  Suppose $H$ and $Q$ are both solvable.
  Since $Q$ is solvable, there exists a filtration $\{ e \} = Q_0 \trianglelefteq \cdots \trianglelefteq Q_n = Q$.
  Let $\phi$ be an isomorphism from $Q$ to $G / H$.
  Then the $\phi(Q_i)$'s form a filtration of $G / H$ and $\phi(Q_i) = G_i / H$ for some subgroup $G_i$ by the same theorems that we used in Problem 2.
  Moreover, $G_i$'s form a filtration from $H$ to $G$.
  Since $H$ is solvable, there exists a filtration from $\{ e \}$ to $H$.
  By concatenating them, we obtain a filtration from $\{ e \}$ to $G$, so $G$ is solvable.
\end{exer}

\begin{exer}{(Problem 5)}
  By Problem 3, $G$ has a unique group $H$ of order 7.
  Since conjugation preserves the order of a group, the group must be normal.
  Then $H \trianglelefteq G$ and $G / H \cong \mathbb{Z}_3$.
  Any group of prime order is abelian and thus solvable.
  Therefore, $G$ is an extension of a solvable group $\mathbb{Z}_7$ by a solvable group $\mathbb{Z}_3$, so it must be solvable.
\end{exer}

\begin{lem}\label{twothree}
  A group of order $3 \cdot 2^k$ is solvable for any $k \geq 0$.
\end{lem}

\begin{proof}
  When $k = 0$, this is trivial.
  When $k = 1$, we have a subgroup of order 3 by Cauchy, which is normal because the index is 2.
  Since every abelian group is solvable, Exercise 4 implies that a group of order 6 is solvable.

  Suppose that we have shown this for some $k \in \mathbb{N}$.
  Let $G$ be a group of order $3 \cdot 2^{k + 1}$.
  It suffices to find a proper, nontrivial normal subgroup $N$ of $G$.
  If such an $N$ exists, the orders of $N$ and $G / N$ are either a prime power or of the form $3 \cdot 2^l$, so they are both solvable by the inductive hypothesis and Exercise 2.
  By the Sylow theorem, the number $t$ of Sylow-2 group must divide 3, so $t = 1, 3$.
  \begin{itemize}
    \item
      If $t = 1$, then we have a normal subgroup of order $2^{k + 1}$, so we are done.
    \item
      Suppose $t = 3$.
      Let $H_1, H_2, H_3$ be the three Sylow-2 groups.
      Let $g \in G$ be given.
      Then $gH_1g^{-1} = H_i, gH_2g^{-1} = H_j, gH_3g^{-1} = H_k$ where $\{ i, j, k \} = \{ 1, 2, 3 \}$.
      Thus we can associate $g$ to the permutation that sends 1 to $i$, 2 to $j$, and 3 to $k$.
      This association induces a group homomorphism $\Phi: G \rightarrow S_3$.
      By the second Sylow theorem, $\ker(\Phi) \ne G$.
      Since $G / \ker(\Phi)$ is a nontrivial subgroup of $S_3$, $G / \ker(\Phi) \leq 6$.
      Since $\abs{G} \geq 3 \cdot 2^2 = 12$, $\ker(\Phi)$ is a nontrivial, proper normal subgroup of $G$.
  \end{itemize}
  Therefore, in each case, we found a nontrivial, proper normal subgroup of $G$.
  By induction, the statement is true for any $k \geq 0$.
\end{proof}

\begin{exer}{(Problem 8)}
  Lemma \ref{twothree} shows that a group of order 192 is solvable because $192 = 3 \cdot 2^6$.
\end{exer}


\begin{exer}{(Problem 7)}
  Since $\deg(f) = 80$ and $f$ is the minimal polynomial (possibly after canceling out the first coefficient), $[\mathbb{Q}(\alpha):\mathbb{Q}] = 80$.
  Since $\mathbb{Q} \subset \mathbb{Q}(\alpha)$ is Galois, $\abs{\Aut(\mathbb{Q}(\alpha)/\mathbb{Q})} = 80$.
  Therefore, it suffices to show that a group of order 80 is solvable.
  By the Sylow theorems, let $t_2, t_5$ be the number of subgroups of order 16 and 5.
  Then $t_2 \mid 5$ and $t_2 \equiv 1 \pmod 2$, so $t_2 = 1, 5$.
  Similarly, $t_5 \mid 16$ and $t_5 \equiv 1 \pmod 5$, so $t_5 = 1, 16$.
  If $t_2 = 1$ or $t_5 = 1$, then the subgroup is normal.
  Then the quotient group is of order 5 or 16, which, by exercise 2 above, is solvable because they are both a power of a prime.
  Suppose $t_2 = 5$ and $t_5 = 16$.
  Since the intersection of two subgroups is a subgroup, Lagrange implies that the 16 subgroups of order 5 only intersect at the identity element.
  Therefore, we know that at least $16 \cdot (5 - 1) = 64$ elements have order 5.
  Similarly, $t_2 = 5$, so there are at least $5 \cdot (16 - 1) = 75$ non-identity elements whose order divide 16.
  However, this is clearly impossible because 5 and 16 are coprime and we only have 80 elements.
  Therefore, this case is impossible.
\end{exer}

\begin{exer}{(Problem 8)}
  $A_5$ is a simple non-abelian group, so it is not solvable.
  [P.3, Galois Theory VI]

  $\abs{A_5} = 5! / 2 = 60$.
  Let $G = A_5 \times \mathbb{Z}/5\mathbb{Z}$.
  Then $G$ has 300 elements and $H = \{ (x, 0) \in G \}$ is a subgroup of $G$ that is isomorphic to $A_5$.
  By lemma 1 [P.4, Galois Theory V], a solvable group cannot contain an unsolvable subgroup.
  Therefore, $G$ is an unsolvable group of order 300.
\end{exer}

\begin{exer}{(Problem 9)}
  \begin{enumerate}
    \item
      By the third Sylow theorem, the number $t$ of Sylow $p$-subgroups of $G$ satisfies $t \mid q$ and $t \equiv 1 \pmod p$.
      Thus $t = 1$.
      Thus the subgroup $H$ of $G$ with $p$ elements is normal because conjugation preserves the order of a group.
      $G / H$ is a cyclic group of order $q$, so let $x + H$ be a generator.
      Then every element $g \in G$ satisfies $g + H = x^i + H$ for a unique $i \in \{ 0, \cdots, q - 1 \}$.
      Then the map $G \rightarrow \mathbb{Z}_{q}$ such that $g \mapsto i$ is a surjective group homomorphism.
      A surjective homomorphism $G \rightarrow \mathbb{Z}_q$ can be constructed in a similar fashion.
    \item
      The problem statement simply says the existence of a homomorphism, which can be achieved by the ``zero" map $g \mapsto e$.
      We will instead show the existence of a surjective homomorphism.
      In (1), we showed the existence of surjective homomorphisms $\phi_p: G \rightarrow C_p$ and $\phi_q: G \rightarrow C_q$.
      We have trivial homomorphisms $\psi_p: C_p \times C_q \rightarrow C_p$ and $\psi_q: C_p \times C_q \rightarrow C_q$ defined by $\psi_p(a, b) \rightarrow a$ and $\psi_q(a, b) \rightarrow b$.
      By the universal mapping property of the product, there must exist a unique group homomorphism $\Phi: G \rightarrow C_p \times C_q$ such that $\phi_p, \phi_q, \psi_p, \psi_q, \Phi$ all commute.
      Since $\phi_p = \psi_p \circ \Phi$ and $\phi_q = \psi_q \circ \Phi$ are both surjective, $\Phi$ must be surjective.
    \item
      Since $\abs{G} = pq$, $\Phi$ must be bijective, so it is an isomorphism.
    \item
      Clearly, $C_p$ and $C_q$ are isomorphic to $\mathbb{Z}/p$ and $\mathbb{Z}/q$.
      Then the map $(a, b) \mapsto qa + b$ is an isomorphism from $\mathbb{Z}/p \times \mathbb{Z}/q$ into $\mathbb{Z}/pq$.
      $\mathbb{Z}/pq$ is isomorphic to $C_{pq}$.
      Therefore, $G$ is isomorphic to $C_{pq}$.
  \end{enumerate}
\end{exer}

\begin{exer}{(Problem 10)}
  By the Corollary 1 indicated in the hint, we obtain a nontrivial center $C$ of $G$.
  By Lagrange, $\abs{C} = p, p^2$.
  If $\abs{C} = p^2$, then $G$ is abelian, so $G$ must be isomorphic to $\mathbb{Z} / (p^2)$ or $(\mathbb{Z} / p)^2$.
  Suppose $\abs{C} = p$.
  Since $C$ is normal, we will consider $G / C$, which is isomorphic to $\mathbb{Z} / p$.
  Let $x + C$ be a generator of $G / C$ and $y$ be a generator of $C$.
  Then every element in $G$ can be expressed as $x^iy^j$ for some $i, j \in \mathbb{Z}/p$.
  However, this implies that $C = G$ because for any $i, j, k, l$, $(x^iy^j)(x^ky^l) = x^ix^ky^jy^l = x^kx^iy^ly^j = (x^ky^l)(x^iy^j)$ because a power of $y$ commutes with any element.
  This is a contradiction, so $\abs{C} \ne p$.
\end{exer}

\begin{exer}{(Problem 11)}
  It suffices to show that every group of order 132 is solvable because it implies that every subgroup of a group of order 132 is solvable.
  Let $p = 11, m = 12$ and apply the third Sylow theorem.
  Them $t_{11} \mid 12$ and $t_{12} \equiv 1 \pmod p$ is satisfied only by 1 or 12.
  Similarly, $t_2 = 1, 3, 11, 33$ and $t_3 = 1, 4, 22$.
  \begin{itemize}
    \item
      Suppose $t_{11} = 1$.
      Let $H$ be the subgroup of order 11.
      Then $H$ is normal and $G / H$ is a group of order 12.
      A group of order $12 = 3 \cdot 2^2$ is solvable by Lemmam \ref{twothree}.
      By Problem 4, $G$ is solvable.
    \item
      Suppose $t_2 = 1$.
      Then the subgroup $H$ of order 4 is normal.
      $G / H$ is a group of order 33, which is solvable by Problem 9.
    \item
      Suppose $t_3 = 1$.
      Then the subgroup $H$ of order 3 is normal.
      $G / H$ is a group of order 44.
      By the third Sylow theorem, we know that there has to be exactly one subgroup $H'$ of order 11 ($t \mid 4$ and $t \equiv 1 \pmod{11}$) of $G / H$.
      Thus we have $(G / H) / H'$ is a group of order 4, which is solvable.
    \item
      Suppose $1 \notin \{ t_2, t_3, t_{11} \}$.
      Then $t_{11} = 12$, so $G$ contains at least $(11 - 1) \cdot 12 = 120$ elements of order 11.
      Similarly, $t_2 \geq 3$, so $G$ contains at least $(4 - 1) \cdot 3 = 9$ elements of order 2 or 4.
      Finally, $t_3 \geq 4$, so $G$ contains at least $(3 - 1) \cdot 4 = 8$ elements of order 3.
      11, 2, 3 are pairwise coprime, but $120 + 9 + 8 = 137 > 132$, so this is a contradiction.
      Therefore, this case cannot happen.
  \end{itemize}
\end{exer}

\end{document}
