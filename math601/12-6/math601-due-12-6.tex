\documentclass[12pt, psamsfonts]{amsart}

%-------Packages---------
\usepackage{amssymb,amsfonts}
\usepackage{semantic}
\usepackage{fullpage}
\usepackage{tikz-cd}
\usepackage{todonotes}
\usepackage{physics}
\usepackage[all,arc]{xy}
\usepackage{enumerate}
\usepackage{enumitem}
\usepackage{mathrsfs}
\usepackage{theoremref}
\usepackage{graphicx}
\usepackage[bookmarks]{hyperref}

%--------Theorem Environments--------
%theoremstyle{plain} --- default
\newtheorem{thm}{Theorem}[section]
\newtheorem{cor}[thm]{Corollary}
\newtheorem{prop}[thm]{Proposition}
\newtheorem{lem}[thm]{Lemma}
\newtheorem{conj}[thm]{Conjecture}
\newtheorem{quest}[thm]{Question}

\theoremstyle{definition}
\newtheorem{defn}[thm]{Definition}
\newtheorem{defns}[thm]{Definitions}
\newtheorem{con}[thm]{Construction}
\newtheorem{exmp}[thm]{Example}
\newtheorem{exmps}[thm]{Examples}
\newtheorem{notn}[thm]{Notation}
\newtheorem{notns}[thm]{Notations}
\newtheorem{addm}[thm]{Addendum}
\newtheorem*{exer}{Exercise}

\theoremstyle{remark}
\newtheorem{rem}[thm]{Remark}
\newtheorem{rems}[thm]{Remarks}
\newtheorem{warn}[thm]{Warning}
\newtheorem{sch}[thm]{Scholium}

\DeclareMathOperator{\Hom}{Hom}
\DeclareMathOperator{\Id}{Id}
\DeclareMathOperator{\End}{End}
\DeclareMathOperator{\ord}{ord}
\DeclareMathOperator{\Aut}{Aut}
\DeclareMathOperator{\Gal}{Gal}

\makeatletter
\let\c@equation\c@thm
\makeatother
\numberwithin{equation}{section}

\bibliographystyle{plain}

\begin{document}

\title{Math 601 (Due 12/6)}
\author{Hidenori Shinohara}
\maketitle

\tableofcontents

\section{Cauchy's Theorem, Finite $p$-groups, The Sylow theorems}

\begin{exer}{(Problem 2)}
  Let a prime number $p$ be given.
  We will show that any group $G$ of order $p^n$ for some $n$ is solvable by induction on $n$.
  When $n = 1$, $G \cong \mathbb{Z}_p$, which is abelian, so it is solvable.
  Suppose we have shown the proposition for some $n \in \mathbb{N}$, and let $G$ be a group of order $p^{n + 1}$.
  By Corollary 1 right above this problem statement in the handout, the center $H$ of $G$ is a nontrivial subgroup.
  Moreover, $H$ is clearly a normal subgroup of $G$.
  Thus it makes sense to consider $G / H$.
  The order of $G / H$ must be $p^m$ for some $1 \leq m \leq n - 1$.
  By the inductive hypothesis, $G / H$ is solvable.
  Since every subgroup of $G / H$ can be realized as the quotient of a subgroup of $G$ by $H$, there must exist a sequence of subgroups $H = G_0 \leq G_1 \leq \cdots \leq G_l = G$ such that $G_0 / H \trianglelefteq G_1 / H \trianglelefteq \cdots \trianglelefteq G_l / H$ and $(G_{i + 1} / H) / (G_{i} / H)$ is abelian for each $i$.
  By Theorem 19 [P.98, Dummit and Foote], $(G_{i + 1} / H) / (G_{i} / H) \cong G_{i + 1} / G_{i}$, so $G_{i + 1} / G_i$ is abelian for each $i$.
  Let $i$ be chosen arbitrarily, and let $g \in G_{i + 1}, h \in G_i$.
  Since $G_i / H \trianglelefteq G_{i + 1} / H$, $ghg^{-1} + H = (g + H)(h + H)(g^{-1} + H) \in G_i / H$.
  Therefore, $ghg^{-1} \in G_i$.
  Thus $G_i \trianglelefteq G_{i + 1}$ for each $i$.

  We showed the existence of a sequence $H = G_0 \trianglelefteq G_1 \trianglelefteq \cdots \trianglelefteq G_l = G$ such that $G_{i + 1} / G_i$ is abelian for each $i$.
  By the inductive hypothesis, there exists a similar sequence of subgroups from $\{ e \}$ to $H$.
  Therefore, $G$ is solvable.
\end{exer}

\begin{exer}{(Problem 3)}
  Let $m = 3, p = 7$.
  Then $\abs{G} = 21 = pm$ with $p \nmid m$.
  Let $t$ be the number of Sylow $p$-subgroups.
  By the third Sylow theorem, $t \mid m$ and $t \equiv 1 \pmod p$.
  The only number that satisfies this is 1, so every group of order 21 has a unique Sylow 7-subgroup.
\end{exer}

\end{document}
