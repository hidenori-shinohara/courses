\documentclass[12pt, psamsfonts]{amsart}

%-------Packages---------
\usepackage{amssymb,amsfonts}
\usepackage{semantic}
\usepackage{fullpage}
\usepackage{tikz-cd}
\usepackage{todonotes}
\usepackage{physics}
\usepackage[all,arc]{xy}
\usepackage{enumerate}
\usepackage{enumitem}
\usepackage{mathrsfs}
\usepackage{theoremref}
\usepackage{graphicx}
\usepackage[bookmarks]{hyperref}

%--------Theorem Environments--------
%theoremstyle{plain} --- default
\newtheorem{thm}{Theorem}[section]
\newtheorem{cor}[thm]{Corollary}
\newtheorem{prop}[thm]{Proposition}
\newtheorem{lem}[thm]{Lemma}
\newtheorem{conj}[thm]{Conjecture}
\newtheorem{quest}[thm]{Question}

\theoremstyle{definition}
\newtheorem{defn}[thm]{Definition}
\newtheorem{defns}[thm]{Definitions}
\newtheorem{con}[thm]{Construction}
\newtheorem{exmp}[thm]{Example}
\newtheorem{exmps}[thm]{Examples}
\newtheorem{notn}[thm]{Notation}
\newtheorem{notns}[thm]{Notations}
\newtheorem{addm}[thm]{Addendum}
\newtheorem*{exer}{Exercise}

\theoremstyle{remark}
\newtheorem{rem}[thm]{Remark}
\newtheorem{rems}[thm]{Remarks}
\newtheorem{warn}[thm]{Warning}
\newtheorem{sch}[thm]{Scholium}

\DeclareMathOperator{\Hom}{Hom}
\DeclareMathOperator{\Id}{Id}
\DeclareMathOperator{\End}{End}
\DeclareMathOperator{\ord}{ord}
\DeclareMathOperator{\Aut}{Aut}
\DeclareMathOperator{\Gal}{Gal}

\makeatletter
\let\c@equation\c@thm
\makeatother
\numberwithin{equation}{section}

\bibliographystyle{plain}

\begin{document}

\title{Math 602(Homework 1)}
\author{Hidenori Shinohara}
\maketitle

\begin{exer}{1}
  \begin{itemize}
    \item
      Let $p \in V(I \cap J)$.
      For any $\sum_{i=1}^{n} f_ig_i \in IJ$, we have $f_ig_i \in I \cap J$ for each $i$.
      Thus $(\sum_{i=1}^{n} f_ig_i)(p) = 0$, so $p \in V(IJ)$.
      Let $p \in V(IJ)$.
      Let $f \in I \cap J$.
      Then $f^2 \in IJ$, so $(f(p))^2 = 0$.
      Thus $f(p) = 0$, so $p \in V(I \cap J)$.
      Therefore, $V(I \cap J) = V(IJ)$.

      Let $p \in V(I) \cup V(J)$.
      Then either all polynomials in $I$ vanish at $p$ or all polynomials in $J$ vanish at $p$.
      Thus all the polynomials in the intersection must vanish at $p$.
      Thus $V(I) \cup V(J) \subset V(I \cap J)$.
      On the other hand, let $p \in V(I \cap J) \setminus (V(I) \cup V(J))$.
      If no such element exists, we are done.
      Then every polynomial in the intersection vanishes at $p$.
      Let $f \in I$ and $g \in J$ be polynomials that do not vanish at $p$.
      Then $fg \in I \cap J$, so $(fg)(p) = 0$.
      However, this is impossible because $f(p) \ne 0$ and $g(p) \ne 0$.
      Therefore, $V(I) \cup V(J) = V(I \cap J)$.
    \item
      $p \in V(I + J)$ if and only if $\forall f \in I + J, f(p) = 0$
      if and only if $\forall f \in I, f(p) = 0$ and $\forall f \in J, f(p) = 0$ if and only if $p \in V(I) \cap V(J)$.
    \item
      If every polynomial in $J$ vanishes at a point, every polynomial in $I$ must vanish at that point.
    \item
      If a polynomial vanishes in $Y$, then it must vanish in $X$.
    \item
      TODO
  \end{itemize}
\end{exer}

\begin{exer}{2}
  \begin{itemize}
    \item
      \begin{align*}
        y \in (I_1 + I_2)^e
          &\iff y \in f(I_1 + I_2)B \\
          &\iff \exists x_1, x_2 \in I_1, I_2, b \in B, y = f(x_1 + x_2)b \\
          &\iff \exists x_1, x_2 \in I_1, I_2, b \in B, y = f(x_1)b + f(x_2)b \\
          &\iff y \in I_1^e + I_2^e.
      \end{align*}
    \item
      \begin{align*}
        y \in (I_1 \cap I_2)^e
          &\implies y \in f(I_1 \cap I_2)B \\
          &\implies \exists x \in I_1 \cap I_2, b \in B, y = f(x)b \\
          &\implies (\exists x \in I_1, b \in B, y = f(x)b) \text{ and } (\exists x \in I_2, b \in B, y = f(x)b) \\
          &\implies y \in I_1^e, y \in I_2^e \\
          &\implies y \in I_1^e \cap I_2^e.
      \end{align*}
    \item
      $(I_1I_2)^e = f(I_1I_2)B = (f(I_1)f(I_2))B = (f(I_1)B)(f(I_2)B)$.
      $f(I_1)f(I_2) = f(I_1I_2)$ because the product of two ideals consists of a finite sum of elements and $f$ preserves finite sums.
    \item
      Let $x \in J_1^c + J_2^c$.
      Then $x \in f^{-1}(J_1) + f^{-1}(J_2)$.
      Then $x = a + b$ where $a \in f^{-1}(J_1)$ and $b \in f^{-1}(J_2)$.
      This implies $x = a + b$ where $f(a) \in J_1$ and $f(b) \in J_2$.
      Then, $f(x) = f(a + b) = f(a) + f(b) \in J_1 + J_2$, so $x \in f^{-1}(J_1 + J_2)$.
    \item
      $f^{-1}(J_1 \cap J_2) = f^{-1}(J_1) \cap f^{-1}(J_2)$ from set theory.
    \item
      Let $\sum_{i=1}^{n} a_ib_i \in J_1^cJ_2^c$ where $a_i \in J_1^c$ and $b_i \in J_2^c$.
      Then $f(a_i) \in J_1$ and $f(b_i) \in J_2$.
      Thus $\sum f(a_i)f(b_i) \in J_1J_2$.
      Since $f$ preserves product and addition, $f(\sum a_ib_i) \in J_1J_2$.
      Thus $\sum a_ib_i \in f^{-1}(J_1J_2) = (J_1J_2)^c$.
  \end{itemize}
\end{exer}

\begin{exer}{3}
  $(I:J)$ is nonempty because $0 \in (I:J)$.
  $(I:J)$ is closed under addition, and for all $x \in R$, $rJ \subset I \implies x(rJ) = r(xJ) = rJ \subset I$.
  Thus $(I:J)$ is an ideal.
\end{exer}


\end{document}
