\documentclass[12pt, psamsfonts]{amsart}

%-------Packages---------
\usepackage{amssymb,amsfonts}
\usepackage{semantic}
\usepackage{fullpage}
\usepackage{tikz-cd}
\usepackage{todonotes}
\usepackage{physics}
\usepackage[all,arc]{xy}
\usepackage{enumerate}
\usepackage{enumitem}
\usepackage{mathrsfs}
\usepackage{theoremref}
\usepackage{graphicx}
\usepackage[bookmarks]{hyperref}

%--------Theorem Environments--------
%theoremstyle{plain} --- default
\newtheorem{thm}{Theorem}[section]
\newtheorem{cor}[thm]{Corollary}
\newtheorem{prop}[thm]{Proposition}
\newtheorem{lem}[thm]{Lemma}
\newtheorem{conj}[thm]{Conjecture}
\newtheorem{quest}[thm]{Question}

\theoremstyle{definition}
\newtheorem{defn}[thm]{Definition}
\newtheorem{defns}[thm]{Definitions}
\newtheorem{con}[thm]{Construction}
\newtheorem{exmp}[thm]{Example}
\newtheorem{exmps}[thm]{Examples}
\newtheorem{notn}[thm]{Notation}
\newtheorem{notns}[thm]{Notations}
\newtheorem{addm}[thm]{Addendum}
\newtheorem*{exer}{Exercise}

\theoremstyle{remark}
\newtheorem{rem}[thm]{Remark}
\newtheorem{rems}[thm]{Remarks}
\newtheorem{warn}[thm]{Warning}
\newtheorem{sch}[thm]{Scholium}

\DeclareMathOperator{\Hom}{Hom}
\DeclareMathOperator{\Id}{Id}
\DeclareMathOperator{\End}{End}
\DeclareMathOperator{\ord}{ord}
\DeclareMathOperator{\Aut}{Aut}
\DeclareMathOperator{\Gal}{Gal}

\makeatletter
\let\c@equation\c@thm
\makeatother
\numberwithin{equation}{section}

\bibliographystyle{plain}

\begin{document}

\title{Math 602(Homework 1)}
\author{Hidenori Shinohara}
\maketitle

\begin{exer}{1}
  \begin{itemize}
    \item
      Let $p \in V(I \cap J)$.
      For any $\sum_{i=1}^{n} f_ig_i \in IJ$, we have $f_ig_i \in I \cap J$ for each $i$.
      Thus $(\sum_{i=1}^{n} f_ig_i)(p) = 0$, so $p \in V(IJ)$.
      Let $p \in V(IJ)$.
      Let $f \in I \cap J$.
      Then $f^2 \in IJ$, so $(f(p))^2 = 0$.
      Thus $f(p) = 0$, so $p \in V(I \cap J)$.
      Therefore, $V(I \cap J) = V(IJ)$.

      Let $p \in V(I) \cup V(J)$.
      Then either all polynomials in $I$ vanish at $p$ or all polynomials in $J$ vanish at $p$.
      Thus all the polynomials in the intersection must vanish at $p$.
      Thus $V(I) \cup V(J) \subset V(I \cap J)$.
      On the other hand, let $p \in V(I \cap J) \setminus (V(I) \cup V(J))$.
      If no such element exists, we are done.
      Then every polynomial in the intersection vanishes at $p$.
      Let $f \in I$ and $g \in J$ be polynomials that do not vanish at $p$.
      Then $fg \in I \cap J$, so $(fg)(p) = 0$.
      However, this is impossible because $f(p) \ne 0$ and $g(p) \ne 0$.
      Therefore, $V(I) \cup V(J) = V(I \cap J)$.
    \item
      $p \in V(I + J)$ if and only if $\forall f \in I + J, f(p) = 0$
      if and only if $\forall f \in I, f(p) = 0$ and $\forall f \in J, f(p) = 0$ if and only if $p \in V(I) \cap V(J)$.
    \item
      If every polynomial in $J$ vanishes at a point, every polynomial in $I$ must vanish at that point.
    \item
      If a polynomial vanishes in $Y$, then it must vanish in $X$.
    \item
      TODO
  \end{itemize}
\end{exer}

\begin{exer}{2}
  \begin{itemize}
    \item
      \begin{align*}
        y \in (I_1 + I_2)^e
          &\iff y \in f(I_1 + I_2)B \\
          &\iff \exists x_1, x_2 \in I_1, I_2, b \in B, y = f(x_1 + x_2)b \\
          &\iff \exists x_1, x_2 \in I_1, I_2, b \in B, y = f(x_1)b + f(x_2)b \\
          &\iff y \in I_1^e + I_2^e.
      \end{align*}
    \item
      \begin{align*}
        y \in (I_1 \cap I_2)^e
          &\implies y \in f(I_1 \cap I_2)B \\
          &\implies \exists x \in I_1 \cap I_2, b \in B, y = f(x)b \\
          &\implies (\exists x \in I_1, b \in B, y = f(x)b) \text{ and } (\exists x \in I_2, b \in B, y = f(x)b) \\
          &\implies y \in I_1^e, y \in I_2^e \\
          &\implies y \in I_1^e \cap I_2^e.
      \end{align*}
    \item
      $(I_1I_2)^e = f(I_1I_2)B = (f(I_1)f(I_2))B = (f(I_1)B)(f(I_2)B)$.
      $f(I_1)f(I_2) = f(I_1I_2)$ because the product of two ideals consists of a finite sum of elements and $f$ preserves finite sums.
    \item
      Let $x \in J_1^c + J_2^c$.
      Then $x \in f^{-1}(J_1) + f^{-1}(J_2)$.
      Then $x = a + b$ where $a \in f^{-1}(J_1)$ and $b \in f^{-1}(J_2)$.
      This implies $x = a + b$ where $f(a) \in J_1$ and $f(b) \in J_2$.
      Then, $f(x) = f(a + b) = f(a) + f(b) \in J_1 + J_2$, so $x \in f^{-1}(J_1 + J_2)$.
    \item
      $f^{-1}(J_1 \cap J_2) = f^{-1}(J_1) \cap f^{-1}(J_2)$ from set theory.
    \item
      Let $\sum_{i=1}^{n} a_ib_i \in J_1^cJ_2^c$ where $a_i \in J_1^c$ and $b_i \in J_2^c$.
      Then $f(a_i) \in J_1$ and $f(b_i) \in J_2$.
      Thus $\sum f(a_i)f(b_i) \in J_1J_2$.
      Since $f$ preserves product and addition, $f(\sum a_ib_i) \in J_1J_2$.
      Thus $\sum a_ib_i \in f^{-1}(J_1J_2) = (J_1J_2)^c$.
  \end{itemize}
\end{exer}

\begin{exer}{3}
  $(I:J)$ is nonempty because $0 \in (I:J)$.
  $(I:J)$ is closed under addition, and for all $x \in R$, $rJ \subset I \implies x(rJ) = r(xJ) = rJ \subset I$.
  Thus $(I:J)$ is an ideal.

  \begin{itemize}
    \item
      Lemma:
      Let $a, b, c$ be ideas.
      If $\forall x \in a, xb \subset c$, then $ab \subset c$.

      Proof:
      Let $\sum a_ib_i \in ab$ be given.
      Then each $a_ib_i \in c$.
      Since $c$ is closed under addition, $\sum a_ib_i \in c$.
      Therefore, $ab \subset c$.
    \item
      Let $x \in a$.
      Then $\forall y \in b, xy \in a$ since $a$ is an ideal.
      Then $xb \subset a$, so $x \in (a:b)$.
    \item
      For all $x \in (a:b)$, $xb \subset a$.
      By the Lemma above, $(a:b)b \subset a$.
    \item
      Let $x \in ((a:b):c)$.
      Then $xc \subset (a:b)$.
      For all $xz \in xc, (xz)b \subset a$.
      Therefore, $(xc)b \subset a$ by the Lemma above.
      Then $x(cb) \subset a$, so $x(bc) \subset a$.
      Hence, $x \in (a:bc)$.

      On the other hand, suppose $x \in (a:bc)$.
      Then $x(bc) \subset a$.
      $x(bc) \subset a \implies (xb)c \subset a \implies xb \subset (a:c) \implies x \in ((a:c):b)$.

      Therefore, $((a:b):c) = (a:bc)$.

      We showed that $((a:b):c) = (a:bc)$.
      This implies $(a:cb) = ((a:c):b)$.
      Since $(a:bc) = (a:cb)$, we have $((a:b):c) = (a:bc) = (a:cb) = ((a:c):b)$.
    \item
      For any $x \in A$,
      \begin{align*}
        x \in (\cap_i a_i:b)
          &\iff xb \subset \cap_i a_i \\
          &\iff \forall i, xb \subset a_i \\
          &\iff \forall i, x \subset (a_i:b) \\
          &\iff x \subset \cap_i (a_i:b).
      \end{align*}
    \item
      For any $x \in A$,

      $$
      \begin{align*}
        x \in (a:\sum_i b_i)
          &\iff x(\sum_i b_i) \subset a \\
          &\implies \forall i, xb_i \subset a \\
          &\iff \forall i, x \subset (a:b_i) \\
          &\iff x \subset \cap_i(a:b_i).
      \end{align*}
      $$

      Therefore, it suffices to show that $\forall i, xb_i \subset a \implies x(\sum_i b_i) \subset a$.
      Let $y_{i_1} + \cdots + y_{i_n} \in \sum_i b_i$ be given where $y_{i_j} \in b_{i_j}$.
      For each $j$, since $xb_{i_j} \subset a$, $xy_{i_j} \in a$.
      Since $a$ is closed under finite addition, $xy_{i_1} + \cdots + xy_{i_n} \in a$.
      Therefore, $\forall i, xb_i \subset a \implies x(\sum_i b_i) \subset a$, so $(a:\sum_i b_i) = \cap_i(a:b_i)$.
    \item
      Let $bf(x) \in (a_1:a_2)^e$ where $b \in B$ and $x \in (a_1:a_2)$.

      \begin{align*}
        xa_2 \subset a_1
          &\implies f(xa_2) \subset f(a_1) \\
          &\implies f(x)f(a_2) \subset f(a_1) \\
          &\implies B(f(x)f(a_2)) \subset Bf(a_1) \\
          &\implies f(x)(Bf(a_2)) \subset Bf(a_1) \\
          &\implies f(x)a_2^e \subset a_1^e \\
          &\implies f(x) \in (a_1^e:a_2^e) \\
          &\implies bf(x) \in (a_1^e:a_2^e). \\
        x \in (b_1:b_2)^c
          &\implies f(x) \in (b_1:b_2) \\
          &\implies f(x)b_2 \in b_1 \\
          &\implies f^{-1}(f(x)b_2) \subset f^{-1}(b_1) \\
          &\implies xf^{-1}(b_2) \subset f^{-1}(f(x)b_2) \subset f^{-1}(b_1) \\
          &\implies xf^{-1}(b_2) \subset f^{-1}(b_1) \\
          &\implies x \in (f^{-1}(b_1):f^{-1}(b_2)) \\
          &\implies x \in (b_1^c:b_2^c).
      \end{align*}
  \end{itemize}

\end{exer}

\begin{exer}{(Problem 4)}
  Let $f = \sum_{i=1}^{m} a_ix^i, g = \sum_{i=1}^{n} b_ix^i \notin p[x]$.
  Let $m', n'$ be the smallest integer such that $a_{m'}, b_{n'} \notin p[x]$.
  Such $m', n'$ must exist because $f, g \notin p[x]$.
  Then the coefficient of $x^{m' + n'}$ in $fg$ is $\sum_{i=0}^{m' + n'} a_ib_{m' + n' - i}$.
  Then $a_ib_{m' + n' - i} \in p$ if and only if $i \ne m'$.
  The coefficient of $x^{m' + n'}$ in $fg$ is not in $p[x]$.
  Therefore, $fg \notin p[x]$, so $p[x]$ is a prime ideal.

  $(0)$ is a maximal ideal of $Q$.
  However, $(0)$ is not a maximal ideal in $\mathbb{Q}[x]$ because $(x)$ is a proper ideal of $\mathbb{Q}[x]$ that properly contains $(0)$.
\end{exer}

\begin{exer}{(Problem 5)}
  \begin{enumerate}
    \item 
      Since $IJ \subset I$ and $IJ \subset J$, $IJ \subset I \cap J$.
      Let $x \in I \cap J$.
      Let $a \in I$ and $b \in J$ such that $a + b = 1$.
      Then $x, a \in I$ and $x, b \in J$.
      Thus $ax + xb \in IJ$.
      In other words, $x = (a + b)x \in IJ$.

      The kernel of $\phi: R \rightarrow R / I \times R / J$ defined by $x \mapsto (x + I, x + J)$ is $I \cap J$.
      Thus $R / (I \cap J)$ is isomorphic to $R / I \times R / J$ and $I \cap J = IJ$ as shown above.
    \item
      Since $I + J = I + J' = (1)$, $a + b = c + d = 1$ for some $a, c \in I, b \in J, d \in J'$.
      Then $1 = (a + b)(c + d) = (ac + bc + ad) + bd \in I + JJ'$ because $ac, bc, ad \in I$ and $bd \in JJ'$.
    \item
      Suppose $I + J = (1)$.
      Then $a + b = 1$ for some $a \in I$ and $b \in J$.
      Let $m, n \in \mathbb{N}$ be given.
      Then $1 = (a + b)^{m + n} \in I^m + J^n$ because $a^ib^{m + n - i} \in I^m$ if $i \geq m$ and $a^ib^{m + n - i} \in J^n$ if $i \leq m$.
      The other direction is trivial.
    \item
      $I_1$ and $I_1 \cdots I_n$ are not comaximal because $I_1 \cdots I_n \subset I_1$.
      Choose $a_2, \cdots, a_n \in I_1$ and $b_2, \cdots, b_n \in I_2, \cdots, I_n$ such that $a_i + b_i = 1$ for each $i$.
      Then $1 = (a_2 + b_2) \cdots (a_n + b_n)$.
      After expanding $(a_2 + b_2) \cdots (a_n + b_n)$, every expression containing some $a_i$ belongs to $I_1$.
      The only expression that does not contain $a_i$ is $b_2 \cdots b_n$, and it is contained in $I_2 \cdots I_n$.
      Thus $I_1 + I_2 \cdots I_n = (1)$.
  \end{enumerate}
\end{exer}

\begin{exer}{(Problem 6)}
  $(1 + x)(1 - x + x^2 + ... + (-x)^{n - 1}) = 1 + (-x)^n = 1 - 0 = 1$.
\end{exer}

\end{document}
