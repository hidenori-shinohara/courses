\documentclass[12pt, psamsfonts]{amsart}

%-------Packages---------
\usepackage{amssymb,amsfonts}
\usepackage{semantic}
\usepackage{fullpage}
\usepackage{tikz-cd}
\usepackage{todonotes}
\usepackage{physics}
\usepackage[all,arc]{xy}
\usepackage{enumerate}
\usepackage{enumitem}
\usepackage{mathrsfs}
\usepackage{theoremref}
\usepackage{graphicx}
\usepackage[bookmarks]{hyperref}

%--------Theorem Environments--------
%theoremstyle{plain} --- default
\newtheorem{thm}{Theorem}[section]
\newtheorem{cor}[thm]{Corollary}
\newtheorem{prop}[thm]{Proposition}
\newtheorem{lem}[thm]{Lemma}
\newtheorem{conj}[thm]{Conjecture}
\newtheorem{quest}[thm]{Question}

\theoremstyle{definition}
\newtheorem{defn}[thm]{Definition}
\newtheorem{defns}[thm]{Definitions}
\newtheorem{con}[thm]{Construction}
\newtheorem{exmp}[thm]{Example}
\newtheorem{exmps}[thm]{Examples}
\newtheorem{notn}[thm]{Notation}
\newtheorem{notns}[thm]{Notations}
\newtheorem{addm}[thm]{Addendum}
\newtheorem*{exer}{Exercise}

\theoremstyle{remark}
\newtheorem{rem}[thm]{Remark}
\newtheorem{rems}[thm]{Remarks}
\newtheorem{warn}[thm]{Warning}
\newtheorem{sch}[thm]{Scholium}

\DeclareMathOperator{\Hom}{Hom}
\DeclareMathOperator{\Id}{Id}
\DeclareMathOperator{\End}{End}
\DeclareMathOperator{\ord}{ord}
\DeclareMathOperator{\Aut}{Aut}
\DeclareMathOperator{\Gal}{Gal}

\makeatletter
\let\c@equation\c@thm
\makeatother
\numberwithin{equation}{section}

\bibliographystyle{plain}

\begin{document}

\title{Math 602(Homework 1)}
\author{Hidenori Shinohara}
\maketitle

\begin{exer}{1}
  \begin{itemize}
    \item
      Let $p \in V(I \cap J)$.
      For any $\sum_{i=1}^{n} f_ig_i \in IJ$, we have $f_ig_i \in I \cap J$ for each $i$.
      Thus $(\sum_{i=1}^{n} f_ig_i)(p) = 0$, so $p \in V(IJ)$.
      Let $p \in V(IJ)$.
      Let $f \in I \cap J$.
      Then $f^2 \in IJ$, so $(f(p))^2 = 0$.
      Thus $f(p) = 0$, so $p \in V(I \cap J)$.
      Therefore, $V(I \cap J) = V(IJ)$.

      Let $p \in V(I) \cup V(J)$.
      Then either all polynomials in $I$ vanish at $p$ or all polynomials in $J$ vanish at $p$.
      Thus all the polynomials in the intersection must vanish at $p$.
      Thus $V(I) \cup V(J) \subset V(I \cap J)$.
      On the other hand, let $p \in V(I \cap J) \setminus (V(I) \cup V(J))$.
      If no such element exists, we are done.
      Then every polynomial in the intersection vanishes at $p$.
      Let $f \in I$ and $g \in J$ be polynomials that do not vanish at $p$.
      Then $fg \in I \cap J$, so $(fg)(p) = 0$.
      However, this is impossible because $f(p) \ne 0$ and $g(p) \ne 0$.
      Therefore, $V(I) \cup V(J) = V(I \cap J)$.
    \item
      $p \in V(I + J)$ if and only if $\forall f \in I + J, f(p) = 0$
      if and only if $\forall f \in I, f(p) = 0$ and $\forall f \in J, f(p) = 0$ if and only if $p \in V(I) \cap V(J)$.
  \end{itemize}
\end{exer}


\end{document}
