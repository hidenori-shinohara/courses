\documentclass[12pt, psamsfonts]{amsart}

%-------Packages---------
\usepackage{amssymb,amsfonts}
\usepackage{semantic}
\usepackage{fullpage}
\usepackage{tikz-cd}
\usepackage{todonotes}
\usepackage{physics}
\usepackage[all,arc]{xy}
\usepackage{enumerate}
\usepackage{enumitem}
\usepackage{mathrsfs}
\usepackage{theoremref}
\usepackage{graphicx}
\usepackage[bookmarks]{hyperref}

%--------Theorem Environments--------
%theoremstyle{plain} --- default
\newtheorem{thm}{Theorem}[section]
\newtheorem{cor}[thm]{Corollary}
\newtheorem{prop}[thm]{Proposition}
\newtheorem{lem}[thm]{Lemma}
\newtheorem{conj}[thm]{Conjecture}
\newtheorem{quest}[thm]{Question}

\theoremstyle{definition}
\newtheorem{defn}[thm]{Definition}
\newtheorem{defns}[thm]{Definitions}
\newtheorem{con}[thm]{Construction}
\newtheorem{exmp}[thm]{Example}
\newtheorem{exmps}[thm]{Examples}
\newtheorem{notn}[thm]{Notation}
\newtheorem{notns}[thm]{Notations}
\newtheorem{addm}[thm]{Addendum}
\newtheorem*{exer}{Exercise}

\theoremstyle{remark}
\newtheorem{rem}[thm]{Remark}
\newtheorem{rems}[thm]{Remarks}
\newtheorem{warn}[thm]{Warning}
\newtheorem{sch}[thm]{Scholium}

\DeclareMathOperator{\Hom}{Hom}
\DeclareMathOperator{\Id}{Id}
\DeclareMathOperator{\End}{End}
\DeclareMathOperator{\ord}{ord}
\DeclareMathOperator{\Aut}{Aut}
\DeclareMathOperator{\Gal}{Gal}

\makeatletter
\let\c@equation\c@thm
\makeatother
\numberwithin{equation}{section}

\bibliographystyle{plain}

\begin{document}

\title{Math 602 (Homework 5)}
\author{Hidenori Shinohara}
\maketitle

\begin{exer}{(1)}
  This can be proved using induction.
  The base case $m = 1$ is trivial.
  Suppose that the proposition has been shown for some $m \in \mathbb{N}$.
  We will show the $(m + 1)$ case.
  By the definition of a determinant,
  \begin{align*}
    \Delta = \sum_{k=1}^{m + 1} (-1)^{k + 1}\det(M_{k, 1})
  \end{align*}
  where $M_{k, 1}$ is the matrix obtained by deleting the $k$th row and 1st column.
  We can apply the inductive hypothesis to each $M_{k, 1}$ because, for instance, when $k = 1$,
  \begin{align*}
    \det(M_{1, 1})
      &= \det\begin{bmatrix}
        \alpha_2 & \alpha_2^2 & \cdots & \alpha_2^m \\
        & \ddots & \\
        \alpha_{m + 1} & \alpha_{m + 1}^2 & \cdots & \alpha_{m + 1}^m
        \end{bmatrix} \\
      &= \alpha_2 \cdots \alpha_{m + 1}\det\begin{bmatrix}
        1 & \alpha_2 & \alpha_2^2 & \cdots & \alpha_2^{m - 1} \\
        & \ddots & \\
        1 & \alpha_{m + 1} & \alpha_{m + 1}^2 & \cdots & \alpha_{m + 1}^{m - 1}
        \end{bmatrix} \\
      &= \alpha_2 \cdots \alpha_{m + 1}\prod_{2 \leq i < j \leq m} (\alpha_j - \alpha_i).
  \end{align*}
  A similar argument can be applied to other cases and we obtain
  \begin{align*}
    \Delta = \sum_{k=1}^{m + 1} (-1)^{k + 1}(\alpha_1 \cdots \hat{\alpha_k} \cdots \alpha_m)\prod_{i < j, i \ne k, j \ne k} (\alpha_j - \alpha_i).
  \end{align*}
  It can be observed that, for each $k = 1, \cdots, m + 1$, the $k$th term $(\alpha_1 \cdots \hat{\alpha_k} \cdots \alpha_m)\prod_{i < j, i \ne k, j \ne k} (\alpha_j - \alpha_i)$ does not contain any $\alpha_k$.
  On the other hand, for any $l \ne k$, every term that we obtain when expanding the $l$th term contains $\alpha_k$.
  Therefore, it suffices to show that, for each $k$, the sum of all the terms in $\prod_{1 \leq i < j \leq m + 1}(\alpha_j - \alpha_i)$ that do not contain $\alpha_k$ is equal to the $k$th term in the above expression.
  \begin{align*}
    \prod_{1 \leq i < j \leq m + 1} (\alpha_j - \alpha_i)
      &= \prod_{k + 1 \leq j} (\alpha_j - \alpha_k)\prod_{j \leq k - 1} (\alpha_k - \alpha_j)\prod_{1 \leq i < j \leq m + 1, i \ne k, j \ne k}(\alpha_j - \alpha_i) \\
      &= (-1)^{k - 1}\prod_{j \ne k} (\alpha_j - \alpha_k)\prod_{1 \leq i < j \leq m + 1, i \ne k, j \ne k}(\alpha_j - \alpha_i) \\
      &= (-1)^{k - 1}(\alpha_1 \cdots \hat{\alpha_k} \cdots \alpha_{m + 1})\prod_{1 \leq i < j \leq m + 1, i \ne k, j \ne k}(\alpha_j - \alpha_i) + \alpha_kF(\alpha_1, \cdots, \alpha_{m + 1}) \\
      &= (-1)^{k + 1}(\alpha_1 \cdots \hat{\alpha_k} \cdots \alpha_{m + 1})\prod_{1 \leq i < j \leq m + 1, i \ne k, j \ne k}(\alpha_j - \alpha_i) + \alpha_kF(\alpha_1, \cdots, \alpha_{m + 1})
  \end{align*}
  for some polynomial $F$.

  $\Delta^2 \ne \prod_{i \ne j}(\alpha_j - \alpha_i)$ in general.
  Let $\alpha_1 = 0, \alpha_2 = 1$.
  Then $\det(A)^2 = \det\begin{bmatrix} 1 & 0 \\ 1 & 1 \end{bmatrix}^2 = 1$.
  On the other hand, $\prod_{i \ne j}(\alpha_j - \alpha_i) = (0 - 1)(1 - 0) = -1$.
\end{exer}

\begin{exer}{(5)}
  Since $R$ is Noetherian, $\sqrt{I}$ is generated by finitely many elements.
  Let $g_1, \cdots, g_n$ denote a set of generators of $\sqrt{I}$.

  For each $i$, there exists $m_i \geq 1$ such that $g_i^{m_i} \in I$.
  Let $N = \sum m_i$.
  Then $(\sqrt{I})^N = \sqrt{I} \cdots \sqrt{I}$ consists of elements of the form $(\sum_{i=1}^{n} x_{1, i}g_{i}) \cdots (\sum_{i=1}^{n} x_{N, i}g_{i})$.
  Each term that we obtain by expanding it is of the from $xg_1^{k_1} \cdots g_n^{k_n}$ for some $k_1, \cdots, k_n$ with $k_1 + \cdots + k_n = N$.
  This implies that for at least one $i$, $m_i \geq k_i$, so each term in the expansion belongs to $I$.
  Therefore, every element in $(\sqrt{I})^N$ is in $I$.
\end{exer}

\begin{exer}{(6)}
  Let $ab \in \sqrt{q}$.
  Then $a^nb^n \in q$ for some $n \in \mathbb{N}$.
  Then $a^n \in q$ or $(b^n)^m \in q$ for some $m \in \mathbb{N}$.
  If $a^n \in q$, then $a \in \sqrt{q}$.
  If $b^{nm} \in q$, then $b \in \sqrt{q}$.
  Therefore, $\sqrt{q}$ is prime.

  Let $f:A \rightarrow B$ be given and $q$ be a primary ideal of $B$.
  Let $ab \in f^{-1}(q)$.
  Then $f(a)f(b) \in q$, so $f(a) \in q$ or $(f(b))^m \in q$ for some $m \geq 1$.
  If $f(a) \in q$, then $a \in f^{-1}(q)$.
  If $f(b^m) \in q$, then $b^m \in f^{-1}(q)$.
  Therefore, $f^{-1}(q)$ is primary.
\end{exer}

\begin{exer}{(7)}
  Since $\sqrt{I}$ is maximal, $I \ne R$.

  Let $x + I, y + I \in A / I$ be two nonzero elements such that $(x + I)(y + I) = 0$.
  In other words, $xy \in I$.
  Since $I \subset \sqrt{I}$, $(x + \sqrt{I})(y + \sqrt{I}) = 0$.
  Since $\sqrt{I}$ is maximal, $A / \sqrt{I}$ is a field.
  Therefore, $x + \sqrt{I} = 0$ or $y + \sqrt{I} = 0$.
  In other words, $x \in \sqrt{I}$ or $y \in \sqrt{I}$.
  If $x \in \sqrt{I}$, then $x + I$ is nilpotent in $A + I$.
  Suppose $x \notin \sqrt{I}$.
  Since $\sqrt{I}$ is maximal, $(x) + \sqrt{I} = (1)$.
  Therefore, $ax + b = 1$ for some $a \in R$ and $b \in \sqrt{I}$.
  Since $b \in \sqrt{I}$, $b^n \in I$ for some $n \geq 1$.
  Therefore $1 = ((ax + b) + I)^n = (ax + b)^n + I = xc + I$ for some element $c$ since $b^n + I = 0$.
  However, this implies $0 = (x + I)(y + I)(c + I) = y + I$, which is a contradiction.
  Therefore, $x + I$ must be nilpotent in $A + I$.
  By symmetry, $y + I$ must be nilpotent in $A + I$.
  
  We have shown that every zero divisor in $A / I$ is nilpotent, which is precisely the definition of a primary ideal.
\end{exer}

\begin{exer}{(9)}
  Let $x \in (q:b)$.
  Then $xb \in q$.
  Since $b \notin q$, $x^n \in q$ for some $n \geq 1$.
  However, this implies $x \in p$.
  Since $(q:b) \subset p$, $\sqrt{(q:b)} \subset \sqrt{p} = p$.
  Clearly, $q \subset (q:b)$, so $p = \sqrt{q} \subset \sqrt{(q:b)}$.
  Therefore, $p = \sqrt{(q:b)}$.

  We will now show that $\sqrt{(q:b)}$ is primary.
  Let $x, y$ be chosen such that $xy \in (q:b)$.
  If $y^n \in (q:b)$ for some $n \geq 1$, we are done.
  In other words, if $y \in \sqrt{(q:b)} = p$, then we are done.
  Suppose otherwise.
  Then $xyb \in q$, so $(xb)y \in q$.
  This implies $xb \in q$ because $y \notin \sqrt{q}$.
  This implies $x \in (q:b)$, and we are done.
\end{exer}

\end{document}


