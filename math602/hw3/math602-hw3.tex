\documentclass[12pt, psamsfonts]{amsart}

%-------Packages---------
\usepackage{amssymb,amsfonts}
\usepackage{semantic}
\usepackage{fullpage}
\usepackage{tikz-cd}
\usepackage{todonotes}
\usepackage{physics}
\usepackage[all,arc]{xy}
\usepackage{enumerate}
\usepackage{enumitem}
\usepackage{mathrsfs}
\usepackage{theoremref}
\usepackage{graphicx}
\usepackage[bookmarks]{hyperref}

%--------Theorem Environments--------
%theoremstyle{plain} --- default
\newtheorem{thm}{Theorem}[section]
\newtheorem{cor}[thm]{Corollary}
\newtheorem{prop}[thm]{Proposition}
\newtheorem{lem}[thm]{Lemma}
\newtheorem{conj}[thm]{Conjecture}
\newtheorem{quest}[thm]{Question}

\theoremstyle{definition}
\newtheorem{defn}[thm]{Definition}
\newtheorem{defns}[thm]{Definitions}
\newtheorem{con}[thm]{Construction}
\newtheorem{exmp}[thm]{Example}
\newtheorem{exmps}[thm]{Examples}
\newtheorem{notn}[thm]{Notation}
\newtheorem{notns}[thm]{Notations}
\newtheorem{addm}[thm]{Addendum}
\newtheorem*{exer}{Exercise}

\theoremstyle{remark}
\newtheorem{rem}[thm]{Remark}
\newtheorem{rems}[thm]{Remarks}
\newtheorem{warn}[thm]{Warning}
\newtheorem{sch}[thm]{Scholium}

\DeclareMathOperator{\Hom}{Hom}
\DeclareMathOperator{\Id}{Id}
\DeclareMathOperator{\End}{End}
\DeclareMathOperator{\ord}{ord}
\DeclareMathOperator{\Aut}{Aut}
\DeclareMathOperator{\Gal}{Gal}

\makeatletter
\let\c@equation\c@thm
\makeatother
\numberwithin{equation}{section}

\bibliographystyle{plain}

\begin{document}

\title{Math 602(Homework 3)}
\author{Hidenori Shinohara}
\maketitle

\begin{section}{Exercises}

\begin{exer}{(Exercise 1)}
  The ideal generated by the three polynomials contains $-yz^4 + yz^2 + y = (xy^2 - xz + y) - y(xy - z^2) + z(x - yz^4)$.
  However, its leading term $-yz^4$ is not in the ideal generated by the leading terms of the three polynomials. 
\end{exer}

\begin{exer}{(Exercise 2)}
 \todo[inline,caption={}]{
   Solve this.
 }
\end{exer}

\begin{exer}{(Exercise 3)}
 \todo[inline,caption={}]{
   Solve this.
 }
\end{exer}

\begin{exer}{(Exercise 4)}
  $0 \in \sqrt{0}$, $a, b \in \sqrt{0} \implies (a + b)^{m + n - 1} = \sum_{i=0}^{m + n - 1} \binom{m + n - 1}{i} a^ib^{m + n - 1 - i} = 0$, and $\forall a \in \sqrt{0}, \forall x \in R, (ax)^n = a^nx^n = 0$, so $\sqrt{0}$ is an ideal.
\end{exer}

\begin{exer}{(Exercise 5)}
 \todo[inline,caption={}]{
   Solve this.
 }
\end{exer}

\begin{exer}{(Exercise 6)}
  Tensoring an exact sequence with $M \otimes_A N$ is the same as tensoring it with $M$ first and tensoring the resulting sequence with $N$ later.
\end{exer}

\end{document}


