\documentclass[12pt, psamsfonts]{amsart}

%-------Packages---------
\usepackage{amssymb,amsfonts}
\usepackage{semantic}
\usepackage{fullpage}
\usepackage{tikz-cd}
\usepackage{todonotes}
\usepackage{physics}
\usepackage[all,arc]{xy}
\usepackage{enumerate}
\usepackage{enumitem}
\usepackage{mathrsfs}
\usepackage{theoremref}
\usepackage{graphicx}
\usepackage[bookmarks]{hyperref}

%--------Theorem Environments--------
%theoremstyle{plain} --- default
\newtheorem{thm}{Theorem}[section]
\newtheorem{cor}[thm]{Corollary}
\newtheorem{prop}[thm]{Proposition}
\newtheorem{lem}[thm]{Lemma}
\newtheorem{conj}[thm]{Conjecture}
\newtheorem{quest}[thm]{Question}

\theoremstyle{definition}
\newtheorem{defn}[thm]{Definition}
\newtheorem{defns}[thm]{Definitions}
\newtheorem{con}[thm]{Construction}
\newtheorem{exmp}[thm]{Example}
\newtheorem{exmps}[thm]{Examples}
\newtheorem{notn}[thm]{Notation}
\newtheorem{notns}[thm]{Notations}
\newtheorem{addm}[thm]{Addendum}
\newtheorem*{exer}{Exercise}

\theoremstyle{remark}
\newtheorem{rem}[thm]{Remark}
\newtheorem{rems}[thm]{Remarks}
\newtheorem{warn}[thm]{Warning}
\newtheorem{sch}[thm]{Scholium}

\DeclareMathOperator{\Hom}{Hom}
\DeclareMathOperator{\Id}{Id}
\DeclareMathOperator{\im}{im}
\DeclareMathOperator{\End}{End}
\DeclareMathOperator{\ord}{ord}
\DeclareMathOperator{\Aut}{Aut}
\DeclareMathOperator{\Gal}{Gal}

\makeatletter
\let\c@equation\c@thm
\makeatother
\numberwithin{equation}{section}

\bibliographystyle{plain}

\begin{document}

\title{Math 602(Homework 3)}
\author{Hidenori Shinohara}
\maketitle

\section{Exercises}

\begin{exer}{(Exercise 1)}
  The ideal generated by the three polynomials contains $-yz^4 + yz^2 + y = (xy^2 - xz + y) - y(xy - z^2) + z(x - yz^4)$.
  However, its leading term $-yz^4$ is not in the ideal generated by the leading terms of the three polynomials. 
\end{exer}

\begin{exer}{(Exercise 2)}
Remainder = $-y^{15} + y^{14} + 7 y^{13} - 7 y^{12} - 21 y^{11} + 21 y^{10} + 35 y^{9} - 35 y^{8} - 35 y^{7} + 35 y^{6} + 21 y^{5} - 21 y^{4} - 7 y^{3} + 7 y^{2} + y - 1$,
$q_1 = x^{6}y^{14} - 6 x^{6}y^{12} + 15 x^{6}y^{10} - 20 x^{6}y^{8} + 15 x^{6}y^{6} - 6 x^{6}y^{4} + x^{6}y^{2} + x^{2}y^{14} - 6 x^{2}y^{12} + 15 x^{2}y^{10} - 20 x^{2}y^{8} + 15 x^{2}y^{6} - 6 x^{2}y^{4} + x^{2}y^{2}$,
$q_2 = 0$
 
Remainder = $y^{23} + y^{11} - y + 1$,
$q_1 = x^{6}y^{2} + x^{5}y^{5} + x^{4}y^{8} + x^{3}y^{11} + x^{2}y^{14} + x^{2}y^{2} + xy^{17} + xy^{5} + y^{20} + y^{8}$,
$q_2 = 0$
\end{exer}

\begin{exer}{(Exercise 3)}
  $z^3 - x^2 = (y - x^2) - (y - z^3)$ is in the ideal generated by the two polynomials, but the leading term is not in the ideal generated by the two polynomials.
\end{exer}

\begin{exer}{(Exercise 4)}
  $0 \in \sqrt{0}$, $a, b \in \sqrt{0} \implies (a + b)^{m + n - 1} = \sum_{i=0}^{m + n - 1} \binom{m + n - 1}{i} a^ib^{m + n - 1 - i} = 0$, and $\forall a \in \sqrt{0}, \forall x \in R, (ax)^n = a^nx^n = 0$, so $\sqrt{0}$ is an ideal.
\end{exer}

\begin{exer}{(Exercise 5)}
  \begin{itemize}
    \item
    \item
      Suppose otherwise.
      Pick $x_j \in I_j \setminus p$ for each $j$.
      $x = \prod_{j=1}^{n} x_j \in I_i$ for each $i$, so $x \in p$.
      Since $p$ is a prime, $p$ must contain at least one $x_j$.
      Contradiction.
  \end{itemize}
\end{exer}

\begin{exer}{(Exercise 6)}
  Tensoring an exact sequence with $M \otimes_A N$ is the same as tensoring it with $M$ first and tensoring the resulting sequence with $N$ later.
\end{exer}

\begin{exer}{(Exercise 7)}
  Since $0 \rightarrow I \xrightarrow{i} R \xrightarrow{q} R / I \rightarrow 0$ is exact, $I \otimes M \rightarrow R \otimes M \rightarrow (R / I) \otimes M \rightarrow 0$ is exact.
  \begin{align*}
    (R / I) \otimes M
      &= \im(q \otimes \Id) \\
      &\cong R \otimes M / \ker(q \otimes \Id) \\
      &\cong R \otimes M / \im(i \otimes \Id) \\
      &\cong R \otimes M / I \otimes M.
  \end{align*}
  Now consider $\phi: R \otimes M \rightarrow M / IM$ that is the composition of $R \otimes M \rightarrow M: x \otimes y \mapsto xy$ and $M \rightarrow M / IM: x \mapsto x + IM$.
  In other words, $\phi$ is $x \otimes y \mapsto xy + IM$.
  Because the two maps are both surjective, $\phi$ must be surjective.
  The kernel of $\phi$ is $I \otimes M$ because
  \begin{itemize}
    \item
      For any $x \otimes y \in I \otimes M$, $\phi(x \otimes y) = xy + IM = 0$ since $xy \in IM$.
    \item
      If $\phi(x \otimes y) = 0$, then $xy \in IM$.
      In other words, $xy = x'y'$ for some $x' \in I$ and $y' \in M$.
      Then $x \otimes y = 1 \otimes xy = 1 \otimes x'y' = x' \otimes y' \in I \otimes M$.
  \end{itemize}
  Therefore, $M / IM \cong (R \otimes M) / (I \otimes M) \cong (R / I) \otimes M$.
\end{exer}

\begin{exer}{(Exercise 8)}
  Let $pa + qb = 1$ for some $p, q \in \mathbb{Z}$.
  Then $1 \otimes 1 = (pa + qb) \otimes (pa + qb) = pa \otimes pa + pa \otimes qb + qb \otimes pa + qb \otimes qb = 0 + 0 + 0 + 0 = 0$.
\end{exer}

\begin{exer}{(Exercise 9)}
  Let $T$ be a $\mathbb{Z}$-module and $f: \mathbb{Q} \times \mathbb{Q} \rightarrow T$ be a bilinear map.
  Then $f(a/b, c/d) = acf(1/b, 1/d) = acbf(1/b, 1/bd) = acf(1, 1/bd) = f(1, ac/bd)$.
  Define a bilinear map $h: \mathbb{Q} \times \mathbb{Q} \rightarrow \mathbb{Q}$ by $(a, b) \mapsto ab$ and a linear map $g: \mathbb{Q} \rightarrow T$ by $a/b \mapsto f(1, a/b)$.
  Then $f = g \circ h$.
  The universal property of a tensor product is satisfied by $\mathbb{Q}$, so $\mathbb{Q} \otimes_{\mathbb{Z}} \mathbb{Q} \cong \mathbb{Q}$.
\end{exer}

\begin{exer}{(Exercise 10)}
  Let $a_1, \cdots, a_n, b_1, \cdots, b_m$ generate $M'$ and $M''$, respectively.
  Let $x_1, \cdots, x_n, y_1, \cdots, y_m \in M$ be chosen such that $x_i$ is the image of $a_i$ and the image of $y_j$ is $b_j$.
  We claim that $x_i, y_j$ generate $M$.
  Let $x \in M$ be given.
  Then $q(x) = d_1b_1 + \cdots + d_mb_m$ for some $d_i \in M$, and thus $q(x - d_1y_1 - \cdots - d_my_m) = 0$.
  Therefore, $x - d_1y_1 - \cdots - d_my_m = i(c_1a_1 + \cdots + c_na_n) = c_1x_1 + \cdots + c_nx_n$, so $x = c_1x_1 + \cdots + c_nx_n + d_1y_1 + \cdots + d_my_m$.
\end{exer}

\begin{exer}{(Exercise 11)}
  This statement is not true.
  When $R = \mathbb{Z}$ and $I = (0)$, $I \otimes_{\mathbb{Z}} \mathbb{Q} = 0$.

  However, the statement is true if $I \ne 0$.
  Let $u \in I$ be a nonzero element.

  Define $h: I \times K \rightarrow K$ by $(a, x / y) \mapsto ax / y$.
  Let $f \in \Hom(I \times K, T)$ be given.

  Define $g: K \rightarrow T$ by $x / y \mapsto f(u, x / uy)$.
  Then 
  \begin{align*}
    (g \circ h)(a, x / y)
      &= g(h(a, x / y)) \\
      &= g(ax / y) \\
      &= f(u, \frac{ax}{yu}) \\
      &= af(u, \frac{x}{yu}) \\
      &= f(au, \frac{x}{yu}) \\
      &= uf(a, \frac{x}{yu}) \\
      &= f(a, \frac{xu}{yu}) \\
      &= f(a, x/y).
  \end{align*}

  Thus $f, g, h$ commute and thus $K \cong I \otimes K$.
\end{exer}

\section{Computation}

\begin{exer}{(Exercise 2)}
  $xy^2$ and $4z^3$.
\end{exer}

\begin{exer}{(Exercise 3)}
  The Sympy code \textbf{groebner([x * y + y**2 - 1, x * y], z, y, x, order = `grlex')} gives $\{ y^2 - 1, x \}$
\end{exer}


\end{document}


