\documentclass[12pt, psamsfonts]{amsart}

%-------Packages---------
\usepackage{amssymb,amsfonts}
\usepackage{semantic}
\usepackage{fullpage}
\usepackage{tikz-cd}
\usepackage{todonotes}
\usepackage{physics}
\usepackage[all,arc]{xy}
\usepackage{enumerate}
\usepackage{enumitem}
\usepackage{mathrsfs}
\usepackage{theoremref}
\usepackage{graphicx}
\usepackage[bookmarks]{hyperref}

%--------Theorem Environments--------
%theoremstyle{plain} --- default
\newtheorem{thm}{Theorem}[section]
\newtheorem{cor}[thm]{Corollary}
\newtheorem{prop}[thm]{Proposition}
\newtheorem{lem}[thm]{Lemma}
\newtheorem{conj}[thm]{Conjecture}
\newtheorem{quest}[thm]{Question}

\theoremstyle{definition}
\newtheorem{defn}[thm]{Definition}
\newtheorem{defns}[thm]{Definitions}
\newtheorem{con}[thm]{Construction}
\newtheorem{exmp}[thm]{Example}
\newtheorem{exmps}[thm]{Examples}
\newtheorem{notn}[thm]{Notation}
\newtheorem{notns}[thm]{Notations}
\newtheorem{addm}[thm]{Addendum}
\newtheorem*{exer}{Exercise}

\theoremstyle{remark}
\newtheorem{rem}[thm]{Remark}
\newtheorem{rems}[thm]{Remarks}
\newtheorem{warn}[thm]{Warning}
\newtheorem{sch}[thm]{Scholium}

\DeclareMathOperator{\Hom}{Hom}
\DeclareMathOperator{\Id}{Id}
\DeclareMathOperator{\End}{End}
\DeclareMathOperator{\ord}{ord}
\DeclareMathOperator{\Aut}{Aut}
\DeclareMathOperator{\Gal}{Gal}
\DeclareMathOperator{\Ann}{Ann}
\DeclareMathOperator{\Spec}{Spec}

\makeatletter
\let\c@equation\c@thm
\makeatother
\numberwithin{equation}{section}

\bibliographystyle{plain}

\begin{document}

\title{Math 602 Homework 4}
\author{Hidenori Shinohara}
\maketitle

\begin{exer}{(1)}
  Let $a / s \in S^{-1}\sqrt{I}$.
  Then $a^n \in I$ and $s \in S$ for some $n \in \mathbb{N}$.
  This implies $(a / s)^n \in S^{-1}I$, so $a / s \in \sqrt{S^{-1}I}$.

  Let $a / s \in \sqrt{S^{-1}I}$.
  Then $a^n / s^n \in S^{-1}I$ for some $n \in \mathbb{N}$.
  Then $a^n \in I$, so $a \in \sqrt{I}$.
  Since $s \in S$, $a / s \in S^{-1}\sqrt{I}$.
\end{exer}

\begin{exer}{(2)}
  Let $\{ V_{\alpha} \}$ be an open cover of $\Spec(R)$.
  For each $\alpha$, $\Spec(R) \setminus V_{\alpha} = V(a_{\alpha})$ for some ideal $a_{\alpha}$ of $R$.
  $\Spec(R) = \cup_{\alpha \in I} V_{\alpha} = U_{\alpha \in I} (\Spec(R) \setminus V(a_{\alpha})) = \Spec(R) \setminus V(\cup a_{\alpha}) = \Spec(R) \setminus V(\sum a_{\alpha})$.
  In other words, $V(\sum a_{\alpha}) = \emptyset$.
  Since every proper ideal is contained in a maximal ideal, $\sum a_{\alpha} = (1)$.
  This implies $1 = x_{\alpha_1} + \cdots + x_{\alpha_n}$ for some $\alpha_1, \cdots, \alpha_n \in I$ and $x_{\alpha_i} \in a_{\alpha_i}$.
  Then $\cup V_{\alpha_i} = \Spec(R) \setminus V(\cup a_{\alpha_i}) = \Spec(R) \setminus V(1) = \Spec(R)$.
  Thus $\Spec(R)$ is indeed compact.
\end{exer}

\begin{exer}{(3)}
  Suppose that $I$ is generated by one element $x$.
  Then $ax = 0 \implies a = 0$ because $A$ is an integral domain.
  Therefore, $I$ is a free module with a basis $\{ x \}$.
  
  On the other hand, suppose that $I$ is a free module with a basis $\{ x_{\alpha} \}$.
  Since it is a basis, each $x_{\alpha} \ne 0$.
  Moreover, if the basis contains more than 2 elements, $(-x_{\alpha'})x_{\alpha} + x_{\alpha}x_{\alpha'} = 0$, so it is not linearly independent.
  Therefore, the basis must contain exactly one element.
\end{exer}

\begin{exer}{(4a)}
  If $m = 0$ in each $M_{f_i}$, then, for each $i$, $f_i^{k_i}m = 0$ for some $k_i \geq 0$.
  Let $k = \max\{ k_1, \cdots, k_n \}$.
  Since $1 \in \ev{f_1, \cdots, f_n}$, 1 can be expressed as a linear combination of $N$ monomials consisting of $f_i$'s.
  Then $m = 1m = 1^{Nk}m = 0$ because each monomial in the $Nk$th power of such a linear combination of $N$ monomials contains at least $k$ appearances of one monomial, which kills $m$.
\end{exer}

\begin{exer}{(5)}
  We will consider the $R / I$-module $M / IM$.
  Let $\hat{m}$ be a maximal ideal in $R / I$.
  Then $\hat{m}$ is of the form $\{ x + I \mid x \in m \}$ for some maximal ideal $m$ of $R$ containing $I$.
  $(M / IM)_{m} = M_m / (IM)_m$ by Corollary 3.4(iii).
  Since $M_m = 0$ for any maximal ideal $m$ containing $I$, $(M / IM)_{m} = 0$.
  By Proposition 3.8[Atiyah], $M / IM = 0$, so $M = IM$.
\end{exer}

\begin{exer}{(6a)}
  $(M:N)$ is nonempty.
  For any $a, b \in (M:N)$, $(a - b)N = aN + (-b)N = aN + bN \subset M$, so $a - b \in (M:N)$.
  Finally, for any $a \in (M:N), x \in R$, $(xa)N = a(xN) \subset aN \subset M$, $ax \in (M:N)$.
\end{exer}

\begin{exer}{(6b)}
  $ $
  \begin{align*}
    a \in \Ann((M + N) / M)
      &\iff a((M + N) / M) = 0 \\
      &\iff \forall (m + n) + M \in (M + N) / M, a((m + n) + M) = 0 \\
      &\iff \forall (m + n) + M \in (M + N) / M, am + an \in M \\
      &\iff \forall n \in N, an \in M \\
      &\iff aN \subset M \\
      &\iff a \in (M:N).
  \end{align*}
\end{exer}

\begin{exer}{(6c)}
  First, we assume that $J$ is generated by a single element $x$.
  Then $Rx = R / \Ann(x)$.
  Then $S^{-1}(Rx) = S^{-1}R / S^{-1}\Ann(x)$.
  On the other hand, $S^{-1}(Rx)$ is an ideal of $S^{-1}R$ generated by $x$, so $(S^{-1}R)x \cong S^{-1}R / \Ann(S^{-1}Rx)$.
  Therefore, $S^{-1}\Ann(x) = \Ann(S^{-1}Rx)$.
  In other words, $S^{-1}\Ann(J) = \Ann(S^{-1}J)$.

  Moreover, if $J_1, J_2$ are generated by single elements, 
  \begin{align*}
    S^{-1}\Ann(J_1 + J_2)
      &= S^{-1}(\Ann(J_1) \cap \Ann(J_2)) \\
      &= S^{-1}\Ann(J_1) \cap S^{-1}\Ann(J_2) \\
      &= \Ann(S^{-1}J_1) \cap \Ann(S^{-1}J_2) \\
      &= \Ann(S^{-1}J_1 + S^{-1}J_2) \\
      &= \Ann(S^{-1}(J_1 + J_2)).
  \end{align*}

  By induction, $S^{-1}\Ann(J) = \Ann(S^{-1}J)$ for any finitely generated ideal.
  Then 
  \begin{align*}
    S^{-1}(I:J)
      &= S^{-1}\Ann((I + J) / I) \\
      &= \Ann(S^{-1}(I + J) / S^{-1}I) \\
      &= \Ann((S^{-1}I + S^{-1}J) / S^{-1}I) \\
      &= (S^{-1}I:S^{-1}J).
  \end{align*}
\end{exer}

\begin{exer}{(7)}
  Let $q \in V(p)$.
  Suppose $M_q = 0$.
  Let $m / s \in (A - q)^{-1}M$.
  Then $tm = 0$ for some $t \in A - q$.
  In other words, for each $m \in M$, there exists $t \in A - q$ such that $tm = 0$.

  Since $p \subset q$, for each $m$, the $t$ must live in $A - p$.
  Therefore, $M_p = 0$.
  However, this is a contradiction because $p \in \text{Supp}(M)$.
  Thus $q \in \text{Supp}(M)$.
\end{exer}

\begin{exer}{(8)}
  Let $b / s \in S^{-1}B$.
  Then $b \in B$, so $b^n + a_{n - 1}b^{n - 1} + \cdots + a_1b + a_0 = 0$ where $a_i \in A$.
  This implies that $(b / s)^n + (a_{n - 1} / s)(b / s)^{n - 1} + \cdots + (a_1 / s^{n - 1})(b / s) + a_0 / s^n = 0$, thus $b / s$ is integral over $S^{-1}A$.
\end{exer}

\end{document}


