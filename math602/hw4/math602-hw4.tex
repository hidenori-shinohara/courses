\documentclass[12pt, psamsfonts]{amsart}

%-------Packages---------
\usepackage{amssymb,amsfonts}
\usepackage{semantic}
\usepackage{fullpage}
\usepackage{tikz-cd}
\usepackage{todonotes}
\usepackage{physics}
\usepackage[all,arc]{xy}
\usepackage{enumerate}
\usepackage{enumitem}
\usepackage{mathrsfs}
\usepackage{theoremref}
\usepackage{graphicx}
\usepackage[bookmarks]{hyperref}

%--------Theorem Environments--------
%theoremstyle{plain} --- default
\newtheorem{thm}{Theorem}[section]
\newtheorem{cor}[thm]{Corollary}
\newtheorem{prop}[thm]{Proposition}
\newtheorem{lem}[thm]{Lemma}
\newtheorem{conj}[thm]{Conjecture}
\newtheorem{quest}[thm]{Question}

\theoremstyle{definition}
\newtheorem{defn}[thm]{Definition}
\newtheorem{defns}[thm]{Definitions}
\newtheorem{con}[thm]{Construction}
\newtheorem{exmp}[thm]{Example}
\newtheorem{exmps}[thm]{Examples}
\newtheorem{notn}[thm]{Notation}
\newtheorem{notns}[thm]{Notations}
\newtheorem{addm}[thm]{Addendum}
\newtheorem*{exer}{Exercise}

\theoremstyle{remark}
\newtheorem{rem}[thm]{Remark}
\newtheorem{rems}[thm]{Remarks}
\newtheorem{warn}[thm]{Warning}
\newtheorem{sch}[thm]{Scholium}

\DeclareMathOperator{\Hom}{Hom}
\DeclareMathOperator{\Id}{Id}
\DeclareMathOperator{\End}{End}
\DeclareMathOperator{\ord}{ord}
\DeclareMathOperator{\Aut}{Aut}
\DeclareMathOperator{\Gal}{Gal}
\DeclareMathOperator{\Ann}{Ann}

\makeatletter
\let\c@equation\c@thm
\makeatother
\numberwithin{equation}{section}

\bibliographystyle{plain}

\begin{document}

\title{Math 602 Homework 4}
\author{Hidenori Shinohara}
\maketitle

\begin{exer}{(1)}
  Let $a / s \in S^{-1}\sqrt{I}$.
  Then $a^n \in I$ and $s \in S$ for some $n \in \mathbb{N}$.
  This implies $(a / s)^n \in S^{-1}I$, so $a / s \in \sqrt{S^{-1}I}$.

  Let $a / s \in \sqrt{S^{-1}I}$.
  Then $a^n / s^n \in S^{-1}I$ for some $n \in \mathbb{N}$.
  Then $a^n \in I$, so $a \in \sqrt{I}$.
  Since $s \in S$, $a / s \in S^{-1}\sqrt{I}$.
\end{exer}

\begin{exer}{(6a)}
  $(M:N)$ is nonempty.
  For any $a, b \in (M:N)$, $(a - b)N = aN + (-b)N = aN + bN \subset M$, so $a - b \in (M:N)$.
  Finally, for any $a \in (M:N), x \in R$, $(xa)N = a(xN) \subset aN \subset M$, $ax \in (M:N)$.
\end{exer}

\begin{exer}{(6b)}
  $ $
  \begin{align*}
    a \in \Ann((M + N) / M)
      &\iff a((M + N) / M) = 0 \\
      &\iff \forall (m + n) + M \in (M + N) / M, a((m + n) + M) = 0 \\
      &\iff \forall (m + n) + M \in (M + N) / M, am + an \in M \\
      &\iff \forall n \in N, an \in M \\
      &\iff aN \subset M \\
      &\iff a \in (M:N).
  \end{align*}
\end{exer}

\end{document}


