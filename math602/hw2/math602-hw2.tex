\documentclass[12pt, psamsfonts]{amsart}

%-------Packages---------
\usepackage{amssymb,amsfonts}
\usepackage{semantic}
\usepackage{fullpage}
\usepackage{tikz-cd}
\usepackage{todonotes}
\usepackage{physics}
\usepackage[all,arc]{xy}
\usepackage{enumerate}
\usepackage{enumitem}
\usepackage{mathrsfs}
\usepackage{theoremref}
\usepackage{graphicx}
\usepackage[bookmarks]{hyperref}

%--------Theorem Environments--------
%theoremstyle{plain} --- default
\newtheorem{thm}{Theorem}[section]
\newtheorem{cor}[thm]{Corollary}
\newtheorem{prop}[thm]{Proposition}
\newtheorem{lem}[thm]{Lemma}
\newtheorem{conj}[thm]{Conjecture}
\newtheorem{quest}[thm]{Question}

\theoremstyle{definition}
\newtheorem{defn}[thm]{Definition}
\newtheorem{defns}[thm]{Definitions}
\newtheorem{con}[thm]{Construction}
\newtheorem{exmp}[thm]{Example}
\newtheorem{exmps}[thm]{Examples}
\newtheorem{notn}[thm]{Notation}
\newtheorem{notns}[thm]{Notations}
\newtheorem{addm}[thm]{Addendum}
\newtheorem*{exer}{Exercise}

\theoremstyle{remark}
\newtheorem{rem}[thm]{Remark}
\newtheorem{rems}[thm]{Remarks}
\newtheorem{warn}[thm]{Warning}
\newtheorem{sch}[thm]{Scholium}

\DeclareMathOperator{\Hom}{Hom}
\DeclareMathOperator{\Id}{Id}
\DeclareMathOperator{\End}{End}
\DeclareMathOperator{\ord}{ord}
\DeclareMathOperator{\Aut}{Aut}
\DeclareMathOperator{\Spec}{Spec}
\DeclareMathOperator{\Gal}{Gal}

\makeatletter
\let\c@equation\c@thm
\makeatother
\numberwithin{equation}{section}

\bibliographystyle{plain}

\begin{document}

\title{Math 602 Homework 2}
\author{Hidenori Shinohara}
\maketitle

\begin{exer}{(Problem 1)}
  We will assume that the problem meant to say ``$su$ with $s \in S \setminus \{ 0 \}$" because it would be trivial otherwise.
  Choose $a_{n - 1}, \cdots, a_0 \in R$ such that $u^n + a_{n - 1}u^{n - 1} + \cdots + a_1u^1 + a_0 = 0$.
  If $a_0 = 0$, then $u(u^{n - 1} + a_{n - 1}u^{n - 2} + \cdots + a_1) = 0$.
  Since we are dealing with integral domains, this implies $u^{n - 1} + a_{n - 1}u^{n - 2} + \cdots + a_1 = 0$.
  By repeating this process, we obtain a monic polynomial with coefficients in $R$ and a nonzero constant term that $u$ satisfies.

  Therefore, we may assume $a_0 \ne 0$.
  Then $u(a_1 + a_2u + \cdots + a_{n - 1}u^{n - 2} + u^{n - 1}) = -a_0 \in R$.
  Since $a_0 \ne 0$, $a_1 + a_2u + \cdots + a_{n - 1}u^{n - 2} + u^{n - 1}$ is a nonzero element in $S$.
  Hence, we showed that some multiple of $u$ lives in $R$.
\end{exer}

\begin{exer}{(Problem 2)}
  Let $R = \mathbb{Z}$ and $S = 2\mathbb{Z}$.
  $R \setminus S$ is not even an ideal because $0 \notin R \setminus S$.
  Thus $R \setminus S$ is not a prime ideal.
\end{exer}

\begin{exer}{(Problem 3)}
  \todo[inline,caption={}]{
    Solve this!
  }
\end{exer}

\begin{exer}{(Problem 4)}
  Let $p \in \Spec(R)$ such that $I \subset p$.
  Define $p / I = \{ x + I \mid x \in p \} \subset R / I$.
  By the third isomorphism theorem of rings, $p / I$ is an ideal of $R / I$.
  Let $x + I, y + I \in R / I$ and suppose $(x + I)(y + I) \in p / I$.
  Then $xy + I = z + I$ for some $z \in p$.
  Thus $xy - z \in I \subset p$ and $z \in p$.
  Thus $xy \in p$, so $x \in p$ or $y \in p$.
  This implies $x + I \in p / I$ or $y + I \in p / I$, so $p / I \in \Spec(R / I)$.

  On the other hand, let $A / I \subset \Spec(R / I)$ be given.
  By the third isomorphism theorem of rings, every ideal of $R / I$ must be of the form $A / I$ where $A$ is an ideal of $R$ that contains $I$.
  Let $x, y \in R$ and suppose $xy \in A$.
  Then $xy + I \in A / I$, so $(x + I)(y + I) \in A / I$.
  Without loss of generality, $x + I \in A / I$.
  Then $x + I = a + I$ for some $a \in A$.
  Thus $x - a \in I \subset A$ and $a \in A$, so $x \in A$.
  Therefore, $A$ is a prime ideal of $R$ containing $I$.
\end{exer}

\end{document}


