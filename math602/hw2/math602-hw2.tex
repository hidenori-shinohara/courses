\documentclass[12pt, psamsfonts]{amsart}

%-------Packages---------
\usepackage{amssymb,amsfonts}
\usepackage{semantic}
\usepackage{fullpage}
\usepackage{tikz-cd}
\usepackage{todonotes}
\usepackage{physics}
\usepackage[all,arc]{xy}
\usepackage{enumerate}
\usepackage{enumitem}
\usepackage{mathrsfs}
\usepackage{theoremref}
\usepackage{graphicx}
\usepackage[bookmarks]{hyperref}

%--------Theorem Environments--------
%theoremstyle{plain} --- default
\newtheorem{thm}{Theorem}[section]
\newtheorem{cor}[thm]{Corollary}
\newtheorem{prop}[thm]{Proposition}
\newtheorem{lem}[thm]{Lemma}
\newtheorem{conj}[thm]{Conjecture}
\newtheorem{quest}[thm]{Question}

\theoremstyle{definition}
\newtheorem{defn}[thm]{Definition}
\newtheorem{defns}[thm]{Definitions}
\newtheorem{con}[thm]{Construction}
\newtheorem{exmp}[thm]{Example}
\newtheorem{exmps}[thm]{Examples}
\newtheorem{notn}[thm]{Notation}
\newtheorem{notns}[thm]{Notations}
\newtheorem{addm}[thm]{Addendum}
\newtheorem*{exer}{Exercise}

\theoremstyle{remark}
\newtheorem{rem}[thm]{Remark}
\newtheorem{rems}[thm]{Remarks}
\newtheorem{warn}[thm]{Warning}
\newtheorem{sch}[thm]{Scholium}

\DeclareMathOperator{\Hom}{Hom}
\DeclareMathOperator{\Id}{Id}
\DeclareMathOperator{\End}{End}
\DeclareMathOperator{\ord}{ord}
\DeclareMathOperator{\Aut}{Aut}
\DeclareMathOperator{\Gal}{Gal}

\makeatletter
\let\c@equation\c@thm
\makeatother
\numberwithin{equation}{section}

\bibliographystyle{plain}

\begin{document}

\title{Math 602 Homework 2}
\author{Hidenori Shinohara}
\maketitle

\begin{exer}{(Problem 1)}
  We will assume that the problem meant to say ``$su$ with $s \in S \setminus \{ 0 \}$" because it would be trivial otherwise.
  Choose $a_{n - 1}, \cdots, a_0 \in R$ such that $u^n + a_{n - 1}u^{n - 1} + \cdots + a_1u^1 + a_0 = 0$.
  If $a_0 = 0$, then $u(u^{n - 1} + a_{n - 1}u^{n - 2} + \cdots + a_1) = 0$.
  Since we are dealing with integral domains, this implies $u^{n - 1} + a_{n - 1}u^{n - 2} + \cdots + a_1 = 0$.
  By repeating this process, we obtain a monic polynomial with coefficients in $R$ and a nonzero constant term that $u$ satisfies.

  Therefore, we may assume $a_0 \ne 0$.
  Then $u(a_1 + a_2u + \cdots + a_{n - 1}u^{n - 2} + u^{n - 1}) = -a_0 \in R$.
  Since $a_0 \ne 0$, $a_1 + a_2u + \cdots + a_{n - 1}u^{n - 2} + u^{n - 1}$ is a nonzero element in $S$.
  Hence, we showed that some multiple of $u$ lives in $R$.
\end{exer}

\end{document}


