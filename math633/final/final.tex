\documentclass[12pt, psamsfonts]{amsart}

%-------Packages---------
\usepackage{amssymb,amsfonts}
\usepackage{semantic}
\usepackage{fullpage}
\usepackage{tikz-cd}
\usepackage{todonotes}
\usepackage{physics}
\usepackage[all,arc]{xy}
\usepackage{enumerate}
\usepackage{enumitem}
\usepackage{mathrsfs}
\usepackage{theoremref}
\usepackage{graphicx}
\usepackage[bookmarks]{hyperref}

%--------Theorem Environments--------
%theoremstyle{plain} --- default
\newtheorem{thm}{Theorem}[section]
\newtheorem{cor}[thm]{Corollary}
\newtheorem{prop}[thm]{Proposition}
\newtheorem{lem}[thm]{Lemma}
\newtheorem{conj}[thm]{Conjecture}
\newtheorem{quest}[thm]{Question}

\theoremstyle{definition}
\newtheorem{defn}[thm]{Definition}
\newtheorem{defns}[thm]{Definitions}
\newtheorem{con}[thm]{Construction}
\newtheorem{exmp}[thm]{Example}
\newtheorem{exmps}[thm]{Examples}
\newtheorem{notn}[thm]{Notation}
\newtheorem{notns}[thm]{Notations}
\newtheorem{addm}[thm]{Addendum}
\newtheorem*{exer}{Exercise}

\theoremstyle{remark}
\newtheorem{rem}[thm]{Remark}
\newtheorem{rems}[thm]{Remarks}
\newtheorem{warn}[thm]{Warning}
\newtheorem{sch}[thm]{Scholium}

\DeclareMathOperator{\Hom}{Hom}
\DeclareMathOperator{\Id}{Id}
\DeclareMathOperator{\End}{End}
\DeclareMathOperator{\ord}{ord}
\DeclareMathOperator{\Aut}{Aut}
\DeclareMathOperator{\Gal}{Gal}

\makeatletter
\let\c@equation\c@thm
\makeatother
\numberwithin{equation}{section}

\bibliographystyle{plain}

\begin{document}

\title{Math 633 (Final Exam)}
\author{Hidenori Shinohara}
\maketitle

\begin{exer}{(1)}
  Since $f$ is holomorphic and $f \ne 0$, $1 / f$ is a non-constant, holomorphic function on the region $\Omega$.
  By the maximum modulus principle, $1 / f$ cannot attain a maximum value in $\Omega$.
  Therefore, $f$ cannot attain a minimum value in $\Omega$.
\end{exer}

\begin{exer}{(2)}
  It suffices to show that, for every $R > 0$, $f$ is holomorphic on the open disk centered at 0 with radius $R$.
  Let $R > 0$ be given.
  Let $T$ be a triangle inside the open disk $D$ centered at 0 with radius $R$.
  If none of the three edges of $T$ lies on the $x$ or $y$ axis, then $\int_{T} f(z) dz = 0$.
  Suppose some of the three edges of $T$ lies on the $x$ and/or $y$ axis.
  Then $T_n = T + (1 + i) / n$ lies in $D$ for any $n \geq N$ for a sufficiently large $N$.
  Since none of the three edges of $T_n$ lies on the $x$ or $y$ axis, $\int_{T_n} f = 0$ for any $n \geq N$.
  Then $\int_{T} f = \lim_{n \rightarrow \infty} \int_{T_n} f = 0$.
\end{exer}

\end{document}


