\documentclass[12pt, psamsfonts]{amsart}

%-------Packages---------
\usepackage{amssymb,amsfonts}
\usepackage{semantic}
\usepackage{fullpage}
\usepackage{tikz-cd}
\usepackage{todonotes}
\usepackage{physics}
\usepackage[all,arc]{xy}
\usepackage{enumerate}
\usepackage{enumitem}
\usepackage{mathrsfs}
\usepackage{theoremref}
\usepackage{graphicx}
\usepackage[bookmarks]{hyperref}

%--------Theorem Environments--------
%theoremstyle{plain} --- default
\newtheorem{thm}{Theorem}[section]
\newtheorem{cor}[thm]{Corollary}
\newtheorem{prop}[thm]{Proposition}
\newtheorem{lem}[thm]{Lemma}
\newtheorem{conj}[thm]{Conjecture}
\newtheorem{quest}[thm]{Question}

\theoremstyle{definition}
\newtheorem{defn}[thm]{Definition}
\newtheorem{defns}[thm]{Definitions}
\newtheorem{con}[thm]{Construction}
\newtheorem{exmp}[thm]{Example}
\newtheorem{exmps}[thm]{Examples}
\newtheorem{notn}[thm]{Notation}
\newtheorem{notns}[thm]{Notations}
\newtheorem{addm}[thm]{Addendum}
\newtheorem*{exer}{Exercise}

\theoremstyle{remark}
\newtheorem{rem}[thm]{Remark}
\newtheorem{rems}[thm]{Remarks}
\newtheorem{warn}[thm]{Warning}
\newtheorem{sch}[thm]{Scholium}

\DeclareMathOperator{\Hom}{Hom}
\DeclareMathOperator{\Id}{Id}
\DeclareMathOperator{\End}{End}
\DeclareMathOperator{\ord}{ord}
\DeclareMathOperator{\Aut}{Aut}
\DeclareMathOperator{\Gal}{Gal}

\makeatletter
\let\c@equation\c@thm
\makeatother
\numberwithin{equation}{section}

\bibliographystyle{plain}

\begin{document}

\title{Math 633 (Final Exam)}
\author{Hidenori Shinohara}
\maketitle

\begin{exer}{(1)}
  Since $f$ is holomorphic and $f \ne 0$, $1 / f$ is a non-constant, holomorphic function on the region $\Omega$.
  By the maximum modulus principle, $1 / f$ cannot attain a maximum value in $\Omega$.
  Therefore, $f$ cannot attain a minimum value in $\Omega$.
\end{exer}

\begin{exer}{(2)}
  It suffices to show that, for every $R > 0$, $f$ is holomorphic on the open disk centered at 0 with radius $R$.
  Let $R > 0$ be given.
  Let $T$ be a triangle inside the open disk $D$ centered at 0 with radius $R$.
  If none of the three edges of $T$ lies on the $x$ or $y$ axis, then $\int_{T} f(z) dz = 0$.
  Suppose some of the three edges of $T$ lies on the $x$ and/or $y$ axis.
  Then $T_n = T + (1 + i) / n$ lies in $D$ for any $n \geq N$ for a sufficiently large $N$.
  Since none of the three edges of $T_n$ lies on the $x$ or $y$ axis, $\int_{T_n} f = 0$ for any $n \geq N$.
  Then $\int_{T} f = \lim_{n \rightarrow \infty} \int_{T_n} f = 0$.
\end{exer}

\begin{exer}{(6)}
  Let $f = 3z^2$ and $g = z^5 + 1$.
  Then $\abs{f} > \abs{g}$ on the unit circle.
  By Rouche's theorem, $f$ and $f + g$ have the same number of zeros inside the unit circle.
  Clearly, $f$ only has one zero with multiplicity 2.
  Thus $p = f + g$ has exactly two zeros inside the unit circle.

  Let $f = z^5$ and $g = 3z^2 + 1$.
  Then $\abs{f} > \abs{g}$ on the circle centered at 0 with radius 2 because $\abs{g} \leq 3 \cdot 2 \cdot 2 + 1 = 13 < 32 = \abs{f}$.
  By Rouche's theorem, $f$ and $f + g$ have the same number of zeros inside $C$.
  $f$ clearly has one zero with multiplicity 5, so $p = f + g$ has exactly 5 zeros inside $C$.

  Therefore, in the annulus, $p$ has 5 - 2 = 3 zeros.
\end{exer}

\begin{exer}{(7)}
  Let $R > a^2$ be given.
  Let $T_1 = [-R, R]$ and $T_2$ be the upper half of the circle centered at 0 with radius $R$.
  Let $f(z) = \exp(iz) / (z^2 + a^2)$.

  \begin{itemize}
    \item
      $\int_{T_1 + T_2} f(z)$ can be calculated using residues.
      The only singularity of $f$ is $ia$.
      Since it is a simple pole, the residue is $\lim_{z \rightarrow ia} (z - ia)\exp(iz) / (z^2 + a^2) = \exp(-a) / 2ia$ by Theorem 1.4 on P.76.
      By the residue formula, $\int_{T_1 + T_2} f(z) = \pi\exp(-a) / 2a$.
    \item
      \begin{align*}
        \abs{\int_{T_2} f(z)}
          &= \abs{\int_{0}^{1} \frac{\exp(iRe^{\pi i t})}{R^2e^{2\pi it} + a^2} R\pi i e^{\pi i t} dt} \\
          &\leq \int_{0}^{1} \abs{\frac{\exp(iRe^{\pi i t})}{R^2e^{2\pi it} + a^2} R\pi i e^{\pi i t}} dt \\
          &\leq \int_{0}^{1} \frac{\abs{\exp(iRe^{\pi i t})}}{\abs{R^2e^{2\pi it} + a^2}} \abs{R\pi i e^{\pi i t}} dt \\
          &\leq \int_{0}^{1} \frac{\exp(-\Im(R e^{\pi i t}))}{\abs{R^2e^{2\pi it} + a^2}} \abs{R\pi i e^{\pi i t}} dt \\
          &\leq \int_{0}^{1} \frac{1}{\exp(R\sin(\pi t))\abs{R^2e^{2\pi it} + a^2}} \abs{R\pi i e^{\pi i t}} dt \\
          &\leq \int_{0}^{1} \frac{1}{\exp(R\sin(\pi t))\abs{R^2e^{2\pi it} + a^2}} R\pi dt \\
          &\leq \pi \int_{0}^{1} \frac{1}{\exp(R\sin(\pi t))\abs{Re^{2\pi it} + a^2 / R}} dt \\
          &\rightarrow 0. % TODO!
      \end{align*}
  \end{itemize}
  Based on these, we obtain that $\int_{T_1} f(z) = \pi e^{-a} / 2a$ as $R \rightarrow \infty$.
  The desired integral is the real part of $\int_{T_1} f(z)$, and it is simply $\pi e^{-a} / 2a$.
\end{exer}

\begin{exer}{(8)}
  Suppose that $f$ is a polynomial and it has a degree above $d$.
  Then for each $\epsilon > 0$ and each $n \in \mathbb{N}$, we can pick sufficiently large $z$ such that $\abs{f(z) - p_n(z)} > \epsilon$.
  Therefore, if $f$ is a polynomial, it has to have a degree $\leq d$.
\end{exer}

\end{document}


