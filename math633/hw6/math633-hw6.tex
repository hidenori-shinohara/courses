\documentclass[12pt, psamsfonts]{amsart}

%-------Packages---------
\usepackage{amssymb,amsfonts}
\usepackage{semantic}
\usepackage{fullpage}
\usepackage{tikz-cd}
\usepackage{todonotes}
\usepackage{physics}
\usepackage[all,arc]{xy}
\usepackage{enumerate}
\usepackage{enumitem}
\usepackage{mathrsfs}
\usepackage{theoremref}
\usepackage{graphicx}
\usepackage[bookmarks]{hyperref}

%--------Theorem Environments--------
%theoremstyle{plain} --- default
\newtheorem{thm}{Theorem}[section]
\newtheorem{cor}[thm]{Corollary}
\newtheorem{prop}[thm]{Proposition}
\newtheorem{lem}[thm]{Lemma}
\newtheorem{conj}[thm]{Conjecture}
\newtheorem{quest}[thm]{Question}

\theoremstyle{definition}
\newtheorem{defn}[thm]{Definition}
\newtheorem{defns}[thm]{Definitions}
\newtheorem{con}[thm]{Construction}
\newtheorem{exmp}[thm]{Example}
\newtheorem{exmps}[thm]{Examples}
\newtheorem{notn}[thm]{Notation}
\newtheorem{notns}[thm]{Notations}
\newtheorem{addm}[thm]{Addendum}
\newtheorem*{exer}{Exercise}

\theoremstyle{remark}
\newtheorem{rem}[thm]{Remark}
\newtheorem{rems}[thm]{Remarks}
\newtheorem{warn}[thm]{Warning}
\newtheorem{sch}[thm]{Scholium}

\DeclareMathOperator{\Hom}{Hom}
\DeclareMathOperator{\Id}{Id}
\DeclareMathOperator{\End}{End}
\DeclareMathOperator{\ord}{ord}
\DeclareMathOperator{\Aut}{Aut}
\DeclareMathOperator{\Gal}{Gal}

\makeatletter
\let\c@equation\c@thm
\makeatother
\numberwithin{equation}{section}

\bibliographystyle{plain}

\begin{document}

\title{Math 633 Homework 6}
\author{Hidenori Shinohara}
\maketitle


\begin{exer}{(1)}
  \todo[inline,caption={}]{
  }
\end{exer}

\begin{exer}{(2)}
  $z \mapsto az + b$ and $z \mapsto cz + d$ are clearly entire.
  If $c = 0$, then $\phi: z \mapsto (az + b) / (cz + d)$ is entire.
  If $c \ne 0$, then $\phi$ is holomorphic everywhere except for $-d / c$ and at $-d / c$, $\phi$ has a pole because $\phi(-d / c) = \infty$.
  In other words, it is meromorphic.

  Let $\phi: z \mapsto (az + b) / (cz + d)$ and $\psi: z \mapsto (-dz + b) / (cz - a)$.
  Then $\phi(\psi(z)) = z$ and $\psi(\phi(z)) = z$, and $(-d)(-a) - bc = ad - bc \ne 0$.

  \todo[inline,caption={}]{
    Finish the last part.
  }
\end{exer}

\begin{exer}{(3)}
  \todo[inline,caption={}]{
  }
\end{exer}

\end{document}


