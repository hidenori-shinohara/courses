\documentclass[12pt, psamsfonts]{amsart}

%-------Packages---------
\usepackage{amssymb,amsfonts}
\usepackage{semantic}
\usepackage{fullpage}
\usepackage{tikz-cd}
\usepackage{todonotes}
\usepackage{physics}
\usepackage[all,arc]{xy}
\usepackage{enumerate}
\usepackage{enumitem}
\usepackage{mathrsfs}
\usepackage{theoremref}
\usepackage{graphicx}
\usepackage[bookmarks]{hyperref}

%--------Theorem Environments--------
%theoremstyle{plain} --- default
\newtheorem{thm}{Theorem}[section]
\newtheorem{cor}[thm]{Corollary}
\newtheorem{prop}[thm]{Proposition}
\newtheorem{lem}[thm]{Lemma}
\newtheorem{conj}[thm]{Conjecture}
\newtheorem{quest}[thm]{Question}

\theoremstyle{definition}
\newtheorem{defn}[thm]{Definition}
\newtheorem{defns}[thm]{Definitions}
\newtheorem{con}[thm]{Construction}
\newtheorem{exmp}[thm]{Example}
\newtheorem{exmps}[thm]{Examples}
\newtheorem{notn}[thm]{Notation}
\newtheorem{notns}[thm]{Notations}
\newtheorem{addm}[thm]{Addendum}
\newtheorem*{exer}{Exercise}

\theoremstyle{remark}
\newtheorem{rem}[thm]{Remark}
\newtheorem{rems}[thm]{Remarks}
\newtheorem{warn}[thm]{Warning}
\newtheorem{sch}[thm]{Scholium}

\DeclareMathOperator{\Hom}{Hom}
\DeclareMathOperator{\Id}{Id}
\DeclareMathOperator{\End}{End}
\DeclareMathOperator{\ord}{ord}
\DeclareMathOperator{\Aut}{Aut}
\DeclareMathOperator{\Gal}{Gal}

\makeatletter
\let\c@equation\c@thm
\makeatother
\numberwithin{equation}{section}

\bibliographystyle{plain}

\begin{document}

\title{Math 633 Homework 6}
\author{Hidenori Shinohara}
\maketitle


\begin{exer}{(1)}
  Define the map $f: H \rightarrow \Omega_1$ such that $f(z) = \exp(\log(z)/\alpha)$ where $\log$ denotes the principal branch of the complex logarithm function.
  This is well defined because $H$ does not contain the real line.
  Moreover, this is holomorphic because it is the composition of holomorphic functions.
  Finally, $f'(z) = \exp(\log(z)/\alpha) / z \ne 0$ on $H$.
  Thus $f$ is conformal.
\end{exer}

\begin{exer}{(2)}
  $z \mapsto az + b$ and $z \mapsto cz + d$ are clearly entire.
  If $c = 0$, then $\phi: z \mapsto (az + b) / (cz + d)$ is entire.
  If $c \ne 0$, then $\phi$ is holomorphic everywhere except for $-d / c$ and at $-d / c$, $\phi$ has a pole because $\phi(-d / c) = \infty$.
  In other words, it is meromorphic.

  Let $\phi: z \mapsto (az + b) / (cz + d)$ and $\psi: z \mapsto (-dz + b) / (cz - a)$.
  Then $\phi(\psi(z)) = z$ and $\psi(\phi(z)) = z$, and $(-d)(-a) - bc = ad - bc \ne 0$.

  Let $f:\hat{\mathbb{C}} \rightarrow \hat{\mathbb{C}}$ be a bijective meromophism.
  $f$ is actually just holomorphic because $f$ cannot have two poles since it is injective.
  Let $g$ be a mobius transformation that sends $f(\infty)$ to $\infty$.
  Then $g \circ f$ is a bijective map on $\hat{\mathbb{C}}$ into $\hat{\mathbb{C}}$.
  Since $g \circ f$ sends $\infty$ to $\infty$, $g \circ f$ is a bijection on $\mathbb{C}$.
  If $h = g \circ f$ is a polynomial, it must be linear by the previous homework.
  Then $f = g^{-1} \circ h$ is a Mobius transformation.
  Suppose $h = g \circ f$ is not a polynomial.
  Then $(g \circ f)(\{ \abs{z} < 1 \})$ is open because $(g \circ f)$ is a continuous bijection.
  $(g \circ f)(\{ \abs{z} > 1 \})$ is dense in $\mathbb{C}$ because $z \mapsto (g \circ f)(1 / z)$ has an essential singularity around $0$.
  However, this implies there exist $\abs{z_1} > 1, \abs{z_2} < 1$ such that $(g \circ f)(z_1) = (g \circ f)(z_2)$.
  This is a contradiction because $g \circ f$ is bijective.
  Therefore, this case is not possible.
\end{exer}

\begin{exer}{(3)}
  $f$ is entire, so it has a power series expansion $\sum a_nz^n$.
  Then $f$ can be extended to a function on $\hat{\mathbb{C}}$ in a canonical way.

  If $f$ has a removable singularity at $\infty$, then $f$ is bounded in a neighborhood $N$ containing $\infty$.
  Then $N^c$ is a compact subset of $\mathbb{C}$, so $f$ is bounded on $N^c$.
  Therefore, $f$ is bounded on $\mathbb{C}$, so $f$ is constant, which is a contradiction because $f$ must be bijective.

  Suppose $f$ has an essential singularity at $\infty$.
  Then $f(\hat{\mathbb{C}} \setminus D)$ is dense in $\mathbb{C}$ where $D$ is the unit disk.
  This implies that $f(\hat{\mathbb{C}} \setminus D) \cap f(D) \ne \emptyset$, which contradicts the bijectivity of $f$.

  Therefore, $f$ has a pole at $\infty$.
  By Part (c) of Problem 2, $f$ is a mobius transformation.
  $c = 0$ because $-d / c$ would be a pole otherwise.
  Thus $f = (a / d)z + (b / d)$ with $a / d \ne 0$.
\end{exer}

\end{document}


