\documentclass[12pt, psamsfonts]{amsart}

%-------Packages---------
\usepackage{amssymb,amsfonts}
\usepackage{semantic}
\usepackage{fullpage}
\usepackage{tikz-cd}
\usepackage{todonotes}
\usepackage{physics}
\usepackage[all,arc]{xy}
\usepackage{enumerate}
\usepackage{enumitem}
\usepackage{mathrsfs}
\usepackage{theoremref}
\usepackage{graphicx}
\usepackage[bookmarks]{hyperref}

%--------Theorem Environments--------
%theoremstyle{plain} --- default
\newtheorem{thm}{Theorem}[section]
\newtheorem{cor}[thm]{Corollary}
\newtheorem{prop}[thm]{Proposition}
\newtheorem{lem}[thm]{Lemma}
\newtheorem{conj}[thm]{Conjecture}
\newtheorem{quest}[thm]{Question}

\theoremstyle{definition}
\newtheorem{defn}[thm]{Definition}
\newtheorem{defns}[thm]{Definitions}
\newtheorem{con}[thm]{Construction}
\newtheorem{exmp}[thm]{Example}
\newtheorem{exmps}[thm]{Examples}
\newtheorem{notn}[thm]{Notation}
\newtheorem{notns}[thm]{Notations}
\newtheorem{addm}[thm]{Addendum}
\newtheorem*{exer}{Exercise}

\theoremstyle{remark}
\newtheorem{rem}[thm]{Remark}
\newtheorem{rems}[thm]{Remarks}
\newtheorem{warn}[thm]{Warning}
\newtheorem{sch}[thm]{Scholium}

\DeclareMathOperator{\Hom}{Hom}
\DeclareMathOperator{\Id}{Id}
\DeclareMathOperator{\End}{End}
\DeclareMathOperator{\ord}{ord}
\DeclareMathOperator{\Aut}{Aut}
\DeclareMathOperator{\Gal}{Gal}

\makeatletter
\let\c@equation\c@thm
\makeatother
\numberwithin{equation}{section}

\bibliographystyle{plain}

\begin{document}

\title{Math 633 Homework 9}
\author{Hidenori Shinohara}
\maketitle

\begin{exer}{(Problem 1)}
  Let $x \in F_1$.
  Since $\Omega$ is bounded, there exists an $R > 0$ such that $\Omega \subset C(x, R)$.
  Then $F_1 \setminus C(x, R)$ and $F_2 \setminus C(x, R)$ are disjoint, closed sets whose union is $\mathbb{C} \setminus C(x, R)$, which is connected.
  Therefore, either $F_1 \setminus C(x, R)$ or $F_2 \setminus C(x, R)$ is empty.
  In other words, either $F_1 \subset C(x, R)$ or $F_2 \setminus c(x, R)$.
\end{exer}

\begin{exer}{(Problem 2(a))}
  We first assume $\omega = 0$.
  This is reasonable because the following argument can be extended to general cases by translating every function by $\omega$.
  If $r, \theta$ are continuous, it is clear that $\alpha$ is continuous.
  Suppose $\alpha$ is continuous.
  Let $r(t) = \abs{\alpha(t)}$.
  Then $r(t):[0, 1] \rightarrow (0, \infty)$ is continuous.
  Moreover, $\alpha(t) = r(t)e^{i\theta(t)}$, so $r(t) = \abs{r(t)} = \abs{r(t)e^{i\theta(t)}} = \abs{\alpha(t)}$, so this is the only possibility for $r(t)$.

  By using the principal branch of logarithm and translation, we can find $\theta(t)$ locally.
  Since the logarithm function and translation function are both continuous, such local $\theta$'s are continuous.
  Since $[0, 1]$ is compact, we can find a finite cover of $[0, 1]$ such that we have $\theta(t)$ for each open set.
  Two $\theta(t)$ can be patched for any two overlapping open sets by adding $2k\pi$ for an appropriate value of $k$.
  Therefore, we can find $\theta$ that is continuous and satisfies $\alpha(t) = r(t)e^{i\theta(t)}$.
  Any other functions $\gamma(t)$ that satisfy the conditions must satisfy $1 = \alpha(t) / \alpha(t) = (r(t)e^{i\theta(t)}) / (r(t)e^{i\gamma(t)}) = e^{i(\theta(t) - \gamma(t))}$, so $\theta(t) - \gamma(t) = 2k\pi$ for some fixed $k \in \mathbb{Z}$.
  
  Hence, we have shown that $\alpha$ is continuous if and only if such continuous $r, \theta$ exist and the choice of $r, \theta$ are unique up to an additive constant for $\theta$.
\end{exer}

\begin{exer}{(Problem 2(b))}
  Again, we will assume $\omega = 0$.
  $\alpha(1) / \alpha(0) = (r(1)e^{i\theta(1)}) / (r(0)e^{i\theta(0)}) = e^{i(\theta(1) - \theta(0))}$ because $r(1) = \abs{\alpha(1)} = \abs{\alpha(0)} = r(0)$.
  Since $\alpha(1) = \alpha(0)$, $e^{i(\theta(1) - \theta(0))} = 1$.
  This implies that $\theta(1) - \theta(0) = 2k\pi$ for a fixed $k \in \mathbb{Z}$.
  In other words, $(\theta(1) - \theta(0)) / 2\pi$ is always an integer.
\end{exer}

\begin{exer}{(Problem 3(a))}
  As $r \rightarrow 0$ with $r > 0$, $p(\alpha_r(t))$ is dominated by $a_0$ for any $t \in [0, 1]$.
  In other words, $p \circ \alpha_r$ lies in a small disk around $a_0$ that is disjoint from 0.
  Thus the winding number is 0 for a sufficiently small $r$.

  As $r \rightarrow \infty$, $p(\alpha_r(t))$ is dominated by $a_n\alpha_r(t)^n$ for any $t \in [0, 1]$.
  Since multiplication by $a_n \ne 0$ is simply a rotation around the origin, $p \circ \alpha_r$ is homotopic to the function that goes around the origin $n$ times in a positive orientation for a sufficiently large $r$.
  Thus the winding number is $n$ for large $r$.
\end{exer}

\end{document}


