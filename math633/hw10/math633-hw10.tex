\documentclass[12pt, psamsfonts]{amsart}

%-------Packages---------
\usepackage{amssymb,amsfonts}
\usepackage{semantic}
\usepackage{fullpage}
\usepackage{tikz-cd}
\usepackage{todonotes}
\usepackage{physics}
\usepackage[all,arc]{xy}
\usepackage{enumerate}
\usepackage{enumitem}
\usepackage{mathrsfs}
\usepackage{theoremref}
\usepackage{graphicx}
\usepackage[bookmarks]{hyperref}

%--------Theorem Environments--------
%theoremstyle{plain} --- default
\newtheorem{thm}{Theorem}[section]
\newtheorem{cor}[thm]{Corollary}
\newtheorem{prop}[thm]{Proposition}
\newtheorem{lem}[thm]{Lemma}
\newtheorem{conj}[thm]{Conjecture}
\newtheorem{quest}[thm]{Question}

\theoremstyle{definition}
\newtheorem{defn}[thm]{Definition}
\newtheorem{defns}[thm]{Definitions}
\newtheorem{con}[thm]{Construction}
\newtheorem{exmp}[thm]{Example}
\newtheorem{exmps}[thm]{Examples}
\newtheorem{notn}[thm]{Notation}
\newtheorem{notns}[thm]{Notations}
\newtheorem{addm}[thm]{Addendum}
\newtheorem*{exer}{Exercise}

\theoremstyle{remark}
\newtheorem{rem}[thm]{Remark}
\newtheorem{rems}[thm]{Remarks}
\newtheorem{warn}[thm]{Warning}
\newtheorem{sch}[thm]{Scholium}

\DeclareMathOperator{\Hom}{Hom}
\DeclareMathOperator{\Id}{Id}
\DeclareMathOperator{\End}{End}
\DeclareMathOperator{\ord}{ord}
\DeclareMathOperator{\Aut}{Aut}
\DeclareMathOperator{\Gal}{Gal}

\makeatletter
\let\c@equation\c@thm
\makeatother
\numberwithin{equation}{section}

\bibliographystyle{plain}

\begin{document}

\title{Math 633(Homework 10)}
\author{Hidenori Shinohara}
\maketitle

\begin{exer}{(Exercise 1(b))}
  When $t = 0$, $\gamma(t) = \gamma_{\epsilon}(t) = 0$, so $\gamma$ and $\gamma_{\epsilon}$ intersect for any $\epsilon > 0$.
\end{exer}

\begin{exer}{(Exercise 2)}
  By Theorem 2.1 (P.351), $\mathbb{C} \setminus \gamma$ is connected.
  Suppose $\mathbb{C} \setminus C$ is not connected.
  Then $\mathbb{C} \setminus C = A_1 \cup A_2$ where $A_1, A_2$ are nonempty, disjoint closed sets.
  This implies that $\mathbb{C} \setminus \gamma = (A_1 \cup D \setminus \{ 1 \}) \cup A_2$.
  Since $A_1 \cup D \setminus \{ 1 \}$ and $A_2$ are disjoint and closed in $\mathbb{C} \setminus \gamma$, at least one of them has to be nonempty.
  Clearly, $(D \setminus \{ 1 \}) \ne \emptyset$, $A_2 = \emptyset$.
  However, this is a contradiction because $A_2$ is supposed to be nonempty.
\end{exer}

\end{document}


