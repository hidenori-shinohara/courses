\documentclass[12pt, psamsfonts]{amsart}

%-------Packages---------
\usepackage{amssymb,amsfonts}
\usepackage{semantic}
\usepackage{fullpage}
\usepackage{tikz-cd}
\usepackage{todonotes}
\usepackage{physics}
\usepackage[all,arc]{xy}
\usepackage{enumerate}
\usepackage{enumitem}
\usepackage{mathrsfs}
\usepackage{theoremref}
\usepackage{graphicx}
\usepackage[bookmarks]{hyperref}

%--------Theorem Environments--------
%theoremstyle{plain} --- default
\newtheorem{thm}{Theorem}[section]
\newtheorem{cor}[thm]{Corollary}
\newtheorem{prop}[thm]{Proposition}
\newtheorem{lem}[thm]{Lemma}
\newtheorem{conj}[thm]{Conjecture}
\newtheorem{quest}[thm]{Question}

\theoremstyle{definition}
\newtheorem{defn}[thm]{Definition}
\newtheorem{defns}[thm]{Definitions}
\newtheorem{con}[thm]{Construction}
\newtheorem{exmp}[thm]{Example}
\newtheorem{exmps}[thm]{Examples}
\newtheorem{notn}[thm]{Notation}
\newtheorem{notns}[thm]{Notations}
\newtheorem{addm}[thm]{Addendum}
\newtheorem*{exer}{Exercise}

\theoremstyle{remark}
\newtheorem{rem}[thm]{Remark}
\newtheorem{rems}[thm]{Remarks}
\newtheorem{warn}[thm]{Warning}
\newtheorem{sch}[thm]{Scholium}

\DeclareMathOperator{\Hom}{Hom}
\DeclareMathOperator{\Id}{Id}
\DeclareMathOperator{\End}{End}
\DeclareMathOperator{\ord}{ord}
\DeclareMathOperator{\Aut}{Aut}
\DeclareMathOperator{\Gal}{Gal}

\makeatletter
\let\c@equation\c@thm
\makeatother
\numberwithin{equation}{section}

\bibliographystyle{plain}

\begin{document}

\title{math 633 (Homework 8)}
\author{Hidenori Shinohara}
\maketitle

\begin{exer}{(Problem 13)}
  For any $w \in D$, $\phi_w(z) = (z - w) / (1 - \overline{w}z)$ is an automorphism on $D$ that maps $w$ to $0$.
  Fix $w \in D$.
  Then $\phi_{f(w)} \circ f \circ \phi_w^{-1}$ is an automorphism that maps 0 to 0.
  By Lemma 2.1, $\abs{(\phi_{f(w)} \circ f \circ \phi_w^{-1})(z)} \leq \abs{z}$ for all $z \in D$.
  Fix $z \in D$.
  Then $\phi_w(z) \in D$, so $\abs{(\phi_{f(w)} \circ f \circ \phi_w^{-1})(\phi_w(z))} \leq \abs{\phi_w(z)}$.
  This equals to
  \begin{align*}
    \abs{(\phi_{f(w)} \circ f \circ \phi_w^{-1})(\phi_w(z))} \leq \abs{\frac{z - w}{1 - \overline{w}z}}
      &\implies \abs{\phi_{f(w)}(f(z))} \leq \abs{\frac{z - w}{1 - \overline{w}z}} \\
      &\implies \abs{\frac{f(z) - f(w)}{1 - \overline{f(w)}f(z)}} \leq \abs{\frac{z - w}{1 - \overline{w}z}} \\
      &\implies \rho(f(z), f(w)) \leq \rho(z, w).
  \end{align*}

  For any $z \in D$ and for any appropriate value of $h \ne 0$,
  \begin{align*}
    \rho(f(z + h), f(z)) \leq \rho(z + h, z)
      &\implies \abs{\frac{f(z + h) - f(z)}{1 - \overline{f(z + h)}f(z)}} \leq \abs{\frac{z + h - z}{1 - \overline{(z + h)}z}} \\
      &\implies \abs{\frac{f(z + h) - f(z)}{h}} \cdot \frac{1}{\abs{1 - \overline{f(z + h)}f(z)}} \leq \abs{\frac{1}{1 - \overline{(z + h)}z}}.
  \end{align*}
  By letting $h \rightarrow 0$, we obtain the Schwarz-Pick lemma.
\end{exer}

\begin{exer}{(Problem 3)}
  Let $\Omega = \{ z \in \mathbb{C} \mid \abs{z} < 1 \}$.
  Then $\Omega$ is a Jordan domain such that $\partial\Omega$ is parametrized by $\alpha(t) = e^{2\pi it}$.
  Clearly, $\alpha' \ne 0$ everywhere.
  However, for any $R > 0$ and $a < b < a + 2\pi$, $\Omega \cap C_R(1) \ne \{ 1 + Re^{i\theta} \mid a < \theta < b \}$.
  Therefore, $\Omega$ is not nice, so the proposition is false.
\end{exer}

\end{document}


