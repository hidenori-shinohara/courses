\documentclass[12pt, psamsfonts]{amsart}

%-------Packages---------
\usepackage{amssymb,amsfonts}
\usepackage{semantic}
\usepackage{fullpage}
\usepackage{tikz-cd}
\usepackage{todonotes}
\usepackage{physics}
\usepackage[all,arc]{xy}
\usepackage{enumerate}
\usepackage{enumitem}
\usepackage{mathrsfs}
\usepackage{theoremref}
\usepackage{graphicx}
\usepackage[bookmarks]{hyperref}

%--------Theorem Environments--------
%theoremstyle{plain} --- default
\newtheorem{thm}{Theorem}[section]
\newtheorem{cor}[thm]{Corollary}
\newtheorem{prop}[thm]{Proposition}
\newtheorem{lem}[thm]{Lemma}
\newtheorem{conj}[thm]{Conjecture}
\newtheorem{quest}[thm]{Question}

\theoremstyle{definition}
\newtheorem{defn}[thm]{Definition}
\newtheorem{defns}[thm]{Definitions}
\newtheorem{con}[thm]{Construction}
\newtheorem{exmp}[thm]{Example}
\newtheorem{exmps}[thm]{Examples}
\newtheorem{notn}[thm]{Notation}
\newtheorem{notns}[thm]{Notations}
\newtheorem{addm}[thm]{Addendum}
\newtheorem*{exer}{Exercise}

\theoremstyle{remark}
\newtheorem{rem}[thm]{Remark}
\newtheorem{rems}[thm]{Remarks}
\newtheorem{warn}[thm]{Warning}
\newtheorem{sch}[thm]{Scholium}

\DeclareMathOperator{\Hom}{Hom}
\DeclareMathOperator{\Id}{Id}
\DeclareMathOperator{\End}{End}
\DeclareMathOperator{\ord}{ord}
\DeclareMathOperator{\Aut}{Aut}
\DeclareMathOperator{\Gal}{Gal}

\makeatletter
\let\c@equation\c@thm
\makeatother
\numberwithin{equation}{section}

\bibliographystyle{plain}

\begin{document}

\title{Math 633(Homework 7)}
\author{Hidenori Shinohara}
\maketitle

\begin{exer}{(1)}
  Suppose $f$ is locally bijective.
  For every $p \in U$, $f$ is locally injective.
  Therefore, $f'$ is nonzero in the neighborhood around $p$.
  In other words, $f'$ is nonzero on $U$.

  \todo[inline,caption={}]{
    The other direction
  }
\end{exer}

\begin{exer}{(10)}
  Let $\sigma(z) = -i(z + 1) / (z - 1)$.
  Then $\sigma$ sends the unit disk to the upper half-plane with $\infty$ since
  $\sigma(a + bi) = (-2b - (a^2 + b^2 - 1)i) / ((a - 1)^2 + b^2)$.
  On the other hand, $\sigma^{-1}: z \mapsto (z - i) / (z + i)$ sends the upper half plane with $\infty$ to the unit disk because $\abs{a + (b - 1)i} \leq \abs{a + (b + 1)i}$ if $b \geq 0$.
  Therefore, $\sigma$ is a bijection between the unit disk and $H \cup \{ \infty \}$.
  $F \circ \sigma$ sends the unit disk to the unit disk, and $F(\sigma(0)) = 0$.
  By Lemma 2.1, $\abs{(F \circ \sigma)(w)} \leq \abs{w}$ for every $w \in D$.
  Then for every $z \in \mathbb{H}$, $\sigma^{-1}(z) \in D$.
  Then $\abs{F(z)} = \abs{(F \circ \sigma)(\sigma^{-1}(z))} \leq \abs{\sigma^{-1}(z)} = \abs{(z - i) / (z + i)}$, which is the desired result.
\end{exer}

\begin{exer}{(12(a))}
  Let $a \ne b$ be two fixed points.
  Let $\sigma(z) = (z - a) / (1 - \overline{a}z)$.
  Then $\sigma$ sends $a$ to $0$ and maps $D$ to $D$ bijectively.
  Let $g = \sigma \circ f \circ \sigma^{-1}$.
  $g$ has two fixed points, 0 and $\sigma(b)$.
  By applying Lemma 2.1, $g$ is a rotation.
  However, $g$ fixes $\sigma(b) \ne 0$, so $g$ must be the identity map.
  Then $f$ must be the identity.
\end{exer}

\begin{exer}{(12(b))}
  The map $\sigma: z \mapsto (z - i) / (z + i)$ maps the upper half-plane to the unit disk bijectively.
  Then $\sigma \circ f \circ \sigma^{-1}$ where $f(z) = z + 1$ is a holomorphic bijection on $f$ that has no fixed point because $f$ has no fixed point.
\end{exer}

\begin{exer}{(16(a))}
  The composition of mobius transformations corresponds to the multiplication of the corresponding matrices.
  Thus it suffices to calculate
  \begin{align*}
    \begin{bmatrix}
      -1 & i \\
      1 & i
    \end{bmatrix}^{-1}
    \begin{bmatrix}
      e^{i\theta} & 0 \\
      0 & 1
    \end{bmatrix}
    \begin{bmatrix}
      -1 & i \\
      1 & i
    \end{bmatrix}
    &= \frac{1}{2} \begin{bmatrix} e^{i\theta} + 1 & -i(e^{i\theta} - 1) \\ i(e^{i\theta} - 1) & e^{i\theta} + 1 \end{bmatrix} \\
    &\rightarrow \frac{1}{2} \begin{bmatrix} 1 & -i(e^{i\theta} - 1) / (e^{i\theta} + 1) \\ i(e^{i\theta} - 1) / (e^{i\theta} + 1)  & 1 \end{bmatrix} \\
    &\rightarrow \frac{1}{2} \begin{bmatrix} 1 & -i(i\tan(\theta / 2))  \\ i(i\tan(\theta / 2))  & 1 \end{bmatrix} \\
    &\rightarrow \frac{1}{2} \begin{bmatrix} 1 & \tan(\theta / 2)  \\ -\tan(\theta / 2)  & 1 \end{bmatrix}.
  \end{align*}
  Thus the answer is the mobius transformation associated to the last matrix.
\end{exer}

\begin{exer}{(16(b))}
  Let $\alpha = a + bi$.
  \begin{align*}
    \begin{bmatrix}
      -1 & i \\
      1 & i
    \end{bmatrix}^{-1}
    \begin{bmatrix}
      -1 & \alpha \\
      -\overline{\alpha} & 1 \\
    \end{bmatrix}
    \begin{bmatrix}
      -1 & i \\
      1 & i
    \end{bmatrix}
    &= \frac{1}{2} \begin{bmatrix} \overline{\alpha} - \alpha & -i(\alpha + \overline{\alpha} - 2) \\ -i(\alpha + \overline{\alpha} + 2) & \alpha - \overline{\alpha} \end{bmatrix} \\ 
    &\rightarrow \begin{bmatrix} b & a - 1 \\ a + 1 & -b \end{bmatrix}.
  \end{align*}
  After multiplying $1 / (1 - a^2 - b^2)$ to every term, we obtain a matrix associated to the desired mobius transformation.
\end{exer}

\begin{exer}{(16(c))}
  Let $\alpha = g(0)$.
  Then $\psi_{\alpha}$ is an automorphism of the unit disk that sends $\alpha$ to 0.
  Then $\psi_{\alpha} \circ g$ is an automorphism of the unit disk that fixes 0.
  By applying Lemma 2.1 to $\psi_{\alpha} \circ g$ and its inverse, we obtain that $\abs{\psi_{\alpha} \circ g} \leq 1$ and $\abs{(\psi_{\alpha} \circ g)^{-1}} \leq 1$.
  Thus $\abs{\psi_{\alpha} \circ g} = 1$.
  Therefore, $\psi_{\alpha} \circ g$ is a rotation by Lemma 2.1.
  By (a), $h = f^{-1} \circ \psi_{\alpha} \circ g \circ f$ is a Mobius transformation associated to a real matrix with determinant 1.
  Then $f^{-1} \circ g \circ f = f^{-1} \circ \psi_{\alpha}^{-1} \circ f \circ h$.
  By Part (b), $f^{-1} \circ \psi_{\alpha}^{-1} \circ f$ is a Mobius transformation associated to a real matrix with determinant 1 because $\psi_{\alpha}^{-1} = \psi_{\alpha}$.
  Since the composition of two Mobius transformations corresponds to the product of the two associated matrices, the composition corresponds to a real matrix whose determinant is 1.
\end{exer}

\end{document}


