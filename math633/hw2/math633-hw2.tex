\documentclass[12pt, psamsfonts]{amsart}

%-------Packages---------
\usepackage{amssymb,amsfonts}
\usepackage{semantic}
\usepackage{fullpage}
\usepackage{tikz-cd}
\usepackage{todonotes}
\usepackage{physics}
\usepackage[all,arc]{xy}
\usepackage{enumerate}
\usepackage{enumitem}
\usepackage{mathrsfs}
\usepackage{theoremref}
\usepackage{graphicx}
\usepackage[bookmarks]{hyperref}

%--------Theorem Environments--------
%theoremstyle{plain} --- default
\newtheorem{thm}{Theorem}[section]
\newtheorem{cor}[thm]{Corollary}
\newtheorem{prop}[thm]{Proposition}
\newtheorem{lem}[thm]{Lemma}
\newtheorem{conj}[thm]{Conjecture}
\newtheorem{quest}[thm]{Question}

\theoremstyle{definition}
\newtheorem{defn}[thm]{Definition}
\newtheorem{defns}[thm]{Definitions}
\newtheorem{con}[thm]{Construction}
\newtheorem{exmp}[thm]{Example}
\newtheorem{exmps}[thm]{Examples}
\newtheorem{notn}[thm]{Notation}
\newtheorem{notns}[thm]{Notations}
\newtheorem{addm}[thm]{Addendum}
\newtheorem*{exer}{Exercise}

\theoremstyle{remark}
\newtheorem{rem}[thm]{Remark}
\newtheorem{rems}[thm]{Remarks}
\newtheorem{warn}[thm]{Warning}
\newtheorem{sch}[thm]{Scholium}

\DeclareMathOperator{\Hom}{Hom}
\DeclareMathOperator{\Id}{Id}
\DeclareMathOperator{\End}{End}
\DeclareMathOperator{\ord}{ord}
\DeclareMathOperator{\Aut}{Aut}
\DeclareMathOperator{\Gal}{Gal}

\makeatletter
\let\c@equation\c@thm
\makeatother
\numberwithin{equation}{section}

\bibliographystyle{plain}

\begin{document}

\title{math 633(Homework 2)}
\author{Hidenori Shinohara}
\maketitle

\begin{exer}{(Problem 2a)}
  Let $\gamma(t) = Re^{2\pi it}$.
  \begin{align*}
    \int_{\gamma} z^ndz
      &= \int_{0}^{1} R^ne^{2\pi i nt}R2\pi ie^{2\pi it} dt \\
      &= 2\pi i R^{n + 1}\int_{0}^{1} e^{2\pi i (n + 1)t} dt \\
      &= \begin{cases}
        R^{n + 1} \frac{e^{2\pi i(n + 1)t}}{n + 1} = 0 & (n \ne -1) \\
        2\pi i & (n = -1).
      \end{cases}
  \end{align*}
\end{exer}

\begin{exer}{(Problem 2b)}
  Let $\gamma(t) = z_0 + Re^{2\pi it}$ where $\abs{R / z_0} < 1$.
  \begin{align*}
    \int_{\gamma} z^ndz
      &= \int_{0}^{1} (z_0 + Re^{2\pi it})^n(z_0 + Re^{e\pi it})' dt
  \end{align*}
  When $n \ne -1$, $(z_0 + Re^{2\pi it})^{n + 1} / (n + 1)$ is a primitive, so the integral is 0.
  Suppose $n = -1$.

  \begin{align*}
    \int_{0}^{1} \frac{2 \pi i Re^{2\pi it}}{z_0 + Re^{2\pi i t}} dt
      &= \int_{0}^{1} \frac{2 \pi i Re^{2\pi it} / z_0}{1 + Re^{2\pi i t} / z_0} dt \\
      &= \int_{0}^{1} \frac{2\pi i Re^{2\pi it}}{z_0} \cdot \sum_{k=0}^{\infty} (\frac{-Re^{2\pi it}}{z_0})^kdt \\
      &= -2\pi i\sum_{k=0}^{\infty} \int_0^1 (\frac{-Re^{2\pi it}}{z_0})^{k + 1} dt \\
      &= -2\pi i\sum_{k=0}^{\infty} (\frac{-Re^{2\pi it}}{z_0})^{k + 1}\int_0^1 e^{2\pi i t(k + 1)} dt \\
      &= 0.
  \end{align*}
  Each $\int_0^1 e^{2\pi i t(k + 1)} dt = 0$ because $e^{2\pi it(k + 1)} / (2\pi i t(k + 1))$ is a primitive.
\end{exer}

\begin{exer}{(Problem 3)}
  \begin{align*}
    \int_a^b \abs{z'(t)} dt
      &= \int_c^d \abs{z'(t(s))}t'(s) ds \\
      &= \int_c^d \abs{z'(t(s))t'(s)} ds \\
      &= \int_c^d \abs{\tilde{z}'(s)} ds
  \end{align*}
  
  where $\tilde{z}(s): [c, d] \rightarrow \mathbb{C}$ is a reparametrization of $z(t): [a, b] \rightarrow \mathbb{C}$.
\end{exer}

\begin{exer}{(Problem 4a)}
  If $t^{\star} \in \Omega_1$, then there exists an open neighborhood $U$ of $z(t^{\star})$ contained in $\Omega_1$.
  Then $z^{-1}(U)$ is a neighborhood of $t^{\star}$ in $[0, 1]$ because $z$ is continuous.
  Since $z(1) \in \Omega_2$, $t^{\star} \ne 1$.
  However, this implies the existence of $\epsilon > 0$ such that $t^{\star} + \epsilon < 1$ and $z(t^{\star} + \epsilon) \in \Omega_1$.
  This is a contradiction.

  If $t^{\star} \in \Omega_2$, then there exists an open neighborhood $U$ of $z(t^{\star})$ contained in $\Omega_2$.
  Since $U$ is open, $z^{-1}(U)$ is a neighborhood of $t^{\star}$ in $[0, 1]$, so $\exists \epsilon > 0$ such that $z(t^{\star} - \epsilon) \in \Omega_2$.

  In each case, we reached a contradiction, so $\Omega$ is not disconnected.
\end{exer}

\begin{exer}{(Problem 4b)}
  For every $v \in \Omega_1$, there exists an open set $U$ such that $v \in U \subset \Omega_1$.
  Then for any $v' \in U$, $v$ and $v'$ can be joined by $\gamma(t) = tv + (1 - t)v'$.
  Thus $U \subset \Omega_1$, so $\Omega_1$ is open.

  Let $v \in \Omega_2$.
  Suppose that for all $\epsilon > 0$, the open disk at $v$ with the radius $\epsilon$ is not contained in $\Omega_2$.
  Otherwise we are done.
  Let $v_0 = w$.
  For every $n \in \mathbb{N}$, choose $v_n \in D(v, 1/n) \setminus \Omega_2$.
  Then there exists a path between each $v_n$ and $w$.
  Moreover, there exists a path between $v_n$ and $v_{n - 1}$ for each $n$ and we will call it $\gamma_n$.
  Define $\gamma: [0, 1] \rightarrow \Omega$ such that for each $n \in \mathbb{N}$, $\gamma([1 - 1/n, 1 - 1/(n+1)])$ is the path $\gamma_n$ and $\gamma(1) = v$.
  Then $\gamma$ is a well-defined path from $w$ to $v$, which is a contradiction because $v \in \Omega_2$.

  Clearly, $\Omega_1 \cap \Omega_2 = \emptyset$ and $\Omega_1 \cup \Omega_2 = \Omega$.
  Since $w \in \Omega_1$, $\Omega_1 \ne \emptyset$, so $\Omega_2 = \emptyset$.
\end{exer}

\end{document}


