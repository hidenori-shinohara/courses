\documentclass[12pt, psamsfonts]{amsart}

%-------Packages---------
\usepackage{amssymb,amsfonts}
\usepackage{semantic}
\usepackage{fullpage}
\usepackage{tikz-cd}
\usepackage{todonotes}
\usepackage{physics}
\usepackage[all,arc]{xy}
\usepackage{enumerate}
\usepackage{enumitem}
\usepackage{mathrsfs}
\usepackage{theoremref}
\usepackage{graphicx}
\usepackage[bookmarks]{hyperref}

%--------Theorem Environments--------
%theoremstyle{plain} --- default
\newtheorem{thm}{Theorem}[section]
\newtheorem{cor}[thm]{Corollary}
\newtheorem{prop}[thm]{Proposition}
\newtheorem{lem}[thm]{Lemma}
\newtheorem{conj}[thm]{Conjecture}
\newtheorem{quest}[thm]{Question}

\theoremstyle{definition}
\newtheorem{defn}[thm]{Definition}
\newtheorem{defns}[thm]{Definitions}
\newtheorem{con}[thm]{Construction}
\newtheorem{exmp}[thm]{Example}
\newtheorem{exmps}[thm]{Examples}
\newtheorem{notn}[thm]{Notation}
\newtheorem{notns}[thm]{Notations}
\newtheorem{addm}[thm]{Addendum}
\newtheorem*{exer}{Exercise}

\theoremstyle{remark}
\newtheorem{rem}[thm]{Remark}
\newtheorem{rems}[thm]{Remarks}
\newtheorem{warn}[thm]{Warning}
\newtheorem{sch}[thm]{Scholium}

\DeclareMathOperator{\Hom}{Hom}
\DeclareMathOperator{\Id}{Id}
\DeclareMathOperator{\End}{End}
\DeclareMathOperator{\ord}{ord}
\DeclareMathOperator{\Aut}{Aut}
\DeclareMathOperator{\Gal}{Gal}

\makeatletter
\let\c@equation\c@thm
\makeatother
\numberwithin{equation}{section}

\bibliographystyle{plain}

\begin{document}

\title{Math 633 Homework 3}
\author{Hidenori Shinohara}
\maketitle

\begin{exer}{(Problem 1)}
  A simply connected space is clearly piecewise smooth simply connected.
  Let $\Omega$ be piecewise smooth simply connected and $\gamma_1, \gamma_2: [0, 1] \rightarrow \Omega$ be two continuous curves with the same end points.
  Since $\Omega$ is open, $\gamma_1(t)$ has an open ball around it that is contained in $\Omega$ for each $t \in [0, 1]$.
  Since $[0, 1]$ is compact and $\gamma_1$ is continuous, $\gamma_1([0, 1])$ is compact.
  Hence, there is a finite partition $0 = t_0 < t_1 < \cdots < t_{n - 1} < t_n = 1$ such that $\gamma_1([t_i, t_{i + 1}])$ is contained in an open ball $\subset \Omega$ for each $i$.
  Then $\gamma_1$ is homotopic to the curve $\gamma_{1'}$ that consists of $n$ straight lines, $i$th of which is the line between $\gamma_1(t_i)$ and $\gamma_1(t_{i - 1})$ where $i = 1, \cdots, n$.
  This can be shown by the ``straight-line" homotopy because $\gamma_1([t_{i - 1}, t_{i}])$ and the $i$th straight line are in an open ball contained in $\Omega$.

  A similar argument can be applied to show that $\gamma_2$ is homotopic to a curve $\gamma_{2'}$ that consists of finitely many straight lines.
  A curve consisting of finitely many straight lines is clearly piecewise smooth.

  Therefore, $\gamma_1 \sim \gamma_{1'} \sim \gamma_{2'} \sim \gamma_2$.
  Thus $\Omega$ is simply connected.
\end{exer}

\begin{exer}{(Problem 4)}
  \begin{itemize}
    \item
      $\Omega_1$ is simply connected because any two continuous curves with the same end points are joined by the straight-line homotopy.
    \item
      $\Omega_2$ is not simply connected because $\Omega_2$ is homeomorphic to $S^1$ which has a nontrivial fundamental group.
      In other words, $\phi: \theta \mapsto (a + b)e^{2\pi i\theta}/2$ is a continuous curve in $\Omega$ that is not homotopic to the constant curve at $(a + b) / 2$.
    \item
      $\Omega_3$ is simply connected because any two continuous curves with the same end points are joined by the straight-line homotopy.
      This is because those two curves must be both in $D_1(0)$, or they must be both in $D_1(2)$.
  \end{itemize}
\end{exer}

\end{document}


