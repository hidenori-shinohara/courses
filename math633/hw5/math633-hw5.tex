\documentclass[12pt, psamsfonts]{amsart}

%-------Packages---------
\usepackage{amssymb,amsfonts}
\usepackage{semantic}
\usepackage{fullpage}
\usepackage{tikz-cd}
\usepackage{todonotes}
\usepackage{physics}
\usepackage[all,arc]{xy}
\usepackage{enumerate}
\usepackage{enumitem}
\usepackage{mathrsfs}
\usepackage{theoremref}
\usepackage{graphicx}
\usepackage[bookmarks]{hyperref}

%--------Theorem Environments--------
%theoremstyle{plain} --- default
\newtheorem{thm}{Theorem}[section]
\newtheorem{cor}[thm]{Corollary}
\newtheorem{prop}[thm]{Proposition}
\newtheorem{lem}[thm]{Lemma}
\newtheorem{conj}[thm]{Conjecture}
\newtheorem{quest}[thm]{Question}

\theoremstyle{definition}
\newtheorem{defn}[thm]{Definition}
\newtheorem{defns}[thm]{Definitions}
\newtheorem{con}[thm]{Construction}
\newtheorem{exmp}[thm]{Example}
\newtheorem{exmps}[thm]{Examples}
\newtheorem{notn}[thm]{Notation}
\newtheorem{notns}[thm]{Notations}
\newtheorem{addm}[thm]{Addendum}
\newtheorem*{exer}{Exercise}

\theoremstyle{remark}
\newtheorem{rem}[thm]{Remark}
\newtheorem{rems}[thm]{Remarks}
\newtheorem{warn}[thm]{Warning}
\newtheorem{sch}[thm]{Scholium}

\DeclareMathOperator{\Hom}{Hom}
\DeclareMathOperator{\Id}{Id}
\DeclareMathOperator{\End}{End}
\DeclareMathOperator{\ord}{ord}
\DeclareMathOperator{\Aut}{Aut}
\DeclareMathOperator{\Gal}{Gal}

\makeatletter
\let\c@equation\c@thm
\makeatother
\numberwithin{equation}{section}

\bibliographystyle{plain}

\begin{document}

\title{Math 633 (Homework 5)}
\author{Hidenori Shinohara}
\maketitle

\begin{exer}{(Problem 1)}
  \todo[inline,caption={}]{
  }
\end{exer}

\begin{exer}{(Problem 2)}
  \todo[inline,caption={}]{
  }
\end{exer}

\begin{exer}{(Problem 3)}
  $p(z) = az + b$ with $a \ne 0$ are the only bijective polynomials.

  By the fundamental theorem of algebra, every polynomial $p(z)$ with coefficients in $\mathbb{C}$ is of the form $a\prod_{i=1}^{n}(z - a_i)$ for $a \ne 0, a_1, \cdots, a_n \in \mathbb{C}$.
  If $a_i \ne a_j$ for some $i, j$, then $p$ cannot be injective.
  Thus any bijective polynomials must be of the form $a(z - b)^n$ for some $a \ne 0$ and $b \in \mathbb{C}$.
  If $n \geq 2$, then $p(\omega + b) = a\omega^n = a$ where $\omega = e^{2\pi i j / n}$ where $j = 0, \cdots, n - 1$.
  Thus $n = 1$ if the polynomial is injective.
  In other words, any bijective polynomial must be linear.

  On the other hand, it is clear that any non-constant linear function is bijective.
\end{exer}

\end{document}


