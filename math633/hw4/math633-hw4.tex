\documentclass[12pt, psamsfonts]{amsart}

%-------Packages---------
\usepackage{amssymb,amsfonts}
\usepackage{semantic}
\usepackage{fullpage}
\usepackage{tikz-cd}
\usepackage{todonotes}
\usepackage{physics}
\usepackage[all,arc]{xy}
\usepackage{enumerate}
\usepackage{enumitem}
\usepackage{mathrsfs}
\usepackage{theoremref}
\usepackage{graphicx}
\usepackage[bookmarks]{hyperref}

%--------Theorem Environments--------
%theoremstyle{plain} --- default
\newtheorem{thm}{Theorem}[section]
\newtheorem{cor}[thm]{Corollary}
\newtheorem{prop}[thm]{Proposition}
\newtheorem{lem}[thm]{Lemma}
\newtheorem{conj}[thm]{Conjecture}
\newtheorem{quest}[thm]{Question}

\theoremstyle{definition}
\newtheorem{defn}[thm]{Definition}
\newtheorem{defns}[thm]{Definitions}
\newtheorem{con}[thm]{Construction}
\newtheorem{exmp}[thm]{Example}
\newtheorem{exmps}[thm]{Examples}
\newtheorem{notn}[thm]{Notation}
\newtheorem{notns}[thm]{Notations}
\newtheorem{addm}[thm]{Addendum}
\newtheorem*{exer}{Exercise}

\theoremstyle{remark}
\newtheorem{rem}[thm]{Remark}
\newtheorem{rems}[thm]{Remarks}
\newtheorem{warn}[thm]{Warning}
\newtheorem{sch}[thm]{Scholium}

\DeclareMathOperator{\Hom}{Hom}
\DeclareMathOperator{\Id}{Id}
\DeclareMathOperator{\End}{End}
\DeclareMathOperator{\ord}{ord}
\DeclareMathOperator{\Aut}{Aut}
\DeclareMathOperator{\Gal}{Gal}
\DeclareMathOperator{\Re}{Re}

\makeatletter
\let\c@equation\c@thm
\makeatother
\numberwithin{equation}{section}

\bibliographystyle{plain}

\begin{document}

\title{Math 633}
\author{Hidenori Shinohara}
\maketitle

\section{Homework 4}

\begin{exer}{(Problem 1)}
  $\abs{\exp(f)} = \exp(\Re(f))$.
  Since $\Re(f)$ is bounded above, $\exp(f)$ is bounded.
  By Liouville's theorem, $\exp(f)$ is constant.
  Thus $f$ is constant because $f$ is continuous and $\exp(z) = \exp(w)$ if and only if $z - w = 2k\pi i$ for some $k \in \mathbb{Z}$.
\end{exer}

\begin{exer}{(Problem 2)}
  Define
  \begin{align*}
    v(x, y) &= \int_{0}^{y} \frac{\partial u}{\partial x}(x, t) dt - \int_{0}^{x} \frac{\partial u}{\partial y}(t, 0) dt.
  \end{align*}
  This gives us:
  \begin{align*}
    v_x(x, y)
      &= \int_{0}^{y} \frac{\partial^2 u}{\partial x^2}(x, t) dt - \frac{\partial u}{\partial y}(x, 0) \\
      &= -\int_{0}^{y} \frac{\partial^2 u}{\partial t^2}(x, t) dt - \frac{\partial u}{\partial y}(x, 0) \\
      &= -(\frac{\partial u}{\partial y}(x, y) - \frac{\partial u}{\partial y}(x, 0)) - \frac{\partial u}{\partial y}(x, 0) \\
      &= -\frac{\partial u}{\partial y}(x, y) \\
      &= -u_y(x, y). \\
    v_y(x, y)
      &= \frac{\partial u}{\partial x}(x, y) - \int_0^x \frac{\partial^2 u}{\partial y^2}(t, 0) dt \\
      &= \frac{\partial u}{\partial x}(x, y) + \int_0^x \frac{\partial^2 u}{\partial x^2}(t, 0) dt \\
      &= \frac{\partial u}{\partial x}(x, y) + \frac{\partial u}{\partial x}(x, 0) - \frac{\partial u}{\partial x}(x, 0) \\
      &= \frac{\partial u}{\partial x}(x, y) \\
      &= u_x(x, y).
  \end{align*}
  By Theorem 2.4, $u + iv$ is holomorphic on $D$.
  Given two $v_1, v_2:D \rightarrow \mathbb{R}$ satisfying such properties, $(u + v_1i) - (u + v_2i)$ is a holomorphic function whose real value is always 0.
  By the Cauchy-Riemann equation, the derivative of $i(v_1 - v_2)$ must be 0.
  In other words, $v_1 - v_2$ must be constant.
\end{exer}


\end{document}


