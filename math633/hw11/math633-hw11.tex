\documentclass[12pt, psamsfonts]{amsart}

%-------Packages---------
\usepackage{amssymb,amsfonts}
\usepackage{semantic}
\usepackage{fullpage}
\usepackage{tikz-cd}
\usepackage{todonotes}
\usepackage{physics}
\usepackage[all,arc]{xy}
\usepackage{enumerate}
\usepackage{enumitem}
\usepackage{mathrsfs}
\usepackage{theoremref}
\usepackage{graphicx}
\usepackage[bookmarks]{hyperref}

%--------Theorem Environments--------
%theoremstyle{plain} --- default
\newtheorem{thm}{Theorem}[section]
\newtheorem{cor}[thm]{Corollary}
\newtheorem{prop}[thm]{Proposition}
\newtheorem{lem}[thm]{Lemma}
\newtheorem{conj}[thm]{Conjecture}
\newtheorem{quest}[thm]{Question}

\theoremstyle{definition}
\newtheorem{defn}[thm]{Definition}
\newtheorem{defns}[thm]{Definitions}
\newtheorem{con}[thm]{Construction}
\newtheorem{exmp}[thm]{Example}
\newtheorem{exmps}[thm]{Examples}
\newtheorem{notn}[thm]{Notation}
\newtheorem{notns}[thm]{Notations}
\newtheorem{addm}[thm]{Addendum}
\newtheorem*{exer}{Exercise}

\theoremstyle{remark}
\newtheorem{rem}[thm]{Remark}
\newtheorem{rems}[thm]{Remarks}
\newtheorem{warn}[thm]{Warning}
\newtheorem{sch}[thm]{Scholium}

\DeclareMathOperator{\Hom}{Hom}
\DeclareMathOperator{\Id}{Id}
\DeclareMathOperator{\End}{End}
\DeclareMathOperator{\ord}{ord}
\DeclareMathOperator{\Aut}{Aut}
\DeclareMathOperator{\Gal}{Gal}

\makeatletter
\let\c@equation\c@thm
\makeatother
\numberwithin{equation}{section}

\bibliographystyle{plain}

\begin{document}

\title{Math 633 (Homework 11)}
\author{Hidenori Shinohara}
\maketitle

\begin{thm}
  If a function $f$ is holomorphic in an open set that contains a simple closed piecewise-smooth curve $\Gamma$ and its interior, then
  \begin{align*}
    \int_{\Gamma} f = 0.
  \end{align*}
\end{thm}

\begin{proof}
  Let $\Omega$ denote the interior of $\Gamma$.
  Let $\mathcal{O}$ denote an open set on which $f$ is holomorphic and which contains $\Gamma$ and $\Omega$.
  Such an open set must exist as it is given in the problem statement.

  Choose $\epsilon > 0$ such that $N(x, \epsilon) \subset \mathcal{O}$ for each $x \in \Gamma$.
  This is possible because $\Gamma$ is a compact subset of an open set $\mathcal{O}$.

  Next, let $P_1, \cdots, P_n$ denote the consecutive points where smooth parts of $\Gamma$ join.
  We may pick $\delta < \epsilon / 10$ such that each circle $C_j$ centered at a point $P_j$ and of radius $\delta$ intersects $\Gamma$ in precisely two distinct points.
  This is possible by Lemma 2.10 because for each $P_j = \Gamma(t_j)$, $C_j$ intersects $\Gamma([0, t_j])$ and $\Gamma([t_j, 1])$ once each.

  These two points on $C_j$ determine two arcs of circles and one is entirely contained in $\Omega$ and the other one does not intersect $\Omega$.
  TODO

\end{proof}

\end{document}


