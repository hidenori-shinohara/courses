\documentclass[12pt, psamsfonts]{amsart}

%-------Packages---------
\usepackage{amssymb,amsfonts}
\usepackage{semantic}
\usepackage{fullpage}
\usepackage{tikz-cd}
\usepackage{todonotes}
\usepackage{physics}
\usepackage[all,arc]{xy}
\usepackage{enumerate}
\usepackage{enumitem}
\usepackage{mathrsfs}
\usepackage{theoremref}
\usepackage{graphicx}
\usepackage[bookmarks]{hyperref}

%--------Theorem Environments--------
%theoremstyle{plain} --- default
\newtheorem{thm}{Theorem}[section]
\newtheorem{cor}[thm]{Corollary}
\newtheorem{prop}[thm]{Proposition}
\newtheorem{lem}[thm]{Lemma}
\newtheorem{conj}[thm]{Conjecture}
\newtheorem{quest}[thm]{Question}

\theoremstyle{definition}
\newtheorem{defn}[thm]{Definition}
\newtheorem{defns}[thm]{Definitions}
\newtheorem{con}[thm]{Construction}
\newtheorem{exmp}[thm]{Example}
\newtheorem{exmps}[thm]{Examples}
\newtheorem{notn}[thm]{Notation}
\newtheorem{notns}[thm]{Notations}
\newtheorem{addm}[thm]{Addendum}
\newtheorem*{exer}{Exercise}

\theoremstyle{remark}
\newtheorem{rem}[thm]{Remark}
\newtheorem{rems}[thm]{Remarks}
\newtheorem{warn}[thm]{Warning}
\newtheorem{sch}[thm]{Scholium}

\DeclareMathOperator{\Hom}{Hom}
\DeclareMathOperator{\Id}{Id}
\DeclareMathOperator{\End}{End}
\DeclareMathOperator{\ord}{ord}
\DeclareMathOperator{\Aut}{Aut}
\DeclareMathOperator{\Gal}{Gal}

\makeatletter
\let\c@equation\c@thm
\makeatother
\numberwithin{equation}{section}

\bibliographystyle{plain}

\begin{document}

\title{Math 633 (Homework 1)}
\author{Hidenori Shinohara}
\maketitle

\begin{exer}{(Problem 1)}
  \begin{itemize}
    \item
      Let $x$ be a limit point of $\overline{A}$.
      If $x \in A$, $x \in \overline{A}$ and we are done.
      Suppose $x \notin A$.
      Then $x_n \rightarrow x$ for some $\{ x_n \} \subset \overline{A}$.
      If infinitely many terms in $\{ x_n \}$ are in $A$, then $x$ is a limit point of $A$.
      Suppose otherwise.
      Then for all sufficiently large $n \in \mathbb{N}$, there exists $\{ x_{n, i} \} \subset A$ that converges to $x_n$, so we can pick sufficiently large $j_n$ such that $\abs{x_n - x_{n, j_n}} < 1 / n$.
      Then $\{ x_{n, j_n} \}$ is a sequence of points in $A$ that converges to $x$, so $x$ is a limit point of $A$.
      Thus $x \in \overline{A}$.
    \item
      Let $z \in \overline{A} \setminus A$.
      $z$ is a limit point of $A$ and $A \subset B$, so $z$ is a limit point of $B$.
      Since $B$ is closed, $z \in B$.
  \end{itemize}
\end{exer}

\begin{exer}{(Problem 2)}
 \begin{itemize}
   \item
     Not open, not closed, not compact. The boundary is $\{ x + iy | \abs{x} = \abs{y} = 1 \}$.
   \item
     Not open. Closed. Compact. The boundary is $A$.
   \item
     Not open. Closed. Not compact. The boundary is the real line.
   \item
     Open. Not closed. Not compact. The boundary is $\{ 0 \}$.
 \end{itemize}
\end{exer}

\begin{exer}{(Problem 3)}
 \begin{itemize}
   \item
     $f'(z) = -1/z^2$.
   \item
     $\abs{z}^2 \cdot (1/z) = \overline{z}$, which is not differentiable anywhere on $\mathbb{C}$.
     Since $1 / z$ is differentiable everywhere on $z \ne 0$, $\abs{z}^2$ is not differentiable anywhere on $z \ne 0$.
     Thus $\abs{z}$ is not differentiable anywhere on $z \ne 0$.
     \begin{align*}
        \lim_{h \rightarrow 0} \frac{f(0 + h) - f(0)}{h}
     \end{align*}
     does not exist.
     This is because the limit is 1 with $h_n = 1 / n$, but the limit is -1 with $h_n = -1/n$.
     Therefore, $\abs{z}$ is nowhere differentiable.
 \end{itemize}
\end{exer}

\begin{exer}{(Problem 4)}
  \begin{itemize}
    \item
      $a \implies b$.
      Define
      \begin{align*}
       \psi(h) = \begin{cases}
         \frac{f(z_0 + h) - f(z_0)}{h} - w & (h \ne 0) \\
         0 & (h = 0).
       \end{cases}
      \end{align*}
      Then $\psi$ is defined on a neighborhood of $0$ because $z_0$ is an interior point of $U$.
      Moreover, $\lim_{h \rightarrow 0} \psi(h) = 0$.
    \item
      $b \implies c$.
      Locally, $\frac{\abs{f(z_0 + h) - f(z_0) - wh}}{{\abs{h}}} = \psi(h)$, so the limit is 0 as $h \rightarrow 0$.
    \item
      $c \implies a$.
      \begin{align*}
        \lim_{h \rightarrow 0} \frac{f(z_0 + h) - f(z_0)}{h} = w
          &\iff \lim_{h \rightarrow 0} \frac{f(z_0 + h) - f(z_0) - wh}{h} = 0 \\
          &\iff \lim_{h \rightarrow 0} \frac{\abs{f(z_0 + h) - f(z_0) - wh}}{\abs{h}} = 0.
      \end{align*}
      Thus $c \implies a$.
  \end{itemize}
\end{exer}

\begin{exer}{(Problem 5(i)(ii))}
  \begin{align*}
    &\lim_{h \rightarrow 0} \frac{(f + g)(z_0 + h) - (f + g)(z_0)}{h} \\
      &= \lim_{h \rightarrow 0} \frac{f(z_0 + h) - f(z_0)}{h} + \lim_{h \rightarrow 0} \frac{g(z_0 + h) - g(z_0)}{h} \\
      &= f'(z_0) + g'(z_0). \\
    &\lim_{h \rightarrow 0} \frac{(fg)(z_0 + h) - (fg)(z_0)}{h} \\
      &= \lim_{h \rightarrow 0} \frac{f(z_0 + h)g(z_0 + h) - f(z_0 + h)g(z_0) + f(z_0 + h)g(z_0) - f(z_0)g(z_0)}{h} \\
      &= \lim_{h \rightarrow 0} f(z_0 + h)\frac{g(z_0 + h) - g(z_0)}{h} + g(z_0)\lim_{h \rightarrow 0}\frac{f(z_0 + h) - f(z_0)}{h} \\
      &= f(z_0)g'(z_0) + g(z_0)f'(z_0).
  \end{align*}
\end{exer}

\begin{exer}{(Problem 5(iii))}
  \begin{align*}
    &\lim_{h \rightarrow 0} \frac{1/g(z_0 + h) - 1 / g(z_0)}{h} \\
      &= \lim_{h \rightarrow 0} \frac{1}{g(z_0)g(z_0+h)} \frac{g(z_0) - g(z_0 + h)}{h} \\
      &= \frac{g'(z_0)}{g^2(z_0)}.
  \end{align*}
  By applying Problem 5(ii), we obtain the quotient rule.
\end{exer}

\end{document}


