\documentclass[12pt, psamsfonts]{amsart}

%-------Packages---------
\usepackage{amssymb,amsfonts}
\usepackage{semantic}
\usepackage{fullpage}
\usepackage{tikz-cd}
\usepackage{todonotes}
\usepackage{physics}
\usepackage[all,arc]{xy}
\usepackage{enumerate}
\usepackage{enumitem}
\usepackage{mathrsfs}
\usepackage{theoremref}
\usepackage{graphicx}
\usepackage[bookmarks]{hyperref}

%--------Theorem Environments--------
%theoremstyle{plain} --- default
\newtheorem{thm}{Theorem}[section]
\newtheorem{cor}[thm]{Corollary}
\newtheorem{prop}[thm]{Proposition}
\newtheorem{lem}[thm]{Lemma}
\newtheorem{conj}[thm]{Conjecture}
\newtheorem{quest}[thm]{Question}

\theoremstyle{definition}
\newtheorem{defn}[thm]{Definition}
\newtheorem{defns}[thm]{Definitions}
\newtheorem{con}[thm]{Construction}
\newtheorem{exmp}[thm]{Example}
\newtheorem{exmps}[thm]{Examples}
\newtheorem{notn}[thm]{Notation}
\newtheorem{notns}[thm]{Notations}
\newtheorem{addm}[thm]{Addendum}
\newtheorem*{exer}{Exercise}

\theoremstyle{remark}
\newtheorem{rem}[thm]{Remark}
\newtheorem{rems}[thm]{Remarks}
\newtheorem{warn}[thm]{Warning}
\newtheorem{sch}[thm]{Scholium}

\DeclareMathOperator{\Hom}{Hom}
\DeclareMathOperator{\Id}{Id}
\DeclareMathOperator{\End}{End}
\DeclareMathOperator{\ord}{ord}
\DeclareMathOperator{\Aut}{Aut}
\DeclareMathOperator{\Gal}{Gal}

\makeatletter
\let\c@equation\c@thm
\makeatother
\numberwithin{equation}{section}

\bibliographystyle{plain}

\begin{document}

\title{Math 633 Midterm}
\author{Hidenori Shinohara}
\maketitle

\section{Goursat, Cauchy on the disc, and the proofs in Section 5 of Chapter 3.}

\begin{prop}[Goursat's Theorem]
  If $\Omega$ is an open set in $\mathbb{C}$, and $T \subset \Omega$ is a triangle whose interior is also contained in $\Omega$, then
  \begin{align*}
    \int_{T} f(z) dz = 0
  \end{align*}
  whenever $f$ is holomorphic in $\Omega$.
\end{prop}

\begin{proof}
$ $
  \begin{itemize}
    \item
      Let $T^0 = T$.
      Having created $T^i$, create 4 triangles from $T^i$ as shown in the textbook with the natural orientation.
      Then one of the 4 triangles, denoted by $T^{i + 1}$, must satisfy $\abs{\int_{T^i} f(z)dz} \leq 4\abs{\int_{T^{i + 1}} f(z)dz}$.
      Since $\{ T_i \}$ is a sequence of nonempty compact sets whose diameter diminishes, there must exist a unique point $z_0$ that belongs to all $T^i$.
    \item
      Since $f$ is holomorphic at $z_0$, $f(z) = f(z_0) + f'(z_0)(z - z_0) + \psi(z)(z - z_0)$ where $\psi(z) \rightarrow 0$ as $z \rightarrow z_0$.
    \item
      Since $f(z_0) + f'(z_0)(z - z_0)$ has a primitive, $\int_{T^n} f(z)dz = \int_{T^n}\psi(z)(z - z_0)dz$ for any $n$.
      $\abs{\int_{T^n}\psi(z)(z - z_0)dz} \leq \epsilon_ndp / 4^n$ where $\epsilon_n = \sup_{z \in T^n} \abs{\psi(z)}$, $d$ the diameter of $T$, and $p$ the perimeter of $T$.
      $\epsilon_n \rightarrow 0$ as $n \rightarrow \infty$, so $\abs{\int_{T} f(z)dz} \leq \epsilon_ndp = 0$ as $n \rightarrow 0$.
  \end{itemize}
\end{proof}


\end{document}


