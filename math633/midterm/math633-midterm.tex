\documentclass[12pt, psamsfonts]{amsart}

%-------Packages---------
\usepackage{amssymb,amsfonts}
\usepackage{semantic}
\usepackage{fullpage}
\usepackage{tikz-cd}
\usepackage{todonotes}
\usepackage{physics}
\usepackage[all,arc]{xy}
\usepackage{enumerate}
\usepackage{enumitem}
\usepackage{mathrsfs}
\usepackage{theoremref}
\usepackage{graphicx}
\usepackage[bookmarks]{hyperref}

%--------Theorem Environments--------
%theoremstyle{plain} --- default
\newtheorem{thm}{Theorem}[section]
\newtheorem{cor}[thm]{Corollary}
\newtheorem{prop}[thm]{Proposition}
\newtheorem{lem}[thm]{Lemma}
\newtheorem{conj}[thm]{Conjecture}
\newtheorem{quest}[thm]{Question}

\theoremstyle{definition}
\newtheorem{defn}[thm]{Definition}
\newtheorem{defns}[thm]{Definitions}
\newtheorem{con}[thm]{Construction}
\newtheorem{exmp}[thm]{Example}
\newtheorem{exmps}[thm]{Examples}
\newtheorem{notn}[thm]{Notation}
\newtheorem{notns}[thm]{Notations}
\newtheorem{addm}[thm]{Addendum}
\newtheorem*{exer}{Exercise}

\theoremstyle{remark}
\newtheorem{rem}[thm]{Remark}
\newtheorem{rems}[thm]{Remarks}
\newtheorem{warn}[thm]{Warning}
\newtheorem{sch}[thm]{Scholium}

\DeclareMathOperator{\Hom}{Hom}
\DeclareMathOperator{\Id}{Id}
\DeclareMathOperator{\End}{End}
\DeclareMathOperator{\ord}{ord}
\DeclareMathOperator{\Aut}{Aut}
\DeclareMathOperator{\Gal}{Gal}

\makeatletter
\let\c@equation\c@thm
\makeatother
\numberwithin{equation}{section}

\bibliographystyle{plain}

\begin{document}

\title{Math 633 Midterm}
\author{Hidenori Shinohara}
\maketitle

\section{Goursat, Cauchy on the disc, and the proofs in Section 5 of Chapter 3.}

\begin{prop}[Goursat's Theorem]
  If $\Omega$ is an open set in $\mathbb{C}$, and $T \subset \Omega$ is a triangle whose interior is also contained in $\Omega$, then
  \begin{align*}
    \int_{T} f(z) dz = 0
  \end{align*}
  whenever $f$ is holomorphic in $\Omega$.
\end{prop}

\begin{proof}
$ $
  \begin{itemize}
    \item
      Let $T^0 = T$.
      Having created $T^i$, create 4 triangles from $T^i$ as shown in the textbook with the natural orientation.
      Then one of the 4 triangles, denoted by $T^{i + 1}$, must satisfy $\abs{\int_{T^i} f(z)dz} \leq 4\abs{\int_{T^{i + 1}} f(z)dz}$.
      Since $\{ T_i \}$ is a sequence of nonempty compact sets whose diameter diminishes, there must exist a unique point $z_0$ that belongs to all $T^i$.
    \item
      Since $f$ is holomorphic at $z_0$, $f(z) = f(z_0) + f'(z_0)(z - z_0) + \psi(z)(z - z_0)$ where $\psi(z) \rightarrow 0$ as $z \rightarrow z_0$.
    \item
      Since $f(z_0) + f'(z_0)(z - z_0)$ has a primitive, $\int_{T^n} f(z)dz = \int_{T^n}\psi(z)(z - z_0)dz$ for any $n$.
      $\abs{\int_{T^n}\psi(z)(z - z_0)dz} \leq \epsilon_ndp / 4^n$ where $\epsilon_n = \sup_{z \in T^n} \abs{\psi(z)}$, $d$ the diameter of $T$, and $p$ the perimeter of $T$.
      $\epsilon_n \rightarrow 0$ as $n \rightarrow \infty$, so $\abs{\int_{T} f(z)dz} \leq \epsilon_ndp = 0$ as $n \rightarrow 0$.
  \end{itemize}
\end{proof}

\begin{prop}[Cauchy's Theorem for a Disk]
  Suppose $f$ is holomorphic in an open set containing the circle $C$ and its interior.
  Then
  \begin{align*}
    \int_C f(z) dz = 0.
  \end{align*}
\end{prop}

\begin{proof}
  Since $f$ has a primitive, the integral over a closed curve is 0. 
  \todo[inline,caption={}]{
    Do I need more than this?
  }
\end{proof}

\begin{prop}[Theorem 5.1]
  If $f$ is holomorphic in $\Omega$, then
  \begin{align*}
    \int_{\gamma_0} f(z)dz = \int_{\gamma_1}f(z) dz  
  \end{align*}
  whenever the two curves $\gamma_0$ and $\gamma_1$ are homotopic in $\Omega$.
\end{prop}

\begin{proof}
  $ $
  \begin{itemize}
    \item
      Let $F: (s, t) \mapsto \gamma_s(t)$ be a homotopy between $\gamma_0$ and $\gamma_1$.
      Let $\epsilon > 0$ be chosen such that $B(F(s, t), 3\epsilon) \subset \Omega$ for all $s, t$.
      Such an $\epsilon$ must exist because $F([0, 1]^2)$ is compact.
    \item
      Choose $\delta > 0$ such that $\sup_{t \in [0, 1]}\abs{\gamma_{s_1}(t) - \gamma_{s_2}(t)} < \epsilon}$ whenever $\abs{s_1 - s_2} < \delta$.
      Such a $\delta$ must exist because $F$ is uniformly continuous.
    \item
      Pick $\abs{s_1 - s_2} < \delta$.
      Choose discs $D_0, \cdots, D_n$ of radius $2\epsilon$ and points $\{ z_0, \cdots, z_{n + 1} \}$, $\{ w_0, \cdots, w_{n + 1} \}$ on $\gamma_{s_1}, \gamma_{s_2}$, respectively such that $z_i, z_{i + 1}, w_i, w_{i + 1} \in D_i$.
      Let $F_i$ denote the primitive of $f$ on $D_i$.
      Then $F_{i + 1}(z_{i + 1}) - F_i(w_{i + 1}) = F_{i + 1}(z_{i + 1}) - F_{i}(w_{i + 1})$.
    \item
      \begin{align*}
        \int_{\gamma_{s_1}} f - \int_{\gamma_{s_2}} f
          &= \sum_{i=0}^{n} [F_i(z_{i + 1}) - F_i(z_i)] - \sum_{i=0}^{n} [F_i(w_{i + 1}) - F_i(w_i)] \\
          &= \sum_{i=0}^{n} [F_i(z_{i + 1}) - F_i(z_i) - F_i(w_{i + 1}) + F_i(w_i)] \\
          &= F_n(z_{n + 1}) - F_n(w_{n + 1}) - (F_0(z_0) - F_0(w_0)) \\
          &= 0.
      \end{align*}
  \end{itemize}
\end{proof}

\begin{prop}[Theorem 5.2]
  Any holomorphic function in a simply connected domain has a primitive.
\end{prop}

\begin{proof}
  $ $
  \begin{itemize}
    \item
      Fix a point $z_0$ in $\Omega$ and define $F(z) = \int_{\gamma}f(w)dw$ where $\gamma$ is a path from $z_0$ to $z$.
      Then $F(z + h) - F(z) = \int_{\eta} f(w)dw$ where $\eta$ is the path from $z$ to $z + h$.
    \item
      Since $f$ is continuous at $z$, $f(w) = f(z) + \psi(w)$ where $\psi(w) \rightarrow 0$ as $w \rightarrow z$.
      Therefore, $F(z + h) - F(z) = f(z)\int_{\eta}dw + \int_{\eta} \psi(w)dw = f(z)h + \int_{\eta} \psi(w)dw$.
      Since $\abs{(\int_{\eta} \psi(w) dw) / h} \leq \sup_{w \in \eta} \abs{\psi(w)} = 0$ as $h \rightarrow 0$.
      Thus $\lim_{h \rightarrow 0} (F(z + h) - F(z)) / h = f(z)$.
  \end{itemize}
\end{proof}

\section{Lioville’s Theorem and the fundamental theorem of algebra}

\begin{prop}[Corollary 4.5(Liouville's Theorem)]
  If $f$ is entire and bounded, then $f$ is constant.
\end{prop}

\begin{proof}
  It suffices to prove that $f' = 0$ since $\mathcal{C}$ is connected
  $\forall z_0 \in \mathbb{C}, \forall R > 0, \abs{f'(z_0)} \leq B  / R$ by the Cauchy inequalities where $B$ is a bound for $f$.
  Let $R \rightarrow \infty$.
\end{proof}


\end{document}
