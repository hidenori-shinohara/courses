\documentclass[12pt, psamsfonts]{amsart}

%-------Packages---------
\usepackage{amssymb,amsfonts}
\usepackage{fullpage}
\usepackage{tikz-cd}
\usepackage{todonotes}
\usepackage{physics}
\usepackage[all,arc]{xy}
\usepackage{enumerate}
\usepackage{enumitem}
\usepackage{mathrsfs}
\usepackage{theoremref}
\usepackage{graphicx}
\usepackage[bookmarks]{hyperref}

%--------Theorem Environments--------
%theoremstyle{plain} --- default
\newtheorem{thm}{Theorem}[section]
\newtheorem{cor}[thm]{Corollary}
\newtheorem{prop}[thm]{Proposition}
\newtheorem{lem}[thm]{Lemma}
\newtheorem{conj}[thm]{Conjecture}
\newtheorem{quest}[thm]{Question}

\theoremstyle{definition}
\newtheorem{defn}[thm]{Definition}
\newtheorem{defns}[thm]{Definitions}
\newtheorem{con}[thm]{Construction}
\newtheorem{exmp}[thm]{Example}
\newtheorem{exmps}[thm]{Examples}
\newtheorem{notn}[thm]{Notation}
\newtheorem{notns}[thm]{Notations}
\newtheorem{addm}[thm]{Addendum}
\newtheorem*{exer}{Exercise}

\theoremstyle{remark}
\newtheorem{rem}[thm]{Remark}
\newtheorem{rems}[thm]{Remarks}
\newtheorem{warn}[thm]{Warning}
\newtheorem{sch}[thm]{Scholium}

\DeclareMathOperator{\Hom}{Hom}
\DeclareMathOperator{\Id}{Id}
\DeclareMathOperator{\End}{End}
\DeclareMathOperator{\ord}{ord}
\DeclareMathOperator{\Aut}{Aut}

\makeatletter
\let\c@equation\c@thm
\makeatother
\numberwithin{equation}{section}

\bibliographystyle{plain}

\begin{document}

\title{Root Test}
\author{Hidenori Shinohara}
\maketitle

\section{Absolute Convergence}

\begin{exmp}
$ $
  \begin{itemize}
    \item
      Does $\sum_{i=1}^{\infty} (\frac{-1}{3})^n$ converge?
      Yes, geometric.
    \item
      Does $\sum_{i=1}^{\infty} \abs{(\frac{-1}{3})^n}$ converge?
      \begin{align*}
        \sum_{i=1}^{\infty} \abs{\Big(\frac{-1}{3}\Big)^n}
          &= \abs{\frac{-1}{3}} + \abs{\Big(\frac{-1}{3}\Big)^2} + \abs{\Big(\frac{-1}{3}\Big)^3} + \cdots \\
          &= \frac{1}{3} + \Big(\frac{1}{3}\Big)^2 + \Big(\frac{1}{3}\Big)^2 + \cdots \\
          &= \sum_{i=1}^{\infty} \Big(\frac{1}{3}\Big)^n.
      \end{align*}
      Yes, geometric.
    \item
      Does $\sum_{i=1}^{\infty} \frac{(-1)^n}{n}$ converge?
      Yes, geometric.
    \item
      Does $\sum_{i=1}^{\infty} \abs{\frac{(-1)^n}{n}}$ converge?
      \begin{align*}
        \sum_{i=1}^{\infty} \abs{\frac{(-1)^n}{3}}
          &= \abs{\frac{-1}{1}} + \abs{\frac{(-1)^2}{2}} + \abs{\frac{(-1)^3}{3}} + \cdots \\
          &= \frac{1}{1} + \frac{1}{2} + \frac{1}{3} + \cdots \\
          &= \sum_{i=1}^{\infty} \frac{1}{n}.
      \end{align*}
      No, harmonic.
  \end{itemize}
\end{exmp}

\begin{defn}
  $\sum a_n$ is called absolutely convergent if $\sum \abs{a_n}$ is convergent.
\end{defn}

\begin{exmp}
$ $
  \begin{itemize}
    \item
      $\sum_{i=1}^{\infty} (\frac{-1}{3})^n$ converges and absolutely converges.
    \item
      $\sum_{i=1}^{\infty} \frac{(-1)^n}{n}$ converges, but does not absolutely converge.
  \end{itemize}
\end{exmp}

\begin{rem}
  Absolutely convergent $\implies$ Convergence.
  However, the converse is not always true.
  (See the example above.)
\end{rem}

\section{Root Test}

\begin{exmp}
$ $
  \begin{itemize}
    \item
      Does $\sum_{i=1}^{\infty} (\frac{2}{3})^n$ converge?
      Yes, geometric.
    \item
      Does $\sum_{i=1}^{\infty} (\frac{2}{3n - 2})^n$ converge?
      \begin{itemize}
        \item
          $(n = 1) \implies \frac{2}{3 \cdot 1 - 2} = 2$.
        \item
          $(n = 2) \implies (\frac{2}{3 \cdot 2 - 2})^2= \frac{1}{4}$.
        \item
          $(n = 3) \implies (\frac{2}{3 \cdot 3 - 2})^3 = \frac{8}{343}$.
      \end{itemize}
      This doesn't look like a geometric series.
      How can we tell the convergence?
  \end{itemize}
\end{exmp}

\begin{rem}
  But $\sum_{i=1}^{\infty} (\frac{2}{3n - 2})^n$ looks a bit like a geometric series!
  Recall: $\sum (\text{something})^n$ converges when $\abs{\text{something}} < 1$.
  If we were to do the same thing, we would want to check $\abs{\frac{2}{3n - 2}}$.
  This wouldn't make much sense because this would depend on the value of $n$.
  It turns out that we need to take the limit $n \rightarrow \infty$.
\end{rem}

\begin{thm}
  Let $L = \lim_{n \rightarrow \infty} \sqrt[n]{\abs{a_n}}$.
  \begin{itemize}
    \item
      $L < 1 \implies$ absolute convergence.
    \item
      $L > 1 \implies$ divergent.
    \item
      $L = 1 \implies$ inconclusive.
  \end{itemize}
\end{thm}

\begin{exer}
$ $
  \begin{itemize}
    \item
      $\sum_{n=1}^{\infty} (\frac{3n + 1}{4 - 2n})^n$.
      Diverges since $L = 9/4$.
    \item
      $\sum_{n=4}^{\infty} [\frac{(-5)^{1 + 2n}}{2^{5n - 3}}]^n$.
      Absolutely converges since $L = 25/32$.
  \end{itemize}
\end{exer}

\end{document}


